\documentclass[11pt]{amsart}
\usepackage{geometry}                % See geometry.pdf to learn the layout options. There are lots.
\geometry{letterpaper}                   % ... or a4paper or a5paper or ... 
%\geometry{landscape}                % Activate for for rotated page geometry
%\usepackage[parfill]{parskip}    % Activate to begin paragraphs with an empty line rather than an indent
\usepackage{graphicx}
\usepackage{color}
\usepackage{epsfig}
\usepackage{amsmath}
\usepackage{amssymb}
\usepackage{epstopdf}
\usepackage{etoolbox}
\usepackage{tikz}
\usepackage{caption}
\usepackage{subcaption}
\usepackage{xfrac}
\usepackage{import}
\usepackage[section]{placeins}

\usepackage{cleveref}
%\usepackage{autonum}


\DeclareGraphicsRule{.tif}{png}{.png}{`convert #1 `dirname #1`/`basename #1 .tif`.png}

\graphicspath{ {../} }
\graphicspath{ {./images/} }
\newcommand{\diagrampath}{./diagrams/}
\newcommand{\plotpath}{./plots}

\newcommand{\mathbi}[1]{\mathit{\mathbf{#1}}}

\newcommand\vstate[3]{%
	\mathbf{\underline{#1}}%
	\ifstrempty{#2}{}{[#2]}%
	\ifstrempty{#3}{}{\langle #3 \rangle}}
\newcommand\sstate[3]{%
	\mathit{\underline{#1}}%
	\ifstrempty{#2}{}{[#2]}%
	\ifstrempty{#3}{}{\langle #3 \rangle}%
	}


\title{Peridynamic Beams: \\ A Non-ordinary, State Based Model}
\author{James O'Grady \and John T. Foster}
%\date{}                                           % Activate to display a given date or no date

\begin{document}

\begin{abstract}
This paper develops a new peridynamic state based model to represent the bending of an Euler-Bernoulli beam.
This model is non-ordinary and derived from the concept of a rotational spring between bonds.
While multiple peridynamic material models capture the behavior of solid materials, this is the first 1D state based peridynamic model to resist bending.
For sufficiently homogenous and differentiable displacements, the model is shown to be equivalent to Eringen's nonlocal elasticity.
As the peridynamic horizon approaches 0, it reduces to the classical Euler-Bernoulli beam equations.
Simple test cases demonstrate the model's performance.
\end{abstract}

\maketitle

\section{Introduction}
The final goal of many mechanical engineering analyses is the prediction and description of material failure.
When processes such as fracture are modeled, the differential equations of classical mechanics are ill-defined at the resulting discontinuities.
A peridynamic formulation of continuum mechanics casts material behavior in terms of integral functions of displacement, so that discontinuities evolve naturally and require no special treatment.
Various peridynamic models capture the material behavior of 3-dimensional solid objects, but would be very expensive to implement for a thin plate or beam.
Other peridynamic models capture tension and compression in 1D bars and 2D plates, but these models do not resist transverse displacement.
A recent paper by Taylor and Steigmann \cite{taylor2013two} reduces a bond based 3D plate to two dimensions with an integral through the plate's thickness.

This paper begins to remedy that lack in the simplest case by presenting a peridynamic equivalent to an Euler-Bernoulli beam.
In addition to directly modeling a thin beam in bending, the simple beam case lays the theoretical framework for more complex peridynamic beam, plate, and shell bending models.
Because many analyses of interest are partly or wholly comprised of these types of features, their development is an important addition to the capabilities of peridynamic analysis.
%
%The second section of this paper provides a brief introduction to peridynamics, including state based models.
%The third section presents the state based beam model and demonstrates equivalence to classical Euler-Bernoulli beam theory in the limit of shrinking nonlocality.
%The fourth section demonstrates the model's equivalence to Eringen's nonlocal elasticity for small peridynamic horizons, and compares other gradient models.
%The fifth section demonstrates the beam model with simple test cases.
The remainder of this introduction reviews other nonlocal work and provides a brief introduction to peridynamics, including state based models.
\Cref{sec:ModelDev} presents the state based beam model and demonstrates equivalence to classical Euler-Bernoulli beam theory in the limit of shrinking nonlocality.
\Cref{sec:EringenCompare} demonstrates the model's equivalence to Eringen's nonlocal elasticity for small peridynamic horizons, and compares other gradient models.
\Cref{sec:Numerical} demonstrates the beam model with simple test cases.
%
\subsection{Nonlocal Beam Models}
\label{sec:NLbeams}
%
Nonlocal elasticity generally allows for forces at a point that are dependent on the material configuration of an entire body, rather than the configuration at that point.\cite{eringen1972nonlocal}
While long-range forces are obvious at the molecular model, material at larger scales is conventionally modeled as though internal forces are local or contact forces.\cite{kroner1967elasticity}
The result of such approximation is accurate for deformations that are homogenous, but introduces some inaccuracy for inhomogenous deformations like the propagation of waves with short wavelengths.
One way to distinguish between homogenous and inhomogenous deformations is to incorporate higher-order gradients of deformation.
While stress in classical elasticity is a function of the (first) gradient of deformation, Eringen's formulation of a nonlocal modulus in \cite{eringen1983differential} approximates a weighted sum of the first and second order gradients.
This introduces a length scale to the model and has the effect of smearing out local deformation inhomogeneities over the surrounding material, while maintaining the conventional result for homogenous deformations.

Previous work in the nonlocal mechanics of beams is motivated by the observed stiffening of nanoscale cantilevers.
Challamel and Wang demonstrate in \cite{Challamel2008small} that Eringen nonlocal elasticity cannot reproduce the scale stiffening, but that stiffening does result from other gradient-elastic models and models incorporating nonlocal curvature.
Because all of these models incorporate higher-order gradients of deformation, they impose stronger continuity requirements than classical elasticity, and are unsuitable for discontinuous displacements.
Because the gradients are evaluated locally, gradient models are called \textit{weakly nonlocal}.

%
\subsection{Peridynamics}
\label{sec:PDintro}
The term \textit{peridynamic} was coined by Silling in \cite{silling2000reformulation} and alludes to the fact that the force at a point is affected by nearby material configuration.
In contrast to gradient models, the peridynamic model is \textit{strongly nonlocal} and casts material behavior at a point as the \textit{integral} equation 
%
\begin{equation}
\label{eq:PDCoPV}
\rho(\mathbf{x})\ddot{\mathbf{u}}(\mathbf{x}) = \int_\Omega \mathbi{f}(\mathbf{x},\mathbf{q}) dV_\mathbf{q}  + \mathbf{b}(\mathbf{x}) \notag
\end{equation}
%
rather than the classical \textit{differential} equation.
Instead of the divergence of stress, we have the integral of a ``force" function $\mathbi{f}$ of the position vector $\mathbf{x}$ and the position vector $\mathbf{q}$ of a point within the body domain $\Omega$. 
This force function may depend on \(\mathbf{x}\), \(\mathbf{q}\), their deformed positions, the original and deformed positions of other points in \(\Omega\), history, etc.

Constitutive modeling of a wide variety of materials is accomplished by choosing the appropriate form for the force function.
While the simplest force functions recreate a one-parameter linear elastic solid material\cite{silling2000reformulation}, other force functions can be used to model nonlinear elasticity, plasticity, damage, and other behaviors\cite{silling2005peridynamic}.
To describe force functions that incorporate the behavior of all points in the nearby material (not just \(\mathbf{x}\) and \(\mathbf{q}\)), we must introduce the concept of a peridynamic state.

Introduced by Silling et al. in 2007~\cite{silling2007peridynamic}, states are functions of the behavior of the continuum points surrounding each location.
The most common states are scalar states and vector states which are scalar and vector valued, respectively.
Unlike multiplication by a second order tensor, which can only map vectors linearly to other vectors, vector states can produce highly nonlinear or even discontinuous mappings.
Important properties of states are magnitude and direction, while important operations include the addition and decomposition of states, inner and tensor products, and the Fr\'{e}chet derivative of a function with respect to a state~\cite{silling2007peridynamic}.

Conservation of linear momentum in the \textit{state based} peridynamic formulation results in the equation of motion,
%
\begin{equation}
\label{eq:PDstateEoM}
\rho(\mathbf{x})\ddot{\mathbf{u}}(\mathbf{x}) = \int_\Omega (\vstate{T}{\mathbf{x}}{\mathbf{q}-\mathbf{x}}-\vstate{T}{\mathbf{q}}{\mathbf{x}-\mathbf{q}}) dV_\mathbf{q}  + \mathbf{b}(\mathbf{x}),\notag
\end{equation}
%
in which $\vstate{T}{\;}{\;}$ is a \textit{force vector-state} that maps the vector in angle brackets, $\langle \rangle$, originating at the point in square brackets, [ ], to a force vector acting on that point.
The deformed image of the vector $\mathbf{q}-\mathbf{x}$ is defined as the \textit{deformation vector-state}, usually denoted $\vstate{Y}{}{}$ and formulated as shown in \cref{eq:PDdeformation}. 
%
\begin{equation}
\label{eq:PDdeformation}
\vstate{Y}{\mathbf{x}}{\mathbf{q}-\mathbf{x}} = (\mathbf{q}-\mathbf{x}) + (\mathbf{u}(\mathbf{q})-\mathbf{u}(\mathbf{x}))
\end{equation}
%

Just as stress and strain are work conjugate, so too are the force and deformation vector states for hyperelastic materials.
If the force state $\vstate{T}{}{}$ is always in the same direction as the deformation state $\vstate{Y}{}{}$, then the force exerted by a ``bond'' between points is in the same direction as the deformed bond, and the model is called \textit{ordinary}.  
Models in which they are not in the same direction are called \textit{non-ordinary}.
Silling et al. demonstrate the possibility of such models in \cite{silling2010peridynamic}, but very little work has touched on their use.
Foster et al. \cite{foster2010viscoplasticity} and Warren et. al. \cite{warren2009non} show that correspondence models, which approximate the deformation gradient and use it to calculate bond forces, result in non-ordinary state based constitutive models for deformation with large rotation.
%
\FloatBarrier
%
\section{Model Development}
\label{sec:ModelDev}
Consider the material model illustrated in \cref{fig:SimpleBondpair} in which every bond vector emanating from a point is connected by a rotational spring to its opposite emanating from that same point.
%
\begin{figure}[h]
\subinputfrom{\diagrampath}{simpleBondPair.eps_tex}
\caption{Illustration of a bond pair model that resists angular deformation}
\label{fig:SimpleBondpair}
\end{figure}
%
If we call the deformed angle between these bonds \(\theta\), and choose the potential energy of that spring to be \( w = \alpha [1 + \cos(\theta) ] \), we can recover the non-ordinary force state proposed by Silling in \cite{silling2007peridynamic}.
%
\begin{equation}
\label{eq:SillingForceNO}
\vstate{T}{}{\boldsymbol{\xi}} =\frac{-\alpha}{|\vstate{Y}{}{\boldsymbol{\xi}}|} \frac{\vstate{Y}{}{\boldsymbol{\xi}}}{|\vstate{Y}{}{\boldsymbol{\xi}}|} \times \left[\frac{\vstate{Y}{}{\boldsymbol{\xi}}}{|\vstate{Y}{}{\boldsymbol{\xi}}|} \times \frac{\vstate{Y}{}{-\boldsymbol{\xi}}}{|\vstate{Y}{}{-\boldsymbol{\xi}}|}\right]
\end{equation}
%
Though it looks complex, \cref{eq:SillingForceNO} indicates a bond force perpendicular to the deformed bond and in the plane containing both the deformed bond and its partner as illustrated in \cref{fig:Bondpair}. 
The force magnitude is proportional to the sine of the angle between the bonds divided by the length of the deformed bond. 
%
\begin{figure}[h]
\subinputfrom{\diagrampath}{bondPair.eps_tex}
\caption{Deformation and force vector states}
\label{fig:Bondpair}
\end{figure}
%
This response is consistent with the idea of a rotational spring between bonds as long as the change in angle is small. 
Because the potential energy and force states are functions of \textit{pairs} of peridynamic bonds, we will call this formulation a \textit{bond-pair model}. 
Other choices for the bond-pair potential function, such as \( w = (\pi - \theta)^2 \), are also possible, but result in more mathematically complex analysis.
\subsection{Energy Equivalence}
\label{sec:EnergyEq}
%
To determine an appropriate choice of $\alpha$, we desire our peridynamic model to have an equivalent strain energy density to a classical Euler-Bernoulli beam in the \emph{local limit}, i.e. when the nonlocal length scale vanishes.  We will begin with the assumptions from Euler beam theory: the length of the beam is much greater than thickness, vertical displacements are small, and rotations are small. For small vertical displacments we have
%
\begin{equation}
\theta(\vstate{Y}{}{\xi},\vstate{Y}{}{\mathbf{-\xi}}) \approx \pi-\frac{v(x+\xi)-2v(x)+v(x-\xi)}{\xi},
\label{eq:beamdtheta}
\end{equation}
%
where $v$ is the vertical displacement of material point.  Using a Taylor series to expand the right-hand-side of eq.~(\ref{eq:beamdtheta})  
%
\begin{equation}
\theta(\vstate{Y}{}{\xi},\vstate{Y}{}{\mathbf{-\xi}}) \approx \pi-\xi \frac{\partial^2 v}{\partial x^2}+\mathcal{O}(\xi^3) =  \pi-\xi \kappa +\mathcal{O}(\xi^3); 
\label{eq:beamdtheta2}
\end{equation}
with
\begin{equation}
\kappa = \frac{\partial^2 v}{\partial x^2}.\notag
\end{equation}
%
Substituting eq.~(\ref{eq:beamdtheta2}) into the equation for the strain energy density of a single bond-pair,
%
\begin{align}
\label{eq:continuousBeamw}
w = \omega(\xi) \alpha \left[1+\cos(\theta(\vstate{Y}{}{\xi},\vstate{Y}{}{\mathbf{-\xi}})) \right] = \omega(\xi) \alpha\frac{\xi^2}{2}(\kappa)^2 +\mathcal{O}(\xi^4).\notag
\end{align}
%
If we use a weighting function \(\omega(\xi)=\omega(|\xi|)\) and assume that the $\omega$ plays the role of a localization kernel, i.e. $\omega = 0 \, \forall \, \xi > \delta$, the resulting strain energy density, $W$, for any material point in the peridynamic beam is
%
\begin{equation}
W = \frac{\alpha}{2}\kappa^2 \int_{-\delta}^\delta \omega(\xi)\xi^2 {\rm d}\xi + \mathcal{O}(\delta^5).\notag
\end{equation}
%
Equating $W$ with the classical Euler-Bernoulli beam strain-energy density, $\Omega$, and taking the limit as $\delta \to 0$ we can solve for $\alpha$
%
\begin{align}
    \lim_{\delta \to 0}  W &= \Omega, \notag \\
    \frac{\alpha}{2} m \kappa^2 &= \frac{EI}{2} \kappa^2, \notag \\
    \alpha &= \frac{EI}{m},
\label{eq:alpha}
\end{align}
%
with 
\begin{equation}
    m = \int_{-\delta}^{\delta} \omega(\xi) \xi^2 {\rm d}\xi \notag.
\end{equation}


\subsection{Weighting Function and Perfect Plasticity}
\label{sec:WeightFunction}
The weighting function \(\omega(\xi)\) describes the relative contribution of each bond-pair, and can be defined according to physical or mathematical considerations. 
Consider an Euler beam in bending with curvature \(\kappa\). 
Fibers running parallel to the neutral axis of the beam are stretched in proportion to their distance from the neutral axis, with strain \(\epsilon = y\kappa\). 
If the fibers are linearly elastic, then the axial stress at each location is \(\sigma = E\epsilon = Ey\kappa\), and the contribution to supported moment \(dM = \kappa E y^2 dA\). 
By comparing the formulations for the moments carried by the Euler beam in \cref{fig:EulerBending} and those of the bond-pair beam in \cref{fig:BPBending}, we see some definite parallels.
%
\begin{figure}[h]
\subinputfrom{\diagrampath}{EulerBending.eps_tex}
\caption{Euler beam moment contribution}
\label{fig:EulerBending}
\end{figure}
%%
\begin{figure}[h]
\subinputfrom{\diagrampath}{BondPairBending.eps_tex}
\caption{Bond-pair moment contribution}
\label{fig:BPBending}
\end{figure}
%
\begin{align}
M_\text{classical}&=\int_{-\frac{t}{2}}^{\frac{t}{2}} \sigma \; y \; dA && = \int_{-\frac{t}{2}}^{\frac{t}{2}} E \kappa \; y^2 \; b(y) dy\notag \\
%
M_\text{PD}&=\int_{-\delta}^{\delta} \vstate{T}{}{\xi}\; \xi \; d\xi & = \int_{-\delta}^{\delta} \alpha \frac{\sin(\Delta\theta)}{|\xi|} \; \xi \; \omega(\xi) d\xi\: & \approx \int_{-\delta}^{\delta} \alpha \kappa |\xi| \; \omega(\xi) d\xi\notag
\end{align}
%
The term \(y\) is the distance from the beam's neutral axis and \(b(y)\) is the width of the beam at that distance from the neutral axis. 
The similarity between classical and peridynamic moment formulations suggests a possible formulation for the weighting function:
%
\begin{equation}
\omega(\xi) = |\xi| b\left(y\right) \text{at}\; y=\frac{\xi}{\delta} \frac{t}{2}\notag
\end{equation}
%
\begin{figure}[h]
\minipage[t]{0.5\textwidth}
 \centering
  \subinputfrom{\diagrampath}{WeightProfile_Uniform.eps_tex}
\caption{Weight function for a beam of rectangular cross-section}
\label{fig:WeightProfileUniform}
\endminipage\hfill
\minipage[t]{0.5\textwidth}%
  \centering
  \subinputfrom{\diagrampath}{WeightProfile_Ibeam.eps_tex}
\caption{Weight function for an I-beam}
\label{fig:WeightProfileIbeam}
\endminipage
\end{figure}
%
This weight function analogizes the relative contributions of bond pairs of different lengths to the relative contributions of fibers at different distances from the centerline. 
Examples are shown in \cref{fig:WeightProfileUniform,fig:WeightProfileIbeam}.
While this weighting function offers no advantages over a uniform weight function in the case of the linearly elastic beam, it offers a way to model advancing plasticity.

In a bent Elastic Perfectly Plastic beam, axial fibers are still stretched in proportion to their distance from the neutral axis, but the relationship \(\sigma = E\epsilon = Ey\kappa\) only holds for \(|\epsilon| = |y\kappa| < \epsilon_c\). 
For greater stretches, the relationship becomes \(\sigma = \pm E\epsilon_c \). 
To model this behavior, consider a bond pair with similar behavior: for angular deformation less than some critical angle, the model behaves as previously described, but the magnitude of the force remains constant above a critical deformation
%
\[ 
|\vstate{T}{}{\xi}| = 
  \begin{cases}
    \alpha \frac{\sin(\theta(\vstate{Y}{}{\xi},\vstate{Y}{}{\mathbf{-\xi}}))}{|\vstate{Y}{}{\xi}|} & \quad \text{if } \theta < \theta_c\\
    \alpha \frac{\sin(\theta_c)}{|\vstate{Y}{}{\xi}|} & \quad \text{if } \theta \geq \theta_c\
  \end{cases}
\]
%
to determine the critical angle \(\theta_c\), we let the onset of plasticity in pairs of the longest bonds to coincide with the onset of plasticity in the fibers at the top and bottom surfaces of the classical beam. 
For small curvatures \(\Delta\theta = \xi\kappa\implies\Delta\theta_c = \frac{2\delta\epsilon_c}{t}\). 
For curvatures \(|\kappa| > \kappa_c=\frac{2\epsilon_c}{t}\), the radius within which bonds are in the elastic region is \(\delta_e = \delta \frac{\kappa_c}{\kappa}\), and parallels the distance from the beam centerline that fibers are in the elastic region \(y_e = \frac{t}{2} \frac{\kappa_c}{\kappa}\)
%
\begin{align}
  M_\text{classical} &= 2 \int_{0}^{y_e}E b(y)y^2 \kappa dy +2 \int_{y_e}^{\frac{t}{2}}E b(y) \epsilon_c y dy\notag \\
  M_\text{PD} &= 2 \int_{0}^{\delta_e}\alpha \omega(\xi) \xi^2 \kappa d\xi +2 \int_{\delta_e}^{\delta}\alpha \omega(\xi) \Delta\theta_c \xi d\xi \notag
\end{align}
%
Of course, as long as the force is independent of history, this model only represents a nonlinear elastic material. 
By keeping track of the plastic deformation \(\theta^p (\xi) = \theta-\theta_c\) of each bond-pair, and applying it as an offset, we can reproduce the hysteresis associated with elastic-perfectly-plastic deformation.
%
%
\section{Relation to Eringen Nonlocality}
\label{sec:EringenCompare}
%If we relax our homogeneity assumption somewhat, we recover from \cref{eq:beamdtheta} slightly more complex expressions for change in angle
If we keep an additional term from the Taylor series approximation of \cref{eq:beamdtheta}, we recover a slightly more complex expressions for change in angle
%
\begin{equation}
\label{eq:beamdthetaHOT}
%\theta(\vstate{Y}{}{\xi},\vstate{Y}{}{\mathbf{-\xi}}) \approx \pi-\xi \frac{\partial^2 y}{\partial x^2} -\frac{\xi^3}{12} \frac{\partial^4 y}{\partial x^4}  +\mathcal{O}(\xi^5)=  \pi-\xi \kappa-\frac{\xi^3}{12} \kappa''+\mathcal{O}(\xi^5)\notag
\theta(\vstate{Y}{}{\xi},\vstate{Y}{}{\mathbf{-\xi}}) \approx \arctan\left(\pi-\xi \frac{\partial^2 v}{\partial x^2} -\frac{\xi^3}{12} \frac{\partial^4 v}{\partial x^4}  +\mathcal{O}(\xi^5)\right)\notag
\end{equation}
%
and for the strain energy (again substituting \(\kappa = v''\) for readability),
%
%\begin{equation}
%W \approx \int_{-\delta}^\delta \omega(\xi)\alpha(\frac{\xi^2}{2}\kappa^2+\frac{\xi^4}{12}\kappa\kappa''-\frac{3\; \xi^4}{8}\kappa^4+\mathcal{O}(\xi^6)) d\xi .
%\end{equation}
%
%
\begin{equation}
%W \approx \int_{-\delta}^\delta \omega(\xi)\alpha(\frac{\xi^2}{2}\kappa^2+\frac{\xi^4}{12}\kappa\kappa''-\frac{\xi^4}{24}\kappa^4+\mathcal{O}(\xi^6)) d\xi .\notag
W \approx \int_{-\delta}^\delta \omega(\xi)\alpha(\frac{\xi^2}{2}\kappa^2+\frac{\xi^4}{12}\kappa\kappa''-\frac{3\; \xi^4}{8}\kappa^4+\mathcal{O}(\xi^6)) d\xi .\notag
\end{equation}
%
As the horizon \(\delta\) becomes small, higher-order \(\xi\) terms become relatively less important, and \(\xi^4\kappa^4\) is dominated by \(\xi^2\kappa^2\) for large \(\kappa\) and by \(\xi^4\kappa\kappa''\) for small \(\kappa\).
We can picture a bending resistance based on the bond length and the nonlocal curvature  \(\bar{\kappa}\), so that 
%
\begin{align}
\label{eq:NLbending}
 \omega(\xi)\alpha\bar{\kappa}&=  \omega(\xi)\alpha(\kappa + \frac{\xi^2}{6}\kappa'') \implies  \\
W &\approx \int_{-\delta}^\delta \omega(\xi)\alpha \frac{\xi^2}{2}\kappa\bar{\kappa}d\xi .\notag
\end{align}
%  
The same analysis can be taken further to obtain higher-order energy terms with even powers of \(\xi\) and even order derivatives of \(\kappa\). 
Not all of these higher-order terms can be separated into the product of a local curvature and nonlocal bending resistance.

Eringen's model for nonlocal elasticity in \cite{eringen1983differential} begins with a nonlocal modulus (denoted here as \(K(|\mathbf{x}'-\mathbf{x}|,\tau)\)) that relates the nonlocal stress \(\mathbf{t}\) at a point to the classical (local) stress \(\boldsymbol{\sigma}\) in the nearby material through the integral
\begin{equation}
\mathbf{t} = \int_\mathbf{V} K(|\mathbf{x}'-\mathbf{x}|,\tau)\sigma(\mathbf{x}')dv(\mathbf{x}').\notag
\end{equation}
In the local limit these relationships take the form of higher-order gradients.
Using a 1-dimensional decaying exponential nonlocal modulus \(K(|x|,\tau)=\frac{1}{l\tau}e^{-\frac{|x|}{l\tau}}\)results in a relationship between \(t\) and \(\sigma\) 
\begin{align}
\left(1-\tau^2l^2\frac{\partial^2}{\partial x^2}\right)t&=\sigma,\notag
\end{align}
in which \(\tau^2l^2\) is a scale-based material parameter.
For well-behaved \(t\) and \(\sigma\) and small values of \(\sigma''''\) and \(\tau^2l^2\), we can see that this relationship could be reformulated as
\begin{equation}
\label{eq:NLstress}
t=\left(1+\tau^2l^2\frac{\partial^2}{\partial x^2}\right)\sigma.\notag
\end{equation}
If we consider the results of the previous section and let \(dM = y\sigma dA\) and \(\sigma = Ey\kappa\), the contribution to moment resulting from Eringen's nonlocal elasticity in a fiber at \(y\)
\begin{equation}
\label{eq:EringenMoment}
E y^2 (\kappa+\tau^2l^2\kappa''), \\
\end{equation}
and the resulting strain energy
\begin{equation}
\label{eq:EringenEnergy}
\int_{-\frac{t}{2}}^{\frac{t}{2}} b(y) E \frac{y^2}{2} \kappa (\kappa+\tau^2l^2\kappa'')  dy,\notag
\end{equation}
bear a striking resemblance to \cref{eq:NLbending}.

%This is consistent with Eringen's observation of nonlocal modulus as Green's function of a linear differential operator in~\cite{eringen1983differential}. 
The similarity between \cref{eq:NLbending,eq:EringenMoment} is not accidental; Eringen's gradient elasticity is the solution to the integral formulation of the nonlocal stress integral equation just as the peridynamic energy is an integral function of nonlocal displacements.

\section{Numerical Simulation}
\label{sec:Numerical}
\subsection{Discretized Model}
\label{sec:Discretized}
For analysis purposes, it is often useful to represent a continuum part with a discrete version. 
Discretizing the bond-pair model is primarily matter of exchanging integrals for sums. 
%
\begin{align}
%\label{eq:discreteBeamw}
w = \alpha \left[1+\cos(\theta(\vstate{Y}{}{\xi},\vstate{Y}{}{\mathbf{-\xi}})) \right] \approx \frac{\alpha}{2}\left(\frac{v(x+\xi)-2v(x)+v(x-\xi)}{\xi}\right)^2 \notag
\end{align}
%
\begin{align}
\label{eq:discretebeam}
\alpha &= \frac{c\; \Delta x}{m} ;\; c= EI ;\; m=\sum_{i=1}^n \omega(\xi_i)\xi_i^2 \implies \nonumber \\
W&=\Delta x \sum_{i=1}^n \frac{EI}{2}\left(\frac{v(x+\xi_i)-2v(x)+v(x-\xi_i)}{\xi_i}\right)^2\notag
\end{align}
%
Discretization of the original model results in the equation of motion
\begin{equation}
\rho(\mathbf{x})\mathbf{\ddot{u}}(\mathbf{x}) = \mathbf{f}(\mathbf{x})+\sum_i \omega(\xi_i)\left\{\frac{\alpha(\mathbf{x})}{|\mathbf{p}_i |}\frac{\mathbf{p}_i}{|\mathbf{p}_i |}\times\left[\frac{\mathbf{p}_i}{|\mathbf{p}_i |}\times \frac{\mathbf{q}_i}{|\mathbf{q}_i |}\right]-\frac{\alpha(\mathbf{x}+\boldsymbol{\xi}_i)}{|\mathbf{p}_i |}\frac{(-\mathbf{p}_i)}{|\mathbf{p}_i |}\times\left[\frac{(-\mathbf{p}_i)}{|\mathbf{p}_i |}\times \frac{\mathbf{r}_i}{|\mathbf{r}_i |}\right]\right\}\notag
\end{equation}
with
\begin{align}
\mathbf{p}_i &= \boldsymbol{\xi}_i+\mathbf{u}(\mathbf{x}+\boldsymbol{\xi}_i)-\mathbf{u}(\mathbf{x});\notag\\
\mathbf{q}_i &= -\boldsymbol{\xi}_i+\mathbf{u}(\mathbf{x}-\boldsymbol{\xi}_i)-\mathbf{u}(\mathbf{x});\notag\\
\mathbf{r}_i &= \boldsymbol{\xi}_i+\mathbf{u}(\mathbf{x}+2\boldsymbol{\xi}_i)-\mathbf{u}(\mathbf{x}+\boldsymbol{\xi}_i).\notag
\end{align}
and for small displacements and rotations in a uniform beam,
\begin{equation}
\rho(x)\ddot{v}(x) = f(x)+\alpha \sum_i 2\omega(\xi_i)\frac{v(x-2\xi_i)-4v(x-\xi_i)+6v(x)-4v(x+\xi_i)+v(x+2\xi_i)}{\xi_i^2}\notag
\end{equation}
Discretizing the bond-pair model as proposed above requires that nodes be evenly spaced along the entire beam, otherwise the displacement \(v(x-\xi_i)\) is ill-defined. 
For this reason, the discretization does not allow for areas of higher and lower ``resolution''. 
%
\subsection{Numerical Method}
\label{sec:NumMethod}
Model behavior was evaluated by implementing the discretized equation of motion.
The case of a beam simply supported at both ends was chosen for simplicity in both evaluation and comparison.
Boundary conditions such as clamped supports and applied moments require careful treatment to ensure both meaningful results and ease of computation.

Deformations for quasistatic loading were computed by implicitly solving the zero-acceleration discrete equation of motion with a Newton's Method solver.
%
\subsection{Results}
\label{sec:Results}
The simplest test case for this model is a linear-elastic beam with a square profile.
The models below were constrained to purely vertical movement of each point in order to reduce the total number of degrees of freedom, speeding evaluation and improving convergence.
For comparison, equivalent models are created and analyzed in Abaqus 6.12 to verify simple cases.
Even a coarse discretization successfully reproduces the elastically deformed beam shape in \cref{fig:elastic_g2000}.

\begin{figure}[h]
  \centering
  \scalebox{.65}{\input{\plotpath/elastic_h20_g2000_edit.pgf}}
  \caption{The uniform-load elastic beam is accurately modeled with few nodes}
  \label{fig:elastic_g2000}
\end{figure}

As an elastic-perfectly-plastic beam exceeds the elastic limit of its material, plastic zones begin to grow on the top and bottom of the beam's cross section.
This behavior is mimicked by the plasticity of the longest bond-pairs, producing the results shown in \cref{fig:eppu_h10_g2000}.
To accurately capture this phenomenon and model beam plasticity, a finer discretization is required.
\begin{figure}[h]
  \centering
  \scalebox{.65}{\input{\plotpath/eppu_h10_g2000_edit.pgf}}
  \caption{The elastic perfectly-plastic beam requires finer discretization}
  \label{fig:eppu_h10_g2000}
\end{figure}

A material that is plastically deformed does not return to its original state when unloaded.
For a beam in bending, the residual deformations can be seen in a beam that has been loaded beyond the onset of plastic deformation and then unloaded.
This result is observed in the bond-pair plasticity model, shown in \cref{fig:ResidualPlasticity}. 
Accurate residual deformation modeling requires both a relatively small horizon and a fairly large number of nodes.

%
\begin{figure}[h]
\minipage[t]{0.5\textwidth}
 \centering
  \scalebox{.45}{%% Creator: Matplotlib, PGF backend
%%
%% To include the figure in your LaTeX document, write
%%   \input{<filename>.pgf}
%%
%% Make sure the required packages are loaded in your preamble
%%   \usepackage{pgf}
%%
%% Figures using additional raster images can only be included by \input if
%% they are in the same directory as the main LaTeX file. For loading figures
%% from other directories you can use the `import` package
%%   \usepackage{import}
%% and then include the figures with
%%   \import{<path to file>}{<filename>.pgf}
%%
%% Matplotlib used the following preamble
%%
\begingroup%
\makeatletter%
\begin{pgfpicture}%
\pgfpathrectangle{\pgfpointorigin}{\pgfqpoint{6.000000in}{7.000000in}}%
\pgfusepath{use as bounding box}%
\begin{pgfscope}%
\pgfsetrectcap%
\pgfsetroundjoin%
\definecolor{currentfill}{rgb}{1.000000,1.000000,1.000000}%
\pgfsetfillcolor{currentfill}%
\pgfsetlinewidth{0.000000pt}%
\definecolor{currentstroke}{rgb}{1.000000,1.000000,1.000000}%
\pgfsetstrokecolor{currentstroke}%
\pgfsetdash{}{0pt}%
\pgfpathmoveto{\pgfqpoint{0.000000in}{0.000000in}}%
\pgfpathlineto{\pgfqpoint{6.000000in}{0.000000in}}%
\pgfpathlineto{\pgfqpoint{6.000000in}{7.000000in}}%
\pgfpathlineto{\pgfqpoint{0.000000in}{7.000000in}}%
\pgfpathclose%
\pgfusepath{fill}%
\end{pgfscope}%
\begin{pgfscope}%
\pgfsetrectcap%
\pgfsetroundjoin%
\definecolor{currentfill}{rgb}{1.000000,1.000000,1.000000}%
\pgfsetfillcolor{currentfill}%
\pgfsetlinewidth{0.000000pt}%
\definecolor{currentstroke}{rgb}{0.000000,0.000000,0.000000}%
\pgfsetstrokecolor{currentstroke}%
\pgfsetdash{}{0pt}%
\pgfpathmoveto{\pgfqpoint{0.750000in}{0.700000in}}%
\pgfpathlineto{\pgfqpoint{5.400000in}{0.700000in}}%
\pgfpathlineto{\pgfqpoint{5.400000in}{6.300000in}}%
\pgfpathlineto{\pgfqpoint{0.750000in}{6.300000in}}%
\pgfpathclose%
\pgfusepath{fill}%
\end{pgfscope}%
\begin{pgfscope}%
\pgfpathrectangle{\pgfqpoint{0.750000in}{0.700000in}}{\pgfqpoint{4.650000in}{5.600000in}} %
\pgfusepath{clip}%
\pgfsetrectcap%
\pgfsetroundjoin%
\pgfsetlinewidth{1.003750pt}%
\definecolor{currentstroke}{rgb}{0.000000,0.000000,1.000000}%
\pgfsetstrokecolor{currentstroke}%
\pgfsetdash{}{0pt}%
\pgfpathmoveto{\pgfqpoint{0.750000in}{6.300000in}}%
\pgfpathlineto{\pgfqpoint{2.168250in}{4.709869in}}%
\pgfpathlineto{\pgfqpoint{2.261250in}{4.610004in}}%
\pgfpathlineto{\pgfqpoint{2.331000in}{4.538555in}}%
\pgfpathlineto{\pgfqpoint{2.400750in}{4.470935in}}%
\pgfpathlineto{\pgfqpoint{2.447250in}{4.428340in}}%
\pgfpathlineto{\pgfqpoint{2.493750in}{4.387968in}}%
\pgfpathlineto{\pgfqpoint{2.540250in}{4.350010in}}%
\pgfpathlineto{\pgfqpoint{2.586750in}{4.314625in}}%
\pgfpathlineto{\pgfqpoint{2.633250in}{4.281970in}}%
\pgfpathlineto{\pgfqpoint{2.679750in}{4.252168in}}%
\pgfpathlineto{\pgfqpoint{2.726250in}{4.225375in}}%
\pgfpathlineto{\pgfqpoint{2.772750in}{4.201680in}}%
\pgfpathlineto{\pgfqpoint{2.819250in}{4.181188in}}%
\pgfpathlineto{\pgfqpoint{2.865750in}{4.163985in}}%
\pgfpathlineto{\pgfqpoint{2.912250in}{4.150125in}}%
\pgfpathlineto{\pgfqpoint{2.958750in}{4.139695in}}%
\pgfpathlineto{\pgfqpoint{3.005250in}{4.132713in}}%
\pgfpathlineto{\pgfqpoint{3.051750in}{4.129213in}}%
\pgfpathlineto{\pgfqpoint{3.098250in}{4.129213in}}%
\pgfpathlineto{\pgfqpoint{3.144750in}{4.132713in}}%
\pgfpathlineto{\pgfqpoint{3.191250in}{4.139695in}}%
\pgfpathlineto{\pgfqpoint{3.237750in}{4.150125in}}%
\pgfpathlineto{\pgfqpoint{3.284250in}{4.163985in}}%
\pgfpathlineto{\pgfqpoint{3.330750in}{4.181188in}}%
\pgfpathlineto{\pgfqpoint{3.377250in}{4.201680in}}%
\pgfpathlineto{\pgfqpoint{3.423750in}{4.225375in}}%
\pgfpathlineto{\pgfqpoint{3.470250in}{4.252168in}}%
\pgfpathlineto{\pgfqpoint{3.516750in}{4.281970in}}%
\pgfpathlineto{\pgfqpoint{3.563250in}{4.314625in}}%
\pgfpathlineto{\pgfqpoint{3.609750in}{4.350010in}}%
\pgfpathlineto{\pgfqpoint{3.656250in}{4.387985in}}%
\pgfpathlineto{\pgfqpoint{3.702750in}{4.428340in}}%
\pgfpathlineto{\pgfqpoint{3.749250in}{4.470935in}}%
\pgfpathlineto{\pgfqpoint{3.819000in}{4.538555in}}%
\pgfpathlineto{\pgfqpoint{3.888750in}{4.610006in}}%
\pgfpathlineto{\pgfqpoint{3.958500in}{4.684507in}}%
\pgfpathlineto{\pgfqpoint{4.051500in}{4.787078in}}%
\pgfpathlineto{\pgfqpoint{4.470000in}{5.256517in}}%
\pgfpathlineto{\pgfqpoint{5.400000in}{6.300000in}}%
\pgfpathlineto{\pgfqpoint{5.400000in}{6.300000in}}%
\pgfusepath{stroke}%
\end{pgfscope}%
\begin{pgfscope}%
\pgfpathrectangle{\pgfqpoint{0.750000in}{0.700000in}}{\pgfqpoint{4.650000in}{5.600000in}} %
\pgfusepath{clip}%
\pgfsetrectcap%
\pgfsetroundjoin%
\pgfsetlinewidth{1.003750pt}%
\definecolor{currentstroke}{rgb}{0.000000,0.500000,0.000000}%
\pgfsetstrokecolor{currentstroke}%
\pgfsetdash{}{0pt}%
\pgfpathmoveto{\pgfqpoint{0.750000in}{6.300000in}}%
\pgfpathlineto{\pgfqpoint{2.331000in}{2.162459in}}%
\pgfpathlineto{\pgfqpoint{2.424000in}{1.926583in}}%
\pgfpathlineto{\pgfqpoint{2.517000in}{1.697901in}}%
\pgfpathlineto{\pgfqpoint{2.563500in}{1.588027in}}%
\pgfpathlineto{\pgfqpoint{2.610000in}{1.482246in}}%
\pgfpathlineto{\pgfqpoint{2.656500in}{1.381554in}}%
\pgfpathlineto{\pgfqpoint{2.703000in}{1.286387in}}%
\pgfpathlineto{\pgfqpoint{2.749500in}{1.197288in}}%
\pgfpathlineto{\pgfqpoint{2.796000in}{1.115316in}}%
\pgfpathlineto{\pgfqpoint{2.842500in}{1.042198in}}%
\pgfpathlineto{\pgfqpoint{2.889000in}{0.979780in}}%
\pgfpathlineto{\pgfqpoint{2.935500in}{0.929681in}}%
\pgfpathlineto{\pgfqpoint{2.982000in}{0.893038in}}%
\pgfpathlineto{\pgfqpoint{3.028500in}{0.870713in}}%
\pgfpathlineto{\pgfqpoint{3.075000in}{0.863217in}}%
\pgfpathlineto{\pgfqpoint{3.121500in}{0.870713in}}%
\pgfpathlineto{\pgfqpoint{3.168000in}{0.893038in}}%
\pgfpathlineto{\pgfqpoint{3.214500in}{0.929681in}}%
\pgfpathlineto{\pgfqpoint{3.261000in}{0.979780in}}%
\pgfpathlineto{\pgfqpoint{3.307500in}{1.042198in}}%
\pgfpathlineto{\pgfqpoint{3.354000in}{1.115316in}}%
\pgfpathlineto{\pgfqpoint{3.400500in}{1.197288in}}%
\pgfpathlineto{\pgfqpoint{3.447000in}{1.286387in}}%
\pgfpathlineto{\pgfqpoint{3.493500in}{1.381554in}}%
\pgfpathlineto{\pgfqpoint{3.540000in}{1.482246in}}%
\pgfpathlineto{\pgfqpoint{3.586500in}{1.588027in}}%
\pgfpathlineto{\pgfqpoint{3.633000in}{1.697901in}}%
\pgfpathlineto{\pgfqpoint{3.726000in}{1.926583in}}%
\pgfpathlineto{\pgfqpoint{3.819000in}{2.162459in}}%
\pgfpathlineto{\pgfqpoint{3.958500in}{2.524837in}}%
\pgfpathlineto{\pgfqpoint{4.935000in}{5.082819in}}%
\pgfpathlineto{\pgfqpoint{5.400000in}{6.300000in}}%
\pgfpathlineto{\pgfqpoint{5.400000in}{6.300000in}}%
\pgfusepath{stroke}%
\end{pgfscope}%
\begin{pgfscope}%
\pgfpathrectangle{\pgfqpoint{0.750000in}{0.700000in}}{\pgfqpoint{4.650000in}{5.600000in}} %
\pgfusepath{clip}%
\pgfsetbuttcap%
\pgfsetmiterjoin%
\definecolor{currentfill}{rgb}{0.000000,0.500000,0.000000}%
\pgfsetfillcolor{currentfill}%
\pgfsetlinewidth{0.501875pt}%
\definecolor{currentstroke}{rgb}{0.000000,0.000000,0.000000}%
\pgfsetstrokecolor{currentstroke}%
\pgfsetdash{}{0pt}%
\pgfsys@defobject{currentmarker}{\pgfqpoint{-0.041667in}{-0.041667in}}{\pgfqpoint{0.041667in}{0.041667in}}{%
\pgfpathmoveto{\pgfqpoint{0.000000in}{0.041667in}}%
\pgfpathlineto{\pgfqpoint{-0.041667in}{-0.041667in}}%
\pgfpathlineto{\pgfqpoint{0.041667in}{-0.041667in}}%
\pgfpathclose%
\pgfusepath{stroke,fill}%
}%
\begin{pgfscope}%
\pgfsys@transformshift{1.122000in}{5.326292in}%
\pgfsys@useobject{currentmarker}{}%
\end{pgfscope}%
\begin{pgfscope}%
\pgfsys@transformshift{1.587000in}{4.108728in}%
\pgfsys@useobject{currentmarker}{}%
\end{pgfscope}%
\begin{pgfscope}%
\pgfsys@transformshift{2.052000in}{2.890680in}%
\pgfsys@useobject{currentmarker}{}%
\end{pgfscope}%
\begin{pgfscope}%
\pgfsys@transformshift{2.517000in}{1.697901in}%
\pgfsys@useobject{currentmarker}{}%
\end{pgfscope}%
\begin{pgfscope}%
\pgfsys@transformshift{2.982000in}{0.893038in}%
\pgfsys@useobject{currentmarker}{}%
\end{pgfscope}%
\begin{pgfscope}%
\pgfsys@transformshift{3.447000in}{1.286387in}%
\pgfsys@useobject{currentmarker}{}%
\end{pgfscope}%
\begin{pgfscope}%
\pgfsys@transformshift{3.912000in}{2.403244in}%
\pgfsys@useobject{currentmarker}{}%
\end{pgfscope}%
\begin{pgfscope}%
\pgfsys@transformshift{4.377000in}{3.621570in}%
\pgfsys@useobject{currentmarker}{}%
\end{pgfscope}%
\begin{pgfscope}%
\pgfsys@transformshift{4.842000in}{4.839325in}%
\pgfsys@useobject{currentmarker}{}%
\end{pgfscope}%
\begin{pgfscope}%
\pgfsys@transformshift{5.307000in}{6.056597in}%
\pgfsys@useobject{currentmarker}{}%
\end{pgfscope}%
\end{pgfscope}%
\begin{pgfscope}%
\pgfpathrectangle{\pgfqpoint{0.750000in}{0.700000in}}{\pgfqpoint{4.650000in}{5.600000in}} %
\pgfusepath{clip}%
\pgfsetrectcap%
\pgfsetroundjoin%
\pgfsetlinewidth{1.003750pt}%
\definecolor{currentstroke}{rgb}{1.000000,0.000000,0.000000}%
\pgfsetstrokecolor{currentstroke}%
\pgfsetdash{}{0pt}%
\pgfpathmoveto{\pgfqpoint{0.750000in}{6.300000in}}%
\pgfpathlineto{\pgfqpoint{2.307750in}{3.673852in}}%
\pgfpathlineto{\pgfqpoint{2.400750in}{3.522013in}}%
\pgfpathlineto{\pgfqpoint{2.470500in}{3.411590in}}%
\pgfpathlineto{\pgfqpoint{2.540250in}{3.305722in}}%
\pgfpathlineto{\pgfqpoint{2.586750in}{3.238378in}}%
\pgfpathlineto{\pgfqpoint{2.633250in}{3.174162in}}%
\pgfpathlineto{\pgfqpoint{2.679750in}{3.113661in}}%
\pgfpathlineto{\pgfqpoint{2.726250in}{3.057585in}}%
\pgfpathlineto{\pgfqpoint{2.772750in}{3.006515in}}%
\pgfpathlineto{\pgfqpoint{2.819250in}{2.961132in}}%
\pgfpathlineto{\pgfqpoint{2.842500in}{2.940692in}}%
\pgfpathlineto{\pgfqpoint{2.865750in}{2.921856in}}%
\pgfpathlineto{\pgfqpoint{2.889000in}{2.904731in}}%
\pgfpathlineto{\pgfqpoint{2.912250in}{2.889399in}}%
\pgfpathlineto{\pgfqpoint{2.935500in}{2.875934in}}%
\pgfpathlineto{\pgfqpoint{2.958750in}{2.864407in}}%
\pgfpathlineto{\pgfqpoint{2.982000in}{2.854902in}}%
\pgfpathlineto{\pgfqpoint{3.005250in}{2.847466in}}%
\pgfpathlineto{\pgfqpoint{3.028500in}{2.842125in}}%
\pgfpathlineto{\pgfqpoint{3.051750in}{2.838908in}}%
\pgfpathlineto{\pgfqpoint{3.075000in}{2.837833in}}%
\pgfpathlineto{\pgfqpoint{3.098250in}{2.838908in}}%
\pgfpathlineto{\pgfqpoint{3.121500in}{2.842125in}}%
\pgfpathlineto{\pgfqpoint{3.144750in}{2.847466in}}%
\pgfpathlineto{\pgfqpoint{3.168000in}{2.854902in}}%
\pgfpathlineto{\pgfqpoint{3.191250in}{2.864407in}}%
\pgfpathlineto{\pgfqpoint{3.214500in}{2.875934in}}%
\pgfpathlineto{\pgfqpoint{3.237750in}{2.889399in}}%
\pgfpathlineto{\pgfqpoint{3.261000in}{2.904731in}}%
\pgfpathlineto{\pgfqpoint{3.284250in}{2.921856in}}%
\pgfpathlineto{\pgfqpoint{3.307500in}{2.940692in}}%
\pgfpathlineto{\pgfqpoint{3.330750in}{2.961132in}}%
\pgfpathlineto{\pgfqpoint{3.377250in}{3.006515in}}%
\pgfpathlineto{\pgfqpoint{3.423750in}{3.057585in}}%
\pgfpathlineto{\pgfqpoint{3.470250in}{3.113661in}}%
\pgfpathlineto{\pgfqpoint{3.516750in}{3.174162in}}%
\pgfpathlineto{\pgfqpoint{3.563250in}{3.238378in}}%
\pgfpathlineto{\pgfqpoint{3.609750in}{3.305722in}}%
\pgfpathlineto{\pgfqpoint{3.679500in}{3.411590in}}%
\pgfpathlineto{\pgfqpoint{3.749250in}{3.522013in}}%
\pgfpathlineto{\pgfqpoint{3.842250in}{3.673852in}}%
\pgfpathlineto{\pgfqpoint{3.981750in}{3.907163in}}%
\pgfpathlineto{\pgfqpoint{5.121000in}{5.829335in}}%
\pgfpathlineto{\pgfqpoint{5.400000in}{6.300000in}}%
\pgfpathlineto{\pgfqpoint{5.400000in}{6.300000in}}%
\pgfusepath{stroke}%
\end{pgfscope}%
\begin{pgfscope}%
\pgfpathrectangle{\pgfqpoint{0.750000in}{0.700000in}}{\pgfqpoint{4.650000in}{5.600000in}} %
\pgfusepath{clip}%
\pgfsetbuttcap%
\pgfsetmiterjoin%
\definecolor{currentfill}{rgb}{1.000000,0.000000,0.000000}%
\pgfsetfillcolor{currentfill}%
\pgfsetlinewidth{0.501875pt}%
\definecolor{currentstroke}{rgb}{0.000000,0.000000,0.000000}%
\pgfsetstrokecolor{currentstroke}%
\pgfsetdash{}{0pt}%
\pgfsys@defobject{currentmarker}{\pgfqpoint{-0.041667in}{-0.041667in}}{\pgfqpoint{0.041667in}{0.041667in}}{%
\pgfpathmoveto{\pgfqpoint{0.041667in}{-0.000000in}}%
\pgfpathlineto{\pgfqpoint{-0.041667in}{0.041667in}}%
\pgfpathlineto{\pgfqpoint{-0.041667in}{-0.041667in}}%
\pgfpathclose%
\pgfusepath{stroke,fill}%
}%
\begin{pgfscope}%
\pgfsys@transformshift{0.796500in}{6.221557in}%
\pgfsys@useobject{currentmarker}{}%
\end{pgfscope}%
\begin{pgfscope}%
\pgfsys@transformshift{1.261500in}{5.437096in}%
\pgfsys@useobject{currentmarker}{}%
\end{pgfscope}%
\begin{pgfscope}%
\pgfsys@transformshift{1.726500in}{4.652570in}%
\pgfsys@useobject{currentmarker}{}%
\end{pgfscope}%
\begin{pgfscope}%
\pgfsys@transformshift{2.191500in}{3.868035in}%
\pgfsys@useobject{currentmarker}{}%
\end{pgfscope}%
\begin{pgfscope}%
\pgfsys@transformshift{2.656500in}{3.143402in}%
\pgfsys@useobject{currentmarker}{}%
\end{pgfscope}%
\begin{pgfscope}%
\pgfsys@transformshift{3.121500in}{2.842125in}%
\pgfsys@useobject{currentmarker}{}%
\end{pgfscope}%
\begin{pgfscope}%
\pgfsys@transformshift{3.586500in}{3.271685in}%
\pgfsys@useobject{currentmarker}{}%
\end{pgfscope}%
\begin{pgfscope}%
\pgfsys@transformshift{4.051500in}{4.024823in}%
\pgfsys@useobject{currentmarker}{}%
\end{pgfscope}%
\begin{pgfscope}%
\pgfsys@transformshift{4.516500in}{4.809479in}%
\pgfsys@useobject{currentmarker}{}%
\end{pgfscope}%
\begin{pgfscope}%
\pgfsys@transformshift{4.981500in}{5.593993in}%
\pgfsys@useobject{currentmarker}{}%
\end{pgfscope}%
\end{pgfscope}%
\begin{pgfscope}%
\pgfpathrectangle{\pgfqpoint{0.750000in}{0.700000in}}{\pgfqpoint{4.650000in}{5.600000in}} %
\pgfusepath{clip}%
\pgfsetrectcap%
\pgfsetroundjoin%
\pgfsetlinewidth{1.003750pt}%
\definecolor{currentstroke}{rgb}{0.000000,0.750000,0.750000}%
\pgfsetstrokecolor{currentstroke}%
\pgfsetdash{}{0pt}%
\pgfpathmoveto{\pgfqpoint{0.750000in}{6.300000in}}%
\pgfpathlineto{\pgfqpoint{2.321700in}{4.431354in}}%
\pgfpathlineto{\pgfqpoint{2.414700in}{4.324776in}}%
\pgfpathlineto{\pgfqpoint{2.489100in}{4.242679in}}%
\pgfpathlineto{\pgfqpoint{2.554200in}{4.174200in}}%
\pgfpathlineto{\pgfqpoint{2.610000in}{4.118760in}}%
\pgfpathlineto{\pgfqpoint{2.656500in}{4.075348in}}%
\pgfpathlineto{\pgfqpoint{2.703000in}{4.034912in}}%
\pgfpathlineto{\pgfqpoint{2.749500in}{3.997892in}}%
\pgfpathlineto{\pgfqpoint{2.786700in}{3.970993in}}%
\pgfpathlineto{\pgfqpoint{2.823900in}{3.946772in}}%
\pgfpathlineto{\pgfqpoint{2.861100in}{3.925414in}}%
\pgfpathlineto{\pgfqpoint{2.898300in}{3.907101in}}%
\pgfpathlineto{\pgfqpoint{2.926200in}{3.895492in}}%
\pgfpathlineto{\pgfqpoint{2.954100in}{3.885770in}}%
\pgfpathlineto{\pgfqpoint{2.982000in}{3.877991in}}%
\pgfpathlineto{\pgfqpoint{3.009900in}{3.872204in}}%
\pgfpathlineto{\pgfqpoint{3.037800in}{3.868448in}}%
\pgfpathlineto{\pgfqpoint{3.065700in}{3.866736in}}%
\pgfpathlineto{\pgfqpoint{3.093600in}{3.867078in}}%
\pgfpathlineto{\pgfqpoint{3.121500in}{3.869473in}}%
\pgfpathlineto{\pgfqpoint{3.149400in}{3.873908in}}%
\pgfpathlineto{\pgfqpoint{3.177300in}{3.880365in}}%
\pgfpathlineto{\pgfqpoint{3.205200in}{3.888797in}}%
\pgfpathlineto{\pgfqpoint{3.233100in}{3.899155in}}%
\pgfpathlineto{\pgfqpoint{3.261000in}{3.911379in}}%
\pgfpathlineto{\pgfqpoint{3.298200in}{3.930478in}}%
\pgfpathlineto{\pgfqpoint{3.335400in}{3.952565in}}%
\pgfpathlineto{\pgfqpoint{3.372600in}{3.977476in}}%
\pgfpathlineto{\pgfqpoint{3.409800in}{4.005007in}}%
\pgfpathlineto{\pgfqpoint{3.447000in}{4.034912in}}%
\pgfpathlineto{\pgfqpoint{3.493500in}{4.075347in}}%
\pgfpathlineto{\pgfqpoint{3.540000in}{4.118760in}}%
\pgfpathlineto{\pgfqpoint{3.595800in}{4.174200in}}%
\pgfpathlineto{\pgfqpoint{3.660900in}{4.242678in}}%
\pgfpathlineto{\pgfqpoint{3.735300in}{4.324776in}}%
\pgfpathlineto{\pgfqpoint{3.828300in}{4.431354in}}%
\pgfpathlineto{\pgfqpoint{3.958500in}{4.584603in}}%
\pgfpathlineto{\pgfqpoint{4.442100in}{5.160064in}}%
\pgfpathlineto{\pgfqpoint{5.400000in}{6.300000in}}%
\pgfpathlineto{\pgfqpoint{5.400000in}{6.300000in}}%
\pgfusepath{stroke}%
\end{pgfscope}%
\begin{pgfscope}%
\pgfpathrectangle{\pgfqpoint{0.750000in}{0.700000in}}{\pgfqpoint{4.650000in}{5.600000in}} %
\pgfusepath{clip}%
\pgfsetbuttcap%
\pgfsetmiterjoin%
\definecolor{currentfill}{rgb}{0.000000,0.750000,0.750000}%
\pgfsetfillcolor{currentfill}%
\pgfsetlinewidth{0.501875pt}%
\definecolor{currentstroke}{rgb}{0.000000,0.000000,0.000000}%
\pgfsetstrokecolor{currentstroke}%
\pgfsetdash{}{0pt}%
\pgfsys@defobject{currentmarker}{\pgfqpoint{-0.041667in}{-0.041667in}}{\pgfqpoint{0.041667in}{0.041667in}}{%
\pgfpathmoveto{\pgfqpoint{-0.000000in}{-0.041667in}}%
\pgfpathlineto{\pgfqpoint{0.041667in}{0.041667in}}%
\pgfpathlineto{\pgfqpoint{-0.041667in}{0.041667in}}%
\pgfpathclose%
\pgfusepath{stroke,fill}%
}%
\begin{pgfscope}%
\pgfsys@transformshift{1.001100in}{6.001186in}%
\pgfsys@useobject{currentmarker}{}%
\end{pgfscope}%
\begin{pgfscope}%
\pgfsys@transformshift{1.466100in}{5.447819in}%
\pgfsys@useobject{currentmarker}{}%
\end{pgfscope}%
\begin{pgfscope}%
\pgfsys@transformshift{1.931100in}{4.894446in}%
\pgfsys@useobject{currentmarker}{}%
\end{pgfscope}%
\begin{pgfscope}%
\pgfsys@transformshift{2.396100in}{4.345801in}%
\pgfsys@useobject{currentmarker}{}%
\end{pgfscope}%
\begin{pgfscope}%
\pgfsys@transformshift{2.861100in}{3.925414in}%
\pgfsys@useobject{currentmarker}{}%
\end{pgfscope}%
\begin{pgfscope}%
\pgfsys@transformshift{3.326100in}{3.946771in}%
\pgfsys@useobject{currentmarker}{}%
\end{pgfscope}%
\begin{pgfscope}%
\pgfsys@transformshift{3.791100in}{4.388312in}%
\pgfsys@useobject{currentmarker}{}%
\end{pgfscope}%
\begin{pgfscope}%
\pgfsys@transformshift{4.256100in}{4.938714in}%
\pgfsys@useobject{currentmarker}{}%
\end{pgfscope}%
\begin{pgfscope}%
\pgfsys@transformshift{4.721100in}{5.492089in}%
\pgfsys@useobject{currentmarker}{}%
\end{pgfscope}%
\begin{pgfscope}%
\pgfsys@transformshift{5.186100in}{6.045455in}%
\pgfsys@useobject{currentmarker}{}%
\end{pgfscope}%
\end{pgfscope}%
\begin{pgfscope}%
\pgfpathrectangle{\pgfqpoint{0.750000in}{0.700000in}}{\pgfqpoint{4.650000in}{5.600000in}} %
\pgfusepath{clip}%
\pgfsetbuttcap%
\pgfsetroundjoin%
\pgfsetlinewidth{0.501875pt}%
\definecolor{currentstroke}{rgb}{0.000000,0.000000,0.000000}%
\pgfsetstrokecolor{currentstroke}%
\pgfsetdash{{1.000000pt}{3.000000pt}}{0.000000pt}%
\pgfpathmoveto{\pgfqpoint{0.750000in}{0.700000in}}%
\pgfpathlineto{\pgfqpoint{0.750000in}{6.300000in}}%
\pgfusepath{stroke}%
\end{pgfscope}%
\begin{pgfscope}%
\pgfsetbuttcap%
\pgfsetroundjoin%
\definecolor{currentfill}{rgb}{0.000000,0.000000,0.000000}%
\pgfsetfillcolor{currentfill}%
\pgfsetlinewidth{0.501875pt}%
\definecolor{currentstroke}{rgb}{0.000000,0.000000,0.000000}%
\pgfsetstrokecolor{currentstroke}%
\pgfsetdash{}{0pt}%
\pgfsys@defobject{currentmarker}{\pgfqpoint{0.000000in}{0.000000in}}{\pgfqpoint{0.000000in}{0.055556in}}{%
\pgfpathmoveto{\pgfqpoint{0.000000in}{0.000000in}}%
\pgfpathlineto{\pgfqpoint{0.000000in}{0.055556in}}%
\pgfusepath{stroke,fill}%
}%
\begin{pgfscope}%
\pgfsys@transformshift{0.750000in}{0.700000in}%
\pgfsys@useobject{currentmarker}{}%
\end{pgfscope}%
\end{pgfscope}%
\begin{pgfscope}%
\pgfsetbuttcap%
\pgfsetroundjoin%
\definecolor{currentfill}{rgb}{0.000000,0.000000,0.000000}%
\pgfsetfillcolor{currentfill}%
\pgfsetlinewidth{0.501875pt}%
\definecolor{currentstroke}{rgb}{0.000000,0.000000,0.000000}%
\pgfsetstrokecolor{currentstroke}%
\pgfsetdash{}{0pt}%
\pgfsys@defobject{currentmarker}{\pgfqpoint{0.000000in}{-0.055556in}}{\pgfqpoint{0.000000in}{0.000000in}}{%
\pgfpathmoveto{\pgfqpoint{0.000000in}{0.000000in}}%
\pgfpathlineto{\pgfqpoint{0.000000in}{-0.055556in}}%
\pgfusepath{stroke,fill}%
}%
\begin{pgfscope}%
\pgfsys@transformshift{0.750000in}{6.300000in}%
\pgfsys@useobject{currentmarker}{}%
\end{pgfscope}%
\end{pgfscope}%
\begin{pgfscope}%
\pgftext[left,bottom,x=0.645738in,y=0.537037in,rotate=0.000000]{{\rmfamily\fontsize{12.000000}{14.400000}\selectfont \(\displaystyle 0.0\)}}
%
\end{pgfscope}%
\begin{pgfscope}%
\pgfpathrectangle{\pgfqpoint{0.750000in}{0.700000in}}{\pgfqpoint{4.650000in}{5.600000in}} %
\pgfusepath{clip}%
\pgfsetbuttcap%
\pgfsetroundjoin%
\pgfsetlinewidth{0.501875pt}%
\definecolor{currentstroke}{rgb}{0.000000,0.000000,0.000000}%
\pgfsetstrokecolor{currentstroke}%
\pgfsetdash{{1.000000pt}{3.000000pt}}{0.000000pt}%
\pgfpathmoveto{\pgfqpoint{1.912500in}{0.700000in}}%
\pgfpathlineto{\pgfqpoint{1.912500in}{6.300000in}}%
\pgfusepath{stroke}%
\end{pgfscope}%
\begin{pgfscope}%
\pgfsetbuttcap%
\pgfsetroundjoin%
\definecolor{currentfill}{rgb}{0.000000,0.000000,0.000000}%
\pgfsetfillcolor{currentfill}%
\pgfsetlinewidth{0.501875pt}%
\definecolor{currentstroke}{rgb}{0.000000,0.000000,0.000000}%
\pgfsetstrokecolor{currentstroke}%
\pgfsetdash{}{0pt}%
\pgfsys@defobject{currentmarker}{\pgfqpoint{0.000000in}{0.000000in}}{\pgfqpoint{0.000000in}{0.055556in}}{%
\pgfpathmoveto{\pgfqpoint{0.000000in}{0.000000in}}%
\pgfpathlineto{\pgfqpoint{0.000000in}{0.055556in}}%
\pgfusepath{stroke,fill}%
}%
\begin{pgfscope}%
\pgfsys@transformshift{1.912500in}{0.700000in}%
\pgfsys@useobject{currentmarker}{}%
\end{pgfscope}%
\end{pgfscope}%
\begin{pgfscope}%
\pgfsetbuttcap%
\pgfsetroundjoin%
\definecolor{currentfill}{rgb}{0.000000,0.000000,0.000000}%
\pgfsetfillcolor{currentfill}%
\pgfsetlinewidth{0.501875pt}%
\definecolor{currentstroke}{rgb}{0.000000,0.000000,0.000000}%
\pgfsetstrokecolor{currentstroke}%
\pgfsetdash{}{0pt}%
\pgfsys@defobject{currentmarker}{\pgfqpoint{0.000000in}{-0.055556in}}{\pgfqpoint{0.000000in}{0.000000in}}{%
\pgfpathmoveto{\pgfqpoint{0.000000in}{0.000000in}}%
\pgfpathlineto{\pgfqpoint{0.000000in}{-0.055556in}}%
\pgfusepath{stroke,fill}%
}%
\begin{pgfscope}%
\pgfsys@transformshift{1.912500in}{6.300000in}%
\pgfsys@useobject{currentmarker}{}%
\end{pgfscope}%
\end{pgfscope}%
\begin{pgfscope}%
\pgftext[left,bottom,x=1.808238in,y=0.537037in,rotate=0.000000]{{\rmfamily\fontsize{12.000000}{14.400000}\selectfont \(\displaystyle 0.5\)}}
%
\end{pgfscope}%
\begin{pgfscope}%
\pgfpathrectangle{\pgfqpoint{0.750000in}{0.700000in}}{\pgfqpoint{4.650000in}{5.600000in}} %
\pgfusepath{clip}%
\pgfsetbuttcap%
\pgfsetroundjoin%
\pgfsetlinewidth{0.501875pt}%
\definecolor{currentstroke}{rgb}{0.000000,0.000000,0.000000}%
\pgfsetstrokecolor{currentstroke}%
\pgfsetdash{{1.000000pt}{3.000000pt}}{0.000000pt}%
\pgfpathmoveto{\pgfqpoint{3.075000in}{0.700000in}}%
\pgfpathlineto{\pgfqpoint{3.075000in}{6.300000in}}%
\pgfusepath{stroke}%
\end{pgfscope}%
\begin{pgfscope}%
\pgfsetbuttcap%
\pgfsetroundjoin%
\definecolor{currentfill}{rgb}{0.000000,0.000000,0.000000}%
\pgfsetfillcolor{currentfill}%
\pgfsetlinewidth{0.501875pt}%
\definecolor{currentstroke}{rgb}{0.000000,0.000000,0.000000}%
\pgfsetstrokecolor{currentstroke}%
\pgfsetdash{}{0pt}%
\pgfsys@defobject{currentmarker}{\pgfqpoint{0.000000in}{0.000000in}}{\pgfqpoint{0.000000in}{0.055556in}}{%
\pgfpathmoveto{\pgfqpoint{0.000000in}{0.000000in}}%
\pgfpathlineto{\pgfqpoint{0.000000in}{0.055556in}}%
\pgfusepath{stroke,fill}%
}%
\begin{pgfscope}%
\pgfsys@transformshift{3.075000in}{0.700000in}%
\pgfsys@useobject{currentmarker}{}%
\end{pgfscope}%
\end{pgfscope}%
\begin{pgfscope}%
\pgfsetbuttcap%
\pgfsetroundjoin%
\definecolor{currentfill}{rgb}{0.000000,0.000000,0.000000}%
\pgfsetfillcolor{currentfill}%
\pgfsetlinewidth{0.501875pt}%
\definecolor{currentstroke}{rgb}{0.000000,0.000000,0.000000}%
\pgfsetstrokecolor{currentstroke}%
\pgfsetdash{}{0pt}%
\pgfsys@defobject{currentmarker}{\pgfqpoint{0.000000in}{-0.055556in}}{\pgfqpoint{0.000000in}{0.000000in}}{%
\pgfpathmoveto{\pgfqpoint{0.000000in}{0.000000in}}%
\pgfpathlineto{\pgfqpoint{0.000000in}{-0.055556in}}%
\pgfusepath{stroke,fill}%
}%
\begin{pgfscope}%
\pgfsys@transformshift{3.075000in}{6.300000in}%
\pgfsys@useobject{currentmarker}{}%
\end{pgfscope}%
\end{pgfscope}%
\begin{pgfscope}%
\pgftext[left,bottom,x=2.970738in,y=0.537037in,rotate=0.000000]{{\rmfamily\fontsize{12.000000}{14.400000}\selectfont \(\displaystyle 1.0\)}}
%
\end{pgfscope}%
\begin{pgfscope}%
\pgfpathrectangle{\pgfqpoint{0.750000in}{0.700000in}}{\pgfqpoint{4.650000in}{5.600000in}} %
\pgfusepath{clip}%
\pgfsetbuttcap%
\pgfsetroundjoin%
\pgfsetlinewidth{0.501875pt}%
\definecolor{currentstroke}{rgb}{0.000000,0.000000,0.000000}%
\pgfsetstrokecolor{currentstroke}%
\pgfsetdash{{1.000000pt}{3.000000pt}}{0.000000pt}%
\pgfpathmoveto{\pgfqpoint{4.237500in}{0.700000in}}%
\pgfpathlineto{\pgfqpoint{4.237500in}{6.300000in}}%
\pgfusepath{stroke}%
\end{pgfscope}%
\begin{pgfscope}%
\pgfsetbuttcap%
\pgfsetroundjoin%
\definecolor{currentfill}{rgb}{0.000000,0.000000,0.000000}%
\pgfsetfillcolor{currentfill}%
\pgfsetlinewidth{0.501875pt}%
\definecolor{currentstroke}{rgb}{0.000000,0.000000,0.000000}%
\pgfsetstrokecolor{currentstroke}%
\pgfsetdash{}{0pt}%
\pgfsys@defobject{currentmarker}{\pgfqpoint{0.000000in}{0.000000in}}{\pgfqpoint{0.000000in}{0.055556in}}{%
\pgfpathmoveto{\pgfqpoint{0.000000in}{0.000000in}}%
\pgfpathlineto{\pgfqpoint{0.000000in}{0.055556in}}%
\pgfusepath{stroke,fill}%
}%
\begin{pgfscope}%
\pgfsys@transformshift{4.237500in}{0.700000in}%
\pgfsys@useobject{currentmarker}{}%
\end{pgfscope}%
\end{pgfscope}%
\begin{pgfscope}%
\pgfsetbuttcap%
\pgfsetroundjoin%
\definecolor{currentfill}{rgb}{0.000000,0.000000,0.000000}%
\pgfsetfillcolor{currentfill}%
\pgfsetlinewidth{0.501875pt}%
\definecolor{currentstroke}{rgb}{0.000000,0.000000,0.000000}%
\pgfsetstrokecolor{currentstroke}%
\pgfsetdash{}{0pt}%
\pgfsys@defobject{currentmarker}{\pgfqpoint{0.000000in}{-0.055556in}}{\pgfqpoint{0.000000in}{0.000000in}}{%
\pgfpathmoveto{\pgfqpoint{0.000000in}{0.000000in}}%
\pgfpathlineto{\pgfqpoint{0.000000in}{-0.055556in}}%
\pgfusepath{stroke,fill}%
}%
\begin{pgfscope}%
\pgfsys@transformshift{4.237500in}{6.300000in}%
\pgfsys@useobject{currentmarker}{}%
\end{pgfscope}%
\end{pgfscope}%
\begin{pgfscope}%
\pgftext[left,bottom,x=4.133238in,y=0.537037in,rotate=0.000000]{{\rmfamily\fontsize{12.000000}{14.400000}\selectfont \(\displaystyle 1.5\)}}
%
\end{pgfscope}%
\begin{pgfscope}%
\pgfpathrectangle{\pgfqpoint{0.750000in}{0.700000in}}{\pgfqpoint{4.650000in}{5.600000in}} %
\pgfusepath{clip}%
\pgfsetbuttcap%
\pgfsetroundjoin%
\pgfsetlinewidth{0.501875pt}%
\definecolor{currentstroke}{rgb}{0.000000,0.000000,0.000000}%
\pgfsetstrokecolor{currentstroke}%
\pgfsetdash{{1.000000pt}{3.000000pt}}{0.000000pt}%
\pgfpathmoveto{\pgfqpoint{5.400000in}{0.700000in}}%
\pgfpathlineto{\pgfqpoint{5.400000in}{6.300000in}}%
\pgfusepath{stroke}%
\end{pgfscope}%
\begin{pgfscope}%
\pgfsetbuttcap%
\pgfsetroundjoin%
\definecolor{currentfill}{rgb}{0.000000,0.000000,0.000000}%
\pgfsetfillcolor{currentfill}%
\pgfsetlinewidth{0.501875pt}%
\definecolor{currentstroke}{rgb}{0.000000,0.000000,0.000000}%
\pgfsetstrokecolor{currentstroke}%
\pgfsetdash{}{0pt}%
\pgfsys@defobject{currentmarker}{\pgfqpoint{0.000000in}{0.000000in}}{\pgfqpoint{0.000000in}{0.055556in}}{%
\pgfpathmoveto{\pgfqpoint{0.000000in}{0.000000in}}%
\pgfpathlineto{\pgfqpoint{0.000000in}{0.055556in}}%
\pgfusepath{stroke,fill}%
}%
\begin{pgfscope}%
\pgfsys@transformshift{5.400000in}{0.700000in}%
\pgfsys@useobject{currentmarker}{}%
\end{pgfscope}%
\end{pgfscope}%
\begin{pgfscope}%
\pgfsetbuttcap%
\pgfsetroundjoin%
\definecolor{currentfill}{rgb}{0.000000,0.000000,0.000000}%
\pgfsetfillcolor{currentfill}%
\pgfsetlinewidth{0.501875pt}%
\definecolor{currentstroke}{rgb}{0.000000,0.000000,0.000000}%
\pgfsetstrokecolor{currentstroke}%
\pgfsetdash{}{0pt}%
\pgfsys@defobject{currentmarker}{\pgfqpoint{0.000000in}{-0.055556in}}{\pgfqpoint{0.000000in}{0.000000in}}{%
\pgfpathmoveto{\pgfqpoint{0.000000in}{0.000000in}}%
\pgfpathlineto{\pgfqpoint{0.000000in}{-0.055556in}}%
\pgfusepath{stroke,fill}%
}%
\begin{pgfscope}%
\pgfsys@transformshift{5.400000in}{6.300000in}%
\pgfsys@useobject{currentmarker}{}%
\end{pgfscope}%
\end{pgfscope}%
\begin{pgfscope}%
\pgftext[left,bottom,x=5.295738in,y=0.537037in,rotate=0.000000]{{\rmfamily\fontsize{12.000000}{14.400000}\selectfont \(\displaystyle 2.0\)}}
%
\end{pgfscope}%
\begin{pgfscope}%
\pgftext[left,bottom,x=2.319809in,y=0.319445in,rotate=0.000000]{{\rmfamily\fontsize{12.000000}{14.400000}\selectfont Distance along Beam}}
%
\end{pgfscope}%
\begin{pgfscope}%
\pgfpathrectangle{\pgfqpoint{0.750000in}{0.700000in}}{\pgfqpoint{4.650000in}{5.600000in}} %
\pgfusepath{clip}%
\pgfsetbuttcap%
\pgfsetroundjoin%
\pgfsetlinewidth{0.501875pt}%
\definecolor{currentstroke}{rgb}{0.000000,0.000000,0.000000}%
\pgfsetstrokecolor{currentstroke}%
\pgfsetdash{{1.000000pt}{3.000000pt}}{0.000000pt}%
\pgfpathmoveto{\pgfqpoint{0.750000in}{6.300000in}}%
\pgfpathlineto{\pgfqpoint{5.400000in}{6.300000in}}%
\pgfusepath{stroke}%
\end{pgfscope}%
\begin{pgfscope}%
\pgfsetbuttcap%
\pgfsetroundjoin%
\definecolor{currentfill}{rgb}{0.000000,0.000000,0.000000}%
\pgfsetfillcolor{currentfill}%
\pgfsetlinewidth{0.501875pt}%
\definecolor{currentstroke}{rgb}{0.000000,0.000000,0.000000}%
\pgfsetstrokecolor{currentstroke}%
\pgfsetdash{}{0pt}%
\pgfsys@defobject{currentmarker}{\pgfqpoint{0.000000in}{0.000000in}}{\pgfqpoint{0.055556in}{0.000000in}}{%
\pgfpathmoveto{\pgfqpoint{0.000000in}{0.000000in}}%
\pgfpathlineto{\pgfqpoint{0.055556in}{0.000000in}}%
\pgfusepath{stroke,fill}%
}%
\begin{pgfscope}%
\pgfsys@transformshift{0.750000in}{6.300000in}%
\pgfsys@useobject{currentmarker}{}%
\end{pgfscope}%
\end{pgfscope}%
\begin{pgfscope}%
\pgfsetbuttcap%
\pgfsetroundjoin%
\definecolor{currentfill}{rgb}{0.000000,0.000000,0.000000}%
\pgfsetfillcolor{currentfill}%
\pgfsetlinewidth{0.501875pt}%
\definecolor{currentstroke}{rgb}{0.000000,0.000000,0.000000}%
\pgfsetstrokecolor{currentstroke}%
\pgfsetdash{}{0pt}%
\pgfsys@defobject{currentmarker}{\pgfqpoint{-0.055556in}{0.000000in}}{\pgfqpoint{0.000000in}{0.000000in}}{%
\pgfpathmoveto{\pgfqpoint{0.000000in}{0.000000in}}%
\pgfpathlineto{\pgfqpoint{-0.055556in}{0.000000in}}%
\pgfusepath{stroke,fill}%
}%
\begin{pgfscope}%
\pgfsys@transformshift{5.400000in}{6.300000in}%
\pgfsys@useobject{currentmarker}{}%
\end{pgfscope}%
\end{pgfscope}%
\begin{pgfscope}%
\pgftext[left,bottom,x=0.485920in,y=6.246296in,rotate=0.000000]{{\rmfamily\fontsize{12.000000}{14.400000}\selectfont \(\displaystyle 0.0\)}}
%
\end{pgfscope}%
\begin{pgfscope}%
\pgfpathrectangle{\pgfqpoint{0.750000in}{0.700000in}}{\pgfqpoint{4.650000in}{5.600000in}} %
\pgfusepath{clip}%
\pgfsetbuttcap%
\pgfsetroundjoin%
\pgfsetlinewidth{0.501875pt}%
\definecolor{currentstroke}{rgb}{0.000000,0.000000,0.000000}%
\pgfsetstrokecolor{currentstroke}%
\pgfsetdash{{1.000000pt}{3.000000pt}}{0.000000pt}%
\pgfpathmoveto{\pgfqpoint{0.750000in}{4.900000in}}%
\pgfpathlineto{\pgfqpoint{5.400000in}{4.900000in}}%
\pgfusepath{stroke}%
\end{pgfscope}%
\begin{pgfscope}%
\pgfsetbuttcap%
\pgfsetroundjoin%
\definecolor{currentfill}{rgb}{0.000000,0.000000,0.000000}%
\pgfsetfillcolor{currentfill}%
\pgfsetlinewidth{0.501875pt}%
\definecolor{currentstroke}{rgb}{0.000000,0.000000,0.000000}%
\pgfsetstrokecolor{currentstroke}%
\pgfsetdash{}{0pt}%
\pgfsys@defobject{currentmarker}{\pgfqpoint{0.000000in}{0.000000in}}{\pgfqpoint{0.055556in}{0.000000in}}{%
\pgfpathmoveto{\pgfqpoint{0.000000in}{0.000000in}}%
\pgfpathlineto{\pgfqpoint{0.055556in}{0.000000in}}%
\pgfusepath{stroke,fill}%
}%
\begin{pgfscope}%
\pgfsys@transformshift{0.750000in}{4.900000in}%
\pgfsys@useobject{currentmarker}{}%
\end{pgfscope}%
\end{pgfscope}%
\begin{pgfscope}%
\pgfsetbuttcap%
\pgfsetroundjoin%
\definecolor{currentfill}{rgb}{0.000000,0.000000,0.000000}%
\pgfsetfillcolor{currentfill}%
\pgfsetlinewidth{0.501875pt}%
\definecolor{currentstroke}{rgb}{0.000000,0.000000,0.000000}%
\pgfsetstrokecolor{currentstroke}%
\pgfsetdash{}{0pt}%
\pgfsys@defobject{currentmarker}{\pgfqpoint{-0.055556in}{0.000000in}}{\pgfqpoint{0.000000in}{0.000000in}}{%
\pgfpathmoveto{\pgfqpoint{0.000000in}{0.000000in}}%
\pgfpathlineto{\pgfqpoint{-0.055556in}{0.000000in}}%
\pgfusepath{stroke,fill}%
}%
\begin{pgfscope}%
\pgfsys@transformshift{5.400000in}{4.900000in}%
\pgfsys@useobject{currentmarker}{}%
\end{pgfscope}%
\end{pgfscope}%
\begin{pgfscope}%
\pgftext[left,bottom,x=0.356290in,y=4.839352in,rotate=0.000000]{{\rmfamily\fontsize{12.000000}{14.400000}\selectfont \(\displaystyle -0.8\)}}
%
\end{pgfscope}%
\begin{pgfscope}%
\pgfpathrectangle{\pgfqpoint{0.750000in}{0.700000in}}{\pgfqpoint{4.650000in}{5.600000in}} %
\pgfusepath{clip}%
\pgfsetbuttcap%
\pgfsetroundjoin%
\pgfsetlinewidth{0.501875pt}%
\definecolor{currentstroke}{rgb}{0.000000,0.000000,0.000000}%
\pgfsetstrokecolor{currentstroke}%
\pgfsetdash{{1.000000pt}{3.000000pt}}{0.000000pt}%
\pgfpathmoveto{\pgfqpoint{0.750000in}{3.500000in}}%
\pgfpathlineto{\pgfqpoint{5.400000in}{3.500000in}}%
\pgfusepath{stroke}%
\end{pgfscope}%
\begin{pgfscope}%
\pgfsetbuttcap%
\pgfsetroundjoin%
\definecolor{currentfill}{rgb}{0.000000,0.000000,0.000000}%
\pgfsetfillcolor{currentfill}%
\pgfsetlinewidth{0.501875pt}%
\definecolor{currentstroke}{rgb}{0.000000,0.000000,0.000000}%
\pgfsetstrokecolor{currentstroke}%
\pgfsetdash{}{0pt}%
\pgfsys@defobject{currentmarker}{\pgfqpoint{0.000000in}{0.000000in}}{\pgfqpoint{0.055556in}{0.000000in}}{%
\pgfpathmoveto{\pgfqpoint{0.000000in}{0.000000in}}%
\pgfpathlineto{\pgfqpoint{0.055556in}{0.000000in}}%
\pgfusepath{stroke,fill}%
}%
\begin{pgfscope}%
\pgfsys@transformshift{0.750000in}{3.500000in}%
\pgfsys@useobject{currentmarker}{}%
\end{pgfscope}%
\end{pgfscope}%
\begin{pgfscope}%
\pgfsetbuttcap%
\pgfsetroundjoin%
\definecolor{currentfill}{rgb}{0.000000,0.000000,0.000000}%
\pgfsetfillcolor{currentfill}%
\pgfsetlinewidth{0.501875pt}%
\definecolor{currentstroke}{rgb}{0.000000,0.000000,0.000000}%
\pgfsetstrokecolor{currentstroke}%
\pgfsetdash{}{0pt}%
\pgfsys@defobject{currentmarker}{\pgfqpoint{-0.055556in}{0.000000in}}{\pgfqpoint{0.000000in}{0.000000in}}{%
\pgfpathmoveto{\pgfqpoint{0.000000in}{0.000000in}}%
\pgfpathlineto{\pgfqpoint{-0.055556in}{0.000000in}}%
\pgfusepath{stroke,fill}%
}%
\begin{pgfscope}%
\pgfsys@transformshift{5.400000in}{3.500000in}%
\pgfsys@useobject{currentmarker}{}%
\end{pgfscope}%
\end{pgfscope}%
\begin{pgfscope}%
\pgftext[left,bottom,x=0.356290in,y=3.439352in,rotate=0.000000]{{\rmfamily\fontsize{12.000000}{14.400000}\selectfont \(\displaystyle -1.6\)}}
%
\end{pgfscope}%
\begin{pgfscope}%
\pgfpathrectangle{\pgfqpoint{0.750000in}{0.700000in}}{\pgfqpoint{4.650000in}{5.600000in}} %
\pgfusepath{clip}%
\pgfsetbuttcap%
\pgfsetroundjoin%
\pgfsetlinewidth{0.501875pt}%
\definecolor{currentstroke}{rgb}{0.000000,0.000000,0.000000}%
\pgfsetstrokecolor{currentstroke}%
\pgfsetdash{{1.000000pt}{3.000000pt}}{0.000000pt}%
\pgfpathmoveto{\pgfqpoint{0.750000in}{2.100000in}}%
\pgfpathlineto{\pgfqpoint{5.400000in}{2.100000in}}%
\pgfusepath{stroke}%
\end{pgfscope}%
\begin{pgfscope}%
\pgfsetbuttcap%
\pgfsetroundjoin%
\definecolor{currentfill}{rgb}{0.000000,0.000000,0.000000}%
\pgfsetfillcolor{currentfill}%
\pgfsetlinewidth{0.501875pt}%
\definecolor{currentstroke}{rgb}{0.000000,0.000000,0.000000}%
\pgfsetstrokecolor{currentstroke}%
\pgfsetdash{}{0pt}%
\pgfsys@defobject{currentmarker}{\pgfqpoint{0.000000in}{0.000000in}}{\pgfqpoint{0.055556in}{0.000000in}}{%
\pgfpathmoveto{\pgfqpoint{0.000000in}{0.000000in}}%
\pgfpathlineto{\pgfqpoint{0.055556in}{0.000000in}}%
\pgfusepath{stroke,fill}%
}%
\begin{pgfscope}%
\pgfsys@transformshift{0.750000in}{2.100000in}%
\pgfsys@useobject{currentmarker}{}%
\end{pgfscope}%
\end{pgfscope}%
\begin{pgfscope}%
\pgfsetbuttcap%
\pgfsetroundjoin%
\definecolor{currentfill}{rgb}{0.000000,0.000000,0.000000}%
\pgfsetfillcolor{currentfill}%
\pgfsetlinewidth{0.501875pt}%
\definecolor{currentstroke}{rgb}{0.000000,0.000000,0.000000}%
\pgfsetstrokecolor{currentstroke}%
\pgfsetdash{}{0pt}%
\pgfsys@defobject{currentmarker}{\pgfqpoint{-0.055556in}{0.000000in}}{\pgfqpoint{0.000000in}{0.000000in}}{%
\pgfpathmoveto{\pgfqpoint{0.000000in}{0.000000in}}%
\pgfpathlineto{\pgfqpoint{-0.055556in}{0.000000in}}%
\pgfusepath{stroke,fill}%
}%
\begin{pgfscope}%
\pgfsys@transformshift{5.400000in}{2.100000in}%
\pgfsys@useobject{currentmarker}{}%
\end{pgfscope}%
\end{pgfscope}%
\begin{pgfscope}%
\pgftext[left,bottom,x=0.356290in,y=2.039352in,rotate=0.000000]{{\rmfamily\fontsize{12.000000}{14.400000}\selectfont \(\displaystyle -2.4\)}}
%
\end{pgfscope}%
\begin{pgfscope}%
\pgfpathrectangle{\pgfqpoint{0.750000in}{0.700000in}}{\pgfqpoint{4.650000in}{5.600000in}} %
\pgfusepath{clip}%
\pgfsetbuttcap%
\pgfsetroundjoin%
\pgfsetlinewidth{0.501875pt}%
\definecolor{currentstroke}{rgb}{0.000000,0.000000,0.000000}%
\pgfsetstrokecolor{currentstroke}%
\pgfsetdash{{1.000000pt}{3.000000pt}}{0.000000pt}%
\pgfpathmoveto{\pgfqpoint{0.750000in}{0.700000in}}%
\pgfpathlineto{\pgfqpoint{5.400000in}{0.700000in}}%
\pgfusepath{stroke}%
\end{pgfscope}%
\begin{pgfscope}%
\pgfsetbuttcap%
\pgfsetroundjoin%
\definecolor{currentfill}{rgb}{0.000000,0.000000,0.000000}%
\pgfsetfillcolor{currentfill}%
\pgfsetlinewidth{0.501875pt}%
\definecolor{currentstroke}{rgb}{0.000000,0.000000,0.000000}%
\pgfsetstrokecolor{currentstroke}%
\pgfsetdash{}{0pt}%
\pgfsys@defobject{currentmarker}{\pgfqpoint{0.000000in}{0.000000in}}{\pgfqpoint{0.055556in}{0.000000in}}{%
\pgfpathmoveto{\pgfqpoint{0.000000in}{0.000000in}}%
\pgfpathlineto{\pgfqpoint{0.055556in}{0.000000in}}%
\pgfusepath{stroke,fill}%
}%
\begin{pgfscope}%
\pgfsys@transformshift{0.750000in}{0.700000in}%
\pgfsys@useobject{currentmarker}{}%
\end{pgfscope}%
\end{pgfscope}%
\begin{pgfscope}%
\pgfsetbuttcap%
\pgfsetroundjoin%
\definecolor{currentfill}{rgb}{0.000000,0.000000,0.000000}%
\pgfsetfillcolor{currentfill}%
\pgfsetlinewidth{0.501875pt}%
\definecolor{currentstroke}{rgb}{0.000000,0.000000,0.000000}%
\pgfsetstrokecolor{currentstroke}%
\pgfsetdash{}{0pt}%
\pgfsys@defobject{currentmarker}{\pgfqpoint{-0.055556in}{0.000000in}}{\pgfqpoint{0.000000in}{0.000000in}}{%
\pgfpathmoveto{\pgfqpoint{0.000000in}{0.000000in}}%
\pgfpathlineto{\pgfqpoint{-0.055556in}{0.000000in}}%
\pgfusepath{stroke,fill}%
}%
\begin{pgfscope}%
\pgfsys@transformshift{5.400000in}{0.700000in}%
\pgfsys@useobject{currentmarker}{}%
\end{pgfscope}%
\end{pgfscope}%
\begin{pgfscope}%
\pgftext[left,bottom,x=0.356290in,y=0.639352in,rotate=0.000000]{{\rmfamily\fontsize{12.000000}{14.400000}\selectfont \(\displaystyle -3.2\)}}
%
\end{pgfscope}%
\begin{pgfscope}%
\pgftext[left,bottom,x=0.286846in,y=3.143030in,rotate=90.000000]{{\rmfamily\fontsize{12.000000}{14.400000}\selectfont Deflection}}
%
\end{pgfscope}%
\begin{pgfscope}%
\pgftext[left,bottom,x=0.750000in,y=6.327778in,rotate=0.000000]{{\rmfamily\fontsize{12.000000}{14.400000}\selectfont \(\displaystyle \times10^{-5}\)}}
%
\end{pgfscope}%
\begin{pgfscope}%
\pgfsetrectcap%
\pgfsetroundjoin%
\pgfsetlinewidth{1.003750pt}%
\definecolor{currentstroke}{rgb}{0.000000,0.000000,0.000000}%
\pgfsetstrokecolor{currentstroke}%
\pgfsetdash{}{0pt}%
\pgfpathmoveto{\pgfqpoint{0.750000in}{6.300000in}}%
\pgfpathlineto{\pgfqpoint{5.400000in}{6.300000in}}%
\pgfusepath{stroke}%
\end{pgfscope}%
\begin{pgfscope}%
\pgfsetrectcap%
\pgfsetroundjoin%
\pgfsetlinewidth{1.003750pt}%
\definecolor{currentstroke}{rgb}{0.000000,0.000000,0.000000}%
\pgfsetstrokecolor{currentstroke}%
\pgfsetdash{}{0pt}%
\pgfpathmoveto{\pgfqpoint{5.400000in}{0.700000in}}%
\pgfpathlineto{\pgfqpoint{5.400000in}{6.300000in}}%
\pgfusepath{stroke}%
\end{pgfscope}%
\begin{pgfscope}%
\pgfsetrectcap%
\pgfsetroundjoin%
\pgfsetlinewidth{1.003750pt}%
\definecolor{currentstroke}{rgb}{0.000000,0.000000,0.000000}%
\pgfsetstrokecolor{currentstroke}%
\pgfsetdash{}{0pt}%
\pgfpathmoveto{\pgfqpoint{0.750000in}{0.700000in}}%
\pgfpathlineto{\pgfqpoint{5.400000in}{0.700000in}}%
\pgfusepath{stroke}%
\end{pgfscope}%
\begin{pgfscope}%
\pgfsetrectcap%
\pgfsetroundjoin%
\pgfsetlinewidth{1.003750pt}%
\definecolor{currentstroke}{rgb}{0.000000,0.000000,0.000000}%
\pgfsetstrokecolor{currentstroke}%
\pgfsetdash{}{0pt}%
\pgfpathmoveto{\pgfqpoint{0.750000in}{0.700000in}}%
\pgfpathlineto{\pgfqpoint{0.750000in}{6.300000in}}%
\pgfusepath{stroke}%
\end{pgfscope}%
\begin{pgfscope}%
\pgftext[left,bottom,x=2.161226in,y=6.369444in,rotate=0.000000]{{\rmfamily\fontsize{14.400000}{17.280000}\selectfont Unloaded EPP Beam}}
%
\end{pgfscope}%
\begin{pgfscope}%
\pgfsetrectcap%
\pgfsetroundjoin%
\definecolor{currentfill}{rgb}{1.000000,1.000000,1.000000}%
\pgfsetfillcolor{currentfill}%
\pgfsetlinewidth{1.003750pt}%
\definecolor{currentstroke}{rgb}{0.000000,0.000000,0.000000}%
\pgfsetstrokecolor{currentstroke}%
\pgfsetdash{}{0pt}%
\pgfpathmoveto{\pgfqpoint{1.710823in}{5.124445in}}%
\pgfpathlineto{\pgfqpoint{4.439177in}{5.124445in}}%
\pgfpathlineto{\pgfqpoint{4.439177in}{6.300000in}}%
\pgfpathlineto{\pgfqpoint{1.710823in}{6.300000in}}%
\pgfpathlineto{\pgfqpoint{1.710823in}{5.124445in}}%
\pgfpathclose%
\pgfusepath{stroke,fill}%
\end{pgfscope}%
\begin{pgfscope}%
\pgfsetrectcap%
\pgfsetroundjoin%
\pgfsetlinewidth{1.003750pt}%
\definecolor{currentstroke}{rgb}{0.000000,0.000000,1.000000}%
\pgfsetstrokecolor{currentstroke}%
\pgfsetdash{}{0pt}%
\pgfpathmoveto{\pgfqpoint{1.850823in}{6.150000in}}%
\pgfpathlineto{\pgfqpoint{2.130823in}{6.150000in}}%
\pgfusepath{stroke}%
\end{pgfscope}%
\begin{pgfscope}%
\pgftext[left,bottom,x=2.350823in,y=6.041111in,rotate=0.000000]{{\rmfamily\fontsize{14.400000}{17.280000}\selectfont Abaqus EPP Beam}}
%
\end{pgfscope}%
\begin{pgfscope}%
\pgfsetrectcap%
\pgfsetroundjoin%
\pgfsetlinewidth{1.003750pt}%
\definecolor{currentstroke}{rgb}{0.000000,0.500000,0.000000}%
\pgfsetstrokecolor{currentstroke}%
\pgfsetdash{}{0pt}%
\pgfpathmoveto{\pgfqpoint{1.850823in}{5.871111in}}%
\pgfpathlineto{\pgfqpoint{2.130823in}{5.871111in}}%
\pgfusepath{stroke}%
\end{pgfscope}%
\begin{pgfscope}%
\pgfsetbuttcap%
\pgfsetmiterjoin%
\definecolor{currentfill}{rgb}{0.000000,0.500000,0.000000}%
\pgfsetfillcolor{currentfill}%
\pgfsetlinewidth{0.501875pt}%
\definecolor{currentstroke}{rgb}{0.000000,0.000000,0.000000}%
\pgfsetstrokecolor{currentstroke}%
\pgfsetdash{}{0pt}%
\pgfsys@defobject{currentmarker}{\pgfqpoint{-0.041667in}{-0.041667in}}{\pgfqpoint{0.041667in}{0.041667in}}{%
\pgfpathmoveto{\pgfqpoint{0.000000in}{0.041667in}}%
\pgfpathlineto{\pgfqpoint{-0.041667in}{-0.041667in}}%
\pgfpathlineto{\pgfqpoint{0.041667in}{-0.041667in}}%
\pgfpathclose%
\pgfusepath{stroke,fill}%
}%
\begin{pgfscope}%
\pgfsys@transformshift{1.850823in}{5.871111in}%
\pgfsys@useobject{currentmarker}{}%
\end{pgfscope}%
\begin{pgfscope}%
\pgfsys@transformshift{2.130823in}{5.871111in}%
\pgfsys@useobject{currentmarker}{}%
\end{pgfscope}%
\end{pgfscope}%
\begin{pgfscope}%
\pgftext[left,bottom,x=2.350823in,y=5.762223in,rotate=0.000000]{{\rmfamily\fontsize{14.400000}{17.280000}\selectfont 100 nodes, horizon 0.10}}
%
\end{pgfscope}%
\begin{pgfscope}%
\pgfsetrectcap%
\pgfsetroundjoin%
\pgfsetlinewidth{1.003750pt}%
\definecolor{currentstroke}{rgb}{1.000000,0.000000,0.000000}%
\pgfsetstrokecolor{currentstroke}%
\pgfsetdash{}{0pt}%
\pgfpathmoveto{\pgfqpoint{1.850823in}{5.592223in}}%
\pgfpathlineto{\pgfqpoint{2.130823in}{5.592223in}}%
\pgfusepath{stroke}%
\end{pgfscope}%
\begin{pgfscope}%
\pgfsetbuttcap%
\pgfsetmiterjoin%
\definecolor{currentfill}{rgb}{1.000000,0.000000,0.000000}%
\pgfsetfillcolor{currentfill}%
\pgfsetlinewidth{0.501875pt}%
\definecolor{currentstroke}{rgb}{0.000000,0.000000,0.000000}%
\pgfsetstrokecolor{currentstroke}%
\pgfsetdash{}{0pt}%
\pgfsys@defobject{currentmarker}{\pgfqpoint{-0.041667in}{-0.041667in}}{\pgfqpoint{0.041667in}{0.041667in}}{%
\pgfpathmoveto{\pgfqpoint{0.041667in}{-0.000000in}}%
\pgfpathlineto{\pgfqpoint{-0.041667in}{0.041667in}}%
\pgfpathlineto{\pgfqpoint{-0.041667in}{-0.041667in}}%
\pgfpathclose%
\pgfusepath{stroke,fill}%
}%
\begin{pgfscope}%
\pgfsys@transformshift{1.850823in}{5.592223in}%
\pgfsys@useobject{currentmarker}{}%
\end{pgfscope}%
\begin{pgfscope}%
\pgfsys@transformshift{2.130823in}{5.592223in}%
\pgfsys@useobject{currentmarker}{}%
\end{pgfscope}%
\end{pgfscope}%
\begin{pgfscope}%
\pgftext[left,bottom,x=2.350823in,y=5.483334in,rotate=0.000000]{{\rmfamily\fontsize{14.400000}{17.280000}\selectfont 200 nodes, horizon 0.10}}
%
\end{pgfscope}%
\begin{pgfscope}%
\pgfsetrectcap%
\pgfsetroundjoin%
\pgfsetlinewidth{1.003750pt}%
\definecolor{currentstroke}{rgb}{0.000000,0.750000,0.750000}%
\pgfsetstrokecolor{currentstroke}%
\pgfsetdash{}{0pt}%
\pgfpathmoveto{\pgfqpoint{1.850823in}{5.313334in}}%
\pgfpathlineto{\pgfqpoint{2.130823in}{5.313334in}}%
\pgfusepath{stroke}%
\end{pgfscope}%
\begin{pgfscope}%
\pgfsetbuttcap%
\pgfsetmiterjoin%
\definecolor{currentfill}{rgb}{0.000000,0.750000,0.750000}%
\pgfsetfillcolor{currentfill}%
\pgfsetlinewidth{0.501875pt}%
\definecolor{currentstroke}{rgb}{0.000000,0.000000,0.000000}%
\pgfsetstrokecolor{currentstroke}%
\pgfsetdash{}{0pt}%
\pgfsys@defobject{currentmarker}{\pgfqpoint{-0.041667in}{-0.041667in}}{\pgfqpoint{0.041667in}{0.041667in}}{%
\pgfpathmoveto{\pgfqpoint{-0.000000in}{-0.041667in}}%
\pgfpathlineto{\pgfqpoint{0.041667in}{0.041667in}}%
\pgfpathlineto{\pgfqpoint{-0.041667in}{0.041667in}}%
\pgfpathclose%
\pgfusepath{stroke,fill}%
}%
\begin{pgfscope}%
\pgfsys@transformshift{1.850823in}{5.313334in}%
\pgfsys@useobject{currentmarker}{}%
\end{pgfscope}%
\begin{pgfscope}%
\pgfsys@transformshift{2.130823in}{5.313334in}%
\pgfsys@useobject{currentmarker}{}%
\end{pgfscope}%
\end{pgfscope}%
\begin{pgfscope}%
\pgftext[left,bottom,x=2.350823in,y=5.204445in,rotate=0.000000]{{\rmfamily\fontsize{14.400000}{17.280000}\selectfont 500 nodes, horizon 0.10}}
%
\end{pgfscope}%
\end{pgfpicture}%
\makeatother%
\endgroup%
}
\caption{N-convergence}
\label{fig:ResidualPlasticityN}
\endminipage\hfill
\minipage[t]{0.5\textwidth}%
  \centering
  \scalebox{.45}{%% Creator: Matplotlib, PGF backend
%%
%% To include the figure in your LaTeX document, write
%%   \input{<filename>.pgf}
%%
%% Make sure the required packages are loaded in your preamble
%%   \usepackage{pgf}
%%
%% Figures using additional raster images can only be included by \input if
%% they are in the same directory as the main LaTeX file. For loading figures
%% from other directories you can use the `import` package
%%   \usepackage{import}
%% and then include the figures with
%%   \import{<path to file>}{<filename>.pgf}
%%
%% Matplotlib used the following preamble
%%
\begingroup%
\makeatletter%
\begin{pgfpicture}%
\pgfpathrectangle{\pgfpointorigin}{\pgfqpoint{6.000000in}{6.000000in}}%
\pgfusepath{use as bounding box}%
\begin{pgfscope}%
\pgfsetrectcap%
\pgfsetroundjoin%
\definecolor{currentfill}{rgb}{1.000000,1.000000,1.000000}%
\pgfsetfillcolor{currentfill}%
\pgfsetlinewidth{0.000000pt}%
\definecolor{currentstroke}{rgb}{1.000000,1.000000,1.000000}%
\pgfsetstrokecolor{currentstroke}%
\pgfsetdash{}{0pt}%
\pgfpathmoveto{\pgfqpoint{0.000000in}{0.000000in}}%
\pgfpathlineto{\pgfqpoint{6.000000in}{0.000000in}}%
\pgfpathlineto{\pgfqpoint{6.000000in}{6.000000in}}%
\pgfpathlineto{\pgfqpoint{0.000000in}{6.000000in}}%
\pgfpathclose%
\pgfusepath{fill}%
\end{pgfscope}%
\begin{pgfscope}%
\pgfsetrectcap%
\pgfsetroundjoin%
\definecolor{currentfill}{rgb}{1.000000,1.000000,1.000000}%
\pgfsetfillcolor{currentfill}%
\pgfsetlinewidth{0.000000pt}%
\definecolor{currentstroke}{rgb}{0.000000,0.000000,0.000000}%
\pgfsetstrokecolor{currentstroke}%
\pgfsetdash{}{0pt}%
\pgfpathmoveto{\pgfqpoint{0.750000in}{0.600000in}}%
\pgfpathlineto{\pgfqpoint{5.400000in}{0.600000in}}%
\pgfpathlineto{\pgfqpoint{5.400000in}{5.400000in}}%
\pgfpathlineto{\pgfqpoint{0.750000in}{5.400000in}}%
\pgfpathclose%
\pgfusepath{fill}%
\end{pgfscope}%
\begin{pgfscope}%
\pgfpathrectangle{\pgfqpoint{0.750000in}{0.600000in}}{\pgfqpoint{4.650000in}{4.800000in}} %
\pgfusepath{clip}%
\pgfsetrectcap%
\pgfsetroundjoin%
\pgfsetlinewidth{1.003750pt}%
\definecolor{currentstroke}{rgb}{0.000000,0.000000,1.000000}%
\pgfsetstrokecolor{currentstroke}%
\pgfsetdash{}{0pt}%
\pgfpathmoveto{\pgfqpoint{0.750000in}{5.400000in}}%
\pgfpathlineto{\pgfqpoint{2.168250in}{2.044999in}}%
\pgfpathlineto{\pgfqpoint{2.238000in}{1.886064in}}%
\pgfpathlineto{\pgfqpoint{2.307750in}{1.732966in}}%
\pgfpathlineto{\pgfqpoint{2.354250in}{1.634991in}}%
\pgfpathlineto{\pgfqpoint{2.400750in}{1.540874in}}%
\pgfpathlineto{\pgfqpoint{2.447250in}{1.451003in}}%
\pgfpathlineto{\pgfqpoint{2.493750in}{1.365822in}}%
\pgfpathlineto{\pgfqpoint{2.540250in}{1.285735in}}%
\pgfpathlineto{\pgfqpoint{2.586750in}{1.211077in}}%
\pgfpathlineto{\pgfqpoint{2.633250in}{1.142178in}}%
\pgfpathlineto{\pgfqpoint{2.679750in}{1.079298in}}%
\pgfpathlineto{\pgfqpoint{2.726250in}{1.022769in}}%
\pgfpathlineto{\pgfqpoint{2.749500in}{0.996923in}}%
\pgfpathlineto{\pgfqpoint{2.772750in}{0.972775in}}%
\pgfpathlineto{\pgfqpoint{2.796000in}{0.950289in}}%
\pgfpathlineto{\pgfqpoint{2.819250in}{0.929538in}}%
\pgfpathlineto{\pgfqpoint{2.842500in}{0.910523in}}%
\pgfpathlineto{\pgfqpoint{2.865750in}{0.893243in}}%
\pgfpathlineto{\pgfqpoint{2.889000in}{0.877735in}}%
\pgfpathlineto{\pgfqpoint{2.912250in}{0.864000in}}%
\pgfpathlineto{\pgfqpoint{2.935500in}{0.852111in}}%
\pgfpathlineto{\pgfqpoint{2.958750in}{0.841994in}}%
\pgfpathlineto{\pgfqpoint{2.982000in}{0.833723in}}%
\pgfpathlineto{\pgfqpoint{3.005250in}{0.827262in}}%
\pgfpathlineto{\pgfqpoint{3.028500in}{0.822646in}}%
\pgfpathlineto{\pgfqpoint{3.051750in}{0.819877in}}%
\pgfpathlineto{\pgfqpoint{3.075000in}{0.818954in}}%
\pgfpathlineto{\pgfqpoint{3.098250in}{0.819877in}}%
\pgfpathlineto{\pgfqpoint{3.121500in}{0.822646in}}%
\pgfpathlineto{\pgfqpoint{3.144750in}{0.827262in}}%
\pgfpathlineto{\pgfqpoint{3.168000in}{0.833723in}}%
\pgfpathlineto{\pgfqpoint{3.191250in}{0.841994in}}%
\pgfpathlineto{\pgfqpoint{3.214500in}{0.852111in}}%
\pgfpathlineto{\pgfqpoint{3.237750in}{0.864000in}}%
\pgfpathlineto{\pgfqpoint{3.261000in}{0.877735in}}%
\pgfpathlineto{\pgfqpoint{3.284250in}{0.893243in}}%
\pgfpathlineto{\pgfqpoint{3.307500in}{0.910523in}}%
\pgfpathlineto{\pgfqpoint{3.330750in}{0.929538in}}%
\pgfpathlineto{\pgfqpoint{3.354000in}{0.950289in}}%
\pgfpathlineto{\pgfqpoint{3.377250in}{0.972775in}}%
\pgfpathlineto{\pgfqpoint{3.400500in}{0.996923in}}%
\pgfpathlineto{\pgfqpoint{3.447000in}{1.050240in}}%
\pgfpathlineto{\pgfqpoint{3.493500in}{1.109982in}}%
\pgfpathlineto{\pgfqpoint{3.540000in}{1.175889in}}%
\pgfpathlineto{\pgfqpoint{3.586500in}{1.247705in}}%
\pgfpathlineto{\pgfqpoint{3.633000in}{1.325132in}}%
\pgfpathlineto{\pgfqpoint{3.679500in}{1.407803in}}%
\pgfpathlineto{\pgfqpoint{3.726000in}{1.495385in}}%
\pgfpathlineto{\pgfqpoint{3.772500in}{1.587434in}}%
\pgfpathlineto{\pgfqpoint{3.819000in}{1.683545in}}%
\pgfpathlineto{\pgfqpoint{3.888750in}{1.834298in}}%
\pgfpathlineto{\pgfqpoint{3.958500in}{1.991487in}}%
\pgfpathlineto{\pgfqpoint{4.051500in}{2.207900in}}%
\pgfpathlineto{\pgfqpoint{4.400250in}{3.033242in}}%
\pgfpathlineto{\pgfqpoint{5.400000in}{5.400000in}}%
\pgfpathlineto{\pgfqpoint{5.400000in}{5.400000in}}%
\pgfusepath{stroke}%
\end{pgfscope}%
\begin{pgfscope}%
\pgfpathrectangle{\pgfqpoint{0.750000in}{0.600000in}}{\pgfqpoint{4.650000in}{4.800000in}} %
\pgfusepath{clip}%
\pgfsetrectcap%
\pgfsetroundjoin%
\pgfsetlinewidth{1.003750pt}%
\definecolor{currentstroke}{rgb}{0.000000,0.500000,0.000000}%
\pgfsetstrokecolor{currentstroke}%
\pgfsetdash{}{0pt}%
\pgfpathmoveto{\pgfqpoint{0.754650in}{5.390751in}}%
\pgfpathlineto{\pgfqpoint{2.372850in}{2.174229in}}%
\pgfpathlineto{\pgfqpoint{2.451900in}{2.023865in}}%
\pgfpathlineto{\pgfqpoint{2.517000in}{1.904678in}}%
\pgfpathlineto{\pgfqpoint{2.572800in}{1.807242in}}%
\pgfpathlineto{\pgfqpoint{2.619300in}{1.730248in}}%
\pgfpathlineto{\pgfqpoint{2.661150in}{1.664795in}}%
\pgfpathlineto{\pgfqpoint{2.703000in}{1.603527in}}%
\pgfpathlineto{\pgfqpoint{2.740200in}{1.552984in}}%
\pgfpathlineto{\pgfqpoint{2.772750in}{1.512092in}}%
\pgfpathlineto{\pgfqpoint{2.805300in}{1.474566in}}%
\pgfpathlineto{\pgfqpoint{2.837850in}{1.440638in}}%
\pgfpathlineto{\pgfqpoint{2.865750in}{1.414593in}}%
\pgfpathlineto{\pgfqpoint{2.893650in}{1.391492in}}%
\pgfpathlineto{\pgfqpoint{2.921550in}{1.371447in}}%
\pgfpathlineto{\pgfqpoint{2.944800in}{1.357161in}}%
\pgfpathlineto{\pgfqpoint{2.968050in}{1.345131in}}%
\pgfpathlineto{\pgfqpoint{2.991300in}{1.335394in}}%
\pgfpathlineto{\pgfqpoint{3.014550in}{1.328000in}}%
\pgfpathlineto{\pgfqpoint{3.037800in}{1.322973in}}%
\pgfpathlineto{\pgfqpoint{3.061050in}{1.320334in}}%
\pgfpathlineto{\pgfqpoint{3.084300in}{1.320096in}}%
\pgfpathlineto{\pgfqpoint{3.107550in}{1.322258in}}%
\pgfpathlineto{\pgfqpoint{3.130800in}{1.326809in}}%
\pgfpathlineto{\pgfqpoint{3.154050in}{1.333733in}}%
\pgfpathlineto{\pgfqpoint{3.177300in}{1.343007in}}%
\pgfpathlineto{\pgfqpoint{3.200550in}{1.354583in}}%
\pgfpathlineto{\pgfqpoint{3.223800in}{1.368424in}}%
\pgfpathlineto{\pgfqpoint{3.251700in}{1.387950in}}%
\pgfpathlineto{\pgfqpoint{3.279600in}{1.410550in}}%
\pgfpathlineto{\pgfqpoint{3.307500in}{1.436118in}}%
\pgfpathlineto{\pgfqpoint{3.335400in}{1.464513in}}%
\pgfpathlineto{\pgfqpoint{3.367950in}{1.501035in}}%
\pgfpathlineto{\pgfqpoint{3.400500in}{1.540992in}}%
\pgfpathlineto{\pgfqpoint{3.437700in}{1.590551in}}%
\pgfpathlineto{\pgfqpoint{3.474900in}{1.643911in}}%
\pgfpathlineto{\pgfqpoint{3.516750in}{1.708025in}}%
\pgfpathlineto{\pgfqpoint{3.563250in}{1.783737in}}%
\pgfpathlineto{\pgfqpoint{3.614400in}{1.871678in}}%
\pgfpathlineto{\pgfqpoint{3.670200in}{1.972200in}}%
\pgfpathlineto{\pgfqpoint{3.739950in}{2.102870in}}%
\pgfpathlineto{\pgfqpoint{3.823650in}{2.264718in}}%
\pgfpathlineto{\pgfqpoint{3.944550in}{2.503858in}}%
\pgfpathlineto{\pgfqpoint{4.958250in}{4.521312in}}%
\pgfpathlineto{\pgfqpoint{5.400000in}{5.400000in}}%
\pgfpathlineto{\pgfqpoint{5.400000in}{5.400000in}}%
\pgfusepath{stroke}%
\end{pgfscope}%
\begin{pgfscope}%
\pgfpathrectangle{\pgfqpoint{0.750000in}{0.600000in}}{\pgfqpoint{4.650000in}{4.800000in}} %
\pgfusepath{clip}%
\pgfsetbuttcap%
\pgfsetmiterjoin%
\definecolor{currentfill}{rgb}{0.000000,0.500000,0.000000}%
\pgfsetfillcolor{currentfill}%
\pgfsetlinewidth{0.501875pt}%
\definecolor{currentstroke}{rgb}{0.000000,0.000000,0.000000}%
\pgfsetstrokecolor{currentstroke}%
\pgfsetdash{}{0pt}%
\pgfsys@defobject{currentmarker}{\pgfqpoint{-0.041667in}{-0.041667in}}{\pgfqpoint{0.041667in}{0.041667in}}{%
\pgfpathmoveto{\pgfqpoint{0.000000in}{0.041667in}}%
\pgfpathlineto{\pgfqpoint{-0.041667in}{-0.041667in}}%
\pgfpathlineto{\pgfqpoint{0.041667in}{-0.041667in}}%
\pgfpathclose%
\pgfusepath{stroke,fill}%
}%
\begin{pgfscope}%
\pgfsys@transformshift{0.829050in}{5.242761in}%
\pgfsys@useobject{currentmarker}{}%
\end{pgfscope}%
\begin{pgfscope}%
\pgfsys@transformshift{1.294050in}{4.317823in}%
\pgfsys@useobject{currentmarker}{}%
\end{pgfscope}%
\begin{pgfscope}%
\pgfsys@transformshift{1.759050in}{3.392947in}%
\pgfsys@useobject{currentmarker}{}%
\end{pgfscope}%
\begin{pgfscope}%
\pgfsys@transformshift{2.224050in}{2.466807in}%
\pgfsys@useobject{currentmarker}{}%
\end{pgfscope}%
\begin{pgfscope}%
\pgfsys@transformshift{2.689050in}{1.623453in}%
\pgfsys@useobject{currentmarker}{}%
\end{pgfscope}%
\begin{pgfscope}%
\pgfsys@transformshift{3.154050in}{1.333733in}%
\pgfsys@useobject{currentmarker}{}%
\end{pgfscope}%
\begin{pgfscope}%
\pgfsys@transformshift{3.619050in}{1.879886in}%
\pgfsys@useobject{currentmarker}{}%
\end{pgfscope}%
\begin{pgfscope}%
\pgfsys@transformshift{4.084050in}{2.782202in}%
\pgfsys@useobject{currentmarker}{}%
\end{pgfscope}%
\begin{pgfscope}%
\pgfsys@transformshift{4.549050in}{3.707286in}%
\pgfsys@useobject{currentmarker}{}%
\end{pgfscope}%
\begin{pgfscope}%
\pgfsys@transformshift{5.014050in}{4.632305in}%
\pgfsys@useobject{currentmarker}{}%
\end{pgfscope}%
\end{pgfscope}%
\begin{pgfscope}%
\pgfpathrectangle{\pgfqpoint{0.750000in}{0.600000in}}{\pgfqpoint{4.650000in}{4.800000in}} %
\pgfusepath{clip}%
\pgfsetrectcap%
\pgfsetroundjoin%
\pgfsetlinewidth{1.003750pt}%
\definecolor{currentstroke}{rgb}{1.000000,0.000000,0.000000}%
\pgfsetstrokecolor{currentstroke}%
\pgfsetdash{}{0pt}%
\pgfpathmoveto{\pgfqpoint{0.750000in}{5.400000in}}%
\pgfpathlineto{\pgfqpoint{2.340300in}{2.142406in}}%
\pgfpathlineto{\pgfqpoint{2.428650in}{1.968050in}}%
\pgfpathlineto{\pgfqpoint{2.493750in}{1.844224in}}%
\pgfpathlineto{\pgfqpoint{2.549550in}{1.742502in}}%
\pgfpathlineto{\pgfqpoint{2.600700in}{1.653805in}}%
\pgfpathlineto{\pgfqpoint{2.647200in}{1.577718in}}%
\pgfpathlineto{\pgfqpoint{2.689050in}{1.513554in}}%
\pgfpathlineto{\pgfqpoint{2.726250in}{1.460398in}}%
\pgfpathlineto{\pgfqpoint{2.763450in}{1.411281in}}%
\pgfpathlineto{\pgfqpoint{2.796000in}{1.371909in}}%
\pgfpathlineto{\pgfqpoint{2.828550in}{1.336153in}}%
\pgfpathlineto{\pgfqpoint{2.856450in}{1.308562in}}%
\pgfpathlineto{\pgfqpoint{2.884350in}{1.283932in}}%
\pgfpathlineto{\pgfqpoint{2.912250in}{1.262398in}}%
\pgfpathlineto{\pgfqpoint{2.940150in}{1.244059in}}%
\pgfpathlineto{\pgfqpoint{2.963400in}{1.231297in}}%
\pgfpathlineto{\pgfqpoint{2.986650in}{1.220878in}}%
\pgfpathlineto{\pgfqpoint{3.009900in}{1.212843in}}%
\pgfpathlineto{\pgfqpoint{3.033150in}{1.207221in}}%
\pgfpathlineto{\pgfqpoint{3.056400in}{1.204036in}}%
\pgfpathlineto{\pgfqpoint{3.079650in}{1.203297in}}%
\pgfpathlineto{\pgfqpoint{3.102900in}{1.205018in}}%
\pgfpathlineto{\pgfqpoint{3.126150in}{1.209179in}}%
\pgfpathlineto{\pgfqpoint{3.149400in}{1.215769in}}%
\pgfpathlineto{\pgfqpoint{3.172650in}{1.224762in}}%
\pgfpathlineto{\pgfqpoint{3.195900in}{1.236121in}}%
\pgfpathlineto{\pgfqpoint{3.219150in}{1.249809in}}%
\pgfpathlineto{\pgfqpoint{3.242400in}{1.265768in}}%
\pgfpathlineto{\pgfqpoint{3.270300in}{1.287826in}}%
\pgfpathlineto{\pgfqpoint{3.298200in}{1.312960in}}%
\pgfpathlineto{\pgfqpoint{3.326100in}{1.341030in}}%
\pgfpathlineto{\pgfqpoint{3.358650in}{1.377319in}}%
\pgfpathlineto{\pgfqpoint{3.391200in}{1.417187in}}%
\pgfpathlineto{\pgfqpoint{3.423750in}{1.460398in}}%
\pgfpathlineto{\pgfqpoint{3.460950in}{1.513553in}}%
\pgfpathlineto{\pgfqpoint{3.502800in}{1.577718in}}%
\pgfpathlineto{\pgfqpoint{3.544650in}{1.645987in}}%
\pgfpathlineto{\pgfqpoint{3.591150in}{1.726025in}}%
\pgfpathlineto{\pgfqpoint{3.642300in}{1.818364in}}%
\pgfpathlineto{\pgfqpoint{3.702750in}{1.932189in}}%
\pgfpathlineto{\pgfqpoint{3.777150in}{2.077515in}}%
\pgfpathlineto{\pgfqpoint{3.874800in}{2.273783in}}%
\pgfpathlineto{\pgfqpoint{4.042200in}{2.616340in}}%
\pgfpathlineto{\pgfqpoint{4.855950in}{4.284723in}}%
\pgfpathlineto{\pgfqpoint{5.400000in}{5.400000in}}%
\pgfpathlineto{\pgfqpoint{5.400000in}{5.400000in}}%
\pgfusepath{stroke}%
\end{pgfscope}%
\begin{pgfscope}%
\pgfpathrectangle{\pgfqpoint{0.750000in}{0.600000in}}{\pgfqpoint{4.650000in}{4.800000in}} %
\pgfusepath{clip}%
\pgfsetbuttcap%
\pgfsetmiterjoin%
\definecolor{currentfill}{rgb}{1.000000,0.000000,0.000000}%
\pgfsetfillcolor{currentfill}%
\pgfsetlinewidth{0.501875pt}%
\definecolor{currentstroke}{rgb}{0.000000,0.000000,0.000000}%
\pgfsetstrokecolor{currentstroke}%
\pgfsetdash{}{0pt}%
\pgfsys@defobject{currentmarker}{\pgfqpoint{-0.041667in}{-0.041667in}}{\pgfqpoint{0.041667in}{0.041667in}}{%
\pgfpathmoveto{\pgfqpoint{0.041667in}{-0.000000in}}%
\pgfpathlineto{\pgfqpoint{-0.041667in}{0.041667in}}%
\pgfpathlineto{\pgfqpoint{-0.041667in}{-0.041667in}}%
\pgfpathclose%
\pgfusepath{stroke,fill}%
}%
\begin{pgfscope}%
\pgfsys@transformshift{0.968550in}{4.951984in}%
\pgfsys@useobject{currentmarker}{}%
\end{pgfscope}%
\begin{pgfscope}%
\pgfsys@transformshift{1.433550in}{3.998752in}%
\pgfsys@useobject{currentmarker}{}%
\end{pgfscope}%
\begin{pgfscope}%
\pgfsys@transformshift{1.898550in}{3.045560in}%
\pgfsys@useobject{currentmarker}{}%
\end{pgfscope}%
\begin{pgfscope}%
\pgfsys@transformshift{2.363550in}{2.095990in}%
\pgfsys@useobject{currentmarker}{}%
\end{pgfscope}%
\begin{pgfscope}%
\pgfsys@transformshift{2.828550in}{1.336153in}%
\pgfsys@useobject{currentmarker}{}%
\end{pgfscope}%
\begin{pgfscope}%
\pgfsys@transformshift{3.293550in}{1.308562in}%
\pgfsys@useobject{currentmarker}{}%
\end{pgfscope}%
\begin{pgfscope}%
\pgfsys@transformshift{3.758550in}{2.040745in}%
\pgfsys@useobject{currentmarker}{}%
\end{pgfscope}%
\begin{pgfscope}%
\pgfsys@transformshift{4.223550in}{2.988362in}%
\pgfsys@useobject{currentmarker}{}%
\end{pgfscope}%
\begin{pgfscope}%
\pgfsys@transformshift{4.688550in}{3.941558in}%
\pgfsys@useobject{currentmarker}{}%
\end{pgfscope}%
\begin{pgfscope}%
\pgfsys@transformshift{5.153550in}{4.894791in}%
\pgfsys@useobject{currentmarker}{}%
\end{pgfscope}%
\end{pgfscope}%
\begin{pgfscope}%
\pgfpathrectangle{\pgfqpoint{0.750000in}{0.600000in}}{\pgfqpoint{4.650000in}{4.800000in}} %
\pgfusepath{clip}%
\pgfsetrectcap%
\pgfsetroundjoin%
\pgfsetlinewidth{1.003750pt}%
\definecolor{currentstroke}{rgb}{0.000000,0.750000,0.750000}%
\pgfsetstrokecolor{currentstroke}%
\pgfsetdash{}{0pt}%
\pgfpathmoveto{\pgfqpoint{0.750000in}{5.400000in}}%
\pgfpathlineto{\pgfqpoint{2.312400in}{1.961387in}}%
\pgfpathlineto{\pgfqpoint{2.400750in}{1.773059in}}%
\pgfpathlineto{\pgfqpoint{2.470500in}{1.629204in}}%
\pgfpathlineto{\pgfqpoint{2.530950in}{1.509520in}}%
\pgfpathlineto{\pgfqpoint{2.582100in}{1.412958in}}%
\pgfpathlineto{\pgfqpoint{2.628600in}{1.329739in}}%
\pgfpathlineto{\pgfqpoint{2.670450in}{1.259191in}}%
\pgfpathlineto{\pgfqpoint{2.707650in}{1.200414in}}%
\pgfpathlineto{\pgfqpoint{2.744850in}{1.145738in}}%
\pgfpathlineto{\pgfqpoint{2.777400in}{1.101583in}}%
\pgfpathlineto{\pgfqpoint{2.809950in}{1.061146in}}%
\pgfpathlineto{\pgfqpoint{2.837850in}{1.029639in}}%
\pgfpathlineto{\pgfqpoint{2.865750in}{1.001185in}}%
\pgfpathlineto{\pgfqpoint{2.893650in}{0.975949in}}%
\pgfpathlineto{\pgfqpoint{2.921550in}{0.954054in}}%
\pgfpathlineto{\pgfqpoint{2.944800in}{0.938443in}}%
\pgfpathlineto{\pgfqpoint{2.968050in}{0.925290in}}%
\pgfpathlineto{\pgfqpoint{2.991300in}{0.914648in}}%
\pgfpathlineto{\pgfqpoint{3.014550in}{0.906567in}}%
\pgfpathlineto{\pgfqpoint{3.037800in}{0.901072in}}%
\pgfpathlineto{\pgfqpoint{3.061050in}{0.898185in}}%
\pgfpathlineto{\pgfqpoint{3.084300in}{0.897922in}}%
\pgfpathlineto{\pgfqpoint{3.107550in}{0.900286in}}%
\pgfpathlineto{\pgfqpoint{3.130800in}{0.905261in}}%
\pgfpathlineto{\pgfqpoint{3.154050in}{0.912827in}}%
\pgfpathlineto{\pgfqpoint{3.177300in}{0.922959in}}%
\pgfpathlineto{\pgfqpoint{3.200550in}{0.935614in}}%
\pgfpathlineto{\pgfqpoint{3.223800in}{0.950738in}}%
\pgfpathlineto{\pgfqpoint{3.247050in}{0.968274in}}%
\pgfpathlineto{\pgfqpoint{3.274950in}{0.992409in}}%
\pgfpathlineto{\pgfqpoint{3.302850in}{1.019807in}}%
\pgfpathlineto{\pgfqpoint{3.330750in}{1.050310in}}%
\pgfpathlineto{\pgfqpoint{3.363300in}{1.089639in}}%
\pgfpathlineto{\pgfqpoint{3.395850in}{1.132760in}}%
\pgfpathlineto{\pgfqpoint{3.428400in}{1.179406in}}%
\pgfpathlineto{\pgfqpoint{3.465600in}{1.236688in}}%
\pgfpathlineto{\pgfqpoint{3.507450in}{1.305737in}}%
\pgfpathlineto{\pgfqpoint{3.549300in}{1.379107in}}%
\pgfpathlineto{\pgfqpoint{3.595800in}{1.465037in}}%
\pgfpathlineto{\pgfqpoint{3.646950in}{1.564090in}}%
\pgfpathlineto{\pgfqpoint{3.707400in}{1.686103in}}%
\pgfpathlineto{\pgfqpoint{3.777150in}{1.831928in}}%
\pgfpathlineto{\pgfqpoint{3.870150in}{2.031924in}}%
\pgfpathlineto{\pgfqpoint{4.018950in}{2.358156in}}%
\pgfpathlineto{\pgfqpoint{5.400000in}{5.400000in}}%
\pgfpathlineto{\pgfqpoint{5.400000in}{5.400000in}}%
\pgfusepath{stroke}%
\end{pgfscope}%
\begin{pgfscope}%
\pgfpathrectangle{\pgfqpoint{0.750000in}{0.600000in}}{\pgfqpoint{4.650000in}{4.800000in}} %
\pgfusepath{clip}%
\pgfsetbuttcap%
\pgfsetmiterjoin%
\definecolor{currentfill}{rgb}{0.000000,0.750000,0.750000}%
\pgfsetfillcolor{currentfill}%
\pgfsetlinewidth{0.501875pt}%
\definecolor{currentstroke}{rgb}{0.000000,0.000000,0.000000}%
\pgfsetstrokecolor{currentstroke}%
\pgfsetdash{}{0pt}%
\pgfsys@defobject{currentmarker}{\pgfqpoint{-0.041667in}{-0.041667in}}{\pgfqpoint{0.041667in}{0.041667in}}{%
\pgfpathmoveto{\pgfqpoint{-0.000000in}{-0.041667in}}%
\pgfpathlineto{\pgfqpoint{0.041667in}{0.041667in}}%
\pgfpathlineto{\pgfqpoint{-0.041667in}{0.041667in}}%
\pgfpathclose%
\pgfusepath{stroke,fill}%
}%
\begin{pgfscope}%
\pgfsys@transformshift{1.108050in}{4.611403in}%
\pgfsys@useobject{currentmarker}{}%
\end{pgfscope}%
\begin{pgfscope}%
\pgfsys@transformshift{1.573050in}{3.587241in}%
\pgfsys@useobject{currentmarker}{}%
\end{pgfscope}%
\begin{pgfscope}%
\pgfsys@transformshift{2.038050in}{2.563056in}%
\pgfsys@useobject{currentmarker}{}%
\end{pgfscope}%
\begin{pgfscope}%
\pgfsys@transformshift{2.503050in}{1.564090in}%
\pgfsys@useobject{currentmarker}{}%
\end{pgfscope}%
\begin{pgfscope}%
\pgfsys@transformshift{2.968050in}{0.925290in}%
\pgfsys@useobject{currentmarker}{}%
\end{pgfscope}%
\begin{pgfscope}%
\pgfsys@transformshift{3.433050in}{1.186343in}%
\pgfsys@useobject{currentmarker}{}%
\end{pgfscope}%
\begin{pgfscope}%
\pgfsys@transformshift{3.898050in}{2.092727in}%
\pgfsys@useobject{currentmarker}{}%
\end{pgfscope}%
\begin{pgfscope}%
\pgfsys@transformshift{4.363050in}{3.116120in}%
\pgfsys@useobject{currentmarker}{}%
\end{pgfscope}%
\begin{pgfscope}%
\pgfsys@transformshift{4.828050in}{4.140290in}%
\pgfsys@useobject{currentmarker}{}%
\end{pgfscope}%
\begin{pgfscope}%
\pgfsys@transformshift{5.293050in}{5.164446in}%
\pgfsys@useobject{currentmarker}{}%
\end{pgfscope}%
\end{pgfscope}%
\begin{pgfscope}%
\pgfpathrectangle{\pgfqpoint{0.750000in}{0.600000in}}{\pgfqpoint{4.650000in}{4.800000in}} %
\pgfusepath{clip}%
\pgfsetbuttcap%
\pgfsetroundjoin%
\pgfsetlinewidth{0.501875pt}%
\definecolor{currentstroke}{rgb}{0.000000,0.000000,0.000000}%
\pgfsetstrokecolor{currentstroke}%
\pgfsetdash{{1.000000pt}{3.000000pt}}{0.000000pt}%
\pgfpathmoveto{\pgfqpoint{0.750000in}{0.600000in}}%
\pgfpathlineto{\pgfqpoint{0.750000in}{5.400000in}}%
\pgfusepath{stroke}%
\end{pgfscope}%
\begin{pgfscope}%
\pgfsetbuttcap%
\pgfsetroundjoin%
\definecolor{currentfill}{rgb}{0.000000,0.000000,0.000000}%
\pgfsetfillcolor{currentfill}%
\pgfsetlinewidth{0.501875pt}%
\definecolor{currentstroke}{rgb}{0.000000,0.000000,0.000000}%
\pgfsetstrokecolor{currentstroke}%
\pgfsetdash{}{0pt}%
\pgfsys@defobject{currentmarker}{\pgfqpoint{0.000000in}{0.000000in}}{\pgfqpoint{0.000000in}{0.055556in}}{%
\pgfpathmoveto{\pgfqpoint{0.000000in}{0.000000in}}%
\pgfpathlineto{\pgfqpoint{0.000000in}{0.055556in}}%
\pgfusepath{stroke,fill}%
}%
\begin{pgfscope}%
\pgfsys@transformshift{0.750000in}{0.600000in}%
\pgfsys@useobject{currentmarker}{}%
\end{pgfscope}%
\end{pgfscope}%
\begin{pgfscope}%
\pgfsetbuttcap%
\pgfsetroundjoin%
\definecolor{currentfill}{rgb}{0.000000,0.000000,0.000000}%
\pgfsetfillcolor{currentfill}%
\pgfsetlinewidth{0.501875pt}%
\definecolor{currentstroke}{rgb}{0.000000,0.000000,0.000000}%
\pgfsetstrokecolor{currentstroke}%
\pgfsetdash{}{0pt}%
\pgfsys@defobject{currentmarker}{\pgfqpoint{0.000000in}{-0.055556in}}{\pgfqpoint{0.000000in}{0.000000in}}{%
\pgfpathmoveto{\pgfqpoint{0.000000in}{0.000000in}}%
\pgfpathlineto{\pgfqpoint{0.000000in}{-0.055556in}}%
\pgfusepath{stroke,fill}%
}%
\begin{pgfscope}%
\pgfsys@transformshift{0.750000in}{5.400000in}%
\pgfsys@useobject{currentmarker}{}%
\end{pgfscope}%
\end{pgfscope}%
\begin{pgfscope}%
\pgftext[x=0.750000in,y=0.544444in,,top]{{\rmfamily\fontsize{12.000000}{14.400000}\selectfont \(\displaystyle 0.0\)}}%
\end{pgfscope}%
\begin{pgfscope}%
\pgfpathrectangle{\pgfqpoint{0.750000in}{0.600000in}}{\pgfqpoint{4.650000in}{4.800000in}} %
\pgfusepath{clip}%
\pgfsetbuttcap%
\pgfsetroundjoin%
\pgfsetlinewidth{0.501875pt}%
\definecolor{currentstroke}{rgb}{0.000000,0.000000,0.000000}%
\pgfsetstrokecolor{currentstroke}%
\pgfsetdash{{1.000000pt}{3.000000pt}}{0.000000pt}%
\pgfpathmoveto{\pgfqpoint{1.912500in}{0.600000in}}%
\pgfpathlineto{\pgfqpoint{1.912500in}{5.400000in}}%
\pgfusepath{stroke}%
\end{pgfscope}%
\begin{pgfscope}%
\pgfsetbuttcap%
\pgfsetroundjoin%
\definecolor{currentfill}{rgb}{0.000000,0.000000,0.000000}%
\pgfsetfillcolor{currentfill}%
\pgfsetlinewidth{0.501875pt}%
\definecolor{currentstroke}{rgb}{0.000000,0.000000,0.000000}%
\pgfsetstrokecolor{currentstroke}%
\pgfsetdash{}{0pt}%
\pgfsys@defobject{currentmarker}{\pgfqpoint{0.000000in}{0.000000in}}{\pgfqpoint{0.000000in}{0.055556in}}{%
\pgfpathmoveto{\pgfqpoint{0.000000in}{0.000000in}}%
\pgfpathlineto{\pgfqpoint{0.000000in}{0.055556in}}%
\pgfusepath{stroke,fill}%
}%
\begin{pgfscope}%
\pgfsys@transformshift{1.912500in}{0.600000in}%
\pgfsys@useobject{currentmarker}{}%
\end{pgfscope}%
\end{pgfscope}%
\begin{pgfscope}%
\pgfsetbuttcap%
\pgfsetroundjoin%
\definecolor{currentfill}{rgb}{0.000000,0.000000,0.000000}%
\pgfsetfillcolor{currentfill}%
\pgfsetlinewidth{0.501875pt}%
\definecolor{currentstroke}{rgb}{0.000000,0.000000,0.000000}%
\pgfsetstrokecolor{currentstroke}%
\pgfsetdash{}{0pt}%
\pgfsys@defobject{currentmarker}{\pgfqpoint{0.000000in}{-0.055556in}}{\pgfqpoint{0.000000in}{0.000000in}}{%
\pgfpathmoveto{\pgfqpoint{0.000000in}{0.000000in}}%
\pgfpathlineto{\pgfqpoint{0.000000in}{-0.055556in}}%
\pgfusepath{stroke,fill}%
}%
\begin{pgfscope}%
\pgfsys@transformshift{1.912500in}{5.400000in}%
\pgfsys@useobject{currentmarker}{}%
\end{pgfscope}%
\end{pgfscope}%
\begin{pgfscope}%
\pgftext[x=1.912500in,y=0.544444in,,top]{{\rmfamily\fontsize{12.000000}{14.400000}\selectfont \(\displaystyle 0.5\)}}%
\end{pgfscope}%
\begin{pgfscope}%
\pgfpathrectangle{\pgfqpoint{0.750000in}{0.600000in}}{\pgfqpoint{4.650000in}{4.800000in}} %
\pgfusepath{clip}%
\pgfsetbuttcap%
\pgfsetroundjoin%
\pgfsetlinewidth{0.501875pt}%
\definecolor{currentstroke}{rgb}{0.000000,0.000000,0.000000}%
\pgfsetstrokecolor{currentstroke}%
\pgfsetdash{{1.000000pt}{3.000000pt}}{0.000000pt}%
\pgfpathmoveto{\pgfqpoint{3.075000in}{0.600000in}}%
\pgfpathlineto{\pgfqpoint{3.075000in}{5.400000in}}%
\pgfusepath{stroke}%
\end{pgfscope}%
\begin{pgfscope}%
\pgfsetbuttcap%
\pgfsetroundjoin%
\definecolor{currentfill}{rgb}{0.000000,0.000000,0.000000}%
\pgfsetfillcolor{currentfill}%
\pgfsetlinewidth{0.501875pt}%
\definecolor{currentstroke}{rgb}{0.000000,0.000000,0.000000}%
\pgfsetstrokecolor{currentstroke}%
\pgfsetdash{}{0pt}%
\pgfsys@defobject{currentmarker}{\pgfqpoint{0.000000in}{0.000000in}}{\pgfqpoint{0.000000in}{0.055556in}}{%
\pgfpathmoveto{\pgfqpoint{0.000000in}{0.000000in}}%
\pgfpathlineto{\pgfqpoint{0.000000in}{0.055556in}}%
\pgfusepath{stroke,fill}%
}%
\begin{pgfscope}%
\pgfsys@transformshift{3.075000in}{0.600000in}%
\pgfsys@useobject{currentmarker}{}%
\end{pgfscope}%
\end{pgfscope}%
\begin{pgfscope}%
\pgfsetbuttcap%
\pgfsetroundjoin%
\definecolor{currentfill}{rgb}{0.000000,0.000000,0.000000}%
\pgfsetfillcolor{currentfill}%
\pgfsetlinewidth{0.501875pt}%
\definecolor{currentstroke}{rgb}{0.000000,0.000000,0.000000}%
\pgfsetstrokecolor{currentstroke}%
\pgfsetdash{}{0pt}%
\pgfsys@defobject{currentmarker}{\pgfqpoint{0.000000in}{-0.055556in}}{\pgfqpoint{0.000000in}{0.000000in}}{%
\pgfpathmoveto{\pgfqpoint{0.000000in}{0.000000in}}%
\pgfpathlineto{\pgfqpoint{0.000000in}{-0.055556in}}%
\pgfusepath{stroke,fill}%
}%
\begin{pgfscope}%
\pgfsys@transformshift{3.075000in}{5.400000in}%
\pgfsys@useobject{currentmarker}{}%
\end{pgfscope}%
\end{pgfscope}%
\begin{pgfscope}%
\pgftext[x=3.075000in,y=0.544444in,,top]{{\rmfamily\fontsize{12.000000}{14.400000}\selectfont \(\displaystyle 1.0\)}}%
\end{pgfscope}%
\begin{pgfscope}%
\pgfpathrectangle{\pgfqpoint{0.750000in}{0.600000in}}{\pgfqpoint{4.650000in}{4.800000in}} %
\pgfusepath{clip}%
\pgfsetbuttcap%
\pgfsetroundjoin%
\pgfsetlinewidth{0.501875pt}%
\definecolor{currentstroke}{rgb}{0.000000,0.000000,0.000000}%
\pgfsetstrokecolor{currentstroke}%
\pgfsetdash{{1.000000pt}{3.000000pt}}{0.000000pt}%
\pgfpathmoveto{\pgfqpoint{4.237500in}{0.600000in}}%
\pgfpathlineto{\pgfqpoint{4.237500in}{5.400000in}}%
\pgfusepath{stroke}%
\end{pgfscope}%
\begin{pgfscope}%
\pgfsetbuttcap%
\pgfsetroundjoin%
\definecolor{currentfill}{rgb}{0.000000,0.000000,0.000000}%
\pgfsetfillcolor{currentfill}%
\pgfsetlinewidth{0.501875pt}%
\definecolor{currentstroke}{rgb}{0.000000,0.000000,0.000000}%
\pgfsetstrokecolor{currentstroke}%
\pgfsetdash{}{0pt}%
\pgfsys@defobject{currentmarker}{\pgfqpoint{0.000000in}{0.000000in}}{\pgfqpoint{0.000000in}{0.055556in}}{%
\pgfpathmoveto{\pgfqpoint{0.000000in}{0.000000in}}%
\pgfpathlineto{\pgfqpoint{0.000000in}{0.055556in}}%
\pgfusepath{stroke,fill}%
}%
\begin{pgfscope}%
\pgfsys@transformshift{4.237500in}{0.600000in}%
\pgfsys@useobject{currentmarker}{}%
\end{pgfscope}%
\end{pgfscope}%
\begin{pgfscope}%
\pgfsetbuttcap%
\pgfsetroundjoin%
\definecolor{currentfill}{rgb}{0.000000,0.000000,0.000000}%
\pgfsetfillcolor{currentfill}%
\pgfsetlinewidth{0.501875pt}%
\definecolor{currentstroke}{rgb}{0.000000,0.000000,0.000000}%
\pgfsetstrokecolor{currentstroke}%
\pgfsetdash{}{0pt}%
\pgfsys@defobject{currentmarker}{\pgfqpoint{0.000000in}{-0.055556in}}{\pgfqpoint{0.000000in}{0.000000in}}{%
\pgfpathmoveto{\pgfqpoint{0.000000in}{0.000000in}}%
\pgfpathlineto{\pgfqpoint{0.000000in}{-0.055556in}}%
\pgfusepath{stroke,fill}%
}%
\begin{pgfscope}%
\pgfsys@transformshift{4.237500in}{5.400000in}%
\pgfsys@useobject{currentmarker}{}%
\end{pgfscope}%
\end{pgfscope}%
\begin{pgfscope}%
\pgftext[x=4.237500in,y=0.544444in,,top]{{\rmfamily\fontsize{12.000000}{14.400000}\selectfont \(\displaystyle 1.5\)}}%
\end{pgfscope}%
\begin{pgfscope}%
\pgfpathrectangle{\pgfqpoint{0.750000in}{0.600000in}}{\pgfqpoint{4.650000in}{4.800000in}} %
\pgfusepath{clip}%
\pgfsetbuttcap%
\pgfsetroundjoin%
\pgfsetlinewidth{0.501875pt}%
\definecolor{currentstroke}{rgb}{0.000000,0.000000,0.000000}%
\pgfsetstrokecolor{currentstroke}%
\pgfsetdash{{1.000000pt}{3.000000pt}}{0.000000pt}%
\pgfpathmoveto{\pgfqpoint{5.400000in}{0.600000in}}%
\pgfpathlineto{\pgfqpoint{5.400000in}{5.400000in}}%
\pgfusepath{stroke}%
\end{pgfscope}%
\begin{pgfscope}%
\pgfsetbuttcap%
\pgfsetroundjoin%
\definecolor{currentfill}{rgb}{0.000000,0.000000,0.000000}%
\pgfsetfillcolor{currentfill}%
\pgfsetlinewidth{0.501875pt}%
\definecolor{currentstroke}{rgb}{0.000000,0.000000,0.000000}%
\pgfsetstrokecolor{currentstroke}%
\pgfsetdash{}{0pt}%
\pgfsys@defobject{currentmarker}{\pgfqpoint{0.000000in}{0.000000in}}{\pgfqpoint{0.000000in}{0.055556in}}{%
\pgfpathmoveto{\pgfqpoint{0.000000in}{0.000000in}}%
\pgfpathlineto{\pgfqpoint{0.000000in}{0.055556in}}%
\pgfusepath{stroke,fill}%
}%
\begin{pgfscope}%
\pgfsys@transformshift{5.400000in}{0.600000in}%
\pgfsys@useobject{currentmarker}{}%
\end{pgfscope}%
\end{pgfscope}%
\begin{pgfscope}%
\pgfsetbuttcap%
\pgfsetroundjoin%
\definecolor{currentfill}{rgb}{0.000000,0.000000,0.000000}%
\pgfsetfillcolor{currentfill}%
\pgfsetlinewidth{0.501875pt}%
\definecolor{currentstroke}{rgb}{0.000000,0.000000,0.000000}%
\pgfsetstrokecolor{currentstroke}%
\pgfsetdash{}{0pt}%
\pgfsys@defobject{currentmarker}{\pgfqpoint{0.000000in}{-0.055556in}}{\pgfqpoint{0.000000in}{0.000000in}}{%
\pgfpathmoveto{\pgfqpoint{0.000000in}{0.000000in}}%
\pgfpathlineto{\pgfqpoint{0.000000in}{-0.055556in}}%
\pgfusepath{stroke,fill}%
}%
\begin{pgfscope}%
\pgfsys@transformshift{5.400000in}{5.400000in}%
\pgfsys@useobject{currentmarker}{}%
\end{pgfscope}%
\end{pgfscope}%
\begin{pgfscope}%
\pgftext[x=5.400000in,y=0.544444in,,top]{{\rmfamily\fontsize{12.000000}{14.400000}\selectfont \(\displaystyle 2.0\)}}%
\end{pgfscope}%
\begin{pgfscope}%
\pgftext[x=3.075000in,y=0.367593in,,top]{{\rmfamily\fontsize{12.000000}{14.400000}\selectfont Distance along Beam}}%
\end{pgfscope}%
\begin{pgfscope}%
\pgfpathrectangle{\pgfqpoint{0.750000in}{0.600000in}}{\pgfqpoint{4.650000in}{4.800000in}} %
\pgfusepath{clip}%
\pgfsetbuttcap%
\pgfsetroundjoin%
\pgfsetlinewidth{0.501875pt}%
\definecolor{currentstroke}{rgb}{0.000000,0.000000,0.000000}%
\pgfsetstrokecolor{currentstroke}%
\pgfsetdash{{1.000000pt}{3.000000pt}}{0.000000pt}%
\pgfpathmoveto{\pgfqpoint{0.750000in}{5.400000in}}%
\pgfpathlineto{\pgfqpoint{5.400000in}{5.400000in}}%
\pgfusepath{stroke}%
\end{pgfscope}%
\begin{pgfscope}%
\pgfsetbuttcap%
\pgfsetroundjoin%
\definecolor{currentfill}{rgb}{0.000000,0.000000,0.000000}%
\pgfsetfillcolor{currentfill}%
\pgfsetlinewidth{0.501875pt}%
\definecolor{currentstroke}{rgb}{0.000000,0.000000,0.000000}%
\pgfsetstrokecolor{currentstroke}%
\pgfsetdash{}{0pt}%
\pgfsys@defobject{currentmarker}{\pgfqpoint{0.000000in}{0.000000in}}{\pgfqpoint{0.055556in}{0.000000in}}{%
\pgfpathmoveto{\pgfqpoint{0.000000in}{0.000000in}}%
\pgfpathlineto{\pgfqpoint{0.055556in}{0.000000in}}%
\pgfusepath{stroke,fill}%
}%
\begin{pgfscope}%
\pgfsys@transformshift{0.750000in}{5.400000in}%
\pgfsys@useobject{currentmarker}{}%
\end{pgfscope}%
\end{pgfscope}%
\begin{pgfscope}%
\pgfsetbuttcap%
\pgfsetroundjoin%
\definecolor{currentfill}{rgb}{0.000000,0.000000,0.000000}%
\pgfsetfillcolor{currentfill}%
\pgfsetlinewidth{0.501875pt}%
\definecolor{currentstroke}{rgb}{0.000000,0.000000,0.000000}%
\pgfsetstrokecolor{currentstroke}%
\pgfsetdash{}{0pt}%
\pgfsys@defobject{currentmarker}{\pgfqpoint{-0.055556in}{0.000000in}}{\pgfqpoint{0.000000in}{0.000000in}}{%
\pgfpathmoveto{\pgfqpoint{0.000000in}{0.000000in}}%
\pgfpathlineto{\pgfqpoint{-0.055556in}{0.000000in}}%
\pgfusepath{stroke,fill}%
}%
\begin{pgfscope}%
\pgfsys@transformshift{5.400000in}{5.400000in}%
\pgfsys@useobject{currentmarker}{}%
\end{pgfscope}%
\end{pgfscope}%
\begin{pgfscope}%
\pgftext[x=0.694444in,y=5.400000in,right,]{{\rmfamily\fontsize{12.000000}{14.400000}\selectfont \(\displaystyle 0.0\)}}%
\end{pgfscope}%
\begin{pgfscope}%
\pgfpathrectangle{\pgfqpoint{0.750000in}{0.600000in}}{\pgfqpoint{4.650000in}{4.800000in}} %
\pgfusepath{clip}%
\pgfsetbuttcap%
\pgfsetroundjoin%
\pgfsetlinewidth{0.501875pt}%
\definecolor{currentstroke}{rgb}{0.000000,0.000000,0.000000}%
\pgfsetstrokecolor{currentstroke}%
\pgfsetdash{{1.000000pt}{3.000000pt}}{0.000000pt}%
\pgfpathmoveto{\pgfqpoint{0.750000in}{4.292308in}}%
\pgfpathlineto{\pgfqpoint{5.400000in}{4.292308in}}%
\pgfusepath{stroke}%
\end{pgfscope}%
\begin{pgfscope}%
\pgfsetbuttcap%
\pgfsetroundjoin%
\definecolor{currentfill}{rgb}{0.000000,0.000000,0.000000}%
\pgfsetfillcolor{currentfill}%
\pgfsetlinewidth{0.501875pt}%
\definecolor{currentstroke}{rgb}{0.000000,0.000000,0.000000}%
\pgfsetstrokecolor{currentstroke}%
\pgfsetdash{}{0pt}%
\pgfsys@defobject{currentmarker}{\pgfqpoint{0.000000in}{0.000000in}}{\pgfqpoint{0.055556in}{0.000000in}}{%
\pgfpathmoveto{\pgfqpoint{0.000000in}{0.000000in}}%
\pgfpathlineto{\pgfqpoint{0.055556in}{0.000000in}}%
\pgfusepath{stroke,fill}%
}%
\begin{pgfscope}%
\pgfsys@transformshift{0.750000in}{4.292308in}%
\pgfsys@useobject{currentmarker}{}%
\end{pgfscope}%
\end{pgfscope}%
\begin{pgfscope}%
\pgfsetbuttcap%
\pgfsetroundjoin%
\definecolor{currentfill}{rgb}{0.000000,0.000000,0.000000}%
\pgfsetfillcolor{currentfill}%
\pgfsetlinewidth{0.501875pt}%
\definecolor{currentstroke}{rgb}{0.000000,0.000000,0.000000}%
\pgfsetstrokecolor{currentstroke}%
\pgfsetdash{}{0pt}%
\pgfsys@defobject{currentmarker}{\pgfqpoint{-0.055556in}{0.000000in}}{\pgfqpoint{0.000000in}{0.000000in}}{%
\pgfpathmoveto{\pgfqpoint{0.000000in}{0.000000in}}%
\pgfpathlineto{\pgfqpoint{-0.055556in}{0.000000in}}%
\pgfusepath{stroke,fill}%
}%
\begin{pgfscope}%
\pgfsys@transformshift{5.400000in}{4.292308in}%
\pgfsys@useobject{currentmarker}{}%
\end{pgfscope}%
\end{pgfscope}%
\begin{pgfscope}%
\pgftext[x=0.694444in,y=4.292308in,right,]{{\rmfamily\fontsize{12.000000}{14.400000}\selectfont \(\displaystyle -0.3\)}}%
\end{pgfscope}%
\begin{pgfscope}%
\pgfpathrectangle{\pgfqpoint{0.750000in}{0.600000in}}{\pgfqpoint{4.650000in}{4.800000in}} %
\pgfusepath{clip}%
\pgfsetbuttcap%
\pgfsetroundjoin%
\pgfsetlinewidth{0.501875pt}%
\definecolor{currentstroke}{rgb}{0.000000,0.000000,0.000000}%
\pgfsetstrokecolor{currentstroke}%
\pgfsetdash{{1.000000pt}{3.000000pt}}{0.000000pt}%
\pgfpathmoveto{\pgfqpoint{0.750000in}{3.184615in}}%
\pgfpathlineto{\pgfqpoint{5.400000in}{3.184615in}}%
\pgfusepath{stroke}%
\end{pgfscope}%
\begin{pgfscope}%
\pgfsetbuttcap%
\pgfsetroundjoin%
\definecolor{currentfill}{rgb}{0.000000,0.000000,0.000000}%
\pgfsetfillcolor{currentfill}%
\pgfsetlinewidth{0.501875pt}%
\definecolor{currentstroke}{rgb}{0.000000,0.000000,0.000000}%
\pgfsetstrokecolor{currentstroke}%
\pgfsetdash{}{0pt}%
\pgfsys@defobject{currentmarker}{\pgfqpoint{0.000000in}{0.000000in}}{\pgfqpoint{0.055556in}{0.000000in}}{%
\pgfpathmoveto{\pgfqpoint{0.000000in}{0.000000in}}%
\pgfpathlineto{\pgfqpoint{0.055556in}{0.000000in}}%
\pgfusepath{stroke,fill}%
}%
\begin{pgfscope}%
\pgfsys@transformshift{0.750000in}{3.184615in}%
\pgfsys@useobject{currentmarker}{}%
\end{pgfscope}%
\end{pgfscope}%
\begin{pgfscope}%
\pgfsetbuttcap%
\pgfsetroundjoin%
\definecolor{currentfill}{rgb}{0.000000,0.000000,0.000000}%
\pgfsetfillcolor{currentfill}%
\pgfsetlinewidth{0.501875pt}%
\definecolor{currentstroke}{rgb}{0.000000,0.000000,0.000000}%
\pgfsetstrokecolor{currentstroke}%
\pgfsetdash{}{0pt}%
\pgfsys@defobject{currentmarker}{\pgfqpoint{-0.055556in}{0.000000in}}{\pgfqpoint{0.000000in}{0.000000in}}{%
\pgfpathmoveto{\pgfqpoint{0.000000in}{0.000000in}}%
\pgfpathlineto{\pgfqpoint{-0.055556in}{0.000000in}}%
\pgfusepath{stroke,fill}%
}%
\begin{pgfscope}%
\pgfsys@transformshift{5.400000in}{3.184615in}%
\pgfsys@useobject{currentmarker}{}%
\end{pgfscope}%
\end{pgfscope}%
\begin{pgfscope}%
\pgftext[x=0.694444in,y=3.184615in,right,]{{\rmfamily\fontsize{12.000000}{14.400000}\selectfont \(\displaystyle -0.6\)}}%
\end{pgfscope}%
\begin{pgfscope}%
\pgfpathrectangle{\pgfqpoint{0.750000in}{0.600000in}}{\pgfqpoint{4.650000in}{4.800000in}} %
\pgfusepath{clip}%
\pgfsetbuttcap%
\pgfsetroundjoin%
\pgfsetlinewidth{0.501875pt}%
\definecolor{currentstroke}{rgb}{0.000000,0.000000,0.000000}%
\pgfsetstrokecolor{currentstroke}%
\pgfsetdash{{1.000000pt}{3.000000pt}}{0.000000pt}%
\pgfpathmoveto{\pgfqpoint{0.750000in}{2.076923in}}%
\pgfpathlineto{\pgfqpoint{5.400000in}{2.076923in}}%
\pgfusepath{stroke}%
\end{pgfscope}%
\begin{pgfscope}%
\pgfsetbuttcap%
\pgfsetroundjoin%
\definecolor{currentfill}{rgb}{0.000000,0.000000,0.000000}%
\pgfsetfillcolor{currentfill}%
\pgfsetlinewidth{0.501875pt}%
\definecolor{currentstroke}{rgb}{0.000000,0.000000,0.000000}%
\pgfsetstrokecolor{currentstroke}%
\pgfsetdash{}{0pt}%
\pgfsys@defobject{currentmarker}{\pgfqpoint{0.000000in}{0.000000in}}{\pgfqpoint{0.055556in}{0.000000in}}{%
\pgfpathmoveto{\pgfqpoint{0.000000in}{0.000000in}}%
\pgfpathlineto{\pgfqpoint{0.055556in}{0.000000in}}%
\pgfusepath{stroke,fill}%
}%
\begin{pgfscope}%
\pgfsys@transformshift{0.750000in}{2.076923in}%
\pgfsys@useobject{currentmarker}{}%
\end{pgfscope}%
\end{pgfscope}%
\begin{pgfscope}%
\pgfsetbuttcap%
\pgfsetroundjoin%
\definecolor{currentfill}{rgb}{0.000000,0.000000,0.000000}%
\pgfsetfillcolor{currentfill}%
\pgfsetlinewidth{0.501875pt}%
\definecolor{currentstroke}{rgb}{0.000000,0.000000,0.000000}%
\pgfsetstrokecolor{currentstroke}%
\pgfsetdash{}{0pt}%
\pgfsys@defobject{currentmarker}{\pgfqpoint{-0.055556in}{0.000000in}}{\pgfqpoint{0.000000in}{0.000000in}}{%
\pgfpathmoveto{\pgfqpoint{0.000000in}{0.000000in}}%
\pgfpathlineto{\pgfqpoint{-0.055556in}{0.000000in}}%
\pgfusepath{stroke,fill}%
}%
\begin{pgfscope}%
\pgfsys@transformshift{5.400000in}{2.076923in}%
\pgfsys@useobject{currentmarker}{}%
\end{pgfscope}%
\end{pgfscope}%
\begin{pgfscope}%
\pgftext[x=0.694444in,y=2.076923in,right,]{{\rmfamily\fontsize{12.000000}{14.400000}\selectfont \(\displaystyle -0.9\)}}%
\end{pgfscope}%
\begin{pgfscope}%
\pgfpathrectangle{\pgfqpoint{0.750000in}{0.600000in}}{\pgfqpoint{4.650000in}{4.800000in}} %
\pgfusepath{clip}%
\pgfsetbuttcap%
\pgfsetroundjoin%
\pgfsetlinewidth{0.501875pt}%
\definecolor{currentstroke}{rgb}{0.000000,0.000000,0.000000}%
\pgfsetstrokecolor{currentstroke}%
\pgfsetdash{{1.000000pt}{3.000000pt}}{0.000000pt}%
\pgfpathmoveto{\pgfqpoint{0.750000in}{0.969231in}}%
\pgfpathlineto{\pgfqpoint{5.400000in}{0.969231in}}%
\pgfusepath{stroke}%
\end{pgfscope}%
\begin{pgfscope}%
\pgfsetbuttcap%
\pgfsetroundjoin%
\definecolor{currentfill}{rgb}{0.000000,0.000000,0.000000}%
\pgfsetfillcolor{currentfill}%
\pgfsetlinewidth{0.501875pt}%
\definecolor{currentstroke}{rgb}{0.000000,0.000000,0.000000}%
\pgfsetstrokecolor{currentstroke}%
\pgfsetdash{}{0pt}%
\pgfsys@defobject{currentmarker}{\pgfqpoint{0.000000in}{0.000000in}}{\pgfqpoint{0.055556in}{0.000000in}}{%
\pgfpathmoveto{\pgfqpoint{0.000000in}{0.000000in}}%
\pgfpathlineto{\pgfqpoint{0.055556in}{0.000000in}}%
\pgfusepath{stroke,fill}%
}%
\begin{pgfscope}%
\pgfsys@transformshift{0.750000in}{0.969231in}%
\pgfsys@useobject{currentmarker}{}%
\end{pgfscope}%
\end{pgfscope}%
\begin{pgfscope}%
\pgfsetbuttcap%
\pgfsetroundjoin%
\definecolor{currentfill}{rgb}{0.000000,0.000000,0.000000}%
\pgfsetfillcolor{currentfill}%
\pgfsetlinewidth{0.501875pt}%
\definecolor{currentstroke}{rgb}{0.000000,0.000000,0.000000}%
\pgfsetstrokecolor{currentstroke}%
\pgfsetdash{}{0pt}%
\pgfsys@defobject{currentmarker}{\pgfqpoint{-0.055556in}{0.000000in}}{\pgfqpoint{0.000000in}{0.000000in}}{%
\pgfpathmoveto{\pgfqpoint{0.000000in}{0.000000in}}%
\pgfpathlineto{\pgfqpoint{-0.055556in}{0.000000in}}%
\pgfusepath{stroke,fill}%
}%
\begin{pgfscope}%
\pgfsys@transformshift{5.400000in}{0.969231in}%
\pgfsys@useobject{currentmarker}{}%
\end{pgfscope}%
\end{pgfscope}%
\begin{pgfscope}%
\pgftext[x=0.694444in,y=0.969231in,right,]{{\rmfamily\fontsize{12.000000}{14.400000}\selectfont \(\displaystyle -1.2\)}}%
\end{pgfscope}%
\begin{pgfscope}%
\pgftext[x=0.286846in,y=3.000000in,,bottom,rotate=90.000000]{{\rmfamily\fontsize{12.000000}{14.400000}\selectfont Deflection}}%
\end{pgfscope}%
\begin{pgfscope}%
\pgftext[x=0.750000in,y=5.441667in,left,base]{{\rmfamily\fontsize{12.000000}{14.400000}\selectfont \(\displaystyle \times10^{-5}\)}}%
\end{pgfscope}%
\begin{pgfscope}%
\pgfsetrectcap%
\pgfsetroundjoin%
\pgfsetlinewidth{1.003750pt}%
\definecolor{currentstroke}{rgb}{0.000000,0.000000,0.000000}%
\pgfsetstrokecolor{currentstroke}%
\pgfsetdash{}{0pt}%
\pgfpathmoveto{\pgfqpoint{0.750000in}{5.400000in}}%
\pgfpathlineto{\pgfqpoint{5.400000in}{5.400000in}}%
\pgfusepath{stroke}%
\end{pgfscope}%
\begin{pgfscope}%
\pgfsetrectcap%
\pgfsetroundjoin%
\pgfsetlinewidth{1.003750pt}%
\definecolor{currentstroke}{rgb}{0.000000,0.000000,0.000000}%
\pgfsetstrokecolor{currentstroke}%
\pgfsetdash{}{0pt}%
\pgfpathmoveto{\pgfqpoint{5.400000in}{0.600000in}}%
\pgfpathlineto{\pgfqpoint{5.400000in}{5.400000in}}%
\pgfusepath{stroke}%
\end{pgfscope}%
\begin{pgfscope}%
\pgfsetrectcap%
\pgfsetroundjoin%
\pgfsetlinewidth{1.003750pt}%
\definecolor{currentstroke}{rgb}{0.000000,0.000000,0.000000}%
\pgfsetstrokecolor{currentstroke}%
\pgfsetdash{}{0pt}%
\pgfpathmoveto{\pgfqpoint{0.750000in}{0.600000in}}%
\pgfpathlineto{\pgfqpoint{5.400000in}{0.600000in}}%
\pgfusepath{stroke}%
\end{pgfscope}%
\begin{pgfscope}%
\pgfsetrectcap%
\pgfsetroundjoin%
\pgfsetlinewidth{1.003750pt}%
\definecolor{currentstroke}{rgb}{0.000000,0.000000,0.000000}%
\pgfsetstrokecolor{currentstroke}%
\pgfsetdash{}{0pt}%
\pgfpathmoveto{\pgfqpoint{0.750000in}{0.600000in}}%
\pgfpathlineto{\pgfqpoint{0.750000in}{5.400000in}}%
\pgfusepath{stroke}%
\end{pgfscope}%
\begin{pgfscope}%
\pgftext[x=3.075000in,y=5.469444in,,base]{{\rmfamily\fontsize{14.400000}{17.280000}\selectfont Unloaded EPP Beam}}%
\end{pgfscope}%
\begin{pgfscope}%
\pgfsetrectcap%
\pgfsetroundjoin%
\definecolor{currentfill}{rgb}{1.000000,1.000000,1.000000}%
\pgfsetfillcolor{currentfill}%
\pgfsetlinewidth{1.003750pt}%
\definecolor{currentstroke}{rgb}{0.000000,0.000000,0.000000}%
\pgfsetstrokecolor{currentstroke}%
\pgfsetdash{}{0pt}%
\pgfpathmoveto{\pgfqpoint{1.661865in}{4.224445in}}%
\pgfpathlineto{\pgfqpoint{4.488135in}{4.224445in}}%
\pgfpathlineto{\pgfqpoint{4.488135in}{5.400000in}}%
\pgfpathlineto{\pgfqpoint{1.661865in}{5.400000in}}%
\pgfpathlineto{\pgfqpoint{1.661865in}{4.224445in}}%
\pgfpathclose%
\pgfusepath{stroke,fill}%
\end{pgfscope}%
\begin{pgfscope}%
\pgfsetrectcap%
\pgfsetroundjoin%
\pgfsetlinewidth{1.003750pt}%
\definecolor{currentstroke}{rgb}{0.000000,0.000000,1.000000}%
\pgfsetstrokecolor{currentstroke}%
\pgfsetdash{}{0pt}%
\pgfpathmoveto{\pgfqpoint{1.801865in}{5.250000in}}%
\pgfpathlineto{\pgfqpoint{2.081865in}{5.250000in}}%
\pgfusepath{stroke}%
\end{pgfscope}%
\begin{pgfscope}%
\pgftext[x=2.301865in,y=5.180000in,left,base]{{\rmfamily\fontsize{14.400000}{17.280000}\selectfont Abaqus EPP Beam}}%
\end{pgfscope}%
\begin{pgfscope}%
\pgfsetrectcap%
\pgfsetroundjoin%
\pgfsetlinewidth{1.003750pt}%
\definecolor{currentstroke}{rgb}{0.000000,0.500000,0.000000}%
\pgfsetstrokecolor{currentstroke}%
\pgfsetdash{}{0pt}%
\pgfpathmoveto{\pgfqpoint{1.801865in}{4.971111in}}%
\pgfpathlineto{\pgfqpoint{2.081865in}{4.971111in}}%
\pgfusepath{stroke}%
\end{pgfscope}%
\begin{pgfscope}%
\pgfsetbuttcap%
\pgfsetmiterjoin%
\definecolor{currentfill}{rgb}{0.000000,0.500000,0.000000}%
\pgfsetfillcolor{currentfill}%
\pgfsetlinewidth{0.501875pt}%
\definecolor{currentstroke}{rgb}{0.000000,0.000000,0.000000}%
\pgfsetstrokecolor{currentstroke}%
\pgfsetdash{}{0pt}%
\pgfsys@defobject{currentmarker}{\pgfqpoint{-0.041667in}{-0.041667in}}{\pgfqpoint{0.041667in}{0.041667in}}{%
\pgfpathmoveto{\pgfqpoint{0.000000in}{0.041667in}}%
\pgfpathlineto{\pgfqpoint{-0.041667in}{-0.041667in}}%
\pgfpathlineto{\pgfqpoint{0.041667in}{-0.041667in}}%
\pgfpathclose%
\pgfusepath{stroke,fill}%
}%
\begin{pgfscope}%
\pgfsys@transformshift{1.801865in}{4.971111in}%
\pgfsys@useobject{currentmarker}{}%
\end{pgfscope}%
\begin{pgfscope}%
\pgfsys@transformshift{2.081865in}{4.971111in}%
\pgfsys@useobject{currentmarker}{}%
\end{pgfscope}%
\end{pgfscope}%
\begin{pgfscope}%
\pgftext[x=2.301865in,y=4.901111in,left,base]{{\rmfamily\fontsize{14.400000}{17.280000}\selectfont 1000 nodes, horizon 0.20}}%
\end{pgfscope}%
\begin{pgfscope}%
\pgfsetrectcap%
\pgfsetroundjoin%
\pgfsetlinewidth{1.003750pt}%
\definecolor{currentstroke}{rgb}{1.000000,0.000000,0.000000}%
\pgfsetstrokecolor{currentstroke}%
\pgfsetdash{}{0pt}%
\pgfpathmoveto{\pgfqpoint{1.801865in}{4.692223in}}%
\pgfpathlineto{\pgfqpoint{2.081865in}{4.692223in}}%
\pgfusepath{stroke}%
\end{pgfscope}%
\begin{pgfscope}%
\pgfsetbuttcap%
\pgfsetmiterjoin%
\definecolor{currentfill}{rgb}{1.000000,0.000000,0.000000}%
\pgfsetfillcolor{currentfill}%
\pgfsetlinewidth{0.501875pt}%
\definecolor{currentstroke}{rgb}{0.000000,0.000000,0.000000}%
\pgfsetstrokecolor{currentstroke}%
\pgfsetdash{}{0pt}%
\pgfsys@defobject{currentmarker}{\pgfqpoint{-0.041667in}{-0.041667in}}{\pgfqpoint{0.041667in}{0.041667in}}{%
\pgfpathmoveto{\pgfqpoint{0.041667in}{-0.000000in}}%
\pgfpathlineto{\pgfqpoint{-0.041667in}{0.041667in}}%
\pgfpathlineto{\pgfqpoint{-0.041667in}{-0.041667in}}%
\pgfpathclose%
\pgfusepath{stroke,fill}%
}%
\begin{pgfscope}%
\pgfsys@transformshift{1.801865in}{4.692223in}%
\pgfsys@useobject{currentmarker}{}%
\end{pgfscope}%
\begin{pgfscope}%
\pgfsys@transformshift{2.081865in}{4.692223in}%
\pgfsys@useobject{currentmarker}{}%
\end{pgfscope}%
\end{pgfscope}%
\begin{pgfscope}%
\pgftext[x=2.301865in,y=4.622223in,left,base]{{\rmfamily\fontsize{14.400000}{17.280000}\selectfont 1000 nodes, horizon 0.15}}%
\end{pgfscope}%
\begin{pgfscope}%
\pgfsetrectcap%
\pgfsetroundjoin%
\pgfsetlinewidth{1.003750pt}%
\definecolor{currentstroke}{rgb}{0.000000,0.750000,0.750000}%
\pgfsetstrokecolor{currentstroke}%
\pgfsetdash{}{0pt}%
\pgfpathmoveto{\pgfqpoint{1.801865in}{4.413334in}}%
\pgfpathlineto{\pgfqpoint{2.081865in}{4.413334in}}%
\pgfusepath{stroke}%
\end{pgfscope}%
\begin{pgfscope}%
\pgfsetbuttcap%
\pgfsetmiterjoin%
\definecolor{currentfill}{rgb}{0.000000,0.750000,0.750000}%
\pgfsetfillcolor{currentfill}%
\pgfsetlinewidth{0.501875pt}%
\definecolor{currentstroke}{rgb}{0.000000,0.000000,0.000000}%
\pgfsetstrokecolor{currentstroke}%
\pgfsetdash{}{0pt}%
\pgfsys@defobject{currentmarker}{\pgfqpoint{-0.041667in}{-0.041667in}}{\pgfqpoint{0.041667in}{0.041667in}}{%
\pgfpathmoveto{\pgfqpoint{-0.000000in}{-0.041667in}}%
\pgfpathlineto{\pgfqpoint{0.041667in}{0.041667in}}%
\pgfpathlineto{\pgfqpoint{-0.041667in}{0.041667in}}%
\pgfpathclose%
\pgfusepath{stroke,fill}%
}%
\begin{pgfscope}%
\pgfsys@transformshift{1.801865in}{4.413334in}%
\pgfsys@useobject{currentmarker}{}%
\end{pgfscope}%
\begin{pgfscope}%
\pgfsys@transformshift{2.081865in}{4.413334in}%
\pgfsys@useobject{currentmarker}{}%
\end{pgfscope}%
\end{pgfscope}%
\begin{pgfscope}%
\pgftext[x=2.301865in,y=4.343334in,left,base]{{\rmfamily\fontsize{14.400000}{17.280000}\selectfont 1000 nodes, horizon 0.10}}%
\end{pgfscope}%
\end{pgfpicture}%
\makeatother%
\endgroup%
}
\caption{H-convergence}
\label{fig:ResidualPlasticityH}
\endminipage
\end{figure}

It is more difficult to verify the brittle material model because brittle failure is unstable.
When a crack begins, moment is transferred to other bond pairs, and failure progresses creating a plastic hinge.
This is borne out by the results in \cref{fig:brittleBeam}, in which ``Nodal Health'' represents the fraction of bond-pairs about each node that have not failed.

\begin{figure}[h]
  \centering
  \scalebox{.65}{%% Creator: Matplotlib, PGF backend
%%
%% To include the figure in your LaTeX document, write
%%   \input{<filename>.pgf}
%%
%% Make sure the required packages are loaded in your preamble
%%   \usepackage{pgf}
%%
%% Figures using additional raster images can only be included by \input if
%% they are in the same directory as the main LaTeX file. For loading figures
%% from other directories you can use the `import` package
%%   \usepackage{import}
%% and then include the figures with
%%   \import{<path to file>}{<filename>.pgf}
%%
%% Matplotlib used the following preamble
%%
\begingroup%
\makeatletter%
\begin{pgfpicture}%
\pgfpathrectangle{\pgfpointorigin}{\pgfqpoint{6.000000in}{7.000000in}}%
\pgfusepath{use as bounding box}%
\begin{pgfscope}%
\pgfsetrectcap%
\pgfsetroundjoin%
\definecolor{currentfill}{rgb}{1.000000,1.000000,1.000000}%
\pgfsetfillcolor{currentfill}%
\pgfsetlinewidth{0.000000pt}%
\definecolor{currentstroke}{rgb}{1.000000,1.000000,1.000000}%
\pgfsetstrokecolor{currentstroke}%
\pgfsetdash{}{0pt}%
\pgfpathmoveto{\pgfqpoint{0.000000in}{0.000000in}}%
\pgfpathlineto{\pgfqpoint{6.000000in}{0.000000in}}%
\pgfpathlineto{\pgfqpoint{6.000000in}{7.000000in}}%
\pgfpathlineto{\pgfqpoint{0.000000in}{7.000000in}}%
\pgfpathclose%
\pgfusepath{fill}%
\end{pgfscope}%
\begin{pgfscope}%
\pgfsetrectcap%
\pgfsetroundjoin%
\definecolor{currentfill}{rgb}{1.000000,1.000000,1.000000}%
\pgfsetfillcolor{currentfill}%
\pgfsetlinewidth{0.000000pt}%
\definecolor{currentstroke}{rgb}{0.000000,0.000000,0.000000}%
\pgfsetstrokecolor{currentstroke}%
\pgfsetdash{}{0pt}%
\pgfpathmoveto{\pgfqpoint{0.600000in}{5.082609in}}%
\pgfpathlineto{\pgfqpoint{5.280000in}{5.082609in}}%
\pgfpathlineto{\pgfqpoint{5.280000in}{6.300000in}}%
\pgfpathlineto{\pgfqpoint{0.600000in}{6.300000in}}%
\pgfpathclose%
\pgfusepath{fill}%
\end{pgfscope}%
\begin{pgfscope}%
\pgfpathrectangle{\pgfqpoint{0.600000in}{5.082609in}}{\pgfqpoint{4.680000in}{1.217391in}} %
\pgfusepath{clip}%
\pgfsetbuttcap%
\pgfsetroundjoin%
\pgfsetlinewidth{1.003750pt}%
\definecolor{currentstroke}{rgb}{0.000000,0.000000,1.000000}%
\pgfsetstrokecolor{currentstroke}%
\pgfsetdash{{6.000000pt}{6.000000pt}}{0.000000pt}%
\pgfpathmoveto{\pgfqpoint{0.600000in}{6.300000in}}%
\pgfpathlineto{\pgfqpoint{0.927600in}{6.087904in}}%
\pgfpathlineto{\pgfqpoint{1.091400in}{5.984367in}}%
\pgfpathlineto{\pgfqpoint{1.231800in}{5.898037in}}%
\pgfpathlineto{\pgfqpoint{1.348800in}{5.828301in}}%
\pgfpathlineto{\pgfqpoint{1.465800in}{5.760976in}}%
\pgfpathlineto{\pgfqpoint{1.582800in}{5.696445in}}%
\pgfpathlineto{\pgfqpoint{1.676400in}{5.647088in}}%
\pgfpathlineto{\pgfqpoint{1.770000in}{5.599959in}}%
\pgfpathlineto{\pgfqpoint{1.863600in}{5.555253in}}%
\pgfpathlineto{\pgfqpoint{1.957200in}{5.513165in}}%
\pgfpathlineto{\pgfqpoint{2.050800in}{5.473891in}}%
\pgfpathlineto{\pgfqpoint{2.144400in}{5.437626in}}%
\pgfpathlineto{\pgfqpoint{2.214600in}{5.412520in}}%
\pgfpathlineto{\pgfqpoint{2.284800in}{5.389300in}}%
\pgfpathlineto{\pgfqpoint{2.355000in}{5.368047in}}%
\pgfpathlineto{\pgfqpoint{2.425200in}{5.348843in}}%
\pgfpathlineto{\pgfqpoint{2.495400in}{5.331778in}}%
\pgfpathlineto{\pgfqpoint{2.565600in}{5.316916in}}%
\pgfpathlineto{\pgfqpoint{2.635800in}{5.304352in}}%
\pgfpathlineto{\pgfqpoint{2.706000in}{5.294328in}}%
\pgfpathlineto{\pgfqpoint{2.752800in}{5.288724in}}%
\pgfpathlineto{\pgfqpoint{2.823000in}{5.282738in}}%
\pgfpathlineto{\pgfqpoint{2.893200in}{5.279406in}}%
\pgfpathlineto{\pgfqpoint{2.940000in}{5.277391in}}%
\pgfpathlineto{\pgfqpoint{2.986800in}{5.279406in}}%
\pgfpathlineto{\pgfqpoint{3.057000in}{5.282738in}}%
\pgfpathlineto{\pgfqpoint{3.127200in}{5.288724in}}%
\pgfpathlineto{\pgfqpoint{3.197400in}{5.297316in}}%
\pgfpathlineto{\pgfqpoint{3.267600in}{5.308279in}}%
\pgfpathlineto{\pgfqpoint{3.337800in}{5.321617in}}%
\pgfpathlineto{\pgfqpoint{3.408000in}{5.337230in}}%
\pgfpathlineto{\pgfqpoint{3.478200in}{5.355012in}}%
\pgfpathlineto{\pgfqpoint{3.548400in}{5.374907in}}%
\pgfpathlineto{\pgfqpoint{3.618600in}{5.396826in}}%
\pgfpathlineto{\pgfqpoint{3.688800in}{5.420683in}}%
\pgfpathlineto{\pgfqpoint{3.759000in}{5.446399in}}%
\pgfpathlineto{\pgfqpoint{3.852600in}{5.483435in}}%
\pgfpathlineto{\pgfqpoint{3.946200in}{5.523431in}}%
\pgfpathlineto{\pgfqpoint{4.039800in}{5.566191in}}%
\pgfpathlineto{\pgfqpoint{4.133400in}{5.611522in}}%
\pgfpathlineto{\pgfqpoint{4.227000in}{5.659227in}}%
\pgfpathlineto{\pgfqpoint{4.320600in}{5.709110in}}%
\pgfpathlineto{\pgfqpoint{4.437600in}{5.774230in}}%
\pgfpathlineto{\pgfqpoint{4.554600in}{5.842067in}}%
\pgfpathlineto{\pgfqpoint{4.671600in}{5.912240in}}%
\pgfpathlineto{\pgfqpoint{4.812000in}{5.998993in}}%
\pgfpathlineto{\pgfqpoint{4.975800in}{6.102885in}}%
\pgfpathlineto{\pgfqpoint{5.209800in}{6.254317in}}%
\pgfpathlineto{\pgfqpoint{5.280000in}{6.300000in}}%
\pgfpathlineto{\pgfqpoint{5.280000in}{6.300000in}}%
\pgfusepath{stroke}%
\end{pgfscope}%
\begin{pgfscope}%
\pgfpathrectangle{\pgfqpoint{0.600000in}{5.082609in}}{\pgfqpoint{4.680000in}{1.217391in}} %
\pgfusepath{clip}%
\pgfsetbuttcap%
\pgfsetroundjoin%
\pgfsetlinewidth{0.501875pt}%
\definecolor{currentstroke}{rgb}{0.000000,0.000000,0.000000}%
\pgfsetstrokecolor{currentstroke}%
\pgfsetdash{{1.000000pt}{3.000000pt}}{0.000000pt}%
\pgfpathmoveto{\pgfqpoint{0.600000in}{5.082609in}}%
\pgfpathlineto{\pgfqpoint{0.600000in}{6.300000in}}%
\pgfusepath{stroke}%
\end{pgfscope}%
\begin{pgfscope}%
\pgfsetbuttcap%
\pgfsetroundjoin%
\definecolor{currentfill}{rgb}{0.000000,0.000000,0.000000}%
\pgfsetfillcolor{currentfill}%
\pgfsetlinewidth{0.501875pt}%
\definecolor{currentstroke}{rgb}{0.000000,0.000000,0.000000}%
\pgfsetstrokecolor{currentstroke}%
\pgfsetdash{}{0pt}%
\pgfsys@defobject{currentmarker}{\pgfqpoint{0.000000in}{0.000000in}}{\pgfqpoint{0.000000in}{0.027778in}}{%
\pgfpathmoveto{\pgfqpoint{0.000000in}{0.000000in}}%
\pgfpathlineto{\pgfqpoint{0.000000in}{0.027778in}}%
\pgfusepath{stroke,fill}%
}%
\begin{pgfscope}%
\pgfsys@transformshift{0.600000in}{5.082609in}%
\pgfsys@useobject{currentmarker}{}%
\end{pgfscope}%
\end{pgfscope}%
\begin{pgfscope}%
\pgfsetbuttcap%
\pgfsetroundjoin%
\definecolor{currentfill}{rgb}{0.000000,0.000000,0.000000}%
\pgfsetfillcolor{currentfill}%
\pgfsetlinewidth{0.501875pt}%
\definecolor{currentstroke}{rgb}{0.000000,0.000000,0.000000}%
\pgfsetstrokecolor{currentstroke}%
\pgfsetdash{}{0pt}%
\pgfsys@defobject{currentmarker}{\pgfqpoint{0.000000in}{-0.027778in}}{\pgfqpoint{0.000000in}{0.000000in}}{%
\pgfpathmoveto{\pgfqpoint{0.000000in}{0.000000in}}%
\pgfpathlineto{\pgfqpoint{0.000000in}{-0.027778in}}%
\pgfusepath{stroke,fill}%
}%
\begin{pgfscope}%
\pgfsys@transformshift{0.600000in}{6.300000in}%
\pgfsys@useobject{currentmarker}{}%
\end{pgfscope}%
\end{pgfscope}%
\begin{pgfscope}%
\pgfpathrectangle{\pgfqpoint{0.600000in}{5.082609in}}{\pgfqpoint{4.680000in}{1.217391in}} %
\pgfusepath{clip}%
\pgfsetbuttcap%
\pgfsetroundjoin%
\pgfsetlinewidth{0.501875pt}%
\definecolor{currentstroke}{rgb}{0.000000,0.000000,0.000000}%
\pgfsetstrokecolor{currentstroke}%
\pgfsetdash{{1.000000pt}{3.000000pt}}{0.000000pt}%
\pgfpathmoveto{\pgfqpoint{1.770000in}{5.082609in}}%
\pgfpathlineto{\pgfqpoint{1.770000in}{6.300000in}}%
\pgfusepath{stroke}%
\end{pgfscope}%
\begin{pgfscope}%
\pgfsetbuttcap%
\pgfsetroundjoin%
\definecolor{currentfill}{rgb}{0.000000,0.000000,0.000000}%
\pgfsetfillcolor{currentfill}%
\pgfsetlinewidth{0.501875pt}%
\definecolor{currentstroke}{rgb}{0.000000,0.000000,0.000000}%
\pgfsetstrokecolor{currentstroke}%
\pgfsetdash{}{0pt}%
\pgfsys@defobject{currentmarker}{\pgfqpoint{0.000000in}{0.000000in}}{\pgfqpoint{0.000000in}{0.027778in}}{%
\pgfpathmoveto{\pgfqpoint{0.000000in}{0.000000in}}%
\pgfpathlineto{\pgfqpoint{0.000000in}{0.027778in}}%
\pgfusepath{stroke,fill}%
}%
\begin{pgfscope}%
\pgfsys@transformshift{1.770000in}{5.082609in}%
\pgfsys@useobject{currentmarker}{}%
\end{pgfscope}%
\end{pgfscope}%
\begin{pgfscope}%
\pgfsetbuttcap%
\pgfsetroundjoin%
\definecolor{currentfill}{rgb}{0.000000,0.000000,0.000000}%
\pgfsetfillcolor{currentfill}%
\pgfsetlinewidth{0.501875pt}%
\definecolor{currentstroke}{rgb}{0.000000,0.000000,0.000000}%
\pgfsetstrokecolor{currentstroke}%
\pgfsetdash{}{0pt}%
\pgfsys@defobject{currentmarker}{\pgfqpoint{0.000000in}{-0.027778in}}{\pgfqpoint{0.000000in}{0.000000in}}{%
\pgfpathmoveto{\pgfqpoint{0.000000in}{0.000000in}}%
\pgfpathlineto{\pgfqpoint{0.000000in}{-0.027778in}}%
\pgfusepath{stroke,fill}%
}%
\begin{pgfscope}%
\pgfsys@transformshift{1.770000in}{6.300000in}%
\pgfsys@useobject{currentmarker}{}%
\end{pgfscope}%
\end{pgfscope}%
\begin{pgfscope}%
\pgfpathrectangle{\pgfqpoint{0.600000in}{5.082609in}}{\pgfqpoint{4.680000in}{1.217391in}} %
\pgfusepath{clip}%
\pgfsetbuttcap%
\pgfsetroundjoin%
\pgfsetlinewidth{0.501875pt}%
\definecolor{currentstroke}{rgb}{0.000000,0.000000,0.000000}%
\pgfsetstrokecolor{currentstroke}%
\pgfsetdash{{1.000000pt}{3.000000pt}}{0.000000pt}%
\pgfpathmoveto{\pgfqpoint{2.940000in}{5.082609in}}%
\pgfpathlineto{\pgfqpoint{2.940000in}{6.300000in}}%
\pgfusepath{stroke}%
\end{pgfscope}%
\begin{pgfscope}%
\pgfsetbuttcap%
\pgfsetroundjoin%
\definecolor{currentfill}{rgb}{0.000000,0.000000,0.000000}%
\pgfsetfillcolor{currentfill}%
\pgfsetlinewidth{0.501875pt}%
\definecolor{currentstroke}{rgb}{0.000000,0.000000,0.000000}%
\pgfsetstrokecolor{currentstroke}%
\pgfsetdash{}{0pt}%
\pgfsys@defobject{currentmarker}{\pgfqpoint{0.000000in}{0.000000in}}{\pgfqpoint{0.000000in}{0.027778in}}{%
\pgfpathmoveto{\pgfqpoint{0.000000in}{0.000000in}}%
\pgfpathlineto{\pgfqpoint{0.000000in}{0.027778in}}%
\pgfusepath{stroke,fill}%
}%
\begin{pgfscope}%
\pgfsys@transformshift{2.940000in}{5.082609in}%
\pgfsys@useobject{currentmarker}{}%
\end{pgfscope}%
\end{pgfscope}%
\begin{pgfscope}%
\pgfsetbuttcap%
\pgfsetroundjoin%
\definecolor{currentfill}{rgb}{0.000000,0.000000,0.000000}%
\pgfsetfillcolor{currentfill}%
\pgfsetlinewidth{0.501875pt}%
\definecolor{currentstroke}{rgb}{0.000000,0.000000,0.000000}%
\pgfsetstrokecolor{currentstroke}%
\pgfsetdash{}{0pt}%
\pgfsys@defobject{currentmarker}{\pgfqpoint{0.000000in}{-0.027778in}}{\pgfqpoint{0.000000in}{0.000000in}}{%
\pgfpathmoveto{\pgfqpoint{0.000000in}{0.000000in}}%
\pgfpathlineto{\pgfqpoint{0.000000in}{-0.027778in}}%
\pgfusepath{stroke,fill}%
}%
\begin{pgfscope}%
\pgfsys@transformshift{2.940000in}{6.300000in}%
\pgfsys@useobject{currentmarker}{}%
\end{pgfscope}%
\end{pgfscope}%
\begin{pgfscope}%
\pgfpathrectangle{\pgfqpoint{0.600000in}{5.082609in}}{\pgfqpoint{4.680000in}{1.217391in}} %
\pgfusepath{clip}%
\pgfsetbuttcap%
\pgfsetroundjoin%
\pgfsetlinewidth{0.501875pt}%
\definecolor{currentstroke}{rgb}{0.000000,0.000000,0.000000}%
\pgfsetstrokecolor{currentstroke}%
\pgfsetdash{{1.000000pt}{3.000000pt}}{0.000000pt}%
\pgfpathmoveto{\pgfqpoint{4.110000in}{5.082609in}}%
\pgfpathlineto{\pgfqpoint{4.110000in}{6.300000in}}%
\pgfusepath{stroke}%
\end{pgfscope}%
\begin{pgfscope}%
\pgfsetbuttcap%
\pgfsetroundjoin%
\definecolor{currentfill}{rgb}{0.000000,0.000000,0.000000}%
\pgfsetfillcolor{currentfill}%
\pgfsetlinewidth{0.501875pt}%
\definecolor{currentstroke}{rgb}{0.000000,0.000000,0.000000}%
\pgfsetstrokecolor{currentstroke}%
\pgfsetdash{}{0pt}%
\pgfsys@defobject{currentmarker}{\pgfqpoint{0.000000in}{0.000000in}}{\pgfqpoint{0.000000in}{0.027778in}}{%
\pgfpathmoveto{\pgfqpoint{0.000000in}{0.000000in}}%
\pgfpathlineto{\pgfqpoint{0.000000in}{0.027778in}}%
\pgfusepath{stroke,fill}%
}%
\begin{pgfscope}%
\pgfsys@transformshift{4.110000in}{5.082609in}%
\pgfsys@useobject{currentmarker}{}%
\end{pgfscope}%
\end{pgfscope}%
\begin{pgfscope}%
\pgfsetbuttcap%
\pgfsetroundjoin%
\definecolor{currentfill}{rgb}{0.000000,0.000000,0.000000}%
\pgfsetfillcolor{currentfill}%
\pgfsetlinewidth{0.501875pt}%
\definecolor{currentstroke}{rgb}{0.000000,0.000000,0.000000}%
\pgfsetstrokecolor{currentstroke}%
\pgfsetdash{}{0pt}%
\pgfsys@defobject{currentmarker}{\pgfqpoint{0.000000in}{-0.027778in}}{\pgfqpoint{0.000000in}{0.000000in}}{%
\pgfpathmoveto{\pgfqpoint{0.000000in}{0.000000in}}%
\pgfpathlineto{\pgfqpoint{0.000000in}{-0.027778in}}%
\pgfusepath{stroke,fill}%
}%
\begin{pgfscope}%
\pgfsys@transformshift{4.110000in}{6.300000in}%
\pgfsys@useobject{currentmarker}{}%
\end{pgfscope}%
\end{pgfscope}%
\begin{pgfscope}%
\pgfpathrectangle{\pgfqpoint{0.600000in}{5.082609in}}{\pgfqpoint{4.680000in}{1.217391in}} %
\pgfusepath{clip}%
\pgfsetbuttcap%
\pgfsetroundjoin%
\pgfsetlinewidth{0.501875pt}%
\definecolor{currentstroke}{rgb}{0.000000,0.000000,0.000000}%
\pgfsetstrokecolor{currentstroke}%
\pgfsetdash{{1.000000pt}{3.000000pt}}{0.000000pt}%
\pgfpathmoveto{\pgfqpoint{5.280000in}{5.082609in}}%
\pgfpathlineto{\pgfqpoint{5.280000in}{6.300000in}}%
\pgfusepath{stroke}%
\end{pgfscope}%
\begin{pgfscope}%
\pgfsetbuttcap%
\pgfsetroundjoin%
\definecolor{currentfill}{rgb}{0.000000,0.000000,0.000000}%
\pgfsetfillcolor{currentfill}%
\pgfsetlinewidth{0.501875pt}%
\definecolor{currentstroke}{rgb}{0.000000,0.000000,0.000000}%
\pgfsetstrokecolor{currentstroke}%
\pgfsetdash{}{0pt}%
\pgfsys@defobject{currentmarker}{\pgfqpoint{0.000000in}{0.000000in}}{\pgfqpoint{0.000000in}{0.027778in}}{%
\pgfpathmoveto{\pgfqpoint{0.000000in}{0.000000in}}%
\pgfpathlineto{\pgfqpoint{0.000000in}{0.027778in}}%
\pgfusepath{stroke,fill}%
}%
\begin{pgfscope}%
\pgfsys@transformshift{5.280000in}{5.082609in}%
\pgfsys@useobject{currentmarker}{}%
\end{pgfscope}%
\end{pgfscope}%
\begin{pgfscope}%
\pgfsetbuttcap%
\pgfsetroundjoin%
\definecolor{currentfill}{rgb}{0.000000,0.000000,0.000000}%
\pgfsetfillcolor{currentfill}%
\pgfsetlinewidth{0.501875pt}%
\definecolor{currentstroke}{rgb}{0.000000,0.000000,0.000000}%
\pgfsetstrokecolor{currentstroke}%
\pgfsetdash{}{0pt}%
\pgfsys@defobject{currentmarker}{\pgfqpoint{0.000000in}{-0.027778in}}{\pgfqpoint{0.000000in}{0.000000in}}{%
\pgfpathmoveto{\pgfqpoint{0.000000in}{0.000000in}}%
\pgfpathlineto{\pgfqpoint{0.000000in}{-0.027778in}}%
\pgfusepath{stroke,fill}%
}%
\begin{pgfscope}%
\pgfsys@transformshift{5.280000in}{6.300000in}%
\pgfsys@useobject{currentmarker}{}%
\end{pgfscope}%
\end{pgfscope}%
\begin{pgfscope}%
\pgfpathrectangle{\pgfqpoint{0.600000in}{5.082609in}}{\pgfqpoint{4.680000in}{1.217391in}} %
\pgfusepath{clip}%
\pgfsetbuttcap%
\pgfsetroundjoin%
\pgfsetlinewidth{0.501875pt}%
\definecolor{currentstroke}{rgb}{0.000000,0.000000,0.000000}%
\pgfsetstrokecolor{currentstroke}%
\pgfsetdash{{1.000000pt}{3.000000pt}}{0.000000pt}%
\pgfpathmoveto{\pgfqpoint{0.600000in}{6.300000in}}%
\pgfpathlineto{\pgfqpoint{5.280000in}{6.300000in}}%
\pgfusepath{stroke}%
\end{pgfscope}%
\begin{pgfscope}%
\pgfsetbuttcap%
\pgfsetroundjoin%
\definecolor{currentfill}{rgb}{0.000000,0.000000,0.000000}%
\pgfsetfillcolor{currentfill}%
\pgfsetlinewidth{0.501875pt}%
\definecolor{currentstroke}{rgb}{0.000000,0.000000,0.000000}%
\pgfsetstrokecolor{currentstroke}%
\pgfsetdash{}{0pt}%
\pgfsys@defobject{currentmarker}{\pgfqpoint{0.000000in}{0.000000in}}{\pgfqpoint{0.055556in}{0.000000in}}{%
\pgfpathmoveto{\pgfqpoint{0.000000in}{0.000000in}}%
\pgfpathlineto{\pgfqpoint{0.055556in}{0.000000in}}%
\pgfusepath{stroke,fill}%
}%
\begin{pgfscope}%
\pgfsys@transformshift{0.600000in}{6.300000in}%
\pgfsys@useobject{currentmarker}{}%
\end{pgfscope}%
\end{pgfscope}%
\begin{pgfscope}%
\pgftext[left,bottom,x=0.462848in,y=6.246296in,rotate=0.000000]{{\rmfamily\fontsize{12.000000}{14.400000}\selectfont \(\displaystyle 0\)}}
%
\end{pgfscope}%
\begin{pgfscope}%
\pgfpathrectangle{\pgfqpoint{0.600000in}{5.082609in}}{\pgfqpoint{4.680000in}{1.217391in}} %
\pgfusepath{clip}%
\pgfsetbuttcap%
\pgfsetroundjoin%
\pgfsetlinewidth{0.501875pt}%
\definecolor{currentstroke}{rgb}{0.000000,0.000000,0.000000}%
\pgfsetstrokecolor{currentstroke}%
\pgfsetdash{{1.000000pt}{3.000000pt}}{0.000000pt}%
\pgfpathmoveto{\pgfqpoint{0.600000in}{5.995652in}}%
\pgfpathlineto{\pgfqpoint{5.280000in}{5.995652in}}%
\pgfusepath{stroke}%
\end{pgfscope}%
\begin{pgfscope}%
\pgfsetbuttcap%
\pgfsetroundjoin%
\definecolor{currentfill}{rgb}{0.000000,0.000000,0.000000}%
\pgfsetfillcolor{currentfill}%
\pgfsetlinewidth{0.501875pt}%
\definecolor{currentstroke}{rgb}{0.000000,0.000000,0.000000}%
\pgfsetstrokecolor{currentstroke}%
\pgfsetdash{}{0pt}%
\pgfsys@defobject{currentmarker}{\pgfqpoint{0.000000in}{0.000000in}}{\pgfqpoint{0.055556in}{0.000000in}}{%
\pgfpathmoveto{\pgfqpoint{0.000000in}{0.000000in}}%
\pgfpathlineto{\pgfqpoint{0.055556in}{0.000000in}}%
\pgfusepath{stroke,fill}%
}%
\begin{pgfscope}%
\pgfsys@transformshift{0.600000in}{5.995652in}%
\pgfsys@useobject{currentmarker}{}%
\end{pgfscope}%
\end{pgfscope}%
\begin{pgfscope}%
\pgftext[left,bottom,x=0.333218in,y=5.935004in,rotate=0.000000]{{\rmfamily\fontsize{12.000000}{14.400000}\selectfont \(\displaystyle -2\)}}
%
\end{pgfscope}%
\begin{pgfscope}%
\pgfpathrectangle{\pgfqpoint{0.600000in}{5.082609in}}{\pgfqpoint{4.680000in}{1.217391in}} %
\pgfusepath{clip}%
\pgfsetbuttcap%
\pgfsetroundjoin%
\pgfsetlinewidth{0.501875pt}%
\definecolor{currentstroke}{rgb}{0.000000,0.000000,0.000000}%
\pgfsetstrokecolor{currentstroke}%
\pgfsetdash{{1.000000pt}{3.000000pt}}{0.000000pt}%
\pgfpathmoveto{\pgfqpoint{0.600000in}{5.691304in}}%
\pgfpathlineto{\pgfqpoint{5.280000in}{5.691304in}}%
\pgfusepath{stroke}%
\end{pgfscope}%
\begin{pgfscope}%
\pgfsetbuttcap%
\pgfsetroundjoin%
\definecolor{currentfill}{rgb}{0.000000,0.000000,0.000000}%
\pgfsetfillcolor{currentfill}%
\pgfsetlinewidth{0.501875pt}%
\definecolor{currentstroke}{rgb}{0.000000,0.000000,0.000000}%
\pgfsetstrokecolor{currentstroke}%
\pgfsetdash{}{0pt}%
\pgfsys@defobject{currentmarker}{\pgfqpoint{0.000000in}{0.000000in}}{\pgfqpoint{0.055556in}{0.000000in}}{%
\pgfpathmoveto{\pgfqpoint{0.000000in}{0.000000in}}%
\pgfpathlineto{\pgfqpoint{0.055556in}{0.000000in}}%
\pgfusepath{stroke,fill}%
}%
\begin{pgfscope}%
\pgfsys@transformshift{0.600000in}{5.691304in}%
\pgfsys@useobject{currentmarker}{}%
\end{pgfscope}%
\end{pgfscope}%
\begin{pgfscope}%
\pgftext[left,bottom,x=0.333218in,y=5.630656in,rotate=0.000000]{{\rmfamily\fontsize{12.000000}{14.400000}\selectfont \(\displaystyle -4\)}}
%
\end{pgfscope}%
\begin{pgfscope}%
\pgfpathrectangle{\pgfqpoint{0.600000in}{5.082609in}}{\pgfqpoint{4.680000in}{1.217391in}} %
\pgfusepath{clip}%
\pgfsetbuttcap%
\pgfsetroundjoin%
\pgfsetlinewidth{0.501875pt}%
\definecolor{currentstroke}{rgb}{0.000000,0.000000,0.000000}%
\pgfsetstrokecolor{currentstroke}%
\pgfsetdash{{1.000000pt}{3.000000pt}}{0.000000pt}%
\pgfpathmoveto{\pgfqpoint{0.600000in}{5.386957in}}%
\pgfpathlineto{\pgfqpoint{5.280000in}{5.386957in}}%
\pgfusepath{stroke}%
\end{pgfscope}%
\begin{pgfscope}%
\pgfsetbuttcap%
\pgfsetroundjoin%
\definecolor{currentfill}{rgb}{0.000000,0.000000,0.000000}%
\pgfsetfillcolor{currentfill}%
\pgfsetlinewidth{0.501875pt}%
\definecolor{currentstroke}{rgb}{0.000000,0.000000,0.000000}%
\pgfsetstrokecolor{currentstroke}%
\pgfsetdash{}{0pt}%
\pgfsys@defobject{currentmarker}{\pgfqpoint{0.000000in}{0.000000in}}{\pgfqpoint{0.055556in}{0.000000in}}{%
\pgfpathmoveto{\pgfqpoint{0.000000in}{0.000000in}}%
\pgfpathlineto{\pgfqpoint{0.055556in}{0.000000in}}%
\pgfusepath{stroke,fill}%
}%
\begin{pgfscope}%
\pgfsys@transformshift{0.600000in}{5.386957in}%
\pgfsys@useobject{currentmarker}{}%
\end{pgfscope}%
\end{pgfscope}%
\begin{pgfscope}%
\pgftext[left,bottom,x=0.333218in,y=5.326308in,rotate=0.000000]{{\rmfamily\fontsize{12.000000}{14.400000}\selectfont \(\displaystyle -6\)}}
%
\end{pgfscope}%
\begin{pgfscope}%
\pgfpathrectangle{\pgfqpoint{0.600000in}{5.082609in}}{\pgfqpoint{4.680000in}{1.217391in}} %
\pgfusepath{clip}%
\pgfsetbuttcap%
\pgfsetroundjoin%
\pgfsetlinewidth{0.501875pt}%
\definecolor{currentstroke}{rgb}{0.000000,0.000000,0.000000}%
\pgfsetstrokecolor{currentstroke}%
\pgfsetdash{{1.000000pt}{3.000000pt}}{0.000000pt}%
\pgfpathmoveto{\pgfqpoint{0.600000in}{5.082609in}}%
\pgfpathlineto{\pgfqpoint{5.280000in}{5.082609in}}%
\pgfusepath{stroke}%
\end{pgfscope}%
\begin{pgfscope}%
\pgfsetbuttcap%
\pgfsetroundjoin%
\definecolor{currentfill}{rgb}{0.000000,0.000000,0.000000}%
\pgfsetfillcolor{currentfill}%
\pgfsetlinewidth{0.501875pt}%
\definecolor{currentstroke}{rgb}{0.000000,0.000000,0.000000}%
\pgfsetstrokecolor{currentstroke}%
\pgfsetdash{}{0pt}%
\pgfsys@defobject{currentmarker}{\pgfqpoint{0.000000in}{0.000000in}}{\pgfqpoint{0.055556in}{0.000000in}}{%
\pgfpathmoveto{\pgfqpoint{0.000000in}{0.000000in}}%
\pgfpathlineto{\pgfqpoint{0.055556in}{0.000000in}}%
\pgfusepath{stroke,fill}%
}%
\begin{pgfscope}%
\pgfsys@transformshift{0.600000in}{5.082609in}%
\pgfsys@useobject{currentmarker}{}%
\end{pgfscope}%
\end{pgfscope}%
\begin{pgfscope}%
\pgftext[left,bottom,x=0.333218in,y=5.021961in,rotate=0.000000]{{\rmfamily\fontsize{12.000000}{14.400000}\selectfont \(\displaystyle -8\)}}
%
\end{pgfscope}%
\begin{pgfscope}%
\pgftext[left,bottom,x=0.600000in,y=6.327778in,rotate=0.000000]{{\rmfamily\fontsize{12.000000}{14.400000}\selectfont \(\displaystyle \times10^{-5}\)}}
%
\end{pgfscope}%
\begin{pgfscope}%
\pgfsetrectcap%
\pgfsetroundjoin%
\pgfsetlinewidth{1.003750pt}%
\definecolor{currentstroke}{rgb}{0.000000,0.000000,0.000000}%
\pgfsetstrokecolor{currentstroke}%
\pgfsetdash{}{0pt}%
\pgfpathmoveto{\pgfqpoint{0.600000in}{6.300000in}}%
\pgfpathlineto{\pgfqpoint{5.280000in}{6.300000in}}%
\pgfusepath{stroke}%
\end{pgfscope}%
\begin{pgfscope}%
\pgfsetrectcap%
\pgfsetroundjoin%
\pgfsetlinewidth{1.003750pt}%
\definecolor{currentstroke}{rgb}{0.000000,0.000000,0.000000}%
\pgfsetstrokecolor{currentstroke}%
\pgfsetdash{}{0pt}%
\pgfpathmoveto{\pgfqpoint{5.280000in}{5.082609in}}%
\pgfpathlineto{\pgfqpoint{5.280000in}{6.300000in}}%
\pgfusepath{stroke}%
\end{pgfscope}%
\begin{pgfscope}%
\pgfsetrectcap%
\pgfsetroundjoin%
\pgfsetlinewidth{1.003750pt}%
\definecolor{currentstroke}{rgb}{0.000000,0.000000,0.000000}%
\pgfsetstrokecolor{currentstroke}%
\pgfsetdash{}{0pt}%
\pgfpathmoveto{\pgfqpoint{0.600000in}{5.082609in}}%
\pgfpathlineto{\pgfqpoint{5.280000in}{5.082609in}}%
\pgfusepath{stroke}%
\end{pgfscope}%
\begin{pgfscope}%
\pgfsetrectcap%
\pgfsetroundjoin%
\pgfsetlinewidth{1.003750pt}%
\definecolor{currentstroke}{rgb}{0.000000,0.000000,0.000000}%
\pgfsetstrokecolor{currentstroke}%
\pgfsetdash{}{0pt}%
\pgfpathmoveto{\pgfqpoint{0.600000in}{5.082609in}}%
\pgfpathlineto{\pgfqpoint{0.600000in}{6.300000in}}%
\pgfusepath{stroke}%
\end{pgfscope}%
\begin{pgfscope}%
\pgfpathrectangle{\pgfqpoint{0.600000in}{5.082609in}}{\pgfqpoint{4.680000in}{1.217391in}} %
\pgfusepath{clip}%
\pgfsetrectcap%
\pgfsetroundjoin%
\pgfsetlinewidth{1.003750pt}%
\definecolor{currentstroke}{rgb}{1.000000,0.000000,0.000000}%
\pgfsetstrokecolor{currentstroke}%
\pgfsetdash{}{0pt}%
\pgfpathmoveto{\pgfqpoint{0.600000in}{6.300000in}}%
\pgfpathlineto{\pgfqpoint{2.916600in}{6.300000in}}%
\pgfpathlineto{\pgfqpoint{2.940000in}{6.178261in}}%
\pgfpathlineto{\pgfqpoint{2.963400in}{6.300000in}}%
\pgfpathlineto{\pgfqpoint{5.280000in}{6.300000in}}%
\pgfpathlineto{\pgfqpoint{5.280000in}{6.300000in}}%
\pgfusepath{stroke}%
\end{pgfscope}%
\begin{pgfscope}%
\pgfsetbuttcap%
\pgfsetroundjoin%
\definecolor{currentfill}{rgb}{0.000000,0.000000,0.000000}%
\pgfsetfillcolor{currentfill}%
\pgfsetlinewidth{0.501875pt}%
\definecolor{currentstroke}{rgb}{0.000000,0.000000,0.000000}%
\pgfsetstrokecolor{currentstroke}%
\pgfsetdash{}{0pt}%
\pgfsys@defobject{currentmarker}{\pgfqpoint{-0.055556in}{0.000000in}}{\pgfqpoint{0.000000in}{0.000000in}}{%
\pgfpathmoveto{\pgfqpoint{0.000000in}{0.000000in}}%
\pgfpathlineto{\pgfqpoint{-0.055556in}{0.000000in}}%
\pgfusepath{stroke,fill}%
}%
\begin{pgfscope}%
\pgfsys@transformshift{5.280000in}{5.082609in}%
\pgfsys@useobject{currentmarker}{}%
\end{pgfscope}%
\end{pgfscope}%
\begin{pgfscope}%
\pgftext[left,bottom,x=5.335556in,y=5.028905in,rotate=0.000000]{{\rmfamily\fontsize{12.000000}{14.400000}\selectfont \(\displaystyle 0.00\)}}
%
\end{pgfscope}%
\begin{pgfscope}%
\pgfsetbuttcap%
\pgfsetroundjoin%
\definecolor{currentfill}{rgb}{0.000000,0.000000,0.000000}%
\pgfsetfillcolor{currentfill}%
\pgfsetlinewidth{0.501875pt}%
\definecolor{currentstroke}{rgb}{0.000000,0.000000,0.000000}%
\pgfsetstrokecolor{currentstroke}%
\pgfsetdash{}{0pt}%
\pgfsys@defobject{currentmarker}{\pgfqpoint{-0.055556in}{0.000000in}}{\pgfqpoint{0.000000in}{0.000000in}}{%
\pgfpathmoveto{\pgfqpoint{0.000000in}{0.000000in}}%
\pgfpathlineto{\pgfqpoint{-0.055556in}{0.000000in}}%
\pgfusepath{stroke,fill}%
}%
\begin{pgfscope}%
\pgfsys@transformshift{5.280000in}{5.386957in}%
\pgfsys@useobject{currentmarker}{}%
\end{pgfscope}%
\end{pgfscope}%
\begin{pgfscope}%
\pgftext[left,bottom,x=5.335556in,y=5.333253in,rotate=0.000000]{{\rmfamily\fontsize{12.000000}{14.400000}\selectfont \(\displaystyle 0.25\)}}
%
\end{pgfscope}%
\begin{pgfscope}%
\pgfsetbuttcap%
\pgfsetroundjoin%
\definecolor{currentfill}{rgb}{0.000000,0.000000,0.000000}%
\pgfsetfillcolor{currentfill}%
\pgfsetlinewidth{0.501875pt}%
\definecolor{currentstroke}{rgb}{0.000000,0.000000,0.000000}%
\pgfsetstrokecolor{currentstroke}%
\pgfsetdash{}{0pt}%
\pgfsys@defobject{currentmarker}{\pgfqpoint{-0.055556in}{0.000000in}}{\pgfqpoint{0.000000in}{0.000000in}}{%
\pgfpathmoveto{\pgfqpoint{0.000000in}{0.000000in}}%
\pgfpathlineto{\pgfqpoint{-0.055556in}{0.000000in}}%
\pgfusepath{stroke,fill}%
}%
\begin{pgfscope}%
\pgfsys@transformshift{5.280000in}{5.691304in}%
\pgfsys@useobject{currentmarker}{}%
\end{pgfscope}%
\end{pgfscope}%
\begin{pgfscope}%
\pgftext[left,bottom,x=5.335556in,y=5.637601in,rotate=0.000000]{{\rmfamily\fontsize{12.000000}{14.400000}\selectfont \(\displaystyle 0.50\)}}
%
\end{pgfscope}%
\begin{pgfscope}%
\pgfsetbuttcap%
\pgfsetroundjoin%
\definecolor{currentfill}{rgb}{0.000000,0.000000,0.000000}%
\pgfsetfillcolor{currentfill}%
\pgfsetlinewidth{0.501875pt}%
\definecolor{currentstroke}{rgb}{0.000000,0.000000,0.000000}%
\pgfsetstrokecolor{currentstroke}%
\pgfsetdash{}{0pt}%
\pgfsys@defobject{currentmarker}{\pgfqpoint{-0.055556in}{0.000000in}}{\pgfqpoint{0.000000in}{0.000000in}}{%
\pgfpathmoveto{\pgfqpoint{0.000000in}{0.000000in}}%
\pgfpathlineto{\pgfqpoint{-0.055556in}{0.000000in}}%
\pgfusepath{stroke,fill}%
}%
\begin{pgfscope}%
\pgfsys@transformshift{5.280000in}{5.995652in}%
\pgfsys@useobject{currentmarker}{}%
\end{pgfscope}%
\end{pgfscope}%
\begin{pgfscope}%
\pgftext[left,bottom,x=5.335556in,y=5.941949in,rotate=0.000000]{{\rmfamily\fontsize{12.000000}{14.400000}\selectfont \(\displaystyle 0.75\)}}
%
\end{pgfscope}%
\begin{pgfscope}%
\pgfsetbuttcap%
\pgfsetroundjoin%
\definecolor{currentfill}{rgb}{0.000000,0.000000,0.000000}%
\pgfsetfillcolor{currentfill}%
\pgfsetlinewidth{0.501875pt}%
\definecolor{currentstroke}{rgb}{0.000000,0.000000,0.000000}%
\pgfsetstrokecolor{currentstroke}%
\pgfsetdash{}{0pt}%
\pgfsys@defobject{currentmarker}{\pgfqpoint{-0.055556in}{0.000000in}}{\pgfqpoint{0.000000in}{0.000000in}}{%
\pgfpathmoveto{\pgfqpoint{0.000000in}{0.000000in}}%
\pgfpathlineto{\pgfqpoint{-0.055556in}{0.000000in}}%
\pgfusepath{stroke,fill}%
}%
\begin{pgfscope}%
\pgfsys@transformshift{5.280000in}{6.300000in}%
\pgfsys@useobject{currentmarker}{}%
\end{pgfscope}%
\end{pgfscope}%
\begin{pgfscope}%
\pgftext[left,bottom,x=5.335556in,y=6.246296in,rotate=0.000000]{{\rmfamily\fontsize{12.000000}{14.400000}\selectfont \(\displaystyle 1.00\)}}
%
\end{pgfscope}%
\begin{pgfscope}%
\pgfsetrectcap%
\pgfsetroundjoin%
\definecolor{currentfill}{rgb}{1.000000,1.000000,1.000000}%
\pgfsetfillcolor{currentfill}%
\pgfsetlinewidth{1.003750pt}%
\definecolor{currentstroke}{rgb}{0.000000,0.000000,0.000000}%
\pgfsetstrokecolor{currentstroke}%
\pgfsetdash{}{0pt}%
\pgfpathmoveto{\pgfqpoint{0.600000in}{6.324348in}}%
\pgfpathlineto{\pgfqpoint{5.280000in}{6.324348in}}%
\pgfpathlineto{\pgfqpoint{5.280000in}{6.624348in}}%
\pgfpathlineto{\pgfqpoint{0.600000in}{6.624348in}}%
\pgfpathlineto{\pgfqpoint{0.600000in}{6.324348in}}%
\pgfpathclose%
\pgfusepath{stroke,fill}%
\end{pgfscope}%
\begin{pgfscope}%
\pgfsetbuttcap%
\pgfsetroundjoin%
\pgfsetlinewidth{1.003750pt}%
\definecolor{currentstroke}{rgb}{0.000000,0.000000,1.000000}%
\pgfsetstrokecolor{currentstroke}%
\pgfsetdash{{6.000000pt}{6.000000pt}}{0.000000pt}%
\pgfpathmoveto{\pgfqpoint{0.740000in}{6.474348in}}%
\pgfpathlineto{\pgfqpoint{1.020000in}{6.474348in}}%
\pgfusepath{stroke}%
\end{pgfscope}%
\begin{pgfscope}%
\pgftext[left,bottom,x=1.240000in,y=6.404348in,rotate=0.000000]{{\rmfamily\fontsize{14.400000}{17.280000}\selectfont Deflection}}
%
\end{pgfscope}%
\begin{pgfscope}%
\pgfsetrectcap%
\pgfsetroundjoin%
\pgfsetlinewidth{1.003750pt}%
\definecolor{currentstroke}{rgb}{1.000000,0.000000,0.000000}%
\pgfsetstrokecolor{currentstroke}%
\pgfsetdash{}{0pt}%
\pgfpathmoveto{\pgfqpoint{3.617629in}{6.474348in}}%
\pgfpathlineto{\pgfqpoint{3.897629in}{6.474348in}}%
\pgfusepath{stroke}%
\end{pgfscope}%
\begin{pgfscope}%
\pgftext[left,bottom,x=4.117629in,y=6.404348in,rotate=0.000000]{{\rmfamily\fontsize{14.400000}{17.280000}\selectfont Node Health}}
%
\end{pgfscope}%
\begin{pgfscope}%
\pgfsetrectcap%
\pgfsetroundjoin%
\definecolor{currentfill}{rgb}{1.000000,1.000000,1.000000}%
\pgfsetfillcolor{currentfill}%
\pgfsetlinewidth{0.000000pt}%
\definecolor{currentstroke}{rgb}{0.000000,0.000000,0.000000}%
\pgfsetstrokecolor{currentstroke}%
\pgfsetdash{}{0pt}%
\pgfpathmoveto{\pgfqpoint{0.600000in}{3.621739in}}%
\pgfpathlineto{\pgfqpoint{5.280000in}{3.621739in}}%
\pgfpathlineto{\pgfqpoint{5.280000in}{4.839130in}}%
\pgfpathlineto{\pgfqpoint{0.600000in}{4.839130in}}%
\pgfpathclose%
\pgfusepath{fill}%
\end{pgfscope}%
\begin{pgfscope}%
\pgfpathrectangle{\pgfqpoint{0.600000in}{3.621739in}}{\pgfqpoint{4.680000in}{1.217391in}} %
\pgfusepath{clip}%
\pgfsetbuttcap%
\pgfsetroundjoin%
\pgfsetlinewidth{1.003750pt}%
\definecolor{currentstroke}{rgb}{0.000000,0.000000,1.000000}%
\pgfsetstrokecolor{currentstroke}%
\pgfsetdash{{6.000000pt}{6.000000pt}}{0.000000pt}%
\pgfpathmoveto{\pgfqpoint{0.600000in}{4.839130in}}%
\pgfpathlineto{\pgfqpoint{0.951000in}{4.618335in}}%
\pgfpathlineto{\pgfqpoint{1.114800in}{4.517850in}}%
\pgfpathlineto{\pgfqpoint{1.255200in}{4.434038in}}%
\pgfpathlineto{\pgfqpoint{1.395600in}{4.353002in}}%
\pgfpathlineto{\pgfqpoint{1.512600in}{4.288028in}}%
\pgfpathlineto{\pgfqpoint{1.629600in}{4.225750in}}%
\pgfpathlineto{\pgfqpoint{1.723200in}{4.178102in}}%
\pgfpathlineto{\pgfqpoint{1.816800in}{4.132583in}}%
\pgfpathlineto{\pgfqpoint{1.910400in}{4.089370in}}%
\pgfpathlineto{\pgfqpoint{2.004000in}{4.048641in}}%
\pgfpathlineto{\pgfqpoint{2.097600in}{4.010578in}}%
\pgfpathlineto{\pgfqpoint{2.191200in}{3.975357in}}%
\pgfpathlineto{\pgfqpoint{2.284800in}{3.943156in}}%
\pgfpathlineto{\pgfqpoint{2.355000in}{3.921099in}}%
\pgfpathlineto{\pgfqpoint{2.425200in}{3.900922in}}%
\pgfpathlineto{\pgfqpoint{2.495400in}{3.882692in}}%
\pgfpathlineto{\pgfqpoint{2.565600in}{3.866444in}}%
\pgfpathlineto{\pgfqpoint{2.635800in}{3.852453in}}%
\pgfpathlineto{\pgfqpoint{2.706000in}{3.840621in}}%
\pgfpathlineto{\pgfqpoint{2.799600in}{3.827729in}}%
\pgfpathlineto{\pgfqpoint{2.869800in}{3.820948in}}%
\pgfpathlineto{\pgfqpoint{2.940000in}{3.816522in}}%
\pgfpathlineto{\pgfqpoint{3.080400in}{3.827729in}}%
\pgfpathlineto{\pgfqpoint{3.174000in}{3.840621in}}%
\pgfpathlineto{\pgfqpoint{3.244200in}{3.852453in}}%
\pgfpathlineto{\pgfqpoint{3.314400in}{3.866444in}}%
\pgfpathlineto{\pgfqpoint{3.384600in}{3.882692in}}%
\pgfpathlineto{\pgfqpoint{3.454800in}{3.900922in}}%
\pgfpathlineto{\pgfqpoint{3.525000in}{3.921099in}}%
\pgfpathlineto{\pgfqpoint{3.595200in}{3.943156in}}%
\pgfpathlineto{\pgfqpoint{3.665400in}{3.967017in}}%
\pgfpathlineto{\pgfqpoint{3.759000in}{4.001499in}}%
\pgfpathlineto{\pgfqpoint{3.852600in}{4.038869in}}%
\pgfpathlineto{\pgfqpoint{3.946200in}{4.078947in}}%
\pgfpathlineto{\pgfqpoint{4.039800in}{4.121556in}}%
\pgfpathlineto{\pgfqpoint{4.133400in}{4.166516in}}%
\pgfpathlineto{\pgfqpoint{4.227000in}{4.213648in}}%
\pgfpathlineto{\pgfqpoint{4.344000in}{4.275346in}}%
\pgfpathlineto{\pgfqpoint{4.461000in}{4.339809in}}%
\pgfpathlineto{\pgfqpoint{4.578000in}{4.406688in}}%
\pgfpathlineto{\pgfqpoint{4.718400in}{4.489641in}}%
\pgfpathlineto{\pgfqpoint{4.882200in}{4.589378in}}%
\pgfpathlineto{\pgfqpoint{5.069400in}{4.706077in}}%
\pgfpathlineto{\pgfqpoint{5.280000in}{4.839130in}}%
\pgfpathlineto{\pgfqpoint{5.280000in}{4.839130in}}%
\pgfusepath{stroke}%
\end{pgfscope}%
\begin{pgfscope}%
\pgfpathrectangle{\pgfqpoint{0.600000in}{3.621739in}}{\pgfqpoint{4.680000in}{1.217391in}} %
\pgfusepath{clip}%
\pgfsetbuttcap%
\pgfsetroundjoin%
\pgfsetlinewidth{0.501875pt}%
\definecolor{currentstroke}{rgb}{0.000000,0.000000,0.000000}%
\pgfsetstrokecolor{currentstroke}%
\pgfsetdash{{1.000000pt}{3.000000pt}}{0.000000pt}%
\pgfpathmoveto{\pgfqpoint{0.600000in}{3.621739in}}%
\pgfpathlineto{\pgfqpoint{0.600000in}{4.839130in}}%
\pgfusepath{stroke}%
\end{pgfscope}%
\begin{pgfscope}%
\pgfsetbuttcap%
\pgfsetroundjoin%
\definecolor{currentfill}{rgb}{0.000000,0.000000,0.000000}%
\pgfsetfillcolor{currentfill}%
\pgfsetlinewidth{0.501875pt}%
\definecolor{currentstroke}{rgb}{0.000000,0.000000,0.000000}%
\pgfsetstrokecolor{currentstroke}%
\pgfsetdash{}{0pt}%
\pgfsys@defobject{currentmarker}{\pgfqpoint{0.000000in}{0.000000in}}{\pgfqpoint{0.000000in}{0.027778in}}{%
\pgfpathmoveto{\pgfqpoint{0.000000in}{0.000000in}}%
\pgfpathlineto{\pgfqpoint{0.000000in}{0.027778in}}%
\pgfusepath{stroke,fill}%
}%
\begin{pgfscope}%
\pgfsys@transformshift{0.600000in}{3.621739in}%
\pgfsys@useobject{currentmarker}{}%
\end{pgfscope}%
\end{pgfscope}%
\begin{pgfscope}%
\pgfsetbuttcap%
\pgfsetroundjoin%
\definecolor{currentfill}{rgb}{0.000000,0.000000,0.000000}%
\pgfsetfillcolor{currentfill}%
\pgfsetlinewidth{0.501875pt}%
\definecolor{currentstroke}{rgb}{0.000000,0.000000,0.000000}%
\pgfsetstrokecolor{currentstroke}%
\pgfsetdash{}{0pt}%
\pgfsys@defobject{currentmarker}{\pgfqpoint{0.000000in}{-0.027778in}}{\pgfqpoint{0.000000in}{0.000000in}}{%
\pgfpathmoveto{\pgfqpoint{0.000000in}{0.000000in}}%
\pgfpathlineto{\pgfqpoint{0.000000in}{-0.027778in}}%
\pgfusepath{stroke,fill}%
}%
\begin{pgfscope}%
\pgfsys@transformshift{0.600000in}{4.839130in}%
\pgfsys@useobject{currentmarker}{}%
\end{pgfscope}%
\end{pgfscope}%
\begin{pgfscope}%
\pgfpathrectangle{\pgfqpoint{0.600000in}{3.621739in}}{\pgfqpoint{4.680000in}{1.217391in}} %
\pgfusepath{clip}%
\pgfsetbuttcap%
\pgfsetroundjoin%
\pgfsetlinewidth{0.501875pt}%
\definecolor{currentstroke}{rgb}{0.000000,0.000000,0.000000}%
\pgfsetstrokecolor{currentstroke}%
\pgfsetdash{{1.000000pt}{3.000000pt}}{0.000000pt}%
\pgfpathmoveto{\pgfqpoint{1.770000in}{3.621739in}}%
\pgfpathlineto{\pgfqpoint{1.770000in}{4.839130in}}%
\pgfusepath{stroke}%
\end{pgfscope}%
\begin{pgfscope}%
\pgfsetbuttcap%
\pgfsetroundjoin%
\definecolor{currentfill}{rgb}{0.000000,0.000000,0.000000}%
\pgfsetfillcolor{currentfill}%
\pgfsetlinewidth{0.501875pt}%
\definecolor{currentstroke}{rgb}{0.000000,0.000000,0.000000}%
\pgfsetstrokecolor{currentstroke}%
\pgfsetdash{}{0pt}%
\pgfsys@defobject{currentmarker}{\pgfqpoint{0.000000in}{0.000000in}}{\pgfqpoint{0.000000in}{0.027778in}}{%
\pgfpathmoveto{\pgfqpoint{0.000000in}{0.000000in}}%
\pgfpathlineto{\pgfqpoint{0.000000in}{0.027778in}}%
\pgfusepath{stroke,fill}%
}%
\begin{pgfscope}%
\pgfsys@transformshift{1.770000in}{3.621739in}%
\pgfsys@useobject{currentmarker}{}%
\end{pgfscope}%
\end{pgfscope}%
\begin{pgfscope}%
\pgfsetbuttcap%
\pgfsetroundjoin%
\definecolor{currentfill}{rgb}{0.000000,0.000000,0.000000}%
\pgfsetfillcolor{currentfill}%
\pgfsetlinewidth{0.501875pt}%
\definecolor{currentstroke}{rgb}{0.000000,0.000000,0.000000}%
\pgfsetstrokecolor{currentstroke}%
\pgfsetdash{}{0pt}%
\pgfsys@defobject{currentmarker}{\pgfqpoint{0.000000in}{-0.027778in}}{\pgfqpoint{0.000000in}{0.000000in}}{%
\pgfpathmoveto{\pgfqpoint{0.000000in}{0.000000in}}%
\pgfpathlineto{\pgfqpoint{0.000000in}{-0.027778in}}%
\pgfusepath{stroke,fill}%
}%
\begin{pgfscope}%
\pgfsys@transformshift{1.770000in}{4.839130in}%
\pgfsys@useobject{currentmarker}{}%
\end{pgfscope}%
\end{pgfscope}%
\begin{pgfscope}%
\pgfpathrectangle{\pgfqpoint{0.600000in}{3.621739in}}{\pgfqpoint{4.680000in}{1.217391in}} %
\pgfusepath{clip}%
\pgfsetbuttcap%
\pgfsetroundjoin%
\pgfsetlinewidth{0.501875pt}%
\definecolor{currentstroke}{rgb}{0.000000,0.000000,0.000000}%
\pgfsetstrokecolor{currentstroke}%
\pgfsetdash{{1.000000pt}{3.000000pt}}{0.000000pt}%
\pgfpathmoveto{\pgfqpoint{2.940000in}{3.621739in}}%
\pgfpathlineto{\pgfqpoint{2.940000in}{4.839130in}}%
\pgfusepath{stroke}%
\end{pgfscope}%
\begin{pgfscope}%
\pgfsetbuttcap%
\pgfsetroundjoin%
\definecolor{currentfill}{rgb}{0.000000,0.000000,0.000000}%
\pgfsetfillcolor{currentfill}%
\pgfsetlinewidth{0.501875pt}%
\definecolor{currentstroke}{rgb}{0.000000,0.000000,0.000000}%
\pgfsetstrokecolor{currentstroke}%
\pgfsetdash{}{0pt}%
\pgfsys@defobject{currentmarker}{\pgfqpoint{0.000000in}{0.000000in}}{\pgfqpoint{0.000000in}{0.027778in}}{%
\pgfpathmoveto{\pgfqpoint{0.000000in}{0.000000in}}%
\pgfpathlineto{\pgfqpoint{0.000000in}{0.027778in}}%
\pgfusepath{stroke,fill}%
}%
\begin{pgfscope}%
\pgfsys@transformshift{2.940000in}{3.621739in}%
\pgfsys@useobject{currentmarker}{}%
\end{pgfscope}%
\end{pgfscope}%
\begin{pgfscope}%
\pgfsetbuttcap%
\pgfsetroundjoin%
\definecolor{currentfill}{rgb}{0.000000,0.000000,0.000000}%
\pgfsetfillcolor{currentfill}%
\pgfsetlinewidth{0.501875pt}%
\definecolor{currentstroke}{rgb}{0.000000,0.000000,0.000000}%
\pgfsetstrokecolor{currentstroke}%
\pgfsetdash{}{0pt}%
\pgfsys@defobject{currentmarker}{\pgfqpoint{0.000000in}{-0.027778in}}{\pgfqpoint{0.000000in}{0.000000in}}{%
\pgfpathmoveto{\pgfqpoint{0.000000in}{0.000000in}}%
\pgfpathlineto{\pgfqpoint{0.000000in}{-0.027778in}}%
\pgfusepath{stroke,fill}%
}%
\begin{pgfscope}%
\pgfsys@transformshift{2.940000in}{4.839130in}%
\pgfsys@useobject{currentmarker}{}%
\end{pgfscope}%
\end{pgfscope}%
\begin{pgfscope}%
\pgfpathrectangle{\pgfqpoint{0.600000in}{3.621739in}}{\pgfqpoint{4.680000in}{1.217391in}} %
\pgfusepath{clip}%
\pgfsetbuttcap%
\pgfsetroundjoin%
\pgfsetlinewidth{0.501875pt}%
\definecolor{currentstroke}{rgb}{0.000000,0.000000,0.000000}%
\pgfsetstrokecolor{currentstroke}%
\pgfsetdash{{1.000000pt}{3.000000pt}}{0.000000pt}%
\pgfpathmoveto{\pgfqpoint{4.110000in}{3.621739in}}%
\pgfpathlineto{\pgfqpoint{4.110000in}{4.839130in}}%
\pgfusepath{stroke}%
\end{pgfscope}%
\begin{pgfscope}%
\pgfsetbuttcap%
\pgfsetroundjoin%
\definecolor{currentfill}{rgb}{0.000000,0.000000,0.000000}%
\pgfsetfillcolor{currentfill}%
\pgfsetlinewidth{0.501875pt}%
\definecolor{currentstroke}{rgb}{0.000000,0.000000,0.000000}%
\pgfsetstrokecolor{currentstroke}%
\pgfsetdash{}{0pt}%
\pgfsys@defobject{currentmarker}{\pgfqpoint{0.000000in}{0.000000in}}{\pgfqpoint{0.000000in}{0.027778in}}{%
\pgfpathmoveto{\pgfqpoint{0.000000in}{0.000000in}}%
\pgfpathlineto{\pgfqpoint{0.000000in}{0.027778in}}%
\pgfusepath{stroke,fill}%
}%
\begin{pgfscope}%
\pgfsys@transformshift{4.110000in}{3.621739in}%
\pgfsys@useobject{currentmarker}{}%
\end{pgfscope}%
\end{pgfscope}%
\begin{pgfscope}%
\pgfsetbuttcap%
\pgfsetroundjoin%
\definecolor{currentfill}{rgb}{0.000000,0.000000,0.000000}%
\pgfsetfillcolor{currentfill}%
\pgfsetlinewidth{0.501875pt}%
\definecolor{currentstroke}{rgb}{0.000000,0.000000,0.000000}%
\pgfsetstrokecolor{currentstroke}%
\pgfsetdash{}{0pt}%
\pgfsys@defobject{currentmarker}{\pgfqpoint{0.000000in}{-0.027778in}}{\pgfqpoint{0.000000in}{0.000000in}}{%
\pgfpathmoveto{\pgfqpoint{0.000000in}{0.000000in}}%
\pgfpathlineto{\pgfqpoint{0.000000in}{-0.027778in}}%
\pgfusepath{stroke,fill}%
}%
\begin{pgfscope}%
\pgfsys@transformshift{4.110000in}{4.839130in}%
\pgfsys@useobject{currentmarker}{}%
\end{pgfscope}%
\end{pgfscope}%
\begin{pgfscope}%
\pgfpathrectangle{\pgfqpoint{0.600000in}{3.621739in}}{\pgfqpoint{4.680000in}{1.217391in}} %
\pgfusepath{clip}%
\pgfsetbuttcap%
\pgfsetroundjoin%
\pgfsetlinewidth{0.501875pt}%
\definecolor{currentstroke}{rgb}{0.000000,0.000000,0.000000}%
\pgfsetstrokecolor{currentstroke}%
\pgfsetdash{{1.000000pt}{3.000000pt}}{0.000000pt}%
\pgfpathmoveto{\pgfqpoint{5.280000in}{3.621739in}}%
\pgfpathlineto{\pgfqpoint{5.280000in}{4.839130in}}%
\pgfusepath{stroke}%
\end{pgfscope}%
\begin{pgfscope}%
\pgfsetbuttcap%
\pgfsetroundjoin%
\definecolor{currentfill}{rgb}{0.000000,0.000000,0.000000}%
\pgfsetfillcolor{currentfill}%
\pgfsetlinewidth{0.501875pt}%
\definecolor{currentstroke}{rgb}{0.000000,0.000000,0.000000}%
\pgfsetstrokecolor{currentstroke}%
\pgfsetdash{}{0pt}%
\pgfsys@defobject{currentmarker}{\pgfqpoint{0.000000in}{0.000000in}}{\pgfqpoint{0.000000in}{0.027778in}}{%
\pgfpathmoveto{\pgfqpoint{0.000000in}{0.000000in}}%
\pgfpathlineto{\pgfqpoint{0.000000in}{0.027778in}}%
\pgfusepath{stroke,fill}%
}%
\begin{pgfscope}%
\pgfsys@transformshift{5.280000in}{3.621739in}%
\pgfsys@useobject{currentmarker}{}%
\end{pgfscope}%
\end{pgfscope}%
\begin{pgfscope}%
\pgfsetbuttcap%
\pgfsetroundjoin%
\definecolor{currentfill}{rgb}{0.000000,0.000000,0.000000}%
\pgfsetfillcolor{currentfill}%
\pgfsetlinewidth{0.501875pt}%
\definecolor{currentstroke}{rgb}{0.000000,0.000000,0.000000}%
\pgfsetstrokecolor{currentstroke}%
\pgfsetdash{}{0pt}%
\pgfsys@defobject{currentmarker}{\pgfqpoint{0.000000in}{-0.027778in}}{\pgfqpoint{0.000000in}{0.000000in}}{%
\pgfpathmoveto{\pgfqpoint{0.000000in}{0.000000in}}%
\pgfpathlineto{\pgfqpoint{0.000000in}{-0.027778in}}%
\pgfusepath{stroke,fill}%
}%
\begin{pgfscope}%
\pgfsys@transformshift{5.280000in}{4.839130in}%
\pgfsys@useobject{currentmarker}{}%
\end{pgfscope}%
\end{pgfscope}%
\begin{pgfscope}%
\pgfpathrectangle{\pgfqpoint{0.600000in}{3.621739in}}{\pgfqpoint{4.680000in}{1.217391in}} %
\pgfusepath{clip}%
\pgfsetbuttcap%
\pgfsetroundjoin%
\pgfsetlinewidth{0.501875pt}%
\definecolor{currentstroke}{rgb}{0.000000,0.000000,0.000000}%
\pgfsetstrokecolor{currentstroke}%
\pgfsetdash{{1.000000pt}{3.000000pt}}{0.000000pt}%
\pgfpathmoveto{\pgfqpoint{0.600000in}{4.839130in}}%
\pgfpathlineto{\pgfqpoint{5.280000in}{4.839130in}}%
\pgfusepath{stroke}%
\end{pgfscope}%
\begin{pgfscope}%
\pgfsetbuttcap%
\pgfsetroundjoin%
\definecolor{currentfill}{rgb}{0.000000,0.000000,0.000000}%
\pgfsetfillcolor{currentfill}%
\pgfsetlinewidth{0.501875pt}%
\definecolor{currentstroke}{rgb}{0.000000,0.000000,0.000000}%
\pgfsetstrokecolor{currentstroke}%
\pgfsetdash{}{0pt}%
\pgfsys@defobject{currentmarker}{\pgfqpoint{0.000000in}{0.000000in}}{\pgfqpoint{0.055556in}{0.000000in}}{%
\pgfpathmoveto{\pgfqpoint{0.000000in}{0.000000in}}%
\pgfpathlineto{\pgfqpoint{0.055556in}{0.000000in}}%
\pgfusepath{stroke,fill}%
}%
\begin{pgfscope}%
\pgfsys@transformshift{0.600000in}{4.839130in}%
\pgfsys@useobject{currentmarker}{}%
\end{pgfscope}%
\end{pgfscope}%
\begin{pgfscope}%
\pgftext[left,bottom,x=0.462848in,y=4.785427in,rotate=0.000000]{{\rmfamily\fontsize{12.000000}{14.400000}\selectfont \(\displaystyle 0\)}}
%
\end{pgfscope}%
\begin{pgfscope}%
\pgfpathrectangle{\pgfqpoint{0.600000in}{3.621739in}}{\pgfqpoint{4.680000in}{1.217391in}} %
\pgfusepath{clip}%
\pgfsetbuttcap%
\pgfsetroundjoin%
\pgfsetlinewidth{0.501875pt}%
\definecolor{currentstroke}{rgb}{0.000000,0.000000,0.000000}%
\pgfsetstrokecolor{currentstroke}%
\pgfsetdash{{1.000000pt}{3.000000pt}}{0.000000pt}%
\pgfpathmoveto{\pgfqpoint{0.600000in}{4.534783in}}%
\pgfpathlineto{\pgfqpoint{5.280000in}{4.534783in}}%
\pgfusepath{stroke}%
\end{pgfscope}%
\begin{pgfscope}%
\pgfsetbuttcap%
\pgfsetroundjoin%
\definecolor{currentfill}{rgb}{0.000000,0.000000,0.000000}%
\pgfsetfillcolor{currentfill}%
\pgfsetlinewidth{0.501875pt}%
\definecolor{currentstroke}{rgb}{0.000000,0.000000,0.000000}%
\pgfsetstrokecolor{currentstroke}%
\pgfsetdash{}{0pt}%
\pgfsys@defobject{currentmarker}{\pgfqpoint{0.000000in}{0.000000in}}{\pgfqpoint{0.055556in}{0.000000in}}{%
\pgfpathmoveto{\pgfqpoint{0.000000in}{0.000000in}}%
\pgfpathlineto{\pgfqpoint{0.055556in}{0.000000in}}%
\pgfusepath{stroke,fill}%
}%
\begin{pgfscope}%
\pgfsys@transformshift{0.600000in}{4.534783in}%
\pgfsys@useobject{currentmarker}{}%
\end{pgfscope}%
\end{pgfscope}%
\begin{pgfscope}%
\pgftext[left,bottom,x=0.333218in,y=4.474135in,rotate=0.000000]{{\rmfamily\fontsize{12.000000}{14.400000}\selectfont \(\displaystyle -2\)}}
%
\end{pgfscope}%
\begin{pgfscope}%
\pgfpathrectangle{\pgfqpoint{0.600000in}{3.621739in}}{\pgfqpoint{4.680000in}{1.217391in}} %
\pgfusepath{clip}%
\pgfsetbuttcap%
\pgfsetroundjoin%
\pgfsetlinewidth{0.501875pt}%
\definecolor{currentstroke}{rgb}{0.000000,0.000000,0.000000}%
\pgfsetstrokecolor{currentstroke}%
\pgfsetdash{{1.000000pt}{3.000000pt}}{0.000000pt}%
\pgfpathmoveto{\pgfqpoint{0.600000in}{4.230435in}}%
\pgfpathlineto{\pgfqpoint{5.280000in}{4.230435in}}%
\pgfusepath{stroke}%
\end{pgfscope}%
\begin{pgfscope}%
\pgfsetbuttcap%
\pgfsetroundjoin%
\definecolor{currentfill}{rgb}{0.000000,0.000000,0.000000}%
\pgfsetfillcolor{currentfill}%
\pgfsetlinewidth{0.501875pt}%
\definecolor{currentstroke}{rgb}{0.000000,0.000000,0.000000}%
\pgfsetstrokecolor{currentstroke}%
\pgfsetdash{}{0pt}%
\pgfsys@defobject{currentmarker}{\pgfqpoint{0.000000in}{0.000000in}}{\pgfqpoint{0.055556in}{0.000000in}}{%
\pgfpathmoveto{\pgfqpoint{0.000000in}{0.000000in}}%
\pgfpathlineto{\pgfqpoint{0.055556in}{0.000000in}}%
\pgfusepath{stroke,fill}%
}%
\begin{pgfscope}%
\pgfsys@transformshift{0.600000in}{4.230435in}%
\pgfsys@useobject{currentmarker}{}%
\end{pgfscope}%
\end{pgfscope}%
\begin{pgfscope}%
\pgftext[left,bottom,x=0.333218in,y=4.169787in,rotate=0.000000]{{\rmfamily\fontsize{12.000000}{14.400000}\selectfont \(\displaystyle -4\)}}
%
\end{pgfscope}%
\begin{pgfscope}%
\pgfpathrectangle{\pgfqpoint{0.600000in}{3.621739in}}{\pgfqpoint{4.680000in}{1.217391in}} %
\pgfusepath{clip}%
\pgfsetbuttcap%
\pgfsetroundjoin%
\pgfsetlinewidth{0.501875pt}%
\definecolor{currentstroke}{rgb}{0.000000,0.000000,0.000000}%
\pgfsetstrokecolor{currentstroke}%
\pgfsetdash{{1.000000pt}{3.000000pt}}{0.000000pt}%
\pgfpathmoveto{\pgfqpoint{0.600000in}{3.926087in}}%
\pgfpathlineto{\pgfqpoint{5.280000in}{3.926087in}}%
\pgfusepath{stroke}%
\end{pgfscope}%
\begin{pgfscope}%
\pgfsetbuttcap%
\pgfsetroundjoin%
\definecolor{currentfill}{rgb}{0.000000,0.000000,0.000000}%
\pgfsetfillcolor{currentfill}%
\pgfsetlinewidth{0.501875pt}%
\definecolor{currentstroke}{rgb}{0.000000,0.000000,0.000000}%
\pgfsetstrokecolor{currentstroke}%
\pgfsetdash{}{0pt}%
\pgfsys@defobject{currentmarker}{\pgfqpoint{0.000000in}{0.000000in}}{\pgfqpoint{0.055556in}{0.000000in}}{%
\pgfpathmoveto{\pgfqpoint{0.000000in}{0.000000in}}%
\pgfpathlineto{\pgfqpoint{0.055556in}{0.000000in}}%
\pgfusepath{stroke,fill}%
}%
\begin{pgfscope}%
\pgfsys@transformshift{0.600000in}{3.926087in}%
\pgfsys@useobject{currentmarker}{}%
\end{pgfscope}%
\end{pgfscope}%
\begin{pgfscope}%
\pgftext[left,bottom,x=0.333218in,y=3.865439in,rotate=0.000000]{{\rmfamily\fontsize{12.000000}{14.400000}\selectfont \(\displaystyle -6\)}}
%
\end{pgfscope}%
\begin{pgfscope}%
\pgfpathrectangle{\pgfqpoint{0.600000in}{3.621739in}}{\pgfqpoint{4.680000in}{1.217391in}} %
\pgfusepath{clip}%
\pgfsetbuttcap%
\pgfsetroundjoin%
\pgfsetlinewidth{0.501875pt}%
\definecolor{currentstroke}{rgb}{0.000000,0.000000,0.000000}%
\pgfsetstrokecolor{currentstroke}%
\pgfsetdash{{1.000000pt}{3.000000pt}}{0.000000pt}%
\pgfpathmoveto{\pgfqpoint{0.600000in}{3.621739in}}%
\pgfpathlineto{\pgfqpoint{5.280000in}{3.621739in}}%
\pgfusepath{stroke}%
\end{pgfscope}%
\begin{pgfscope}%
\pgfsetbuttcap%
\pgfsetroundjoin%
\definecolor{currentfill}{rgb}{0.000000,0.000000,0.000000}%
\pgfsetfillcolor{currentfill}%
\pgfsetlinewidth{0.501875pt}%
\definecolor{currentstroke}{rgb}{0.000000,0.000000,0.000000}%
\pgfsetstrokecolor{currentstroke}%
\pgfsetdash{}{0pt}%
\pgfsys@defobject{currentmarker}{\pgfqpoint{0.000000in}{0.000000in}}{\pgfqpoint{0.055556in}{0.000000in}}{%
\pgfpathmoveto{\pgfqpoint{0.000000in}{0.000000in}}%
\pgfpathlineto{\pgfqpoint{0.055556in}{0.000000in}}%
\pgfusepath{stroke,fill}%
}%
\begin{pgfscope}%
\pgfsys@transformshift{0.600000in}{3.621739in}%
\pgfsys@useobject{currentmarker}{}%
\end{pgfscope}%
\end{pgfscope}%
\begin{pgfscope}%
\pgftext[left,bottom,x=0.333218in,y=3.561091in,rotate=0.000000]{{\rmfamily\fontsize{12.000000}{14.400000}\selectfont \(\displaystyle -8\)}}
%
\end{pgfscope}%
\begin{pgfscope}%
\pgftext[left,bottom,x=0.600000in,y=4.866908in,rotate=0.000000]{{\rmfamily\fontsize{12.000000}{14.400000}\selectfont \(\displaystyle \times10^{-5}\)}}
%
\end{pgfscope}%
\begin{pgfscope}%
\pgfsetrectcap%
\pgfsetroundjoin%
\pgfsetlinewidth{1.003750pt}%
\definecolor{currentstroke}{rgb}{0.000000,0.000000,0.000000}%
\pgfsetstrokecolor{currentstroke}%
\pgfsetdash{}{0pt}%
\pgfpathmoveto{\pgfqpoint{0.600000in}{4.839130in}}%
\pgfpathlineto{\pgfqpoint{5.280000in}{4.839130in}}%
\pgfusepath{stroke}%
\end{pgfscope}%
\begin{pgfscope}%
\pgfsetrectcap%
\pgfsetroundjoin%
\pgfsetlinewidth{1.003750pt}%
\definecolor{currentstroke}{rgb}{0.000000,0.000000,0.000000}%
\pgfsetstrokecolor{currentstroke}%
\pgfsetdash{}{0pt}%
\pgfpathmoveto{\pgfqpoint{5.280000in}{3.621739in}}%
\pgfpathlineto{\pgfqpoint{5.280000in}{4.839130in}}%
\pgfusepath{stroke}%
\end{pgfscope}%
\begin{pgfscope}%
\pgfsetrectcap%
\pgfsetroundjoin%
\pgfsetlinewidth{1.003750pt}%
\definecolor{currentstroke}{rgb}{0.000000,0.000000,0.000000}%
\pgfsetstrokecolor{currentstroke}%
\pgfsetdash{}{0pt}%
\pgfpathmoveto{\pgfqpoint{0.600000in}{3.621739in}}%
\pgfpathlineto{\pgfqpoint{5.280000in}{3.621739in}}%
\pgfusepath{stroke}%
\end{pgfscope}%
\begin{pgfscope}%
\pgfsetrectcap%
\pgfsetroundjoin%
\pgfsetlinewidth{1.003750pt}%
\definecolor{currentstroke}{rgb}{0.000000,0.000000,0.000000}%
\pgfsetstrokecolor{currentstroke}%
\pgfsetdash{}{0pt}%
\pgfpathmoveto{\pgfqpoint{0.600000in}{3.621739in}}%
\pgfpathlineto{\pgfqpoint{0.600000in}{4.839130in}}%
\pgfusepath{stroke}%
\end{pgfscope}%
\begin{pgfscope}%
\pgfpathrectangle{\pgfqpoint{0.600000in}{3.621739in}}{\pgfqpoint{4.680000in}{1.217391in}} %
\pgfusepath{clip}%
\pgfsetrectcap%
\pgfsetroundjoin%
\pgfsetlinewidth{1.003750pt}%
\definecolor{currentstroke}{rgb}{1.000000,0.000000,0.000000}%
\pgfsetstrokecolor{currentstroke}%
\pgfsetdash{}{0pt}%
\pgfpathmoveto{\pgfqpoint{0.600000in}{4.839130in}}%
\pgfpathlineto{\pgfqpoint{2.846400in}{4.839130in}}%
\pgfpathlineto{\pgfqpoint{2.869800in}{4.717391in}}%
\pgfpathlineto{\pgfqpoint{2.893200in}{4.717391in}}%
\pgfpathlineto{\pgfqpoint{2.940000in}{4.473913in}}%
\pgfpathlineto{\pgfqpoint{2.986800in}{4.717391in}}%
\pgfpathlineto{\pgfqpoint{3.010200in}{4.717391in}}%
\pgfpathlineto{\pgfqpoint{3.033600in}{4.839130in}}%
\pgfpathlineto{\pgfqpoint{5.280000in}{4.839130in}}%
\pgfpathlineto{\pgfqpoint{5.280000in}{4.839130in}}%
\pgfusepath{stroke}%
\end{pgfscope}%
\begin{pgfscope}%
\pgfsetbuttcap%
\pgfsetroundjoin%
\definecolor{currentfill}{rgb}{0.000000,0.000000,0.000000}%
\pgfsetfillcolor{currentfill}%
\pgfsetlinewidth{0.501875pt}%
\definecolor{currentstroke}{rgb}{0.000000,0.000000,0.000000}%
\pgfsetstrokecolor{currentstroke}%
\pgfsetdash{}{0pt}%
\pgfsys@defobject{currentmarker}{\pgfqpoint{-0.055556in}{0.000000in}}{\pgfqpoint{0.000000in}{0.000000in}}{%
\pgfpathmoveto{\pgfqpoint{0.000000in}{0.000000in}}%
\pgfpathlineto{\pgfqpoint{-0.055556in}{0.000000in}}%
\pgfusepath{stroke,fill}%
}%
\begin{pgfscope}%
\pgfsys@transformshift{5.280000in}{3.621739in}%
\pgfsys@useobject{currentmarker}{}%
\end{pgfscope}%
\end{pgfscope}%
\begin{pgfscope}%
\pgftext[left,bottom,x=5.335556in,y=3.568036in,rotate=0.000000]{{\rmfamily\fontsize{12.000000}{14.400000}\selectfont \(\displaystyle 0.00\)}}
%
\end{pgfscope}%
\begin{pgfscope}%
\pgfsetbuttcap%
\pgfsetroundjoin%
\definecolor{currentfill}{rgb}{0.000000,0.000000,0.000000}%
\pgfsetfillcolor{currentfill}%
\pgfsetlinewidth{0.501875pt}%
\definecolor{currentstroke}{rgb}{0.000000,0.000000,0.000000}%
\pgfsetstrokecolor{currentstroke}%
\pgfsetdash{}{0pt}%
\pgfsys@defobject{currentmarker}{\pgfqpoint{-0.055556in}{0.000000in}}{\pgfqpoint{0.000000in}{0.000000in}}{%
\pgfpathmoveto{\pgfqpoint{0.000000in}{0.000000in}}%
\pgfpathlineto{\pgfqpoint{-0.055556in}{0.000000in}}%
\pgfusepath{stroke,fill}%
}%
\begin{pgfscope}%
\pgfsys@transformshift{5.280000in}{3.926087in}%
\pgfsys@useobject{currentmarker}{}%
\end{pgfscope}%
\end{pgfscope}%
\begin{pgfscope}%
\pgftext[left,bottom,x=5.335556in,y=3.872383in,rotate=0.000000]{{\rmfamily\fontsize{12.000000}{14.400000}\selectfont \(\displaystyle 0.25\)}}
%
\end{pgfscope}%
\begin{pgfscope}%
\pgfsetbuttcap%
\pgfsetroundjoin%
\definecolor{currentfill}{rgb}{0.000000,0.000000,0.000000}%
\pgfsetfillcolor{currentfill}%
\pgfsetlinewidth{0.501875pt}%
\definecolor{currentstroke}{rgb}{0.000000,0.000000,0.000000}%
\pgfsetstrokecolor{currentstroke}%
\pgfsetdash{}{0pt}%
\pgfsys@defobject{currentmarker}{\pgfqpoint{-0.055556in}{0.000000in}}{\pgfqpoint{0.000000in}{0.000000in}}{%
\pgfpathmoveto{\pgfqpoint{0.000000in}{0.000000in}}%
\pgfpathlineto{\pgfqpoint{-0.055556in}{0.000000in}}%
\pgfusepath{stroke,fill}%
}%
\begin{pgfscope}%
\pgfsys@transformshift{5.280000in}{4.230435in}%
\pgfsys@useobject{currentmarker}{}%
\end{pgfscope}%
\end{pgfscope}%
\begin{pgfscope}%
\pgftext[left,bottom,x=5.335556in,y=4.176731in,rotate=0.000000]{{\rmfamily\fontsize{12.000000}{14.400000}\selectfont \(\displaystyle 0.50\)}}
%
\end{pgfscope}%
\begin{pgfscope}%
\pgfsetbuttcap%
\pgfsetroundjoin%
\definecolor{currentfill}{rgb}{0.000000,0.000000,0.000000}%
\pgfsetfillcolor{currentfill}%
\pgfsetlinewidth{0.501875pt}%
\definecolor{currentstroke}{rgb}{0.000000,0.000000,0.000000}%
\pgfsetstrokecolor{currentstroke}%
\pgfsetdash{}{0pt}%
\pgfsys@defobject{currentmarker}{\pgfqpoint{-0.055556in}{0.000000in}}{\pgfqpoint{0.000000in}{0.000000in}}{%
\pgfpathmoveto{\pgfqpoint{0.000000in}{0.000000in}}%
\pgfpathlineto{\pgfqpoint{-0.055556in}{0.000000in}}%
\pgfusepath{stroke,fill}%
}%
\begin{pgfscope}%
\pgfsys@transformshift{5.280000in}{4.534783in}%
\pgfsys@useobject{currentmarker}{}%
\end{pgfscope}%
\end{pgfscope}%
\begin{pgfscope}%
\pgftext[left,bottom,x=5.335556in,y=4.481079in,rotate=0.000000]{{\rmfamily\fontsize{12.000000}{14.400000}\selectfont \(\displaystyle 0.75\)}}
%
\end{pgfscope}%
\begin{pgfscope}%
\pgfsetbuttcap%
\pgfsetroundjoin%
\definecolor{currentfill}{rgb}{0.000000,0.000000,0.000000}%
\pgfsetfillcolor{currentfill}%
\pgfsetlinewidth{0.501875pt}%
\definecolor{currentstroke}{rgb}{0.000000,0.000000,0.000000}%
\pgfsetstrokecolor{currentstroke}%
\pgfsetdash{}{0pt}%
\pgfsys@defobject{currentmarker}{\pgfqpoint{-0.055556in}{0.000000in}}{\pgfqpoint{0.000000in}{0.000000in}}{%
\pgfpathmoveto{\pgfqpoint{0.000000in}{0.000000in}}%
\pgfpathlineto{\pgfqpoint{-0.055556in}{0.000000in}}%
\pgfusepath{stroke,fill}%
}%
\begin{pgfscope}%
\pgfsys@transformshift{5.280000in}{4.839130in}%
\pgfsys@useobject{currentmarker}{}%
\end{pgfscope}%
\end{pgfscope}%
\begin{pgfscope}%
\pgftext[left,bottom,x=5.335556in,y=4.785427in,rotate=0.000000]{{\rmfamily\fontsize{12.000000}{14.400000}\selectfont \(\displaystyle 1.00\)}}
%
\end{pgfscope}%
\begin{pgfscope}%
\pgfsetrectcap%
\pgfsetroundjoin%
\definecolor{currentfill}{rgb}{1.000000,1.000000,1.000000}%
\pgfsetfillcolor{currentfill}%
\pgfsetlinewidth{0.000000pt}%
\definecolor{currentstroke}{rgb}{0.000000,0.000000,0.000000}%
\pgfsetstrokecolor{currentstroke}%
\pgfsetdash{}{0pt}%
\pgfpathmoveto{\pgfqpoint{0.600000in}{2.160870in}}%
\pgfpathlineto{\pgfqpoint{5.280000in}{2.160870in}}%
\pgfpathlineto{\pgfqpoint{5.280000in}{3.378261in}}%
\pgfpathlineto{\pgfqpoint{0.600000in}{3.378261in}}%
\pgfpathclose%
\pgfusepath{fill}%
\end{pgfscope}%
\begin{pgfscope}%
\pgfpathrectangle{\pgfqpoint{0.600000in}{2.160870in}}{\pgfqpoint{4.680000in}{1.217391in}} %
\pgfusepath{clip}%
\pgfsetbuttcap%
\pgfsetroundjoin%
\pgfsetlinewidth{1.003750pt}%
\definecolor{currentstroke}{rgb}{0.000000,0.000000,1.000000}%
\pgfsetstrokecolor{currentstroke}%
\pgfsetdash{{6.000000pt}{6.000000pt}}{0.000000pt}%
\pgfpathmoveto{\pgfqpoint{0.600000in}{3.378261in}}%
\pgfpathlineto{\pgfqpoint{0.974400in}{3.157100in}}%
\pgfpathlineto{\pgfqpoint{1.161600in}{3.049329in}}%
\pgfpathlineto{\pgfqpoint{1.325400in}{2.957805in}}%
\pgfpathlineto{\pgfqpoint{1.465800in}{2.882039in}}%
\pgfpathlineto{\pgfqpoint{1.582800in}{2.821165in}}%
\pgfpathlineto{\pgfqpoint{1.699800in}{2.762655in}}%
\pgfpathlineto{\pgfqpoint{1.816800in}{2.706792in}}%
\pgfpathlineto{\pgfqpoint{1.933800in}{2.653861in}}%
\pgfpathlineto{\pgfqpoint{2.027400in}{2.613817in}}%
\pgfpathlineto{\pgfqpoint{2.121000in}{2.575977in}}%
\pgfpathlineto{\pgfqpoint{2.214600in}{2.540486in}}%
\pgfpathlineto{\pgfqpoint{2.308200in}{2.507485in}}%
\pgfpathlineto{\pgfqpoint{2.401800in}{2.477132in}}%
\pgfpathlineto{\pgfqpoint{2.495400in}{2.449575in}}%
\pgfpathlineto{\pgfqpoint{2.589000in}{2.424822in}}%
\pgfpathlineto{\pgfqpoint{2.659200in}{2.408467in}}%
\pgfpathlineto{\pgfqpoint{2.729400in}{2.393976in}}%
\pgfpathlineto{\pgfqpoint{2.940000in}{2.355652in}}%
\pgfpathlineto{\pgfqpoint{2.986800in}{2.365060in}}%
\pgfpathlineto{\pgfqpoint{3.080400in}{2.381060in}}%
\pgfpathlineto{\pgfqpoint{3.220800in}{2.408467in}}%
\pgfpathlineto{\pgfqpoint{3.314400in}{2.430743in}}%
\pgfpathlineto{\pgfqpoint{3.408000in}{2.456225in}}%
\pgfpathlineto{\pgfqpoint{3.501600in}{2.484459in}}%
\pgfpathlineto{\pgfqpoint{3.595200in}{2.515494in}}%
\pgfpathlineto{\pgfqpoint{3.688800in}{2.549130in}}%
\pgfpathlineto{\pgfqpoint{3.782400in}{2.585222in}}%
\pgfpathlineto{\pgfqpoint{3.876000in}{2.623627in}}%
\pgfpathlineto{\pgfqpoint{3.969600in}{2.664199in}}%
\pgfpathlineto{\pgfqpoint{4.063200in}{2.706792in}}%
\pgfpathlineto{\pgfqpoint{4.180200in}{2.762655in}}%
\pgfpathlineto{\pgfqpoint{4.297200in}{2.821165in}}%
\pgfpathlineto{\pgfqpoint{4.414200in}{2.882039in}}%
\pgfpathlineto{\pgfqpoint{4.554600in}{2.957805in}}%
\pgfpathlineto{\pgfqpoint{4.695000in}{3.036075in}}%
\pgfpathlineto{\pgfqpoint{4.858800in}{3.129899in}}%
\pgfpathlineto{\pgfqpoint{5.069400in}{3.253290in}}%
\pgfpathlineto{\pgfqpoint{5.280000in}{3.378261in}}%
\pgfpathlineto{\pgfqpoint{5.280000in}{3.378261in}}%
\pgfusepath{stroke}%
\end{pgfscope}%
\begin{pgfscope}%
\pgfpathrectangle{\pgfqpoint{0.600000in}{2.160870in}}{\pgfqpoint{4.680000in}{1.217391in}} %
\pgfusepath{clip}%
\pgfsetbuttcap%
\pgfsetroundjoin%
\pgfsetlinewidth{0.501875pt}%
\definecolor{currentstroke}{rgb}{0.000000,0.000000,0.000000}%
\pgfsetstrokecolor{currentstroke}%
\pgfsetdash{{1.000000pt}{3.000000pt}}{0.000000pt}%
\pgfpathmoveto{\pgfqpoint{0.600000in}{2.160870in}}%
\pgfpathlineto{\pgfqpoint{0.600000in}{3.378261in}}%
\pgfusepath{stroke}%
\end{pgfscope}%
\begin{pgfscope}%
\pgfsetbuttcap%
\pgfsetroundjoin%
\definecolor{currentfill}{rgb}{0.000000,0.000000,0.000000}%
\pgfsetfillcolor{currentfill}%
\pgfsetlinewidth{0.501875pt}%
\definecolor{currentstroke}{rgb}{0.000000,0.000000,0.000000}%
\pgfsetstrokecolor{currentstroke}%
\pgfsetdash{}{0pt}%
\pgfsys@defobject{currentmarker}{\pgfqpoint{0.000000in}{0.000000in}}{\pgfqpoint{0.000000in}{0.027778in}}{%
\pgfpathmoveto{\pgfqpoint{0.000000in}{0.000000in}}%
\pgfpathlineto{\pgfqpoint{0.000000in}{0.027778in}}%
\pgfusepath{stroke,fill}%
}%
\begin{pgfscope}%
\pgfsys@transformshift{0.600000in}{2.160870in}%
\pgfsys@useobject{currentmarker}{}%
\end{pgfscope}%
\end{pgfscope}%
\begin{pgfscope}%
\pgfsetbuttcap%
\pgfsetroundjoin%
\definecolor{currentfill}{rgb}{0.000000,0.000000,0.000000}%
\pgfsetfillcolor{currentfill}%
\pgfsetlinewidth{0.501875pt}%
\definecolor{currentstroke}{rgb}{0.000000,0.000000,0.000000}%
\pgfsetstrokecolor{currentstroke}%
\pgfsetdash{}{0pt}%
\pgfsys@defobject{currentmarker}{\pgfqpoint{0.000000in}{-0.027778in}}{\pgfqpoint{0.000000in}{0.000000in}}{%
\pgfpathmoveto{\pgfqpoint{0.000000in}{0.000000in}}%
\pgfpathlineto{\pgfqpoint{0.000000in}{-0.027778in}}%
\pgfusepath{stroke,fill}%
}%
\begin{pgfscope}%
\pgfsys@transformshift{0.600000in}{3.378261in}%
\pgfsys@useobject{currentmarker}{}%
\end{pgfscope}%
\end{pgfscope}%
\begin{pgfscope}%
\pgfpathrectangle{\pgfqpoint{0.600000in}{2.160870in}}{\pgfqpoint{4.680000in}{1.217391in}} %
\pgfusepath{clip}%
\pgfsetbuttcap%
\pgfsetroundjoin%
\pgfsetlinewidth{0.501875pt}%
\definecolor{currentstroke}{rgb}{0.000000,0.000000,0.000000}%
\pgfsetstrokecolor{currentstroke}%
\pgfsetdash{{1.000000pt}{3.000000pt}}{0.000000pt}%
\pgfpathmoveto{\pgfqpoint{1.770000in}{2.160870in}}%
\pgfpathlineto{\pgfqpoint{1.770000in}{3.378261in}}%
\pgfusepath{stroke}%
\end{pgfscope}%
\begin{pgfscope}%
\pgfsetbuttcap%
\pgfsetroundjoin%
\definecolor{currentfill}{rgb}{0.000000,0.000000,0.000000}%
\pgfsetfillcolor{currentfill}%
\pgfsetlinewidth{0.501875pt}%
\definecolor{currentstroke}{rgb}{0.000000,0.000000,0.000000}%
\pgfsetstrokecolor{currentstroke}%
\pgfsetdash{}{0pt}%
\pgfsys@defobject{currentmarker}{\pgfqpoint{0.000000in}{0.000000in}}{\pgfqpoint{0.000000in}{0.027778in}}{%
\pgfpathmoveto{\pgfqpoint{0.000000in}{0.000000in}}%
\pgfpathlineto{\pgfqpoint{0.000000in}{0.027778in}}%
\pgfusepath{stroke,fill}%
}%
\begin{pgfscope}%
\pgfsys@transformshift{1.770000in}{2.160870in}%
\pgfsys@useobject{currentmarker}{}%
\end{pgfscope}%
\end{pgfscope}%
\begin{pgfscope}%
\pgfsetbuttcap%
\pgfsetroundjoin%
\definecolor{currentfill}{rgb}{0.000000,0.000000,0.000000}%
\pgfsetfillcolor{currentfill}%
\pgfsetlinewidth{0.501875pt}%
\definecolor{currentstroke}{rgb}{0.000000,0.000000,0.000000}%
\pgfsetstrokecolor{currentstroke}%
\pgfsetdash{}{0pt}%
\pgfsys@defobject{currentmarker}{\pgfqpoint{0.000000in}{-0.027778in}}{\pgfqpoint{0.000000in}{0.000000in}}{%
\pgfpathmoveto{\pgfqpoint{0.000000in}{0.000000in}}%
\pgfpathlineto{\pgfqpoint{0.000000in}{-0.027778in}}%
\pgfusepath{stroke,fill}%
}%
\begin{pgfscope}%
\pgfsys@transformshift{1.770000in}{3.378261in}%
\pgfsys@useobject{currentmarker}{}%
\end{pgfscope}%
\end{pgfscope}%
\begin{pgfscope}%
\pgfpathrectangle{\pgfqpoint{0.600000in}{2.160870in}}{\pgfqpoint{4.680000in}{1.217391in}} %
\pgfusepath{clip}%
\pgfsetbuttcap%
\pgfsetroundjoin%
\pgfsetlinewidth{0.501875pt}%
\definecolor{currentstroke}{rgb}{0.000000,0.000000,0.000000}%
\pgfsetstrokecolor{currentstroke}%
\pgfsetdash{{1.000000pt}{3.000000pt}}{0.000000pt}%
\pgfpathmoveto{\pgfqpoint{2.940000in}{2.160870in}}%
\pgfpathlineto{\pgfqpoint{2.940000in}{3.378261in}}%
\pgfusepath{stroke}%
\end{pgfscope}%
\begin{pgfscope}%
\pgfsetbuttcap%
\pgfsetroundjoin%
\definecolor{currentfill}{rgb}{0.000000,0.000000,0.000000}%
\pgfsetfillcolor{currentfill}%
\pgfsetlinewidth{0.501875pt}%
\definecolor{currentstroke}{rgb}{0.000000,0.000000,0.000000}%
\pgfsetstrokecolor{currentstroke}%
\pgfsetdash{}{0pt}%
\pgfsys@defobject{currentmarker}{\pgfqpoint{0.000000in}{0.000000in}}{\pgfqpoint{0.000000in}{0.027778in}}{%
\pgfpathmoveto{\pgfqpoint{0.000000in}{0.000000in}}%
\pgfpathlineto{\pgfqpoint{0.000000in}{0.027778in}}%
\pgfusepath{stroke,fill}%
}%
\begin{pgfscope}%
\pgfsys@transformshift{2.940000in}{2.160870in}%
\pgfsys@useobject{currentmarker}{}%
\end{pgfscope}%
\end{pgfscope}%
\begin{pgfscope}%
\pgfsetbuttcap%
\pgfsetroundjoin%
\definecolor{currentfill}{rgb}{0.000000,0.000000,0.000000}%
\pgfsetfillcolor{currentfill}%
\pgfsetlinewidth{0.501875pt}%
\definecolor{currentstroke}{rgb}{0.000000,0.000000,0.000000}%
\pgfsetstrokecolor{currentstroke}%
\pgfsetdash{}{0pt}%
\pgfsys@defobject{currentmarker}{\pgfqpoint{0.000000in}{-0.027778in}}{\pgfqpoint{0.000000in}{0.000000in}}{%
\pgfpathmoveto{\pgfqpoint{0.000000in}{0.000000in}}%
\pgfpathlineto{\pgfqpoint{0.000000in}{-0.027778in}}%
\pgfusepath{stroke,fill}%
}%
\begin{pgfscope}%
\pgfsys@transformshift{2.940000in}{3.378261in}%
\pgfsys@useobject{currentmarker}{}%
\end{pgfscope}%
\end{pgfscope}%
\begin{pgfscope}%
\pgfpathrectangle{\pgfqpoint{0.600000in}{2.160870in}}{\pgfqpoint{4.680000in}{1.217391in}} %
\pgfusepath{clip}%
\pgfsetbuttcap%
\pgfsetroundjoin%
\pgfsetlinewidth{0.501875pt}%
\definecolor{currentstroke}{rgb}{0.000000,0.000000,0.000000}%
\pgfsetstrokecolor{currentstroke}%
\pgfsetdash{{1.000000pt}{3.000000pt}}{0.000000pt}%
\pgfpathmoveto{\pgfqpoint{4.110000in}{2.160870in}}%
\pgfpathlineto{\pgfqpoint{4.110000in}{3.378261in}}%
\pgfusepath{stroke}%
\end{pgfscope}%
\begin{pgfscope}%
\pgfsetbuttcap%
\pgfsetroundjoin%
\definecolor{currentfill}{rgb}{0.000000,0.000000,0.000000}%
\pgfsetfillcolor{currentfill}%
\pgfsetlinewidth{0.501875pt}%
\definecolor{currentstroke}{rgb}{0.000000,0.000000,0.000000}%
\pgfsetstrokecolor{currentstroke}%
\pgfsetdash{}{0pt}%
\pgfsys@defobject{currentmarker}{\pgfqpoint{0.000000in}{0.000000in}}{\pgfqpoint{0.000000in}{0.027778in}}{%
\pgfpathmoveto{\pgfqpoint{0.000000in}{0.000000in}}%
\pgfpathlineto{\pgfqpoint{0.000000in}{0.027778in}}%
\pgfusepath{stroke,fill}%
}%
\begin{pgfscope}%
\pgfsys@transformshift{4.110000in}{2.160870in}%
\pgfsys@useobject{currentmarker}{}%
\end{pgfscope}%
\end{pgfscope}%
\begin{pgfscope}%
\pgfsetbuttcap%
\pgfsetroundjoin%
\definecolor{currentfill}{rgb}{0.000000,0.000000,0.000000}%
\pgfsetfillcolor{currentfill}%
\pgfsetlinewidth{0.501875pt}%
\definecolor{currentstroke}{rgb}{0.000000,0.000000,0.000000}%
\pgfsetstrokecolor{currentstroke}%
\pgfsetdash{}{0pt}%
\pgfsys@defobject{currentmarker}{\pgfqpoint{0.000000in}{-0.027778in}}{\pgfqpoint{0.000000in}{0.000000in}}{%
\pgfpathmoveto{\pgfqpoint{0.000000in}{0.000000in}}%
\pgfpathlineto{\pgfqpoint{0.000000in}{-0.027778in}}%
\pgfusepath{stroke,fill}%
}%
\begin{pgfscope}%
\pgfsys@transformshift{4.110000in}{3.378261in}%
\pgfsys@useobject{currentmarker}{}%
\end{pgfscope}%
\end{pgfscope}%
\begin{pgfscope}%
\pgfpathrectangle{\pgfqpoint{0.600000in}{2.160870in}}{\pgfqpoint{4.680000in}{1.217391in}} %
\pgfusepath{clip}%
\pgfsetbuttcap%
\pgfsetroundjoin%
\pgfsetlinewidth{0.501875pt}%
\definecolor{currentstroke}{rgb}{0.000000,0.000000,0.000000}%
\pgfsetstrokecolor{currentstroke}%
\pgfsetdash{{1.000000pt}{3.000000pt}}{0.000000pt}%
\pgfpathmoveto{\pgfqpoint{5.280000in}{2.160870in}}%
\pgfpathlineto{\pgfqpoint{5.280000in}{3.378261in}}%
\pgfusepath{stroke}%
\end{pgfscope}%
\begin{pgfscope}%
\pgfsetbuttcap%
\pgfsetroundjoin%
\definecolor{currentfill}{rgb}{0.000000,0.000000,0.000000}%
\pgfsetfillcolor{currentfill}%
\pgfsetlinewidth{0.501875pt}%
\definecolor{currentstroke}{rgb}{0.000000,0.000000,0.000000}%
\pgfsetstrokecolor{currentstroke}%
\pgfsetdash{}{0pt}%
\pgfsys@defobject{currentmarker}{\pgfqpoint{0.000000in}{0.000000in}}{\pgfqpoint{0.000000in}{0.027778in}}{%
\pgfpathmoveto{\pgfqpoint{0.000000in}{0.000000in}}%
\pgfpathlineto{\pgfqpoint{0.000000in}{0.027778in}}%
\pgfusepath{stroke,fill}%
}%
\begin{pgfscope}%
\pgfsys@transformshift{5.280000in}{2.160870in}%
\pgfsys@useobject{currentmarker}{}%
\end{pgfscope}%
\end{pgfscope}%
\begin{pgfscope}%
\pgfsetbuttcap%
\pgfsetroundjoin%
\definecolor{currentfill}{rgb}{0.000000,0.000000,0.000000}%
\pgfsetfillcolor{currentfill}%
\pgfsetlinewidth{0.501875pt}%
\definecolor{currentstroke}{rgb}{0.000000,0.000000,0.000000}%
\pgfsetstrokecolor{currentstroke}%
\pgfsetdash{}{0pt}%
\pgfsys@defobject{currentmarker}{\pgfqpoint{0.000000in}{-0.027778in}}{\pgfqpoint{0.000000in}{0.000000in}}{%
\pgfpathmoveto{\pgfqpoint{0.000000in}{0.000000in}}%
\pgfpathlineto{\pgfqpoint{0.000000in}{-0.027778in}}%
\pgfusepath{stroke,fill}%
}%
\begin{pgfscope}%
\pgfsys@transformshift{5.280000in}{3.378261in}%
\pgfsys@useobject{currentmarker}{}%
\end{pgfscope}%
\end{pgfscope}%
\begin{pgfscope}%
\pgfpathrectangle{\pgfqpoint{0.600000in}{2.160870in}}{\pgfqpoint{4.680000in}{1.217391in}} %
\pgfusepath{clip}%
\pgfsetbuttcap%
\pgfsetroundjoin%
\pgfsetlinewidth{0.501875pt}%
\definecolor{currentstroke}{rgb}{0.000000,0.000000,0.000000}%
\pgfsetstrokecolor{currentstroke}%
\pgfsetdash{{1.000000pt}{3.000000pt}}{0.000000pt}%
\pgfpathmoveto{\pgfqpoint{0.600000in}{3.378261in}}%
\pgfpathlineto{\pgfqpoint{5.280000in}{3.378261in}}%
\pgfusepath{stroke}%
\end{pgfscope}%
\begin{pgfscope}%
\pgfsetbuttcap%
\pgfsetroundjoin%
\definecolor{currentfill}{rgb}{0.000000,0.000000,0.000000}%
\pgfsetfillcolor{currentfill}%
\pgfsetlinewidth{0.501875pt}%
\definecolor{currentstroke}{rgb}{0.000000,0.000000,0.000000}%
\pgfsetstrokecolor{currentstroke}%
\pgfsetdash{}{0pt}%
\pgfsys@defobject{currentmarker}{\pgfqpoint{0.000000in}{0.000000in}}{\pgfqpoint{0.055556in}{0.000000in}}{%
\pgfpathmoveto{\pgfqpoint{0.000000in}{0.000000in}}%
\pgfpathlineto{\pgfqpoint{0.055556in}{0.000000in}}%
\pgfusepath{stroke,fill}%
}%
\begin{pgfscope}%
\pgfsys@transformshift{0.600000in}{3.378261in}%
\pgfsys@useobject{currentmarker}{}%
\end{pgfscope}%
\end{pgfscope}%
\begin{pgfscope}%
\pgftext[left,bottom,x=0.462848in,y=3.324557in,rotate=0.000000]{{\rmfamily\fontsize{12.000000}{14.400000}\selectfont \(\displaystyle 0\)}}
%
\end{pgfscope}%
\begin{pgfscope}%
\pgfpathrectangle{\pgfqpoint{0.600000in}{2.160870in}}{\pgfqpoint{4.680000in}{1.217391in}} %
\pgfusepath{clip}%
\pgfsetbuttcap%
\pgfsetroundjoin%
\pgfsetlinewidth{0.501875pt}%
\definecolor{currentstroke}{rgb}{0.000000,0.000000,0.000000}%
\pgfsetstrokecolor{currentstroke}%
\pgfsetdash{{1.000000pt}{3.000000pt}}{0.000000pt}%
\pgfpathmoveto{\pgfqpoint{0.600000in}{3.073913in}}%
\pgfpathlineto{\pgfqpoint{5.280000in}{3.073913in}}%
\pgfusepath{stroke}%
\end{pgfscope}%
\begin{pgfscope}%
\pgfsetbuttcap%
\pgfsetroundjoin%
\definecolor{currentfill}{rgb}{0.000000,0.000000,0.000000}%
\pgfsetfillcolor{currentfill}%
\pgfsetlinewidth{0.501875pt}%
\definecolor{currentstroke}{rgb}{0.000000,0.000000,0.000000}%
\pgfsetstrokecolor{currentstroke}%
\pgfsetdash{}{0pt}%
\pgfsys@defobject{currentmarker}{\pgfqpoint{0.000000in}{0.000000in}}{\pgfqpoint{0.055556in}{0.000000in}}{%
\pgfpathmoveto{\pgfqpoint{0.000000in}{0.000000in}}%
\pgfpathlineto{\pgfqpoint{0.055556in}{0.000000in}}%
\pgfusepath{stroke,fill}%
}%
\begin{pgfscope}%
\pgfsys@transformshift{0.600000in}{3.073913in}%
\pgfsys@useobject{currentmarker}{}%
\end{pgfscope}%
\end{pgfscope}%
\begin{pgfscope}%
\pgftext[left,bottom,x=0.333218in,y=3.013265in,rotate=0.000000]{{\rmfamily\fontsize{12.000000}{14.400000}\selectfont \(\displaystyle -2\)}}
%
\end{pgfscope}%
\begin{pgfscope}%
\pgfpathrectangle{\pgfqpoint{0.600000in}{2.160870in}}{\pgfqpoint{4.680000in}{1.217391in}} %
\pgfusepath{clip}%
\pgfsetbuttcap%
\pgfsetroundjoin%
\pgfsetlinewidth{0.501875pt}%
\definecolor{currentstroke}{rgb}{0.000000,0.000000,0.000000}%
\pgfsetstrokecolor{currentstroke}%
\pgfsetdash{{1.000000pt}{3.000000pt}}{0.000000pt}%
\pgfpathmoveto{\pgfqpoint{0.600000in}{2.769565in}}%
\pgfpathlineto{\pgfqpoint{5.280000in}{2.769565in}}%
\pgfusepath{stroke}%
\end{pgfscope}%
\begin{pgfscope}%
\pgfsetbuttcap%
\pgfsetroundjoin%
\definecolor{currentfill}{rgb}{0.000000,0.000000,0.000000}%
\pgfsetfillcolor{currentfill}%
\pgfsetlinewidth{0.501875pt}%
\definecolor{currentstroke}{rgb}{0.000000,0.000000,0.000000}%
\pgfsetstrokecolor{currentstroke}%
\pgfsetdash{}{0pt}%
\pgfsys@defobject{currentmarker}{\pgfqpoint{0.000000in}{0.000000in}}{\pgfqpoint{0.055556in}{0.000000in}}{%
\pgfpathmoveto{\pgfqpoint{0.000000in}{0.000000in}}%
\pgfpathlineto{\pgfqpoint{0.055556in}{0.000000in}}%
\pgfusepath{stroke,fill}%
}%
\begin{pgfscope}%
\pgfsys@transformshift{0.600000in}{2.769565in}%
\pgfsys@useobject{currentmarker}{}%
\end{pgfscope}%
\end{pgfscope}%
\begin{pgfscope}%
\pgftext[left,bottom,x=0.333218in,y=2.708917in,rotate=0.000000]{{\rmfamily\fontsize{12.000000}{14.400000}\selectfont \(\displaystyle -4\)}}
%
\end{pgfscope}%
\begin{pgfscope}%
\pgfpathrectangle{\pgfqpoint{0.600000in}{2.160870in}}{\pgfqpoint{4.680000in}{1.217391in}} %
\pgfusepath{clip}%
\pgfsetbuttcap%
\pgfsetroundjoin%
\pgfsetlinewidth{0.501875pt}%
\definecolor{currentstroke}{rgb}{0.000000,0.000000,0.000000}%
\pgfsetstrokecolor{currentstroke}%
\pgfsetdash{{1.000000pt}{3.000000pt}}{0.000000pt}%
\pgfpathmoveto{\pgfqpoint{0.600000in}{2.465217in}}%
\pgfpathlineto{\pgfqpoint{5.280000in}{2.465217in}}%
\pgfusepath{stroke}%
\end{pgfscope}%
\begin{pgfscope}%
\pgfsetbuttcap%
\pgfsetroundjoin%
\definecolor{currentfill}{rgb}{0.000000,0.000000,0.000000}%
\pgfsetfillcolor{currentfill}%
\pgfsetlinewidth{0.501875pt}%
\definecolor{currentstroke}{rgb}{0.000000,0.000000,0.000000}%
\pgfsetstrokecolor{currentstroke}%
\pgfsetdash{}{0pt}%
\pgfsys@defobject{currentmarker}{\pgfqpoint{0.000000in}{0.000000in}}{\pgfqpoint{0.055556in}{0.000000in}}{%
\pgfpathmoveto{\pgfqpoint{0.000000in}{0.000000in}}%
\pgfpathlineto{\pgfqpoint{0.055556in}{0.000000in}}%
\pgfusepath{stroke,fill}%
}%
\begin{pgfscope}%
\pgfsys@transformshift{0.600000in}{2.465217in}%
\pgfsys@useobject{currentmarker}{}%
\end{pgfscope}%
\end{pgfscope}%
\begin{pgfscope}%
\pgftext[left,bottom,x=0.333218in,y=2.404569in,rotate=0.000000]{{\rmfamily\fontsize{12.000000}{14.400000}\selectfont \(\displaystyle -6\)}}
%
\end{pgfscope}%
\begin{pgfscope}%
\pgfpathrectangle{\pgfqpoint{0.600000in}{2.160870in}}{\pgfqpoint{4.680000in}{1.217391in}} %
\pgfusepath{clip}%
\pgfsetbuttcap%
\pgfsetroundjoin%
\pgfsetlinewidth{0.501875pt}%
\definecolor{currentstroke}{rgb}{0.000000,0.000000,0.000000}%
\pgfsetstrokecolor{currentstroke}%
\pgfsetdash{{1.000000pt}{3.000000pt}}{0.000000pt}%
\pgfpathmoveto{\pgfqpoint{0.600000in}{2.160870in}}%
\pgfpathlineto{\pgfqpoint{5.280000in}{2.160870in}}%
\pgfusepath{stroke}%
\end{pgfscope}%
\begin{pgfscope}%
\pgfsetbuttcap%
\pgfsetroundjoin%
\definecolor{currentfill}{rgb}{0.000000,0.000000,0.000000}%
\pgfsetfillcolor{currentfill}%
\pgfsetlinewidth{0.501875pt}%
\definecolor{currentstroke}{rgb}{0.000000,0.000000,0.000000}%
\pgfsetstrokecolor{currentstroke}%
\pgfsetdash{}{0pt}%
\pgfsys@defobject{currentmarker}{\pgfqpoint{0.000000in}{0.000000in}}{\pgfqpoint{0.055556in}{0.000000in}}{%
\pgfpathmoveto{\pgfqpoint{0.000000in}{0.000000in}}%
\pgfpathlineto{\pgfqpoint{0.055556in}{0.000000in}}%
\pgfusepath{stroke,fill}%
}%
\begin{pgfscope}%
\pgfsys@transformshift{0.600000in}{2.160870in}%
\pgfsys@useobject{currentmarker}{}%
\end{pgfscope}%
\end{pgfscope}%
\begin{pgfscope}%
\pgftext[left,bottom,x=0.333218in,y=2.100222in,rotate=0.000000]{{\rmfamily\fontsize{12.000000}{14.400000}\selectfont \(\displaystyle -8\)}}
%
\end{pgfscope}%
\begin{pgfscope}%
\pgftext[left,bottom,x=0.600000in,y=3.406039in,rotate=0.000000]{{\rmfamily\fontsize{12.000000}{14.400000}\selectfont \(\displaystyle \times10^{-5}\)}}
%
\end{pgfscope}%
\begin{pgfscope}%
\pgfsetrectcap%
\pgfsetroundjoin%
\pgfsetlinewidth{1.003750pt}%
\definecolor{currentstroke}{rgb}{0.000000,0.000000,0.000000}%
\pgfsetstrokecolor{currentstroke}%
\pgfsetdash{}{0pt}%
\pgfpathmoveto{\pgfqpoint{0.600000in}{3.378261in}}%
\pgfpathlineto{\pgfqpoint{5.280000in}{3.378261in}}%
\pgfusepath{stroke}%
\end{pgfscope}%
\begin{pgfscope}%
\pgfsetrectcap%
\pgfsetroundjoin%
\pgfsetlinewidth{1.003750pt}%
\definecolor{currentstroke}{rgb}{0.000000,0.000000,0.000000}%
\pgfsetstrokecolor{currentstroke}%
\pgfsetdash{}{0pt}%
\pgfpathmoveto{\pgfqpoint{5.280000in}{2.160870in}}%
\pgfpathlineto{\pgfqpoint{5.280000in}{3.378261in}}%
\pgfusepath{stroke}%
\end{pgfscope}%
\begin{pgfscope}%
\pgfsetrectcap%
\pgfsetroundjoin%
\pgfsetlinewidth{1.003750pt}%
\definecolor{currentstroke}{rgb}{0.000000,0.000000,0.000000}%
\pgfsetstrokecolor{currentstroke}%
\pgfsetdash{}{0pt}%
\pgfpathmoveto{\pgfqpoint{0.600000in}{2.160870in}}%
\pgfpathlineto{\pgfqpoint{5.280000in}{2.160870in}}%
\pgfusepath{stroke}%
\end{pgfscope}%
\begin{pgfscope}%
\pgfsetrectcap%
\pgfsetroundjoin%
\pgfsetlinewidth{1.003750pt}%
\definecolor{currentstroke}{rgb}{0.000000,0.000000,0.000000}%
\pgfsetstrokecolor{currentstroke}%
\pgfsetdash{}{0pt}%
\pgfpathmoveto{\pgfqpoint{0.600000in}{2.160870in}}%
\pgfpathlineto{\pgfqpoint{0.600000in}{3.378261in}}%
\pgfusepath{stroke}%
\end{pgfscope}%
\begin{pgfscope}%
\pgfpathrectangle{\pgfqpoint{0.600000in}{2.160870in}}{\pgfqpoint{4.680000in}{1.217391in}} %
\pgfusepath{clip}%
\pgfsetrectcap%
\pgfsetroundjoin%
\pgfsetlinewidth{1.003750pt}%
\definecolor{currentstroke}{rgb}{1.000000,0.000000,0.000000}%
\pgfsetstrokecolor{currentstroke}%
\pgfsetdash{}{0pt}%
\pgfpathmoveto{\pgfqpoint{0.600000in}{3.378261in}}%
\pgfpathlineto{\pgfqpoint{2.846400in}{3.378261in}}%
\pgfpathlineto{\pgfqpoint{2.893200in}{3.134783in}}%
\pgfpathlineto{\pgfqpoint{2.916600in}{2.891304in}}%
\pgfpathlineto{\pgfqpoint{2.940000in}{2.526087in}}%
\pgfpathlineto{\pgfqpoint{2.963400in}{2.891304in}}%
\pgfpathlineto{\pgfqpoint{2.986800in}{3.134783in}}%
\pgfpathlineto{\pgfqpoint{3.033600in}{3.378261in}}%
\pgfpathlineto{\pgfqpoint{5.280000in}{3.378261in}}%
\pgfpathlineto{\pgfqpoint{5.280000in}{3.378261in}}%
\pgfusepath{stroke}%
\end{pgfscope}%
\begin{pgfscope}%
\pgfsetbuttcap%
\pgfsetroundjoin%
\definecolor{currentfill}{rgb}{0.000000,0.000000,0.000000}%
\pgfsetfillcolor{currentfill}%
\pgfsetlinewidth{0.501875pt}%
\definecolor{currentstroke}{rgb}{0.000000,0.000000,0.000000}%
\pgfsetstrokecolor{currentstroke}%
\pgfsetdash{}{0pt}%
\pgfsys@defobject{currentmarker}{\pgfqpoint{-0.055556in}{0.000000in}}{\pgfqpoint{0.000000in}{0.000000in}}{%
\pgfpathmoveto{\pgfqpoint{0.000000in}{0.000000in}}%
\pgfpathlineto{\pgfqpoint{-0.055556in}{0.000000in}}%
\pgfusepath{stroke,fill}%
}%
\begin{pgfscope}%
\pgfsys@transformshift{5.280000in}{2.160870in}%
\pgfsys@useobject{currentmarker}{}%
\end{pgfscope}%
\end{pgfscope}%
\begin{pgfscope}%
\pgftext[left,bottom,x=5.335556in,y=2.107166in,rotate=0.000000]{{\rmfamily\fontsize{12.000000}{14.400000}\selectfont \(\displaystyle 0.00\)}}
%
\end{pgfscope}%
\begin{pgfscope}%
\pgfsetbuttcap%
\pgfsetroundjoin%
\definecolor{currentfill}{rgb}{0.000000,0.000000,0.000000}%
\pgfsetfillcolor{currentfill}%
\pgfsetlinewidth{0.501875pt}%
\definecolor{currentstroke}{rgb}{0.000000,0.000000,0.000000}%
\pgfsetstrokecolor{currentstroke}%
\pgfsetdash{}{0pt}%
\pgfsys@defobject{currentmarker}{\pgfqpoint{-0.055556in}{0.000000in}}{\pgfqpoint{0.000000in}{0.000000in}}{%
\pgfpathmoveto{\pgfqpoint{0.000000in}{0.000000in}}%
\pgfpathlineto{\pgfqpoint{-0.055556in}{0.000000in}}%
\pgfusepath{stroke,fill}%
}%
\begin{pgfscope}%
\pgfsys@transformshift{5.280000in}{2.465217in}%
\pgfsys@useobject{currentmarker}{}%
\end{pgfscope}%
\end{pgfscope}%
\begin{pgfscope}%
\pgftext[left,bottom,x=5.335556in,y=2.411514in,rotate=0.000000]{{\rmfamily\fontsize{12.000000}{14.400000}\selectfont \(\displaystyle 0.25\)}}
%
\end{pgfscope}%
\begin{pgfscope}%
\pgfsetbuttcap%
\pgfsetroundjoin%
\definecolor{currentfill}{rgb}{0.000000,0.000000,0.000000}%
\pgfsetfillcolor{currentfill}%
\pgfsetlinewidth{0.501875pt}%
\definecolor{currentstroke}{rgb}{0.000000,0.000000,0.000000}%
\pgfsetstrokecolor{currentstroke}%
\pgfsetdash{}{0pt}%
\pgfsys@defobject{currentmarker}{\pgfqpoint{-0.055556in}{0.000000in}}{\pgfqpoint{0.000000in}{0.000000in}}{%
\pgfpathmoveto{\pgfqpoint{0.000000in}{0.000000in}}%
\pgfpathlineto{\pgfqpoint{-0.055556in}{0.000000in}}%
\pgfusepath{stroke,fill}%
}%
\begin{pgfscope}%
\pgfsys@transformshift{5.280000in}{2.769565in}%
\pgfsys@useobject{currentmarker}{}%
\end{pgfscope}%
\end{pgfscope}%
\begin{pgfscope}%
\pgftext[left,bottom,x=5.335556in,y=2.715862in,rotate=0.000000]{{\rmfamily\fontsize{12.000000}{14.400000}\selectfont \(\displaystyle 0.50\)}}
%
\end{pgfscope}%
\begin{pgfscope}%
\pgfsetbuttcap%
\pgfsetroundjoin%
\definecolor{currentfill}{rgb}{0.000000,0.000000,0.000000}%
\pgfsetfillcolor{currentfill}%
\pgfsetlinewidth{0.501875pt}%
\definecolor{currentstroke}{rgb}{0.000000,0.000000,0.000000}%
\pgfsetstrokecolor{currentstroke}%
\pgfsetdash{}{0pt}%
\pgfsys@defobject{currentmarker}{\pgfqpoint{-0.055556in}{0.000000in}}{\pgfqpoint{0.000000in}{0.000000in}}{%
\pgfpathmoveto{\pgfqpoint{0.000000in}{0.000000in}}%
\pgfpathlineto{\pgfqpoint{-0.055556in}{0.000000in}}%
\pgfusepath{stroke,fill}%
}%
\begin{pgfscope}%
\pgfsys@transformshift{5.280000in}{3.073913in}%
\pgfsys@useobject{currentmarker}{}%
\end{pgfscope}%
\end{pgfscope}%
\begin{pgfscope}%
\pgftext[left,bottom,x=5.335556in,y=3.020209in,rotate=0.000000]{{\rmfamily\fontsize{12.000000}{14.400000}\selectfont \(\displaystyle 0.75\)}}
%
\end{pgfscope}%
\begin{pgfscope}%
\pgfsetbuttcap%
\pgfsetroundjoin%
\definecolor{currentfill}{rgb}{0.000000,0.000000,0.000000}%
\pgfsetfillcolor{currentfill}%
\pgfsetlinewidth{0.501875pt}%
\definecolor{currentstroke}{rgb}{0.000000,0.000000,0.000000}%
\pgfsetstrokecolor{currentstroke}%
\pgfsetdash{}{0pt}%
\pgfsys@defobject{currentmarker}{\pgfqpoint{-0.055556in}{0.000000in}}{\pgfqpoint{0.000000in}{0.000000in}}{%
\pgfpathmoveto{\pgfqpoint{0.000000in}{0.000000in}}%
\pgfpathlineto{\pgfqpoint{-0.055556in}{0.000000in}}%
\pgfusepath{stroke,fill}%
}%
\begin{pgfscope}%
\pgfsys@transformshift{5.280000in}{3.378261in}%
\pgfsys@useobject{currentmarker}{}%
\end{pgfscope}%
\end{pgfscope}%
\begin{pgfscope}%
\pgftext[left,bottom,x=5.335556in,y=3.324557in,rotate=0.000000]{{\rmfamily\fontsize{12.000000}{14.400000}\selectfont \(\displaystyle 1.00\)}}
%
\end{pgfscope}%
\begin{pgfscope}%
\pgfsetrectcap%
\pgfsetroundjoin%
\definecolor{currentfill}{rgb}{1.000000,1.000000,1.000000}%
\pgfsetfillcolor{currentfill}%
\pgfsetlinewidth{0.000000pt}%
\definecolor{currentstroke}{rgb}{0.000000,0.000000,0.000000}%
\pgfsetstrokecolor{currentstroke}%
\pgfsetdash{}{0pt}%
\pgfpathmoveto{\pgfqpoint{0.600000in}{0.700000in}}%
\pgfpathlineto{\pgfqpoint{5.280000in}{0.700000in}}%
\pgfpathlineto{\pgfqpoint{5.280000in}{1.917391in}}%
\pgfpathlineto{\pgfqpoint{0.600000in}{1.917391in}}%
\pgfpathclose%
\pgfusepath{fill}%
\end{pgfscope}%
\begin{pgfscope}%
\pgfpathrectangle{\pgfqpoint{0.600000in}{0.700000in}}{\pgfqpoint{4.680000in}{1.217391in}} %
\pgfusepath{clip}%
\pgfsetbuttcap%
\pgfsetroundjoin%
\pgfsetlinewidth{1.003750pt}%
\definecolor{currentstroke}{rgb}{0.000000,0.000000,1.000000}%
\pgfsetstrokecolor{currentstroke}%
\pgfsetdash{{6.000000pt}{6.000000pt}}{0.000000pt}%
\pgfpathmoveto{\pgfqpoint{0.600000in}{1.917391in}}%
\pgfpathlineto{\pgfqpoint{1.208400in}{1.636100in}}%
\pgfpathlineto{\pgfqpoint{1.536000in}{1.487226in}}%
\pgfpathlineto{\pgfqpoint{1.793400in}{1.372556in}}%
\pgfpathlineto{\pgfqpoint{2.027400in}{1.270608in}}%
\pgfpathlineto{\pgfqpoint{2.238000in}{1.181106in}}%
\pgfpathlineto{\pgfqpoint{2.425200in}{1.103633in}}%
\pgfpathlineto{\pgfqpoint{2.635800in}{1.018983in}}%
\pgfpathlineto{\pgfqpoint{2.776200in}{0.965161in}}%
\pgfpathlineto{\pgfqpoint{2.823000in}{0.946912in}}%
\pgfpathlineto{\pgfqpoint{2.869800in}{0.925824in}}%
\pgfpathlineto{\pgfqpoint{2.893200in}{0.913773in}}%
\pgfpathlineto{\pgfqpoint{2.916600in}{0.903128in}}%
\pgfpathlineto{\pgfqpoint{2.940000in}{0.894783in}}%
\pgfpathlineto{\pgfqpoint{2.963400in}{0.902579in}}%
\pgfpathlineto{\pgfqpoint{2.986800in}{0.912708in}}%
\pgfpathlineto{\pgfqpoint{3.033600in}{0.935525in}}%
\pgfpathlineto{\pgfqpoint{3.103800in}{0.964018in}}%
\pgfpathlineto{\pgfqpoint{3.244200in}{1.018066in}}%
\pgfpathlineto{\pgfqpoint{3.384600in}{1.074082in}}%
\pgfpathlineto{\pgfqpoint{3.642000in}{1.180251in}}%
\pgfpathlineto{\pgfqpoint{3.852600in}{1.269863in}}%
\pgfpathlineto{\pgfqpoint{4.086600in}{1.371933in}}%
\pgfpathlineto{\pgfqpoint{4.344000in}{1.486738in}}%
\pgfpathlineto{\pgfqpoint{4.648200in}{1.625052in}}%
\pgfpathlineto{\pgfqpoint{5.046000in}{1.808691in}}%
\pgfpathlineto{\pgfqpoint{5.280000in}{1.917391in}}%
\pgfpathlineto{\pgfqpoint{5.280000in}{1.917391in}}%
\pgfusepath{stroke}%
\end{pgfscope}%
\begin{pgfscope}%
\pgfpathrectangle{\pgfqpoint{0.600000in}{0.700000in}}{\pgfqpoint{4.680000in}{1.217391in}} %
\pgfusepath{clip}%
\pgfsetbuttcap%
\pgfsetroundjoin%
\pgfsetlinewidth{0.501875pt}%
\definecolor{currentstroke}{rgb}{0.000000,0.000000,0.000000}%
\pgfsetstrokecolor{currentstroke}%
\pgfsetdash{{1.000000pt}{3.000000pt}}{0.000000pt}%
\pgfpathmoveto{\pgfqpoint{0.600000in}{0.700000in}}%
\pgfpathlineto{\pgfqpoint{0.600000in}{1.917391in}}%
\pgfusepath{stroke}%
\end{pgfscope}%
\begin{pgfscope}%
\pgfsetbuttcap%
\pgfsetroundjoin%
\definecolor{currentfill}{rgb}{0.000000,0.000000,0.000000}%
\pgfsetfillcolor{currentfill}%
\pgfsetlinewidth{0.501875pt}%
\definecolor{currentstroke}{rgb}{0.000000,0.000000,0.000000}%
\pgfsetstrokecolor{currentstroke}%
\pgfsetdash{}{0pt}%
\pgfsys@defobject{currentmarker}{\pgfqpoint{0.000000in}{0.000000in}}{\pgfqpoint{0.000000in}{0.055556in}}{%
\pgfpathmoveto{\pgfqpoint{0.000000in}{0.000000in}}%
\pgfpathlineto{\pgfqpoint{0.000000in}{0.055556in}}%
\pgfusepath{stroke,fill}%
}%
\begin{pgfscope}%
\pgfsys@transformshift{0.600000in}{0.700000in}%
\pgfsys@useobject{currentmarker}{}%
\end{pgfscope}%
\end{pgfscope}%
\begin{pgfscope}%
\pgfsetbuttcap%
\pgfsetroundjoin%
\definecolor{currentfill}{rgb}{0.000000,0.000000,0.000000}%
\pgfsetfillcolor{currentfill}%
\pgfsetlinewidth{0.501875pt}%
\definecolor{currentstroke}{rgb}{0.000000,0.000000,0.000000}%
\pgfsetstrokecolor{currentstroke}%
\pgfsetdash{}{0pt}%
\pgfsys@defobject{currentmarker}{\pgfqpoint{0.000000in}{-0.055556in}}{\pgfqpoint{0.000000in}{0.000000in}}{%
\pgfpathmoveto{\pgfqpoint{0.000000in}{0.000000in}}%
\pgfpathlineto{\pgfqpoint{0.000000in}{-0.055556in}}%
\pgfusepath{stroke,fill}%
}%
\begin{pgfscope}%
\pgfsys@transformshift{0.600000in}{1.917391in}%
\pgfsys@useobject{currentmarker}{}%
\end{pgfscope}%
\end{pgfscope}%
\begin{pgfscope}%
\pgftext[left,bottom,x=0.495738in,y=0.537037in,rotate=0.000000]{{\rmfamily\fontsize{12.000000}{14.400000}\selectfont \(\displaystyle 0.0\)}}
%
\end{pgfscope}%
\begin{pgfscope}%
\pgfpathrectangle{\pgfqpoint{0.600000in}{0.700000in}}{\pgfqpoint{4.680000in}{1.217391in}} %
\pgfusepath{clip}%
\pgfsetbuttcap%
\pgfsetroundjoin%
\pgfsetlinewidth{0.501875pt}%
\definecolor{currentstroke}{rgb}{0.000000,0.000000,0.000000}%
\pgfsetstrokecolor{currentstroke}%
\pgfsetdash{{1.000000pt}{3.000000pt}}{0.000000pt}%
\pgfpathmoveto{\pgfqpoint{1.770000in}{0.700000in}}%
\pgfpathlineto{\pgfqpoint{1.770000in}{1.917391in}}%
\pgfusepath{stroke}%
\end{pgfscope}%
\begin{pgfscope}%
\pgfsetbuttcap%
\pgfsetroundjoin%
\definecolor{currentfill}{rgb}{0.000000,0.000000,0.000000}%
\pgfsetfillcolor{currentfill}%
\pgfsetlinewidth{0.501875pt}%
\definecolor{currentstroke}{rgb}{0.000000,0.000000,0.000000}%
\pgfsetstrokecolor{currentstroke}%
\pgfsetdash{}{0pt}%
\pgfsys@defobject{currentmarker}{\pgfqpoint{0.000000in}{0.000000in}}{\pgfqpoint{0.000000in}{0.055556in}}{%
\pgfpathmoveto{\pgfqpoint{0.000000in}{0.000000in}}%
\pgfpathlineto{\pgfqpoint{0.000000in}{0.055556in}}%
\pgfusepath{stroke,fill}%
}%
\begin{pgfscope}%
\pgfsys@transformshift{1.770000in}{0.700000in}%
\pgfsys@useobject{currentmarker}{}%
\end{pgfscope}%
\end{pgfscope}%
\begin{pgfscope}%
\pgfsetbuttcap%
\pgfsetroundjoin%
\definecolor{currentfill}{rgb}{0.000000,0.000000,0.000000}%
\pgfsetfillcolor{currentfill}%
\pgfsetlinewidth{0.501875pt}%
\definecolor{currentstroke}{rgb}{0.000000,0.000000,0.000000}%
\pgfsetstrokecolor{currentstroke}%
\pgfsetdash{}{0pt}%
\pgfsys@defobject{currentmarker}{\pgfqpoint{0.000000in}{-0.055556in}}{\pgfqpoint{0.000000in}{0.000000in}}{%
\pgfpathmoveto{\pgfqpoint{0.000000in}{0.000000in}}%
\pgfpathlineto{\pgfqpoint{0.000000in}{-0.055556in}}%
\pgfusepath{stroke,fill}%
}%
\begin{pgfscope}%
\pgfsys@transformshift{1.770000in}{1.917391in}%
\pgfsys@useobject{currentmarker}{}%
\end{pgfscope}%
\end{pgfscope}%
\begin{pgfscope}%
\pgftext[left,bottom,x=1.665738in,y=0.537037in,rotate=0.000000]{{\rmfamily\fontsize{12.000000}{14.400000}\selectfont \(\displaystyle 0.5\)}}
%
\end{pgfscope}%
\begin{pgfscope}%
\pgfpathrectangle{\pgfqpoint{0.600000in}{0.700000in}}{\pgfqpoint{4.680000in}{1.217391in}} %
\pgfusepath{clip}%
\pgfsetbuttcap%
\pgfsetroundjoin%
\pgfsetlinewidth{0.501875pt}%
\definecolor{currentstroke}{rgb}{0.000000,0.000000,0.000000}%
\pgfsetstrokecolor{currentstroke}%
\pgfsetdash{{1.000000pt}{3.000000pt}}{0.000000pt}%
\pgfpathmoveto{\pgfqpoint{2.940000in}{0.700000in}}%
\pgfpathlineto{\pgfqpoint{2.940000in}{1.917391in}}%
\pgfusepath{stroke}%
\end{pgfscope}%
\begin{pgfscope}%
\pgfsetbuttcap%
\pgfsetroundjoin%
\definecolor{currentfill}{rgb}{0.000000,0.000000,0.000000}%
\pgfsetfillcolor{currentfill}%
\pgfsetlinewidth{0.501875pt}%
\definecolor{currentstroke}{rgb}{0.000000,0.000000,0.000000}%
\pgfsetstrokecolor{currentstroke}%
\pgfsetdash{}{0pt}%
\pgfsys@defobject{currentmarker}{\pgfqpoint{0.000000in}{0.000000in}}{\pgfqpoint{0.000000in}{0.055556in}}{%
\pgfpathmoveto{\pgfqpoint{0.000000in}{0.000000in}}%
\pgfpathlineto{\pgfqpoint{0.000000in}{0.055556in}}%
\pgfusepath{stroke,fill}%
}%
\begin{pgfscope}%
\pgfsys@transformshift{2.940000in}{0.700000in}%
\pgfsys@useobject{currentmarker}{}%
\end{pgfscope}%
\end{pgfscope}%
\begin{pgfscope}%
\pgfsetbuttcap%
\pgfsetroundjoin%
\definecolor{currentfill}{rgb}{0.000000,0.000000,0.000000}%
\pgfsetfillcolor{currentfill}%
\pgfsetlinewidth{0.501875pt}%
\definecolor{currentstroke}{rgb}{0.000000,0.000000,0.000000}%
\pgfsetstrokecolor{currentstroke}%
\pgfsetdash{}{0pt}%
\pgfsys@defobject{currentmarker}{\pgfqpoint{0.000000in}{-0.055556in}}{\pgfqpoint{0.000000in}{0.000000in}}{%
\pgfpathmoveto{\pgfqpoint{0.000000in}{0.000000in}}%
\pgfpathlineto{\pgfqpoint{0.000000in}{-0.055556in}}%
\pgfusepath{stroke,fill}%
}%
\begin{pgfscope}%
\pgfsys@transformshift{2.940000in}{1.917391in}%
\pgfsys@useobject{currentmarker}{}%
\end{pgfscope}%
\end{pgfscope}%
\begin{pgfscope}%
\pgftext[left,bottom,x=2.835738in,y=0.537037in,rotate=0.000000]{{\rmfamily\fontsize{12.000000}{14.400000}\selectfont \(\displaystyle 1.0\)}}
%
\end{pgfscope}%
\begin{pgfscope}%
\pgfpathrectangle{\pgfqpoint{0.600000in}{0.700000in}}{\pgfqpoint{4.680000in}{1.217391in}} %
\pgfusepath{clip}%
\pgfsetbuttcap%
\pgfsetroundjoin%
\pgfsetlinewidth{0.501875pt}%
\definecolor{currentstroke}{rgb}{0.000000,0.000000,0.000000}%
\pgfsetstrokecolor{currentstroke}%
\pgfsetdash{{1.000000pt}{3.000000pt}}{0.000000pt}%
\pgfpathmoveto{\pgfqpoint{4.110000in}{0.700000in}}%
\pgfpathlineto{\pgfqpoint{4.110000in}{1.917391in}}%
\pgfusepath{stroke}%
\end{pgfscope}%
\begin{pgfscope}%
\pgfsetbuttcap%
\pgfsetroundjoin%
\definecolor{currentfill}{rgb}{0.000000,0.000000,0.000000}%
\pgfsetfillcolor{currentfill}%
\pgfsetlinewidth{0.501875pt}%
\definecolor{currentstroke}{rgb}{0.000000,0.000000,0.000000}%
\pgfsetstrokecolor{currentstroke}%
\pgfsetdash{}{0pt}%
\pgfsys@defobject{currentmarker}{\pgfqpoint{0.000000in}{0.000000in}}{\pgfqpoint{0.000000in}{0.055556in}}{%
\pgfpathmoveto{\pgfqpoint{0.000000in}{0.000000in}}%
\pgfpathlineto{\pgfqpoint{0.000000in}{0.055556in}}%
\pgfusepath{stroke,fill}%
}%
\begin{pgfscope}%
\pgfsys@transformshift{4.110000in}{0.700000in}%
\pgfsys@useobject{currentmarker}{}%
\end{pgfscope}%
\end{pgfscope}%
\begin{pgfscope}%
\pgfsetbuttcap%
\pgfsetroundjoin%
\definecolor{currentfill}{rgb}{0.000000,0.000000,0.000000}%
\pgfsetfillcolor{currentfill}%
\pgfsetlinewidth{0.501875pt}%
\definecolor{currentstroke}{rgb}{0.000000,0.000000,0.000000}%
\pgfsetstrokecolor{currentstroke}%
\pgfsetdash{}{0pt}%
\pgfsys@defobject{currentmarker}{\pgfqpoint{0.000000in}{-0.055556in}}{\pgfqpoint{0.000000in}{0.000000in}}{%
\pgfpathmoveto{\pgfqpoint{0.000000in}{0.000000in}}%
\pgfpathlineto{\pgfqpoint{0.000000in}{-0.055556in}}%
\pgfusepath{stroke,fill}%
}%
\begin{pgfscope}%
\pgfsys@transformshift{4.110000in}{1.917391in}%
\pgfsys@useobject{currentmarker}{}%
\end{pgfscope}%
\end{pgfscope}%
\begin{pgfscope}%
\pgftext[left,bottom,x=4.005738in,y=0.537037in,rotate=0.000000]{{\rmfamily\fontsize{12.000000}{14.400000}\selectfont \(\displaystyle 1.5\)}}
%
\end{pgfscope}%
\begin{pgfscope}%
\pgfpathrectangle{\pgfqpoint{0.600000in}{0.700000in}}{\pgfqpoint{4.680000in}{1.217391in}} %
\pgfusepath{clip}%
\pgfsetbuttcap%
\pgfsetroundjoin%
\pgfsetlinewidth{0.501875pt}%
\definecolor{currentstroke}{rgb}{0.000000,0.000000,0.000000}%
\pgfsetstrokecolor{currentstroke}%
\pgfsetdash{{1.000000pt}{3.000000pt}}{0.000000pt}%
\pgfpathmoveto{\pgfqpoint{5.280000in}{0.700000in}}%
\pgfpathlineto{\pgfqpoint{5.280000in}{1.917391in}}%
\pgfusepath{stroke}%
\end{pgfscope}%
\begin{pgfscope}%
\pgfsetbuttcap%
\pgfsetroundjoin%
\definecolor{currentfill}{rgb}{0.000000,0.000000,0.000000}%
\pgfsetfillcolor{currentfill}%
\pgfsetlinewidth{0.501875pt}%
\definecolor{currentstroke}{rgb}{0.000000,0.000000,0.000000}%
\pgfsetstrokecolor{currentstroke}%
\pgfsetdash{}{0pt}%
\pgfsys@defobject{currentmarker}{\pgfqpoint{0.000000in}{0.000000in}}{\pgfqpoint{0.000000in}{0.055556in}}{%
\pgfpathmoveto{\pgfqpoint{0.000000in}{0.000000in}}%
\pgfpathlineto{\pgfqpoint{0.000000in}{0.055556in}}%
\pgfusepath{stroke,fill}%
}%
\begin{pgfscope}%
\pgfsys@transformshift{5.280000in}{0.700000in}%
\pgfsys@useobject{currentmarker}{}%
\end{pgfscope}%
\end{pgfscope}%
\begin{pgfscope}%
\pgfsetbuttcap%
\pgfsetroundjoin%
\definecolor{currentfill}{rgb}{0.000000,0.000000,0.000000}%
\pgfsetfillcolor{currentfill}%
\pgfsetlinewidth{0.501875pt}%
\definecolor{currentstroke}{rgb}{0.000000,0.000000,0.000000}%
\pgfsetstrokecolor{currentstroke}%
\pgfsetdash{}{0pt}%
\pgfsys@defobject{currentmarker}{\pgfqpoint{0.000000in}{-0.055556in}}{\pgfqpoint{0.000000in}{0.000000in}}{%
\pgfpathmoveto{\pgfqpoint{0.000000in}{0.000000in}}%
\pgfpathlineto{\pgfqpoint{0.000000in}{-0.055556in}}%
\pgfusepath{stroke,fill}%
}%
\begin{pgfscope}%
\pgfsys@transformshift{5.280000in}{1.917391in}%
\pgfsys@useobject{currentmarker}{}%
\end{pgfscope}%
\end{pgfscope}%
\begin{pgfscope}%
\pgftext[left,bottom,x=5.175738in,y=0.537037in,rotate=0.000000]{{\rmfamily\fontsize{12.000000}{14.400000}\selectfont \(\displaystyle 2.0\)}}
%
\end{pgfscope}%
\begin{pgfscope}%
\pgftext[left,bottom,x=2.184809in,y=0.319445in,rotate=0.000000]{{\rmfamily\fontsize{12.000000}{14.400000}\selectfont Distance along Beam}}
%
\end{pgfscope}%
\begin{pgfscope}%
\pgfpathrectangle{\pgfqpoint{0.600000in}{0.700000in}}{\pgfqpoint{4.680000in}{1.217391in}} %
\pgfusepath{clip}%
\pgfsetbuttcap%
\pgfsetroundjoin%
\pgfsetlinewidth{0.501875pt}%
\definecolor{currentstroke}{rgb}{0.000000,0.000000,0.000000}%
\pgfsetstrokecolor{currentstroke}%
\pgfsetdash{{1.000000pt}{3.000000pt}}{0.000000pt}%
\pgfpathmoveto{\pgfqpoint{0.600000in}{1.917391in}}%
\pgfpathlineto{\pgfqpoint{5.280000in}{1.917391in}}%
\pgfusepath{stroke}%
\end{pgfscope}%
\begin{pgfscope}%
\pgfsetbuttcap%
\pgfsetroundjoin%
\definecolor{currentfill}{rgb}{0.000000,0.000000,0.000000}%
\pgfsetfillcolor{currentfill}%
\pgfsetlinewidth{0.501875pt}%
\definecolor{currentstroke}{rgb}{0.000000,0.000000,0.000000}%
\pgfsetstrokecolor{currentstroke}%
\pgfsetdash{}{0pt}%
\pgfsys@defobject{currentmarker}{\pgfqpoint{0.000000in}{0.000000in}}{\pgfqpoint{0.055556in}{0.000000in}}{%
\pgfpathmoveto{\pgfqpoint{0.000000in}{0.000000in}}%
\pgfpathlineto{\pgfqpoint{0.055556in}{0.000000in}}%
\pgfusepath{stroke,fill}%
}%
\begin{pgfscope}%
\pgfsys@transformshift{0.600000in}{1.917391in}%
\pgfsys@useobject{currentmarker}{}%
\end{pgfscope}%
\end{pgfscope}%
\begin{pgfscope}%
\pgftext[left,bottom,x=0.462848in,y=1.863688in,rotate=0.000000]{{\rmfamily\fontsize{12.000000}{14.400000}\selectfont \(\displaystyle 0\)}}
%
\end{pgfscope}%
\begin{pgfscope}%
\pgfpathrectangle{\pgfqpoint{0.600000in}{0.700000in}}{\pgfqpoint{4.680000in}{1.217391in}} %
\pgfusepath{clip}%
\pgfsetbuttcap%
\pgfsetroundjoin%
\pgfsetlinewidth{0.501875pt}%
\definecolor{currentstroke}{rgb}{0.000000,0.000000,0.000000}%
\pgfsetstrokecolor{currentstroke}%
\pgfsetdash{{1.000000pt}{3.000000pt}}{0.000000pt}%
\pgfpathmoveto{\pgfqpoint{0.600000in}{1.613043in}}%
\pgfpathlineto{\pgfqpoint{5.280000in}{1.613043in}}%
\pgfusepath{stroke}%
\end{pgfscope}%
\begin{pgfscope}%
\pgfsetbuttcap%
\pgfsetroundjoin%
\definecolor{currentfill}{rgb}{0.000000,0.000000,0.000000}%
\pgfsetfillcolor{currentfill}%
\pgfsetlinewidth{0.501875pt}%
\definecolor{currentstroke}{rgb}{0.000000,0.000000,0.000000}%
\pgfsetstrokecolor{currentstroke}%
\pgfsetdash{}{0pt}%
\pgfsys@defobject{currentmarker}{\pgfqpoint{0.000000in}{0.000000in}}{\pgfqpoint{0.055556in}{0.000000in}}{%
\pgfpathmoveto{\pgfqpoint{0.000000in}{0.000000in}}%
\pgfpathlineto{\pgfqpoint{0.055556in}{0.000000in}}%
\pgfusepath{stroke,fill}%
}%
\begin{pgfscope}%
\pgfsys@transformshift{0.600000in}{1.613043in}%
\pgfsys@useobject{currentmarker}{}%
\end{pgfscope}%
\end{pgfscope}%
\begin{pgfscope}%
\pgftext[left,bottom,x=0.333218in,y=1.552395in,rotate=0.000000]{{\rmfamily\fontsize{12.000000}{14.400000}\selectfont \(\displaystyle -2\)}}
%
\end{pgfscope}%
\begin{pgfscope}%
\pgfpathrectangle{\pgfqpoint{0.600000in}{0.700000in}}{\pgfqpoint{4.680000in}{1.217391in}} %
\pgfusepath{clip}%
\pgfsetbuttcap%
\pgfsetroundjoin%
\pgfsetlinewidth{0.501875pt}%
\definecolor{currentstroke}{rgb}{0.000000,0.000000,0.000000}%
\pgfsetstrokecolor{currentstroke}%
\pgfsetdash{{1.000000pt}{3.000000pt}}{0.000000pt}%
\pgfpathmoveto{\pgfqpoint{0.600000in}{1.308696in}}%
\pgfpathlineto{\pgfqpoint{5.280000in}{1.308696in}}%
\pgfusepath{stroke}%
\end{pgfscope}%
\begin{pgfscope}%
\pgfsetbuttcap%
\pgfsetroundjoin%
\definecolor{currentfill}{rgb}{0.000000,0.000000,0.000000}%
\pgfsetfillcolor{currentfill}%
\pgfsetlinewidth{0.501875pt}%
\definecolor{currentstroke}{rgb}{0.000000,0.000000,0.000000}%
\pgfsetstrokecolor{currentstroke}%
\pgfsetdash{}{0pt}%
\pgfsys@defobject{currentmarker}{\pgfqpoint{0.000000in}{0.000000in}}{\pgfqpoint{0.055556in}{0.000000in}}{%
\pgfpathmoveto{\pgfqpoint{0.000000in}{0.000000in}}%
\pgfpathlineto{\pgfqpoint{0.055556in}{0.000000in}}%
\pgfusepath{stroke,fill}%
}%
\begin{pgfscope}%
\pgfsys@transformshift{0.600000in}{1.308696in}%
\pgfsys@useobject{currentmarker}{}%
\end{pgfscope}%
\end{pgfscope}%
\begin{pgfscope}%
\pgftext[left,bottom,x=0.333218in,y=1.248048in,rotate=0.000000]{{\rmfamily\fontsize{12.000000}{14.400000}\selectfont \(\displaystyle -4\)}}
%
\end{pgfscope}%
\begin{pgfscope}%
\pgfpathrectangle{\pgfqpoint{0.600000in}{0.700000in}}{\pgfqpoint{4.680000in}{1.217391in}} %
\pgfusepath{clip}%
\pgfsetbuttcap%
\pgfsetroundjoin%
\pgfsetlinewidth{0.501875pt}%
\definecolor{currentstroke}{rgb}{0.000000,0.000000,0.000000}%
\pgfsetstrokecolor{currentstroke}%
\pgfsetdash{{1.000000pt}{3.000000pt}}{0.000000pt}%
\pgfpathmoveto{\pgfqpoint{0.600000in}{1.004348in}}%
\pgfpathlineto{\pgfqpoint{5.280000in}{1.004348in}}%
\pgfusepath{stroke}%
\end{pgfscope}%
\begin{pgfscope}%
\pgfsetbuttcap%
\pgfsetroundjoin%
\definecolor{currentfill}{rgb}{0.000000,0.000000,0.000000}%
\pgfsetfillcolor{currentfill}%
\pgfsetlinewidth{0.501875pt}%
\definecolor{currentstroke}{rgb}{0.000000,0.000000,0.000000}%
\pgfsetstrokecolor{currentstroke}%
\pgfsetdash{}{0pt}%
\pgfsys@defobject{currentmarker}{\pgfqpoint{0.000000in}{0.000000in}}{\pgfqpoint{0.055556in}{0.000000in}}{%
\pgfpathmoveto{\pgfqpoint{0.000000in}{0.000000in}}%
\pgfpathlineto{\pgfqpoint{0.055556in}{0.000000in}}%
\pgfusepath{stroke,fill}%
}%
\begin{pgfscope}%
\pgfsys@transformshift{0.600000in}{1.004348in}%
\pgfsys@useobject{currentmarker}{}%
\end{pgfscope}%
\end{pgfscope}%
\begin{pgfscope}%
\pgftext[left,bottom,x=0.333218in,y=0.943700in,rotate=0.000000]{{\rmfamily\fontsize{12.000000}{14.400000}\selectfont \(\displaystyle -6\)}}
%
\end{pgfscope}%
\begin{pgfscope}%
\pgfpathrectangle{\pgfqpoint{0.600000in}{0.700000in}}{\pgfqpoint{4.680000in}{1.217391in}} %
\pgfusepath{clip}%
\pgfsetbuttcap%
\pgfsetroundjoin%
\pgfsetlinewidth{0.501875pt}%
\definecolor{currentstroke}{rgb}{0.000000,0.000000,0.000000}%
\pgfsetstrokecolor{currentstroke}%
\pgfsetdash{{1.000000pt}{3.000000pt}}{0.000000pt}%
\pgfpathmoveto{\pgfqpoint{0.600000in}{0.700000in}}%
\pgfpathlineto{\pgfqpoint{5.280000in}{0.700000in}}%
\pgfusepath{stroke}%
\end{pgfscope}%
\begin{pgfscope}%
\pgfsetbuttcap%
\pgfsetroundjoin%
\definecolor{currentfill}{rgb}{0.000000,0.000000,0.000000}%
\pgfsetfillcolor{currentfill}%
\pgfsetlinewidth{0.501875pt}%
\definecolor{currentstroke}{rgb}{0.000000,0.000000,0.000000}%
\pgfsetstrokecolor{currentstroke}%
\pgfsetdash{}{0pt}%
\pgfsys@defobject{currentmarker}{\pgfqpoint{0.000000in}{0.000000in}}{\pgfqpoint{0.055556in}{0.000000in}}{%
\pgfpathmoveto{\pgfqpoint{0.000000in}{0.000000in}}%
\pgfpathlineto{\pgfqpoint{0.055556in}{0.000000in}}%
\pgfusepath{stroke,fill}%
}%
\begin{pgfscope}%
\pgfsys@transformshift{0.600000in}{0.700000in}%
\pgfsys@useobject{currentmarker}{}%
\end{pgfscope}%
\end{pgfscope}%
\begin{pgfscope}%
\pgftext[left,bottom,x=0.333218in,y=0.639352in,rotate=0.000000]{{\rmfamily\fontsize{12.000000}{14.400000}\selectfont \(\displaystyle -8\)}}
%
\end{pgfscope}%
\begin{pgfscope}%
\pgftext[left,bottom,x=0.600000in,y=1.945169in,rotate=0.000000]{{\rmfamily\fontsize{12.000000}{14.400000}\selectfont \(\displaystyle \times10^{-5}\)}}
%
\end{pgfscope}%
\begin{pgfscope}%
\pgfsetrectcap%
\pgfsetroundjoin%
\pgfsetlinewidth{1.003750pt}%
\definecolor{currentstroke}{rgb}{0.000000,0.000000,0.000000}%
\pgfsetstrokecolor{currentstroke}%
\pgfsetdash{}{0pt}%
\pgfpathmoveto{\pgfqpoint{0.600000in}{1.917391in}}%
\pgfpathlineto{\pgfqpoint{5.280000in}{1.917391in}}%
\pgfusepath{stroke}%
\end{pgfscope}%
\begin{pgfscope}%
\pgfsetrectcap%
\pgfsetroundjoin%
\pgfsetlinewidth{1.003750pt}%
\definecolor{currentstroke}{rgb}{0.000000,0.000000,0.000000}%
\pgfsetstrokecolor{currentstroke}%
\pgfsetdash{}{0pt}%
\pgfpathmoveto{\pgfqpoint{5.280000in}{0.700000in}}%
\pgfpathlineto{\pgfqpoint{5.280000in}{1.917391in}}%
\pgfusepath{stroke}%
\end{pgfscope}%
\begin{pgfscope}%
\pgfsetrectcap%
\pgfsetroundjoin%
\pgfsetlinewidth{1.003750pt}%
\definecolor{currentstroke}{rgb}{0.000000,0.000000,0.000000}%
\pgfsetstrokecolor{currentstroke}%
\pgfsetdash{}{0pt}%
\pgfpathmoveto{\pgfqpoint{0.600000in}{0.700000in}}%
\pgfpathlineto{\pgfqpoint{5.280000in}{0.700000in}}%
\pgfusepath{stroke}%
\end{pgfscope}%
\begin{pgfscope}%
\pgfsetrectcap%
\pgfsetroundjoin%
\pgfsetlinewidth{1.003750pt}%
\definecolor{currentstroke}{rgb}{0.000000,0.000000,0.000000}%
\pgfsetstrokecolor{currentstroke}%
\pgfsetdash{}{0pt}%
\pgfpathmoveto{\pgfqpoint{0.600000in}{0.700000in}}%
\pgfpathlineto{\pgfqpoint{0.600000in}{1.917391in}}%
\pgfusepath{stroke}%
\end{pgfscope}%
\begin{pgfscope}%
\pgfpathrectangle{\pgfqpoint{0.600000in}{0.700000in}}{\pgfqpoint{4.680000in}{1.217391in}} %
\pgfusepath{clip}%
\pgfsetrectcap%
\pgfsetroundjoin%
\pgfsetlinewidth{1.003750pt}%
\definecolor{currentstroke}{rgb}{1.000000,0.000000,0.000000}%
\pgfsetstrokecolor{currentstroke}%
\pgfsetdash{}{0pt}%
\pgfpathmoveto{\pgfqpoint{0.600000in}{1.917391in}}%
\pgfpathlineto{\pgfqpoint{2.776200in}{1.917391in}}%
\pgfpathlineto{\pgfqpoint{2.823000in}{1.673913in}}%
\pgfpathlineto{\pgfqpoint{2.893200in}{0.943478in}}%
\pgfpathlineto{\pgfqpoint{2.940000in}{0.700000in}}%
\pgfpathlineto{\pgfqpoint{2.986800in}{0.943478in}}%
\pgfpathlineto{\pgfqpoint{3.033600in}{1.430435in}}%
\pgfpathlineto{\pgfqpoint{3.057000in}{1.552174in}}%
\pgfpathlineto{\pgfqpoint{3.080400in}{1.795652in}}%
\pgfpathlineto{\pgfqpoint{3.103800in}{1.917391in}}%
\pgfpathlineto{\pgfqpoint{5.280000in}{1.917391in}}%
\pgfpathlineto{\pgfqpoint{5.280000in}{1.917391in}}%
\pgfusepath{stroke}%
\end{pgfscope}%
\begin{pgfscope}%
\pgfsetbuttcap%
\pgfsetroundjoin%
\definecolor{currentfill}{rgb}{0.000000,0.000000,0.000000}%
\pgfsetfillcolor{currentfill}%
\pgfsetlinewidth{0.501875pt}%
\definecolor{currentstroke}{rgb}{0.000000,0.000000,0.000000}%
\pgfsetstrokecolor{currentstroke}%
\pgfsetdash{}{0pt}%
\pgfsys@defobject{currentmarker}{\pgfqpoint{-0.055556in}{0.000000in}}{\pgfqpoint{0.000000in}{0.000000in}}{%
\pgfpathmoveto{\pgfqpoint{0.000000in}{0.000000in}}%
\pgfpathlineto{\pgfqpoint{-0.055556in}{0.000000in}}%
\pgfusepath{stroke,fill}%
}%
\begin{pgfscope}%
\pgfsys@transformshift{5.280000in}{0.700000in}%
\pgfsys@useobject{currentmarker}{}%
\end{pgfscope}%
\end{pgfscope}%
\begin{pgfscope}%
\pgftext[left,bottom,x=5.335556in,y=0.646296in,rotate=0.000000]{{\rmfamily\fontsize{12.000000}{14.400000}\selectfont \(\displaystyle 0.00\)}}
%
\end{pgfscope}%
\begin{pgfscope}%
\pgfsetbuttcap%
\pgfsetroundjoin%
\definecolor{currentfill}{rgb}{0.000000,0.000000,0.000000}%
\pgfsetfillcolor{currentfill}%
\pgfsetlinewidth{0.501875pt}%
\definecolor{currentstroke}{rgb}{0.000000,0.000000,0.000000}%
\pgfsetstrokecolor{currentstroke}%
\pgfsetdash{}{0pt}%
\pgfsys@defobject{currentmarker}{\pgfqpoint{-0.055556in}{0.000000in}}{\pgfqpoint{0.000000in}{0.000000in}}{%
\pgfpathmoveto{\pgfqpoint{0.000000in}{0.000000in}}%
\pgfpathlineto{\pgfqpoint{-0.055556in}{0.000000in}}%
\pgfusepath{stroke,fill}%
}%
\begin{pgfscope}%
\pgfsys@transformshift{5.280000in}{1.004348in}%
\pgfsys@useobject{currentmarker}{}%
\end{pgfscope}%
\end{pgfscope}%
\begin{pgfscope}%
\pgftext[left,bottom,x=5.335556in,y=0.950644in,rotate=0.000000]{{\rmfamily\fontsize{12.000000}{14.400000}\selectfont \(\displaystyle 0.25\)}}
%
\end{pgfscope}%
\begin{pgfscope}%
\pgfsetbuttcap%
\pgfsetroundjoin%
\definecolor{currentfill}{rgb}{0.000000,0.000000,0.000000}%
\pgfsetfillcolor{currentfill}%
\pgfsetlinewidth{0.501875pt}%
\definecolor{currentstroke}{rgb}{0.000000,0.000000,0.000000}%
\pgfsetstrokecolor{currentstroke}%
\pgfsetdash{}{0pt}%
\pgfsys@defobject{currentmarker}{\pgfqpoint{-0.055556in}{0.000000in}}{\pgfqpoint{0.000000in}{0.000000in}}{%
\pgfpathmoveto{\pgfqpoint{0.000000in}{0.000000in}}%
\pgfpathlineto{\pgfqpoint{-0.055556in}{0.000000in}}%
\pgfusepath{stroke,fill}%
}%
\begin{pgfscope}%
\pgfsys@transformshift{5.280000in}{1.308696in}%
\pgfsys@useobject{currentmarker}{}%
\end{pgfscope}%
\end{pgfscope}%
\begin{pgfscope}%
\pgftext[left,bottom,x=5.335556in,y=1.254992in,rotate=0.000000]{{\rmfamily\fontsize{12.000000}{14.400000}\selectfont \(\displaystyle 0.50\)}}
%
\end{pgfscope}%
\begin{pgfscope}%
\pgfsetbuttcap%
\pgfsetroundjoin%
\definecolor{currentfill}{rgb}{0.000000,0.000000,0.000000}%
\pgfsetfillcolor{currentfill}%
\pgfsetlinewidth{0.501875pt}%
\definecolor{currentstroke}{rgb}{0.000000,0.000000,0.000000}%
\pgfsetstrokecolor{currentstroke}%
\pgfsetdash{}{0pt}%
\pgfsys@defobject{currentmarker}{\pgfqpoint{-0.055556in}{0.000000in}}{\pgfqpoint{0.000000in}{0.000000in}}{%
\pgfpathmoveto{\pgfqpoint{0.000000in}{0.000000in}}%
\pgfpathlineto{\pgfqpoint{-0.055556in}{0.000000in}}%
\pgfusepath{stroke,fill}%
}%
\begin{pgfscope}%
\pgfsys@transformshift{5.280000in}{1.613043in}%
\pgfsys@useobject{currentmarker}{}%
\end{pgfscope}%
\end{pgfscope}%
\begin{pgfscope}%
\pgftext[left,bottom,x=5.335556in,y=1.559340in,rotate=0.000000]{{\rmfamily\fontsize{12.000000}{14.400000}\selectfont \(\displaystyle 0.75\)}}
%
\end{pgfscope}%
\begin{pgfscope}%
\pgfsetbuttcap%
\pgfsetroundjoin%
\definecolor{currentfill}{rgb}{0.000000,0.000000,0.000000}%
\pgfsetfillcolor{currentfill}%
\pgfsetlinewidth{0.501875pt}%
\definecolor{currentstroke}{rgb}{0.000000,0.000000,0.000000}%
\pgfsetstrokecolor{currentstroke}%
\pgfsetdash{}{0pt}%
\pgfsys@defobject{currentmarker}{\pgfqpoint{-0.055556in}{0.000000in}}{\pgfqpoint{0.000000in}{0.000000in}}{%
\pgfpathmoveto{\pgfqpoint{0.000000in}{0.000000in}}%
\pgfpathlineto{\pgfqpoint{-0.055556in}{0.000000in}}%
\pgfusepath{stroke,fill}%
}%
\begin{pgfscope}%
\pgfsys@transformshift{5.280000in}{1.917391in}%
\pgfsys@useobject{currentmarker}{}%
\end{pgfscope}%
\end{pgfscope}%
\begin{pgfscope}%
\pgftext[left,bottom,x=5.335556in,y=1.863688in,rotate=0.000000]{{\rmfamily\fontsize{12.000000}{14.400000}\selectfont \(\displaystyle 1.00\)}}
%
\end{pgfscope}%
\begin{pgfscope}%
\pgftext[left,bottom,x=1.157083in,y=6.682223in,rotate=0.000000]{{\rmfamily\fontsize{14.000000}{16.800000}\selectfont Single Loadstep Brittle Failure Progression}}
%
\end{pgfscope}%
\begin{pgfscope}%
\pgftext[left,bottom,x=0.237870in,y=3.143030in,rotate=90.000000]{{\rmfamily\fontsize{12.000000}{14.400000}\selectfont Deflection}}
%
\end{pgfscope}%
\begin{pgfscope}%
\pgftext[left,bottom,x=5.937870in,y=3.049011in,rotate=90.000000]{{\rmfamily\fontsize{12.000000}{14.400000}\selectfont Node Health}}
%
\end{pgfscope}%
\end{pgfpicture}%
\makeatother%
\endgroup%
}
  \caption{A brittle beam with prescribed center displacement}
  \label{fig:brittleBeam}
\end{figure}

Unlike a local model, partial failure is observed at nodes near the plastic hinge, as pairs of bonds that straddle the hinge are broken.

\section{Conclusion}
\label{sec:Conclusion}

As far as we know, this the first peridynamic state based thin feature model, and results in accurate deformation results for simple beam tests.
The proposed damage model successfully reproduces the impact of nonlinear elasticity on deformation of a rectangular cantilever, and the framework is laid to allow application of the same model to I-beams.
It simplifies treatment of bending in beams, and is extensible to bending in plates.


\bibliographystyle{plain}
\bibliography{jogrady_bibdesk}

\end{document} 
\end
