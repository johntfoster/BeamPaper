%\documentclass[5p,twocolumn]{elsarticle}
\documentclass[preprint,review,12pt]{elsarticle}
\usepackage{geometry}                % See geometry.pdf to learn the layout options. There are lots.
\geometry{letterpaper}                   % ... or a4paper or a5paper or ... 
%\geometry{landscape}                % Activate for for rotated page geometry
%\usepackage[parfill]{parskip}    % Activate to begin paragraphs with an empty line rather than an indent
\usepackage{graphicx}
\usepackage{color}
\usepackage{epsfig}
\usepackage{amsmath}
\usepackage{amssymb}
\usepackage{mathtools}
\usepackage{epstopdf}
\usepackage{etoolbox}
\usepackage{tikz}
\usepackage{caption}
\usepackage{subcaption}
\usepackage{xfrac}
\usepackage{import}
\usepackage[section]{placeins}
\usepackage[inline]{trackchanges}

\usepackage{cleveref}
%\usepackage{autonum}


%\DeclareGraphicsRule{.tif}{png}{.png}{`convert #1 `dirname #1`/`basename #1 .tif`.png}

\graphicspath{{../}}
\graphicspath{{./images/}}
\newcommand{\diagrampath}{./diagrams/}
\newcommand{\plotpath}{./plots}

\newcommand{\mathbi}[1]{\mathit{\mathbf{#1}}}

\newcommand\vstate[3]{%
	\mathbf{\underline{#1}}%
	\ifstrempty{#2}{}{[#2]}%
	\ifstrempty{#3}{}{\langle #3 \rangle}}
\newcommand\sstate[3]{%
	\mathit{\underline{#1}}%
	\ifstrempty{#2}{}{[#2]}%
	\ifstrempty{#3}{}{\langle #3 \rangle}%
	}

\journal{International Journal of Solids and Structures}

\begin{document}

\begin{frontmatter}

\title{Peridynamic beams: A non-ordinary, state-based model}

\author[utsa]{James O'Grady\corref{cor1}}
\ead{jogrady@gmail.com}

\author[utsa]{John Foster\corref{cor2}}
\ead{john.foster@utsa.edu}

\cortext[cor1]{Principal Corresponding author}
\cortext[cor2]{Corresponding author}

\address[utsa]{The University of Texas at San Antonio, One UTSA Circle, San Antonio, TX 78249}

\begin{abstract}
This paper develops a new peridynamic state based model to represent the bending of an Euler-Bernoulli beam.
This model is non-ordinary and derived from the concept of a rotational spring between bonds.
While multiple peridynamic material models capture the behavior of solid materials, this is the first 1D state based peridynamic model to resist bending.
For sufficiently homogeneous and differentiable displacements, the model is shown to be equivalent to Eringen's nonlocal elasticity.
As the peridynamic horizon approaches 0, it reduces to the classical Euler-Bernoulli beam equations.
Simple test cases demonstrate the model's performance.
\end{abstract}

\begin{keyword}
peridynamics \sep non-ordinary model \sep non-local model
\end{keyword}

\end{frontmatter}

\section{Introduction}
\note{Re: Rev. I, Comment 3}\change{A goal of many mechanical engineering analyses is the prediction and description of material failure.}{James, this is just an example of how to use trackchanges, please remove this comment}
When processes such as fracture are modeled, the partial-differential equations of classical mechanics are ill-defined at the resulting discontinuities in displacement.  
A peridynamic formulation of continuum mechanics casts material behavior in terms of integral functions of displacement (as opposed to gradients of displacement), so that discontinuities can evolve naturally and require no special treatment.  
Various peridynamic material models capture the deformation \note{Another example, you have to go around citations} \change{behavior of 3-dimensional solid objects}{You have to stop short of citations} \cite{silling2007peridynamic, silling2005meshfree, gerstle2007peridynamic}, \add{and start more after} but would be very expensive to implement for a thin plate or beam, as the thru-thickness discretization requirement to properly capture resistance to bending would be prohibitively expensive in a computational setting for a long, slender structural object.  
Other peridynamic models capture tension and compression in 1D bars\cite{silling2003deformation} and 2D membranes\cite{silling2005peridynamic}, but these features do not resist transverse displacement.  
A recent paper by Taylor and Steigmann \cite{taylor2013two} reduces a bond based 3D plate to two dimensions with an integral through the plate's thickness. 
This creates a model that can represent thin structures and includes a bending term, but is used to simulate tension loading. 
The model is limited to the 3D bond-based Poisson ratio \(\nu=\sfrac{1}{4}\), though the same technique could be applied to a state-based model at the expense of complexity.

This paper presents a peridynamic equivalent to an Euler-Bernoulli beam, along with a methodology for representing non-uniform cross-sections, plastic behavior, and failure.  
This model is not derived from prior ordinary peridynamic models based on bond extension, but directly resists bending deformation.
In addition to directly modeling a beam in bending, the simple beam case lays the theoretical framework for more complex peridynamic beam, plate, and shell bending models.
Because many analyses of interest are partly or wholly comprised of these types of features, their development is an important addition to the capabilities of peridynamic analysis.
%
%The second section of this paper provides a brief introduction to peridynamics, including state based models.
%The third section presents the state based beam model and demonstrates equivalence to classical Euler-Bernoulli beam theory in the limit of vanishing nonlocality.
%The fourth section demonstrates the model's equivalence to Eringen's nonlocal elasticity for small peridynamic horizons, and compares other gradient models.
%The fifth section demonstrates the beam model with simple test cases.
The remainder of this introduction reviews other nonlocal work and provides a brief introduction to peridynamics, including state based models.
\Cref{sec:ModelDev} presents the state based beam model and demonstrates equivalence to classical Euler-Bernoulli beam theory in the limit of shrinking nonlocality.
\Cref{sec:EringenCompare} demonstrates the model's relationship to Eringen's nonlocal elasticity for small peridynamic horizons.
\Cref{sec:Numerical} demonstrates the beam model with simple numerical examples.
%
\subsection{Nonlocal Beam Models}
\label{sec:NLbeams}
%
Nonlocal elasticity generally allows for forces at a point that are dependent on the material configuration of an entire body, rather than the configuration at that point \cite{eringen1972nonlocal}.  While long-range forces are obvious at the molecular model, material at larger scales is conventionally modeled as though internal forces are local or contact forces \cite{kroner1967elasticity}.
The result of such approximation is accurate for deformations that are homogeneous, but introduces some inaccuracy for inhomogeneous deformations like the propagation of waves with short wavelengths.
One way to distinguish between homogeneous and inhomogeneous deformations is to incorporate higher-order gradients of deformation.
While stress in classical elasticity is a function of the (first) gradient of deformation, Eringen's formulation of a nonlocal modulus in \cite{eringen1983differential} approximates a weighted sum of the first and second order gradients.
This introduces a length scale to the model and has the effect of smearing out local deformation inhomogeneities over the surrounding material, while maintaining the conventional result for homogeneous deformations.

Previous work in the nonlocal mechanics of beams is motivated by the observed stiffening of nanoscale cantilevers.
Challamel and Wang demonstrate in \cite{Challamel2008small} that Eringen nonlocal elasticity cannot reproduce the scale stiffening, but that stiffening does result from other gradient-elastic models and models incorporating nonlocal curvature.
Because all of these models incorporate higher-order gradients of deformation, they impose stronger continuity requirements than classical elasticity, and are unsuitable for discontinuous displacements.
Because the gradients are evaluated locally, gradient models are called \textit{weakly nonlocal}.

%
\subsection{Peridynamics}
\label{sec:PDintro}
The term \textit{peridynamic} was coined by Silling in \cite{silling2000reformulation} and alludes to the fact that the force at a point is affected by nearby material configuration.
In contrast to gradient models, the peridynamic model is \textit{strongly nonlocal} and casts material behavior at a point as the \textit{integral equation} 
%
\begin{equation}
\label{eq:PDCoPV}
\rho(\mathbf{x})\ddot{\mathbf{u}}(\mathbf{x}) = \int_\Omega \mathbi{f}(\mathbf{x},\mathbf{q}) dV_\mathbf{q}  + \mathbf{b}(\mathbf{x}) \notag
\end{equation}
%
rather than the classical \textit{partial-differential equation}.
Instead of the divergence of stress, we have the integral of a ``force" functional $\mathbi{f}$ of the position vectors $\mathbf{x}$ and  $\mathbf{q}$ of a point within the body domain $\Omega$. 
This force functional may depend on $\mathbf{x}$, $\mathbf{q}$, their deformed positions, the original and deformed positions of other points in $\Omega$, history, etc.

Constitutive modeling of a wide variety of materials is accomplished by choosing the appropriate form for the force function.  While the simplest force functions recreate a one-parameter linear elastic solid material \cite{silling2000reformulation}, other force functions can be used to model nonlinear elasticity, plasticity, damage, and other behaviors \cite{silling2005peridynamic}.

To describe force functionals that incorporate the behavior of a totality of points in the nearby material (not just $\mathbf{x}$ and $\mathbf{q}$), we must introduce the concept of a peridynamic state.

Introduced by Silling et al.\ in 2007~\cite{silling2007peridynamic}, states are functions of the behavior of the continuum points surrounding each location.
The most common states are scalar-states and vector-states which are scalar and vector valued, respectively.
Unlike a second order tensor, which can only map vectors linearly to other vectors, vector-states can produce nonlinear or even discontinuous mappings.  Important properties of states are magnitude and direction, while important operations include the addition and decomposition of states, inner and tensor products, and the Fr\'{e}chet derivative of a function with respect to a state \cite{silling2007peridynamic}.

Conservation of linear momentum in the \textit{state-based} peridynamic formulation results in the equation of motion,
%
\begin{equation}
\label{eq:PDstateEoM}
\rho(\mathbf{x})\ddot{\mathbf{u}}(\mathbf{x}) = \int_\Omega (\vstate{T}{\mathbf{x}}{\mathbf{q}-\mathbf{x}}-\vstate{T}{\mathbf{q}}{\mathbf{x}-\mathbf{q}}) dV_\mathbf{q}  + \mathbf{b}(\mathbf{x}),\notag
\end{equation}
%
in which $\vstate{T}{\;}{\;}$ is a \textit{force vector-state} that maps the vector in angle brackets, $\langle \rangle$, originating at the point in square brackets, [ ], to a force vector acting on that point.
The deformed image of the vector $\mathbf{q-x}$) is defined as the \textit{deformation vector-state}, usually denoted $\vstate{Y}{}{}$ and formulated as shown in \cref{eq:PDdeformation}. 
%
\begin{equation}
\label{eq:PDdeformation}
\vstate{Y}{\mathbf{x}}{\mathbf{q}-\mathbf{x}} = (\mathbf{q}-\mathbf{x}) + (\mathbf{u}(\mathbf{q})-\mathbf{u}(\mathbf{x}))
\end{equation}
%

Just as stress and strain are work conjugate, so too are the force and deformation vector states for hyperelastic materials.
If the force state $\vstate{T}{}{}$ is always in the same direction as the deformation state $\vstate{Y}{}{}$, then the force exerted by a ``bond'' (i.e.\ the vector $\mathbf{q-x}$ between points is in the same direction as the deformed bond, and the model is called \textit{ordinary}.  
Models in which the bond-force interactions are not in the same direction as the deformed bond are called \textit{non-ordinary}.
Silling et al.\ demonstrate the possibility of such models in \cite{silling2010peridynamic}, but very little work has touched on their use.  Foster et al.\ \cite{foster2010viscoplasticity} and Warren et al.\ \cite{warren2009non} show that some correspondence models, which approximate the deformation gradient and use it to calculate bond forces, result in non-ordinary state-based constitutive models for finite deformations.
%
\FloatBarrier
%
\section{A non-ordinary beam model}
\label{sec:ModelDev}
Consider the material model illustrated in \cref{fig:SimpleBondpair} in which every bond-vector originating from a point is connected by a rotational spring to its opposite originating from that same point.
%
\begin{figure}[h]
\centering
\resizebox{0.6\linewidth}{!}{\subinputfrom{\diagrampath}{simpleBondPair.eps_tex}}
\caption{Illustration of a bond pair model that resists angular deformation}
\label{fig:SimpleBondpair}
\end{figure}
%
If we call the deformed angle between these bonds \(\theta\), and choose the potential energy of that spring to be \( w = \alpha [1 + \cos(\theta) ] \), we can recover the non-ordinary force state proposed by Silling in \cite{silling2007peridynamic}.
%
\begin{equation}
\label{eq:SillingForceNO}
\vstate{T}{}{\boldsymbol{\xi}} =\frac{-\alpha}{|\vstate{Y}{}{\boldsymbol{\xi}}|} \frac{\vstate{Y}{}{\boldsymbol{\xi}}}{|\vstate{Y}{}{\boldsymbol{\xi}}|} \times \left[\frac{\vstate{Y}{}{\boldsymbol{\xi}}}{|\vstate{Y}{}{\boldsymbol{\xi}}|} \times \frac{\vstate{Y}{}{-\boldsymbol{\xi}}}{|\vstate{Y}{}{-\boldsymbol{\xi}}|}\right]
\end{equation}
%
Though it looks complex, \cref{eq:SillingForceNO} indicates a bond force perpendicular to the deformed bond and in the plane containing both the deformed bond and its partner as illustrated in \cref{fig:Bondpair}. 
The force magnitude is proportional to the sine of the angle between the bonds divided by the length of the deformed bond. 
%
\begin{figure}[h]
\centering
\resizebox{0.6\linewidth}{!}{\subinputfrom{\diagrampath}{bondPair.eps_tex}}
\caption{Deformation and force vector states}
\label{fig:Bondpair}
\end{figure}
%
This response is consistent with the idea of a rotational spring between bonds as long as the change in angle is small. 
Because the potential energy and force states are functions of \textit{pairs} of peridynamic bonds, we will call this formulation a \textit{bond-pair model}. 
Other choices for the bond-pair potential function, such as $w = (\pi - \theta)^2$, are also possible, but result in more mathematically complex analysis.

\subsection{Energy Equivalence}
\label{sec:EnergyEq}
%
To determine an appropriate choice of $\alpha$, we desire our peridynamic model to have an equivalent strain energy density to a classical Euler-Bernoulli beam in the \emph{local limit}, i.e.\ when the nonlocal length scale vanishes.  We will begin with the assumptions from Euler beam theory: the length of the beam is much greater than thickness, vertical displacements are small, and rotations are small. For small vertical displacements (i.e.\ $\sin{\theta} \approx \theta$) we have
%
\begin{equation}
\theta(\vstate{Y}{}{\xi},\vstate{Y}{}{\mathbf{-\xi}}) \approx \pi-\frac{v(x+\xi)-2v(x)+v(x-\xi)}{\xi},
\label{eq:beamdtheta}
\end{equation}
%
where $v$ is the vertical displacement of material point.  Momentarily assuming that $v$ is continuous and using a Taylor series to expand the right-hand-side of eq.~(\ref{eq:beamdtheta})  
%
\begin{align}
\theta(\vstate{Y}{}{\xi},\vstate{Y}{}{\mathbf{-\xi}}) &\approx \pi-\xi \frac{\partial^2 v}{\partial x^2}+\mathcal{O}(\xi^3) \notag \\
&\approx  \pi-\xi \kappa +\mathcal{O}(\xi^3); 
\label{eq:beamdtheta2}
\end{align}
with
\begin{equation}
\kappa = \frac{\partial^2 v}{\partial x^2}.\notag
\end{equation}
%
Substituting eq.~(\ref{eq:beamdtheta2}) into the equation for the strain energy density of a single bond-pair,
%
\begin{align}
\label{eq:continuousBeamw}
w &= \omega(\xi) \alpha \left[1+\cos(\theta(\vstate{Y}{}{\xi},\vstate{Y}{}{\mathbf{-\xi}})) \right] \notag\\
&\approx \omega(\xi) \alpha\frac{\xi^2}{2}(\kappa)^2 +\mathcal{O}(\xi^4).\notag
\end{align}
%
If we use a weighting function \(\omega(\xi)=\omega(|\xi|)\) and assume that the $\omega$ plays the role of a localization kernel, i.e. $\omega = 0 \,\, \forall \,\, \xi > \delta$, the resulting strain energy density, $W$, for any material point in the peridynamic beam is
%
\begin{equation}
W = \frac{\alpha}{2}\kappa^2 \int_{-\delta}^\delta \omega(\xi)\xi^2 {\rm d}\xi + \mathcal{O}(\delta^5).\notag
\end{equation}
%
Equating $W$ with the classical Euler-Bernoulli beam strain-energy density, $\Omega$, and taking the limit as $\delta \to 0$ we can solve for $\alpha$
%
\begin{align}
    \lim_{\delta \to 0}  W &= \Omega, \notag \\
    \frac{\alpha}{2} m \kappa^2 &= \frac{EI}{2} \kappa^2, \notag \\
    \alpha &= \frac{EI}{m},
\label{eq:alpha}
\end{align}
%
with 
\begin{equation}
    m = \int_{-\delta}^{\delta} \omega(\xi) \xi^2 {\rm d}\xi \notag.
\end{equation}


\subsection{Weighting function and perfect plasticity}
\label{sec:WeightFunction}
The weighting function \(\omega(\xi)\) describes the relative contribution of each bond-pair, and can be defined according to physical or mathematical considerations. 
Consider a classical Euler-Bernoulli beam in bending with curvature \(\kappa\). 
Fibers running parallel to the neutral axis of the beam are stretched in proportion to their distance from the neutral axis, with strain \(\epsilon = y\kappa\). 
If the fibers are linearly elastic, then the axial stress at each location is \(\sigma = E\epsilon = Ey\kappa\), and the contribution to supported moment \(dM = \kappa E y^2 dA\). 
By comparing the formulations for the moments carried by the Euler beam in \cref{fig:EulerBending} and those of the bond-pair beam in \cref{fig:BPBending}, we see some definite parallels.
%
\begin{figure}[h]
\centering
\resizebox{0.5\linewidth}{!}{\subinputfrom{\diagrampath}{EulerBending.eps_tex}}
\caption{Euler beam moment contribution}
\label{fig:EulerBending}
\end{figure}
%%
\begin{figure}[h]
\centering
\subinputfrom{\diagrampath}{BondPairBending_edit.eps_tex}
\caption{Bond-pair moment contribution}
\label{fig:BPBending}
\end{figure}
%
\begin{align}
M_\text{E}&=\int_{-\frac{t}{2}}^{\frac{t}{2}} \sigma \; y \; dA &= \int_{-\frac{t}{2}}^{\frac{t}{2}} E \kappa \; y^2 \; b(y) dy\notag \\
%
M_\text{PD}&=\int_{-\delta}^{\delta} \vstate{T}{}{\xi}\; \xi \; d\xi &\notag \\
&= \int_{-\delta}^{\delta} \alpha \frac{\sin(\Delta\theta)}{|\xi|} \; \xi \; \omega(\xi) d\xi\: &\approx \int_{-\delta}^{\delta} \alpha \kappa |\xi| \; \omega(\xi) d\xi\notag
\end{align}
%
The term \(y\) is the distance from the beam's neutral axis and \(b(y)\) is the width of the beam at that distance from the neutral axis. 
The similarity between classical and peridynamic moment formulations suggests a possible formulation for the weighting function:
%
\begin{equation}
\label{eq:WeightFunction}
\omega(\xi) = |\xi| b\left(y\right) \quad \text{at} \quad y=\frac{\xi}{\delta} \frac{t}{2}
\end{equation}
%
\begin{figure}[h]
 \centering
  \subinputfrom{\diagrampath}{WeightProfile_Uniform.eps_tex}
\caption{Weight function for a beam of rectangular cross-section}
\label{fig:WeightProfileUniform}
\end{figure}

\begin{figure}
  \centering
  \subinputfrom{\diagrampath}{WeightProfile_Ibeam.eps_tex}
\caption{Weight function for an I-beam}
\label{fig:WeightProfileIbeam}
\end{figure}
%
This weight function analogizes the relative contributions of bond pairs of different lengths to the relative contributions of fibers at different distances from the centerline. 
An example for a rectangular beam is illustrated in \cref{fig:WeightProfileUniform}.
For an I beam with height \(h_\text{beam}\), width \(w_\text{beam}\), web height \(h_\text{web}\), and web width \(w_\text{web}\), substituting the beam profile
\begin{equation}
b(y) = 
  \begin{dcases}
    w_\text{web} & \text{if } |y| \leq \frac{h_\text{web}}{2} \\
    w_\text{beam} & \text{if } \frac{h_\text{web}}{2} < |y| \leq \frac{h_\text{beam}}{2} \\
    0 &\text{otherwise}
  \end{dcases}\notag
\end{equation}
into \cref{eq:WeightFunction} gives the weight function
\begin{equation}
\omega(\xi) = 
  \begin{dcases}
    w_\text{web} & \text{if } |\xi| \leq \delta\frac{h_\text{web}}{h_\text{beam}} \\
    w_\text{beam} & \text{if } \delta\frac{h_\text{web}}{h_\text{beam}} < |\xi| \leq \delta \\
    0 &\text{otherwise}
  \end{dcases}\notag
\end{equation}
and is ilustrated in \cref{fig:WeightProfileIbeam}.
    While this weighting function offers no advantages over a uniform weight function in the case of the linearly elastic beam, it offers a way to model advancing plasticity.

In a deformed elastic perfectly-plastic beam, axial fibers are still stretched in proportion to their distance from the neutral axis, but the relationship \(\sigma = E\epsilon = Ey\kappa\) only holds for \(|\epsilon| = |y\kappa| < \epsilon_c\). 
For greater stretches, the relationship becomes \(\sigma = \pm E\epsilon_c \). 
To model this behavior, consider a bond pair with similar behavior: for angular deformation less than some critical angle, the model behaves as previously described, but the magnitude of the force remains constant above a critical deformation
%
\[ 
|\vstate{T}{}{\xi}| = 
  \begin{cases}
    \alpha \frac{\sin(\theta(\vstate{Y}{}{\xi},\vstate{Y}{}{\mathbf{-\xi}}))}{|\vstate{Y}{}{\xi}|} & \quad \text{if } \theta < \theta_c\\
    \alpha \frac{\sin(\theta_c)}{|\vstate{Y}{}{\xi}|} & \quad \text{if } \theta \geq \theta_c\
  \end{cases}
\]
%
to determine the critical angle \(\theta_c\), we let the onset of plasticity in pairs of the longest bonds to coincide with the onset of plasticity in the fibers at the top and bottom surfaces of the classical beam. 
For small curvatures \(\Delta\theta = \xi\kappa\implies\Delta\theta_c = \frac{2\delta\epsilon_c}{t}\). 
For curvatures \(|\kappa| > \kappa_c=\frac{2\epsilon_c}{t}\), the radius within which bonds are in the elastic region is \(\delta_e = \delta \frac{\kappa_c}{\kappa}\), and parallels the distance from the beam centerline that fibers are in the elastic region \(y_e = \frac{t}{2} \frac{\kappa_c}{\kappa}\)
%
\begin{align}
  M_\text{classical} &= 2 \int_{0}^{y_e}E b(y)y^2 \kappa dy +2 \int_{y_e}^{\frac{t}{2}}E b(y) \epsilon_c y dy\notag \\
  M_\text{PD} &= 2 \int_{0}^{\delta_e}\alpha \omega(\xi) \xi^2 \kappa d\xi +2 \int_{\delta_e}^{\delta}\alpha \omega(\xi) \Delta\theta_c \xi d\xi \notag
\end{align}
%
Of course, as long as the force is independent of history, this model only represents a nonlinear elastic material. 
By keeping track of the plastic deformation \(\theta^p (\xi) = \theta-\theta_c\) of each bond-pair, and applying it as an offset, we can reproduce the hysteresis associated with elastic-perfectly-plastic deformation.
%
%
\section{Relation to Eringen Nonlocality}
\label{sec:EringenCompare}
%If we relax our homogeneity assumption somewhat, we recover from \cref{eq:beamdtheta} slightly more complex expressions for change in angle
If we keep an additional term from the Taylor series approximation of \cref{eq:beamdtheta}, we recover a slightly more complex expressions for change in angle
%
\begin{equation}
\label{eq:beamdthetaHOT}
%\theta(\vstate{Y}{}{\xi},\vstate{Y}{}{\mathbf{-\xi}}) \approx \pi-\xi \frac{\partial^2 y}{\partial x^2} -\frac{\xi^3}{12} \frac{\partial^4 y}{\partial x^4}  +\mathcal{O}(\xi^5)=  \pi-\xi \kappa-\frac{\xi^3}{12} \kappa''+\mathcal{O}(\xi^5)\notag
\theta(\vstate{Y}{}{\xi},\vstate{Y}{}{\mathbf{-\xi}}) \approx \arctan\left(\pi-\xi \frac{\partial^2 v}{\partial x^2} -\frac{\xi^3}{12} \frac{\partial^4 v}{\partial x^4}  +\mathcal{O}(\xi^5)\right)\notag
\end{equation}
%
and for the strain energy (again substituting \(\kappa = v''\) for readability),
%
%\begin{equation}
%W \approx \int_{-\delta}^\delta \omega(\xi)\alpha(\frac{\xi^2}{2}\kappa^2+\frac{\xi^4}{12}\kappa\kappa''-\frac{3\; \xi^4}{8}\kappa^4+\mathcal{O}(\xi^6)) d\xi .
%\end{equation}
%
%
\begin{equation}
%W \approx \int_{-\delta}^\delta \omega(\xi)\alpha(\frac{\xi^2}{2}\kappa^2+\frac{\xi^4}{12}\kappa\kappa''-\frac{\xi^4}{24}\kappa^4+\mathcal{O}(\xi^6)) d\xi .\notag
W \approx \int_{-\delta}^\delta \omega(\xi)\alpha(\frac{\xi^2}{2}\kappa^2+\frac{\xi^4}{12}\kappa\kappa''-\frac{3\; \xi^4}{8}\kappa^4+\mathcal{O}(\xi^6)) d\xi.\notag
\end{equation}
%
As the horizon \(\delta\) becomes small, higher-order \(\xi\) terms become relatively less important, and \(\xi^4\kappa^4\) is dominated by \(\xi^2\kappa^2\) for large \(\kappa\) and by \(\xi^4\kappa\kappa''\) for small \(\kappa\).
The remaining terms can be rearranged,
\begin{align}
W &\approx \int_{-\delta}^\delta \omega(\xi)\alpha \frac{\xi^2}{2}\kappa(\kappa + \frac{\xi^2}{6}\kappa'') d\xi, \notag
\end{align}
in a manner strongly suggesting an alternative bending resistance term.
We can picture a bending resistance based on the bond length and proportional to the nonlocal curvature  \(\bar{\kappa}=(\kappa + \frac{\xi^2}{6}\kappa'')\), so that 
%
\begin{align}
\label{eq:NLbending}
\bar{\kappa}&=(\kappa + \frac{\xi^2}{6}\kappa'') \implies  \\
W &\approx \int_{-\delta}^\delta \omega(\xi)\alpha \frac{\xi^2}{2}\kappa\bar{\kappa}d\xi .\notag
\end{align}
%  
The same analysis can be taken further to obtain higher-order energy terms with even powers of \(\xi\) and even order derivatives of \(\kappa\). 
Not all of these higher-order terms can be separated into the product of a local curvature and nonlocal bending resistance.

Eringen's model for nonlocal elasticity in \cite{eringen1983differential} begins with a nonlocal modulus (denoted here as \(K(|\mathbf{x}'-\mathbf{x}|,\tau)\)) that relates the nonlocal stress \(\mathbf{t}\) at a point to the classical (local) stress \(\boldsymbol{\sigma}\) in the nearby material through the integral
\begin{equation}
\mathbf{t} = \int_\mathbf{V} K(|\mathbf{x}'-\mathbf{x}|,\tau)\sigma(\mathbf{x}')dv(\mathbf{x}').\notag
\end{equation}
In the local limit these relationships take the form of higher-order gradients.
Using a 1-dimensional decaying exponential nonlocal modulus \(K(|x|,\tau)=\frac{1}{l\tau}e^{-\frac{|x|}{l\tau}}\)results in a relationship between \(t\) and \(\sigma\) 
\begin{align}
\left(1-\tau^2l^2\frac{\partial^2}{\partial x^2}\right)t&=\sigma,\notag
\end{align}
in which \(\tau^2l^2\) is a scale-based material parameter.
For well-behaved \(t\) and \(\sigma\) and small values of \(\sigma''''\) and \(\tau^2l^2\), we can see that this relationship could be reformulated as
\begin{equation}
\label{eq:NLstress}
t=\left(1+\tau^2l^2\frac{\partial^2}{\partial x^2}\right)\sigma.\notag
\end{equation}
If we consider the results of the previous section and let \(dM = y\sigma dA\) and \(\sigma = Ey\kappa\), the contribution to moment resulting from Eringen's nonlocal elasticity in a fiber at \(y\)
\begin{equation}
\label{eq:EringenMoment}
E y^2 (\kappa+\tau^2l^2\kappa''), \\
\end{equation}
and the resulting strain energy
\begin{equation}
\label{eq:EringenEnergy}
\int_{-\frac{t}{2}}^{\frac{t}{2}} b(y) E \frac{y^2}{2} \kappa (\kappa+\tau^2l^2\kappa'')  dy,\notag
\end{equation}
bear a striking resemblance to \cref{eq:NLbending}.
In fact, by carefully choosing peridynamic parameter values, the results can be made identical.
For a rectangular beam of width \(b\) and thickness \(t\), choosing 
\begin{equation}
\omega(\xi) = |\xi|b ;\qquad \delta = \tau l \sqrt{3} ;\qquad \alpha = \frac{E b t^3}{54 \tau^4 l^4}\notag
\end{equation}
results in
\begin{equation}
W \approx E b \frac{t^3}{12} \frac{\kappa}{2}(\kappa+\tau^2 l^2 \kappa''), \notag
\end{equation}
the same result for both models.

The similarity between \cref{eq:NLbending,eq:EringenMoment} is not accidental; Eringen's gradient elasticity is the solution to the integral formulation of the nonlocal stress integral equation just as the peridynamic energy is an integral function of nonlocal displacements.
It is therefore unsurprising that, like Eringen's nonlocal elasticity\cite{Challamel2008small}, this peridynamic bending model fails to predict the stiffening associated with nanoscale cantilevers.
Instead, the advantage of peridynamic models is their natural handling of discontinuities.

\section{Numerical Simulation}
\label{sec:Numerical}
\subsection{Discretized Model}
\label{sec:Discretized}
Discretizing the bond-pair model is primarily matter of exchanging integrals for sums. 
%
\begin{align}
%\label{eq:discreteBeamw}
w &= \alpha \left[1+\cos(\theta(\vstate{Y}{}{\xi},\vstate{Y}{}{\mathbf{-\xi}})) \right] \notag \\
&\approx \frac{\alpha}{2}\left(\frac{v(x+\xi)-2v(x)+v(x-\xi)}{\xi}\right)^2 \notag
\end{align}
%
\begin{align}
\label{eq:discretebeam}
\alpha &= \frac{c\; \Delta x}{m} ;\; c= EI ;\; m=\sum_{i=1}^n \omega(\xi_i)\xi_i^2 \implies \nonumber \\
W&=\Delta x \sum_{i=1}^n \frac{EI}{2}\left(\frac{v(x+\xi_i)-2v(x)+v(x-\xi_i)}{\xi_i}\right)^2\notag
\end{align}
%
Discretization of the original model results in the equation of motion
\begin{align}
\rho(\mathbf{x})\mathbf{\ddot{u}}(\mathbf{x}) = \mathbf{f}(\mathbf{x})&+\sum_i \omega(\xi_i)\left\{\frac{\alpha(\mathbf{x})}{|\mathbf{p}_i |}\frac{\mathbf{p}_i}{|\mathbf{p}_i |}\times \left[ \frac{\mathbf{p}_i}{|\mathbf{p}_i |}\times \frac{\mathbf{q}_i}{|\mathbf{q}_i |}\right] \right. \notag \\
& \left. -\frac{\alpha(\mathbf{x}+\boldsymbol{\xi}_i)}{|\mathbf{p}_i |}\frac{(-\mathbf{p}_i)}{|\mathbf{p}_i |}\times\left[\frac{(-\mathbf{p}_i)}{|\mathbf{p}_i |}\times \frac{\mathbf{r}_i}{|\mathbf{r}_i |} \right] \right\} \notag
\end{align}
with
\begin{align}
\mathbf{p}_i &= \boldsymbol{\xi}_i+\mathbf{u}(\mathbf{x}+\boldsymbol{\xi}_i)-\mathbf{u}(\mathbf{x});\notag\\
\mathbf{q}_i &= -\boldsymbol{\xi}_i+\mathbf{u}(\mathbf{x}-\boldsymbol{\xi}_i)-\mathbf{u}(\mathbf{x});\notag\\
\mathbf{r}_i &= \boldsymbol{\xi}_i+\mathbf{u}(\mathbf{x}+2\boldsymbol{\xi}_i)-\mathbf{u}(\mathbf{x}+\boldsymbol{\xi}_i).\notag
\end{align}
and for small displacements and rotations in a uniform beam,
\begin{align}
&\rho(x)\ddot{v}(x) = f(x)+\alpha \sum_i 2\omega(\xi_i)\bigg( \notag\\
&\left. \frac{v(x-2\xi_i)-4v(x-\xi_i)+6v(x)-4v(x+\xi_i)+v(x+2\xi_i)}{\xi_i^2}\right) \notag
\end{align}
It is worth noting the similarity between this expression and a finite-difference fourth derivative of displacement, a result expected from Euler beam theory.
This discretization requires that nodes be evenly spaced along the entire beam, otherwise the displacement \(v(x-\xi_i)\) is ill-defined. 
For this reason, the discretization does not allow for areas of higher and lower ``resolution''. 
%
\subsection{Numerical Method}
\label{sec:NumMethod}
Model behavior is evaluated by implementing the discretized equation of motion.
The case of a beam simply supported at both ends is chosen for simplicity in both evaluation and comparison.
Boundary conditions such as clamped supports and applied moments require careful treatment to ensure both meaningful results and ease of computation.

Deformations for quasistatic loading are computed by implicitly solving the zero-acceleration discrete equation of motion with a Newton's Method solver.
%
\subsection{Results}
\label{sec:Results}
The simplest test case for this model is a linear-elastic beam with a square profile.
For comparison, equivalent models are created and analyzed in Abaqus 6.12 to verify simple cases.
Even a coarse discretization successfully reproduces the elastically deformed beam shape in \cref{fig:elastic_g2000}.

\begin{figure}[h]
  \centering
  \resizebox{0.5\linewidth}{!}{%% Creator: Matplotlib, PGF backend
%%
%% To include the figure in your LaTeX document, write
%%   \input{<filename>.pgf}
%%
%% Make sure the required packages are loaded in your preamble
%%   \usepackage{pgf}
%%
%% Figures using additional raster images can only be included by \input if
%% they are in the same directory as the main LaTeX file. For loading figures
%% from other directories you can use the `import` package
%%   \usepackage{import}
%% and then include the figures with
%%   \import{<path to file>}{<filename>.pgf}
%%
%% Matplotlib used the following preamble
%%
\begingroup%
\makeatletter%
\begin{pgfpicture}%
\pgfpathrectangle{\pgfpointorigin}{\pgfqpoint{6.000000in}{7.000000in}}%
\pgfusepath{use as bounding box}%
\begin{pgfscope}%
\pgfsetrectcap%
\pgfsetroundjoin%
\definecolor{currentfill}{rgb}{1.000000,1.000000,1.000000}%
\pgfsetfillcolor{currentfill}%
\pgfsetlinewidth{0.000000pt}%
\definecolor{currentstroke}{rgb}{1.000000,1.000000,1.000000}%
\pgfsetstrokecolor{currentstroke}%
\pgfsetdash{}{0pt}%
\pgfpathmoveto{\pgfqpoint{0.000000in}{0.000000in}}%
\pgfpathlineto{\pgfqpoint{6.000000in}{0.000000in}}%
\pgfpathlineto{\pgfqpoint{6.000000in}{7.000000in}}%
\pgfpathlineto{\pgfqpoint{0.000000in}{7.000000in}}%
\pgfpathclose%
\pgfusepath{fill}%
\end{pgfscope}%
\begin{pgfscope}%
\pgfsetrectcap%
\pgfsetroundjoin%
\definecolor{currentfill}{rgb}{1.000000,1.000000,1.000000}%
\pgfsetfillcolor{currentfill}%
\pgfsetlinewidth{0.000000pt}%
\definecolor{currentstroke}{rgb}{0.000000,0.000000,0.000000}%
\pgfsetstrokecolor{currentstroke}%
\pgfsetdash{}{0pt}%
\pgfpathmoveto{\pgfqpoint{0.750000in}{0.700000in}}%
\pgfpathlineto{\pgfqpoint{5.400000in}{0.700000in}}%
\pgfpathlineto{\pgfqpoint{5.400000in}{6.300000in}}%
\pgfpathlineto{\pgfqpoint{0.750000in}{6.300000in}}%
\pgfpathclose%
\pgfusepath{fill}%
\end{pgfscope}%
\begin{pgfscope}%
\pgfpathrectangle{\pgfqpoint{0.750000in}{0.700000in}}{\pgfqpoint{4.650000in}{5.600000in}} %
\pgfusepath{clip}%
\pgfsetrectcap%
\pgfsetroundjoin%
\pgfsetlinewidth{1.003750pt}%
\definecolor{currentstroke}{rgb}{0.000000,0.000000,1.000000}%
\pgfsetstrokecolor{currentstroke}%
\pgfsetdash{}{0pt}%
\pgfpathmoveto{\pgfqpoint{0.750000in}{6.300000in}}%
\pgfpathlineto{\pgfqpoint{0.982500in}{5.525959in}}%
\pgfpathlineto{\pgfqpoint{1.105725in}{5.123309in}}%
\pgfpathlineto{\pgfqpoint{1.215000in}{4.773893in}}%
\pgfpathlineto{\pgfqpoint{1.310325in}{4.476584in}}%
\pgfpathlineto{\pgfqpoint{1.396350in}{4.215381in}}%
\pgfpathlineto{\pgfqpoint{1.477725in}{3.975319in}}%
\pgfpathlineto{\pgfqpoint{1.554450in}{3.755869in}}%
\pgfpathlineto{\pgfqpoint{1.626525in}{3.556331in}}%
\pgfpathlineto{\pgfqpoint{1.696275in}{3.369758in}}%
\pgfpathlineto{\pgfqpoint{1.763700in}{3.195873in}}%
\pgfpathlineto{\pgfqpoint{1.826475in}{3.039983in}}%
\pgfpathlineto{\pgfqpoint{1.886925in}{2.895583in}}%
\pgfpathlineto{\pgfqpoint{1.947375in}{2.757015in}}%
\pgfpathlineto{\pgfqpoint{2.003175in}{2.634450in}}%
\pgfpathlineto{\pgfqpoint{2.058975in}{2.517177in}}%
\pgfpathlineto{\pgfqpoint{2.112450in}{2.409890in}}%
\pgfpathlineto{\pgfqpoint{2.165925in}{2.307718in}}%
\pgfpathlineto{\pgfqpoint{2.217075in}{2.214879in}}%
\pgfpathlineto{\pgfqpoint{2.265900in}{2.130809in}}%
\pgfpathlineto{\pgfqpoint{2.312400in}{2.054943in}}%
\pgfpathlineto{\pgfqpoint{2.358900in}{1.983245in}}%
\pgfpathlineto{\pgfqpoint{2.403075in}{1.919041in}}%
\pgfpathlineto{\pgfqpoint{2.447250in}{1.858696in}}%
\pgfpathlineto{\pgfqpoint{2.491425in}{1.802257in}}%
\pgfpathlineto{\pgfqpoint{2.533275in}{1.752427in}}%
\pgfpathlineto{\pgfqpoint{2.575125in}{1.706175in}}%
\pgfpathlineto{\pgfqpoint{2.614650in}{1.665799in}}%
\pgfpathlineto{\pgfqpoint{2.654175in}{1.628667in}}%
\pgfpathlineto{\pgfqpoint{2.689050in}{1.598613in}}%
\pgfpathlineto{\pgfqpoint{2.726250in}{1.569353in}}%
\pgfpathlineto{\pgfqpoint{2.763450in}{1.543033in}}%
\pgfpathlineto{\pgfqpoint{2.800650in}{1.519653in}}%
\pgfpathlineto{\pgfqpoint{2.828550in}{1.504020in}}%
\pgfpathlineto{\pgfqpoint{2.865750in}{1.485773in}}%
\pgfpathlineto{\pgfqpoint{2.902950in}{1.470513in}}%
\pgfpathlineto{\pgfqpoint{2.940150in}{1.458193in}}%
\pgfpathlineto{\pgfqpoint{2.977350in}{1.448860in}}%
\pgfpathlineto{\pgfqpoint{3.014550in}{1.442513in}}%
\pgfpathlineto{\pgfqpoint{3.049425in}{1.439293in}}%
\pgfpathlineto{\pgfqpoint{3.077325in}{1.438593in}}%
\pgfpathlineto{\pgfqpoint{3.105225in}{1.439573in}}%
\pgfpathlineto{\pgfqpoint{3.133125in}{1.442233in}}%
\pgfpathlineto{\pgfqpoint{3.161025in}{1.446573in}}%
\pgfpathlineto{\pgfqpoint{3.191250in}{1.453153in}}%
\pgfpathlineto{\pgfqpoint{3.228450in}{1.463980in}}%
\pgfpathlineto{\pgfqpoint{3.265650in}{1.477793in}}%
\pgfpathlineto{\pgfqpoint{3.293550in}{1.490067in}}%
\pgfpathlineto{\pgfqpoint{3.323775in}{1.505233in}}%
\pgfpathlineto{\pgfqpoint{3.360975in}{1.526607in}}%
\pgfpathlineto{\pgfqpoint{3.398175in}{1.550920in}}%
\pgfpathlineto{\pgfqpoint{3.437700in}{1.579993in}}%
\pgfpathlineto{\pgfqpoint{3.477225in}{1.612333in}}%
\pgfpathlineto{\pgfqpoint{3.514425in}{1.645733in}}%
\pgfpathlineto{\pgfqpoint{3.553950in}{1.684396in}}%
\pgfpathlineto{\pgfqpoint{3.593475in}{1.726289in}}%
\pgfpathlineto{\pgfqpoint{3.633000in}{1.771383in}}%
\pgfpathlineto{\pgfqpoint{3.674850in}{1.822595in}}%
\pgfpathlineto{\pgfqpoint{3.716700in}{1.877335in}}%
\pgfpathlineto{\pgfqpoint{3.758550in}{1.935565in}}%
\pgfpathlineto{\pgfqpoint{3.802725in}{2.000777in}}%
\pgfpathlineto{\pgfqpoint{3.849225in}{2.073521in}}%
\pgfpathlineto{\pgfqpoint{3.895725in}{2.150419in}}%
\pgfpathlineto{\pgfqpoint{3.944550in}{2.235553in}}%
\pgfpathlineto{\pgfqpoint{3.995700in}{2.329488in}}%
\pgfpathlineto{\pgfqpoint{4.046850in}{2.428188in}}%
\pgfpathlineto{\pgfqpoint{4.100325in}{2.536352in}}%
\pgfpathlineto{\pgfqpoint{4.153800in}{2.649486in}}%
\pgfpathlineto{\pgfqpoint{4.209600in}{2.772700in}}%
\pgfpathlineto{\pgfqpoint{4.267725in}{2.906489in}}%
\pgfpathlineto{\pgfqpoint{4.328175in}{3.051328in}}%
\pgfpathlineto{\pgfqpoint{4.388625in}{3.201758in}}%
\pgfpathlineto{\pgfqpoint{4.453725in}{3.369758in}}%
\pgfpathlineto{\pgfqpoint{4.521150in}{3.550008in}}%
\pgfpathlineto{\pgfqpoint{4.590900in}{3.742797in}}%
\pgfpathlineto{\pgfqpoint{4.665300in}{3.955084in}}%
\pgfpathlineto{\pgfqpoint{4.742025in}{4.180652in}}%
\pgfpathlineto{\pgfqpoint{4.825725in}{4.433744in}}%
\pgfpathlineto{\pgfqpoint{4.914075in}{4.707981in}}%
\pgfpathlineto{\pgfqpoint{5.014050in}{5.025851in}}%
\pgfpathlineto{\pgfqpoint{5.125650in}{5.388357in}}%
\pgfpathlineto{\pgfqpoint{5.265150in}{5.849603in}}%
\pgfpathlineto{\pgfqpoint{5.400000in}{6.300000in}}%
\pgfpathlineto{\pgfqpoint{5.400000in}{6.300000in}}%
\pgfusepath{stroke}%
\end{pgfscope}%
\begin{pgfscope}%
\pgfpathrectangle{\pgfqpoint{0.750000in}{0.700000in}}{\pgfqpoint{4.650000in}{5.600000in}} %
\pgfusepath{clip}%
\pgfsetbuttcap%
\pgfsetroundjoin%
\definecolor{currentfill}{rgb}{0.000000,0.000000,1.000000}%
\pgfsetfillcolor{currentfill}%
\pgfsetlinewidth{0.501875pt}%
\definecolor{currentstroke}{rgb}{0.000000,0.000000,0.000000}%
\pgfsetstrokecolor{currentstroke}%
\pgfsetdash{}{0pt}%
\pgfsys@defobject{currentmarker}{\pgfqpoint{-0.041667in}{-0.041667in}}{\pgfqpoint{0.041667in}{0.041667in}}{%
\pgfpathmoveto{\pgfqpoint{0.000000in}{-0.041667in}}%
\pgfpathcurveto{\pgfqpoint{0.011050in}{-0.041667in}}{\pgfqpoint{0.021649in}{-0.037276in}}{\pgfqpoint{0.029463in}{-0.029463in}}%
\pgfpathcurveto{\pgfqpoint{0.037276in}{-0.021649in}}{\pgfqpoint{0.041667in}{-0.011050in}}{\pgfqpoint{0.041667in}{0.000000in}}%
\pgfpathcurveto{\pgfqpoint{0.041667in}{0.011050in}}{\pgfqpoint{0.037276in}{0.021649in}}{\pgfqpoint{0.029463in}{0.029463in}}%
\pgfpathcurveto{\pgfqpoint{0.021649in}{0.037276in}}{\pgfqpoint{0.011050in}{0.041667in}}{\pgfqpoint{0.000000in}{0.041667in}}%
\pgfpathcurveto{\pgfqpoint{-0.011050in}{0.041667in}}{\pgfqpoint{-0.021649in}{0.037276in}}{\pgfqpoint{-0.029463in}{0.029463in}}%
\pgfpathcurveto{\pgfqpoint{-0.037276in}{0.021649in}}{\pgfqpoint{-0.041667in}{0.011050in}}{\pgfqpoint{-0.041667in}{0.000000in}}%
\pgfpathcurveto{\pgfqpoint{-0.041667in}{-0.011050in}}{\pgfqpoint{-0.037276in}{-0.021649in}}{\pgfqpoint{-0.029463in}{-0.029463in}}%
\pgfpathcurveto{\pgfqpoint{-0.021649in}{-0.037276in}}{\pgfqpoint{-0.011050in}{-0.041667in}}{\pgfqpoint{0.000000in}{-0.041667in}}%
\pgfpathclose%
\pgfusepath{stroke,fill}%
}%
\begin{pgfscope}%
\pgfsys@transformshift{0.982500in}{5.525959in}%
\pgfsys@useobject{currentmarker}{}%
\end{pgfscope}%
\begin{pgfscope}%
\pgfsys@transformshift{1.447500in}{4.063640in}%
\pgfsys@useobject{currentmarker}{}%
\end{pgfscope}%
\begin{pgfscope}%
\pgfsys@transformshift{1.912500in}{2.836237in}%
\pgfsys@useobject{currentmarker}{}%
\end{pgfscope}%
\begin{pgfscope}%
\pgfsys@transformshift{2.377500in}{1.955744in}%
\pgfsys@useobject{currentmarker}{}%
\end{pgfscope}%
\begin{pgfscope}%
\pgfsys@transformshift{2.842500in}{1.496833in}%
\pgfsys@useobject{currentmarker}{}%
\end{pgfscope}%
\begin{pgfscope}%
\pgfsys@transformshift{3.307500in}{1.496833in}%
\pgfsys@useobject{currentmarker}{}%
\end{pgfscope}%
\begin{pgfscope}%
\pgfsys@transformshift{3.772500in}{1.955744in}%
\pgfsys@useobject{currentmarker}{}%
\end{pgfscope}%
\begin{pgfscope}%
\pgfsys@transformshift{4.237500in}{2.836237in}%
\pgfsys@useobject{currentmarker}{}%
\end{pgfscope}%
\begin{pgfscope}%
\pgfsys@transformshift{4.702500in}{4.063640in}%
\pgfsys@useobject{currentmarker}{}%
\end{pgfscope}%
\begin{pgfscope}%
\pgfsys@transformshift{5.167500in}{5.525959in}%
\pgfsys@useobject{currentmarker}{}%
\end{pgfscope}%
\end{pgfscope}%
\begin{pgfscope}%
\pgfpathrectangle{\pgfqpoint{0.750000in}{0.700000in}}{\pgfqpoint{4.650000in}{5.600000in}} %
\pgfusepath{clip}%
\pgfsetrectcap%
\pgfsetroundjoin%
\pgfsetlinewidth{1.003750pt}%
\definecolor{currentstroke}{rgb}{0.000000,0.500000,0.000000}%
\pgfsetstrokecolor{currentstroke}%
\pgfsetdash{}{0pt}%
\pgfpathmoveto{\pgfqpoint{0.843000in}{5.999378in}}%
\pgfpathlineto{\pgfqpoint{0.936000in}{5.699350in}}%
\pgfpathlineto{\pgfqpoint{1.029000in}{5.400867in}}%
\pgfpathlineto{\pgfqpoint{1.122000in}{5.104598in}}%
\pgfpathlineto{\pgfqpoint{1.215000in}{4.811703in}}%
\pgfpathlineto{\pgfqpoint{1.308000in}{4.529093in}}%
\pgfpathlineto{\pgfqpoint{1.401000in}{4.252699in}}%
\pgfpathlineto{\pgfqpoint{1.494000in}{3.984495in}}%
\pgfpathlineto{\pgfqpoint{1.587000in}{3.725948in}}%
\pgfpathlineto{\pgfqpoint{1.680000in}{3.478018in}}%
\pgfpathlineto{\pgfqpoint{1.773000in}{3.241847in}}%
\pgfpathlineto{\pgfqpoint{1.866000in}{3.018048in}}%
\pgfpathlineto{\pgfqpoint{1.959000in}{2.807565in}}%
\pgfpathlineto{\pgfqpoint{2.052000in}{2.611111in}}%
\pgfpathlineto{\pgfqpoint{2.145000in}{2.429434in}}%
\pgfpathlineto{\pgfqpoint{2.238000in}{2.263109in}}%
\pgfpathlineto{\pgfqpoint{2.331000in}{2.112749in}}%
\pgfpathlineto{\pgfqpoint{2.424000in}{1.978879in}}%
\pgfpathlineto{\pgfqpoint{2.517000in}{1.861940in}}%
\pgfpathlineto{\pgfqpoint{2.610000in}{1.762334in}}%
\pgfpathlineto{\pgfqpoint{2.703000in}{1.680390in}}%
\pgfpathlineto{\pgfqpoint{2.796000in}{1.616387in}}%
\pgfpathlineto{\pgfqpoint{2.889000in}{1.570535in}}%
\pgfpathlineto{\pgfqpoint{2.982000in}{1.542987in}}%
\pgfpathlineto{\pgfqpoint{3.075000in}{1.533834in}}%
\pgfpathlineto{\pgfqpoint{3.168000in}{1.543105in}}%
\pgfpathlineto{\pgfqpoint{3.261000in}{1.570770in}}%
\pgfpathlineto{\pgfqpoint{3.354000in}{1.616736in}}%
\pgfpathlineto{\pgfqpoint{3.447000in}{1.680850in}}%
\pgfpathlineto{\pgfqpoint{3.540000in}{1.762901in}}%
\pgfpathlineto{\pgfqpoint{3.633000in}{1.862608in}}%
\pgfpathlineto{\pgfqpoint{3.726000in}{1.979641in}}%
\pgfpathlineto{\pgfqpoint{3.819000in}{2.113598in}}%
\pgfpathlineto{\pgfqpoint{3.912000in}{2.264035in}}%
\pgfpathlineto{\pgfqpoint{4.005000in}{2.430428in}}%
\pgfpathlineto{\pgfqpoint{4.098000in}{2.612161in}}%
\pgfpathlineto{\pgfqpoint{4.191000in}{2.808658in}}%
\pgfpathlineto{\pgfqpoint{4.284000in}{3.019172in}}%
\pgfpathlineto{\pgfqpoint{4.377000in}{3.242988in}}%
\pgfpathlineto{\pgfqpoint{4.470000in}{3.479160in}}%
\pgfpathlineto{\pgfqpoint{4.563000in}{3.727073in}}%
\pgfpathlineto{\pgfqpoint{4.656000in}{3.985583in}}%
\pgfpathlineto{\pgfqpoint{4.749000in}{4.253736in}}%
\pgfpathlineto{\pgfqpoint{4.842000in}{4.530053in}}%
\pgfpathlineto{\pgfqpoint{4.935000in}{4.812588in}}%
\pgfpathlineto{\pgfqpoint{5.028000in}{5.105337in}}%
\pgfpathlineto{\pgfqpoint{5.121000in}{5.401451in}}%
\pgfpathlineto{\pgfqpoint{5.214000in}{5.699866in}}%
\pgfpathlineto{\pgfqpoint{5.307000in}{5.999590in}}%
\pgfpathlineto{\pgfqpoint{5.400000in}{6.300000in}}%
\pgfusepath{stroke}%
\end{pgfscope}%
\begin{pgfscope}%
\pgfpathrectangle{\pgfqpoint{0.750000in}{0.700000in}}{\pgfqpoint{4.650000in}{5.600000in}} %
\pgfusepath{clip}%
\pgfsetbuttcap%
\pgfsetmiterjoin%
\definecolor{currentfill}{rgb}{0.000000,0.500000,0.000000}%
\pgfsetfillcolor{currentfill}%
\pgfsetlinewidth{0.501875pt}%
\definecolor{currentstroke}{rgb}{0.000000,0.000000,0.000000}%
\pgfsetstrokecolor{currentstroke}%
\pgfsetdash{}{0pt}%
\pgfsys@defobject{currentmarker}{\pgfqpoint{-0.041667in}{-0.041667in}}{\pgfqpoint{0.041667in}{0.041667in}}{%
\pgfpathmoveto{\pgfqpoint{0.000000in}{0.041667in}}%
\pgfpathlineto{\pgfqpoint{-0.041667in}{-0.041667in}}%
\pgfpathlineto{\pgfqpoint{0.041667in}{-0.041667in}}%
\pgfpathclose%
\pgfusepath{stroke,fill}%
}%
\begin{pgfscope}%
\pgfsys@transformshift{1.587000in}{3.725948in}%
\pgfsys@useobject{currentmarker}{}%
\end{pgfscope}%
\begin{pgfscope}%
\pgfsys@transformshift{2.517000in}{1.861940in}%
\pgfsys@useobject{currentmarker}{}%
\end{pgfscope}%
\begin{pgfscope}%
\pgfsys@transformshift{3.447000in}{1.680850in}%
\pgfsys@useobject{currentmarker}{}%
\end{pgfscope}%
\begin{pgfscope}%
\pgfsys@transformshift{4.377000in}{3.242988in}%
\pgfsys@useobject{currentmarker}{}%
\end{pgfscope}%
\begin{pgfscope}%
\pgfsys@transformshift{5.307000in}{5.999590in}%
\pgfsys@useobject{currentmarker}{}%
\end{pgfscope}%
\end{pgfscope}%
\begin{pgfscope}%
\pgfpathrectangle{\pgfqpoint{0.750000in}{0.700000in}}{\pgfqpoint{4.650000in}{5.600000in}} %
\pgfusepath{clip}%
\pgfsetrectcap%
\pgfsetroundjoin%
\pgfsetlinewidth{1.003750pt}%
\definecolor{currentstroke}{rgb}{1.000000,0.000000,0.000000}%
\pgfsetstrokecolor{currentstroke}%
\pgfsetdash{}{0pt}%
\pgfpathmoveto{\pgfqpoint{0.796500in}{6.146764in}}%
\pgfpathlineto{\pgfqpoint{1.029000in}{5.385235in}}%
\pgfpathlineto{\pgfqpoint{1.168500in}{4.935536in}}%
\pgfpathlineto{\pgfqpoint{1.215000in}{4.787898in}}%
\pgfpathlineto{\pgfqpoint{1.354500in}{4.360990in}}%
\pgfpathlineto{\pgfqpoint{1.447500in}{4.085119in}}%
\pgfpathlineto{\pgfqpoint{1.540500in}{3.818522in}}%
\pgfpathlineto{\pgfqpoint{1.633500in}{3.562346in}}%
\pgfpathlineto{\pgfqpoint{1.726500in}{3.317624in}}%
\pgfpathlineto{\pgfqpoint{1.773000in}{3.199789in}}%
\pgfpathlineto{\pgfqpoint{1.819500in}{3.085106in}}%
\pgfpathlineto{\pgfqpoint{1.866000in}{2.973708in}}%
\pgfpathlineto{\pgfqpoint{1.912500in}{2.865674in}}%
\pgfpathlineto{\pgfqpoint{1.959000in}{2.761121in}}%
\pgfpathlineto{\pgfqpoint{2.005500in}{2.660123in}}%
\pgfpathlineto{\pgfqpoint{2.052000in}{2.562783in}}%
\pgfpathlineto{\pgfqpoint{2.098500in}{2.469169in}}%
\pgfpathlineto{\pgfqpoint{2.145000in}{2.379375in}}%
\pgfpathlineto{\pgfqpoint{2.191500in}{2.293461in}}%
\pgfpathlineto{\pgfqpoint{2.238000in}{2.211518in}}%
\pgfpathlineto{\pgfqpoint{2.284500in}{2.133613in}}%
\pgfpathlineto{\pgfqpoint{2.331000in}{2.059809in}}%
\pgfpathlineto{\pgfqpoint{2.377500in}{1.990169in}}%
\pgfpathlineto{\pgfqpoint{2.424000in}{1.924750in}}%
\pgfpathlineto{\pgfqpoint{2.470500in}{1.863608in}}%
\pgfpathlineto{\pgfqpoint{2.517000in}{1.806792in}}%
\pgfpathlineto{\pgfqpoint{2.563500in}{1.754350in}}%
\pgfpathlineto{\pgfqpoint{2.610000in}{1.706325in}}%
\pgfpathlineto{\pgfqpoint{2.656500in}{1.662757in}}%
\pgfpathlineto{\pgfqpoint{2.703000in}{1.623682in}}%
\pgfpathlineto{\pgfqpoint{2.749500in}{1.589131in}}%
\pgfpathlineto{\pgfqpoint{2.796000in}{1.559134in}}%
\pgfpathlineto{\pgfqpoint{2.842500in}{1.533714in}}%
\pgfpathlineto{\pgfqpoint{2.889000in}{1.512891in}}%
\pgfpathlineto{\pgfqpoint{2.935500in}{1.496684in}}%
\pgfpathlineto{\pgfqpoint{2.982000in}{1.485105in}}%
\pgfpathlineto{\pgfqpoint{3.028500in}{1.478164in}}%
\pgfpathlineto{\pgfqpoint{3.075000in}{1.475865in}}%
\pgfpathlineto{\pgfqpoint{3.121500in}{1.478211in}}%
\pgfpathlineto{\pgfqpoint{3.168000in}{1.485201in}}%
\pgfpathlineto{\pgfqpoint{3.214500in}{1.496827in}}%
\pgfpathlineto{\pgfqpoint{3.261000in}{1.513081in}}%
\pgfpathlineto{\pgfqpoint{3.307500in}{1.533950in}}%
\pgfpathlineto{\pgfqpoint{3.354000in}{1.559416in}}%
\pgfpathlineto{\pgfqpoint{3.400500in}{1.589459in}}%
\pgfpathlineto{\pgfqpoint{3.447000in}{1.624054in}}%
\pgfpathlineto{\pgfqpoint{3.493500in}{1.663173in}}%
\pgfpathlineto{\pgfqpoint{3.540000in}{1.706784in}}%
\pgfpathlineto{\pgfqpoint{3.586500in}{1.754850in}}%
\pgfpathlineto{\pgfqpoint{3.633000in}{1.807333in}}%
\pgfpathlineto{\pgfqpoint{3.679500in}{1.864188in}}%
\pgfpathlineto{\pgfqpoint{3.726000in}{1.925367in}}%
\pgfpathlineto{\pgfqpoint{3.772500in}{1.990822in}}%
\pgfpathlineto{\pgfqpoint{3.819000in}{2.060496in}}%
\pgfpathlineto{\pgfqpoint{3.865500in}{2.134333in}}%
\pgfpathlineto{\pgfqpoint{3.912000in}{2.212268in}}%
\pgfpathlineto{\pgfqpoint{3.958500in}{2.294240in}}%
\pgfpathlineto{\pgfqpoint{4.005000in}{2.380180in}}%
\pgfpathlineto{\pgfqpoint{4.051500in}{2.469998in}}%
\pgfpathlineto{\pgfqpoint{4.098000in}{2.563634in}}%
\pgfpathlineto{\pgfqpoint{4.144500in}{2.660993in}}%
\pgfpathlineto{\pgfqpoint{4.191000in}{2.762008in}}%
\pgfpathlineto{\pgfqpoint{4.237500in}{2.866574in}}%
\pgfpathlineto{\pgfqpoint{4.284000in}{2.974619in}}%
\pgfpathlineto{\pgfqpoint{4.330500in}{3.086024in}}%
\pgfpathlineto{\pgfqpoint{4.377000in}{3.200713in}}%
\pgfpathlineto{\pgfqpoint{4.423500in}{3.318550in}}%
\pgfpathlineto{\pgfqpoint{4.516500in}{3.563266in}}%
\pgfpathlineto{\pgfqpoint{4.609500in}{3.819421in}}%
\pgfpathlineto{\pgfqpoint{4.702500in}{4.085982in}}%
\pgfpathlineto{\pgfqpoint{4.795500in}{4.361802in}}%
\pgfpathlineto{\pgfqpoint{4.888500in}{4.645556in}}%
\pgfpathlineto{\pgfqpoint{4.935000in}{4.788614in}}%
\pgfpathlineto{\pgfqpoint{5.074500in}{5.234997in}}%
\pgfpathlineto{\pgfqpoint{5.260500in}{5.841347in}}%
\pgfpathlineto{\pgfqpoint{5.400000in}{6.300000in}}%
\pgfpathlineto{\pgfqpoint{5.400000in}{6.300000in}}%
\pgfusepath{stroke}%
\end{pgfscope}%
\begin{pgfscope}%
\pgfpathrectangle{\pgfqpoint{0.750000in}{0.700000in}}{\pgfqpoint{4.650000in}{5.600000in}} %
\pgfusepath{clip}%
\pgfsetbuttcap%
\pgfsetmiterjoin%
\definecolor{currentfill}{rgb}{1.000000,0.000000,0.000000}%
\pgfsetfillcolor{currentfill}%
\pgfsetlinewidth{0.501875pt}%
\definecolor{currentstroke}{rgb}{0.000000,0.000000,0.000000}%
\pgfsetstrokecolor{currentstroke}%
\pgfsetdash{}{0pt}%
\pgfsys@defobject{currentmarker}{\pgfqpoint{-0.041667in}{-0.041667in}}{\pgfqpoint{0.041667in}{0.041667in}}{%
\pgfpathmoveto{\pgfqpoint{0.041667in}{-0.000000in}}%
\pgfpathlineto{\pgfqpoint{-0.041667in}{0.041667in}}%
\pgfpathlineto{\pgfqpoint{-0.041667in}{-0.041667in}}%
\pgfpathclose%
\pgfusepath{stroke,fill}%
}%
\begin{pgfscope}%
\pgfsys@transformshift{1.075500in}{5.234456in}%
\pgfsys@useobject{currentmarker}{}%
\end{pgfscope}%
\begin{pgfscope}%
\pgfsys@transformshift{1.540500in}{3.818522in}%
\pgfsys@useobject{currentmarker}{}%
\end{pgfscope}%
\begin{pgfscope}%
\pgfsys@transformshift{2.005500in}{2.660123in}%
\pgfsys@useobject{currentmarker}{}%
\end{pgfscope}%
\begin{pgfscope}%
\pgfsys@transformshift{2.470500in}{1.863608in}%
\pgfsys@useobject{currentmarker}{}%
\end{pgfscope}%
\begin{pgfscope}%
\pgfsys@transformshift{2.935500in}{1.496684in}%
\pgfsys@useobject{currentmarker}{}%
\end{pgfscope}%
\begin{pgfscope}%
\pgfsys@transformshift{3.400500in}{1.589459in}%
\pgfsys@useobject{currentmarker}{}%
\end{pgfscope}%
\begin{pgfscope}%
\pgfsys@transformshift{3.865500in}{2.134333in}%
\pgfsys@useobject{currentmarker}{}%
\end{pgfscope}%
\begin{pgfscope}%
\pgfsys@transformshift{4.330500in}{3.086024in}%
\pgfsys@useobject{currentmarker}{}%
\end{pgfscope}%
\begin{pgfscope}%
\pgfsys@transformshift{4.795500in}{4.361802in}%
\pgfsys@useobject{currentmarker}{}%
\end{pgfscope}%
\begin{pgfscope}%
\pgfsys@transformshift{5.260500in}{5.841347in}%
\pgfsys@useobject{currentmarker}{}%
\end{pgfscope}%
\end{pgfscope}%
\begin{pgfscope}%
\pgfpathrectangle{\pgfqpoint{0.750000in}{0.700000in}}{\pgfqpoint{4.650000in}{5.600000in}} %
\pgfusepath{clip}%
\pgfsetbuttcap%
\pgfsetroundjoin%
\pgfsetlinewidth{0.501875pt}%
\definecolor{currentstroke}{rgb}{0.000000,0.000000,0.000000}%
\pgfsetstrokecolor{currentstroke}%
\pgfsetdash{{1.000000pt}{3.000000pt}}{0.000000pt}%
\pgfpathmoveto{\pgfqpoint{0.750000in}{0.700000in}}%
\pgfpathlineto{\pgfqpoint{0.750000in}{6.300000in}}%
\pgfusepath{stroke}%
\end{pgfscope}%
\begin{pgfscope}%
\pgfsetbuttcap%
\pgfsetroundjoin%
\definecolor{currentfill}{rgb}{0.000000,0.000000,0.000000}%
\pgfsetfillcolor{currentfill}%
\pgfsetlinewidth{0.501875pt}%
\definecolor{currentstroke}{rgb}{0.000000,0.000000,0.000000}%
\pgfsetstrokecolor{currentstroke}%
\pgfsetdash{}{0pt}%
\pgfsys@defobject{currentmarker}{\pgfqpoint{0.000000in}{0.000000in}}{\pgfqpoint{0.000000in}{0.055556in}}{%
\pgfpathmoveto{\pgfqpoint{0.000000in}{0.000000in}}%
\pgfpathlineto{\pgfqpoint{0.000000in}{0.055556in}}%
\pgfusepath{stroke,fill}%
}%
\begin{pgfscope}%
\pgfsys@transformshift{0.750000in}{0.700000in}%
\pgfsys@useobject{currentmarker}{}%
\end{pgfscope}%
\end{pgfscope}%
\begin{pgfscope}%
\pgfsetbuttcap%
\pgfsetroundjoin%
\definecolor{currentfill}{rgb}{0.000000,0.000000,0.000000}%
\pgfsetfillcolor{currentfill}%
\pgfsetlinewidth{0.501875pt}%
\definecolor{currentstroke}{rgb}{0.000000,0.000000,0.000000}%
\pgfsetstrokecolor{currentstroke}%
\pgfsetdash{}{0pt}%
\pgfsys@defobject{currentmarker}{\pgfqpoint{0.000000in}{-0.055556in}}{\pgfqpoint{0.000000in}{0.000000in}}{%
\pgfpathmoveto{\pgfqpoint{0.000000in}{0.000000in}}%
\pgfpathlineto{\pgfqpoint{0.000000in}{-0.055556in}}%
\pgfusepath{stroke,fill}%
}%
\begin{pgfscope}%
\pgfsys@transformshift{0.750000in}{6.300000in}%
\pgfsys@useobject{currentmarker}{}%
\end{pgfscope}%
\end{pgfscope}%
\begin{pgfscope}%
\pgftext[left,bottom,x=0.645738in,y=0.537037in,rotate=0.000000]{{\rmfamily\fontsize{12.000000}{14.400000}\selectfont \(\displaystyle 0.0\)}}
%
\end{pgfscope}%
\begin{pgfscope}%
\pgfpathrectangle{\pgfqpoint{0.750000in}{0.700000in}}{\pgfqpoint{4.650000in}{5.600000in}} %
\pgfusepath{clip}%
\pgfsetbuttcap%
\pgfsetroundjoin%
\pgfsetlinewidth{0.501875pt}%
\definecolor{currentstroke}{rgb}{0.000000,0.000000,0.000000}%
\pgfsetstrokecolor{currentstroke}%
\pgfsetdash{{1.000000pt}{3.000000pt}}{0.000000pt}%
\pgfpathmoveto{\pgfqpoint{1.912500in}{0.700000in}}%
\pgfpathlineto{\pgfqpoint{1.912500in}{6.300000in}}%
\pgfusepath{stroke}%
\end{pgfscope}%
\begin{pgfscope}%
\pgfsetbuttcap%
\pgfsetroundjoin%
\definecolor{currentfill}{rgb}{0.000000,0.000000,0.000000}%
\pgfsetfillcolor{currentfill}%
\pgfsetlinewidth{0.501875pt}%
\definecolor{currentstroke}{rgb}{0.000000,0.000000,0.000000}%
\pgfsetstrokecolor{currentstroke}%
\pgfsetdash{}{0pt}%
\pgfsys@defobject{currentmarker}{\pgfqpoint{0.000000in}{0.000000in}}{\pgfqpoint{0.000000in}{0.055556in}}{%
\pgfpathmoveto{\pgfqpoint{0.000000in}{0.000000in}}%
\pgfpathlineto{\pgfqpoint{0.000000in}{0.055556in}}%
\pgfusepath{stroke,fill}%
}%
\begin{pgfscope}%
\pgfsys@transformshift{1.912500in}{0.700000in}%
\pgfsys@useobject{currentmarker}{}%
\end{pgfscope}%
\end{pgfscope}%
\begin{pgfscope}%
\pgfsetbuttcap%
\pgfsetroundjoin%
\definecolor{currentfill}{rgb}{0.000000,0.000000,0.000000}%
\pgfsetfillcolor{currentfill}%
\pgfsetlinewidth{0.501875pt}%
\definecolor{currentstroke}{rgb}{0.000000,0.000000,0.000000}%
\pgfsetstrokecolor{currentstroke}%
\pgfsetdash{}{0pt}%
\pgfsys@defobject{currentmarker}{\pgfqpoint{0.000000in}{-0.055556in}}{\pgfqpoint{0.000000in}{0.000000in}}{%
\pgfpathmoveto{\pgfqpoint{0.000000in}{0.000000in}}%
\pgfpathlineto{\pgfqpoint{0.000000in}{-0.055556in}}%
\pgfusepath{stroke,fill}%
}%
\begin{pgfscope}%
\pgfsys@transformshift{1.912500in}{6.300000in}%
\pgfsys@useobject{currentmarker}{}%
\end{pgfscope}%
\end{pgfscope}%
\begin{pgfscope}%
\pgftext[left,bottom,x=1.808238in,y=0.537037in,rotate=0.000000]{{\rmfamily\fontsize{12.000000}{14.400000}\selectfont \(\displaystyle 0.5\)}}
%
\end{pgfscope}%
\begin{pgfscope}%
\pgfpathrectangle{\pgfqpoint{0.750000in}{0.700000in}}{\pgfqpoint{4.650000in}{5.600000in}} %
\pgfusepath{clip}%
\pgfsetbuttcap%
\pgfsetroundjoin%
\pgfsetlinewidth{0.501875pt}%
\definecolor{currentstroke}{rgb}{0.000000,0.000000,0.000000}%
\pgfsetstrokecolor{currentstroke}%
\pgfsetdash{{1.000000pt}{3.000000pt}}{0.000000pt}%
\pgfpathmoveto{\pgfqpoint{3.075000in}{0.700000in}}%
\pgfpathlineto{\pgfqpoint{3.075000in}{6.300000in}}%
\pgfusepath{stroke}%
\end{pgfscope}%
\begin{pgfscope}%
\pgfsetbuttcap%
\pgfsetroundjoin%
\definecolor{currentfill}{rgb}{0.000000,0.000000,0.000000}%
\pgfsetfillcolor{currentfill}%
\pgfsetlinewidth{0.501875pt}%
\definecolor{currentstroke}{rgb}{0.000000,0.000000,0.000000}%
\pgfsetstrokecolor{currentstroke}%
\pgfsetdash{}{0pt}%
\pgfsys@defobject{currentmarker}{\pgfqpoint{0.000000in}{0.000000in}}{\pgfqpoint{0.000000in}{0.055556in}}{%
\pgfpathmoveto{\pgfqpoint{0.000000in}{0.000000in}}%
\pgfpathlineto{\pgfqpoint{0.000000in}{0.055556in}}%
\pgfusepath{stroke,fill}%
}%
\begin{pgfscope}%
\pgfsys@transformshift{3.075000in}{0.700000in}%
\pgfsys@useobject{currentmarker}{}%
\end{pgfscope}%
\end{pgfscope}%
\begin{pgfscope}%
\pgfsetbuttcap%
\pgfsetroundjoin%
\definecolor{currentfill}{rgb}{0.000000,0.000000,0.000000}%
\pgfsetfillcolor{currentfill}%
\pgfsetlinewidth{0.501875pt}%
\definecolor{currentstroke}{rgb}{0.000000,0.000000,0.000000}%
\pgfsetstrokecolor{currentstroke}%
\pgfsetdash{}{0pt}%
\pgfsys@defobject{currentmarker}{\pgfqpoint{0.000000in}{-0.055556in}}{\pgfqpoint{0.000000in}{0.000000in}}{%
\pgfpathmoveto{\pgfqpoint{0.000000in}{0.000000in}}%
\pgfpathlineto{\pgfqpoint{0.000000in}{-0.055556in}}%
\pgfusepath{stroke,fill}%
}%
\begin{pgfscope}%
\pgfsys@transformshift{3.075000in}{6.300000in}%
\pgfsys@useobject{currentmarker}{}%
\end{pgfscope}%
\end{pgfscope}%
\begin{pgfscope}%
\pgftext[left,bottom,x=2.970738in,y=0.537037in,rotate=0.000000]{{\rmfamily\fontsize{12.000000}{14.400000}\selectfont \(\displaystyle 1.0\)}}
%
\end{pgfscope}%
\begin{pgfscope}%
\pgfpathrectangle{\pgfqpoint{0.750000in}{0.700000in}}{\pgfqpoint{4.650000in}{5.600000in}} %
\pgfusepath{clip}%
\pgfsetbuttcap%
\pgfsetroundjoin%
\pgfsetlinewidth{0.501875pt}%
\definecolor{currentstroke}{rgb}{0.000000,0.000000,0.000000}%
\pgfsetstrokecolor{currentstroke}%
\pgfsetdash{{1.000000pt}{3.000000pt}}{0.000000pt}%
\pgfpathmoveto{\pgfqpoint{4.237500in}{0.700000in}}%
\pgfpathlineto{\pgfqpoint{4.237500in}{6.300000in}}%
\pgfusepath{stroke}%
\end{pgfscope}%
\begin{pgfscope}%
\pgfsetbuttcap%
\pgfsetroundjoin%
\definecolor{currentfill}{rgb}{0.000000,0.000000,0.000000}%
\pgfsetfillcolor{currentfill}%
\pgfsetlinewidth{0.501875pt}%
\definecolor{currentstroke}{rgb}{0.000000,0.000000,0.000000}%
\pgfsetstrokecolor{currentstroke}%
\pgfsetdash{}{0pt}%
\pgfsys@defobject{currentmarker}{\pgfqpoint{0.000000in}{0.000000in}}{\pgfqpoint{0.000000in}{0.055556in}}{%
\pgfpathmoveto{\pgfqpoint{0.000000in}{0.000000in}}%
\pgfpathlineto{\pgfqpoint{0.000000in}{0.055556in}}%
\pgfusepath{stroke,fill}%
}%
\begin{pgfscope}%
\pgfsys@transformshift{4.237500in}{0.700000in}%
\pgfsys@useobject{currentmarker}{}%
\end{pgfscope}%
\end{pgfscope}%
\begin{pgfscope}%
\pgfsetbuttcap%
\pgfsetroundjoin%
\definecolor{currentfill}{rgb}{0.000000,0.000000,0.000000}%
\pgfsetfillcolor{currentfill}%
\pgfsetlinewidth{0.501875pt}%
\definecolor{currentstroke}{rgb}{0.000000,0.000000,0.000000}%
\pgfsetstrokecolor{currentstroke}%
\pgfsetdash{}{0pt}%
\pgfsys@defobject{currentmarker}{\pgfqpoint{0.000000in}{-0.055556in}}{\pgfqpoint{0.000000in}{0.000000in}}{%
\pgfpathmoveto{\pgfqpoint{0.000000in}{0.000000in}}%
\pgfpathlineto{\pgfqpoint{0.000000in}{-0.055556in}}%
\pgfusepath{stroke,fill}%
}%
\begin{pgfscope}%
\pgfsys@transformshift{4.237500in}{6.300000in}%
\pgfsys@useobject{currentmarker}{}%
\end{pgfscope}%
\end{pgfscope}%
\begin{pgfscope}%
\pgftext[left,bottom,x=4.133238in,y=0.537037in,rotate=0.000000]{{\rmfamily\fontsize{12.000000}{14.400000}\selectfont \(\displaystyle 1.5\)}}
%
\end{pgfscope}%
\begin{pgfscope}%
\pgfpathrectangle{\pgfqpoint{0.750000in}{0.700000in}}{\pgfqpoint{4.650000in}{5.600000in}} %
\pgfusepath{clip}%
\pgfsetbuttcap%
\pgfsetroundjoin%
\pgfsetlinewidth{0.501875pt}%
\definecolor{currentstroke}{rgb}{0.000000,0.000000,0.000000}%
\pgfsetstrokecolor{currentstroke}%
\pgfsetdash{{1.000000pt}{3.000000pt}}{0.000000pt}%
\pgfpathmoveto{\pgfqpoint{5.400000in}{0.700000in}}%
\pgfpathlineto{\pgfqpoint{5.400000in}{6.300000in}}%
\pgfusepath{stroke}%
\end{pgfscope}%
\begin{pgfscope}%
\pgfsetbuttcap%
\pgfsetroundjoin%
\definecolor{currentfill}{rgb}{0.000000,0.000000,0.000000}%
\pgfsetfillcolor{currentfill}%
\pgfsetlinewidth{0.501875pt}%
\definecolor{currentstroke}{rgb}{0.000000,0.000000,0.000000}%
\pgfsetstrokecolor{currentstroke}%
\pgfsetdash{}{0pt}%
\pgfsys@defobject{currentmarker}{\pgfqpoint{0.000000in}{0.000000in}}{\pgfqpoint{0.000000in}{0.055556in}}{%
\pgfpathmoveto{\pgfqpoint{0.000000in}{0.000000in}}%
\pgfpathlineto{\pgfqpoint{0.000000in}{0.055556in}}%
\pgfusepath{stroke,fill}%
}%
\begin{pgfscope}%
\pgfsys@transformshift{5.400000in}{0.700000in}%
\pgfsys@useobject{currentmarker}{}%
\end{pgfscope}%
\end{pgfscope}%
\begin{pgfscope}%
\pgfsetbuttcap%
\pgfsetroundjoin%
\definecolor{currentfill}{rgb}{0.000000,0.000000,0.000000}%
\pgfsetfillcolor{currentfill}%
\pgfsetlinewidth{0.501875pt}%
\definecolor{currentstroke}{rgb}{0.000000,0.000000,0.000000}%
\pgfsetstrokecolor{currentstroke}%
\pgfsetdash{}{0pt}%
\pgfsys@defobject{currentmarker}{\pgfqpoint{0.000000in}{-0.055556in}}{\pgfqpoint{0.000000in}{0.000000in}}{%
\pgfpathmoveto{\pgfqpoint{0.000000in}{0.000000in}}%
\pgfpathlineto{\pgfqpoint{0.000000in}{-0.055556in}}%
\pgfusepath{stroke,fill}%
}%
\begin{pgfscope}%
\pgfsys@transformshift{5.400000in}{6.300000in}%
\pgfsys@useobject{currentmarker}{}%
\end{pgfscope}%
\end{pgfscope}%
\begin{pgfscope}%
\pgftext[left,bottom,x=5.295738in,y=0.537037in,rotate=0.000000]{{\rmfamily\fontsize{12.000000}{14.400000}\selectfont \(\displaystyle 2.0\)}}
%
\end{pgfscope}%
\begin{pgfscope}%
\pgftext[left,bottom,x=2.319809in,y=0.319445in,rotate=0.000000]{{\rmfamily\fontsize{12.000000}{14.400000}\selectfont Distance along Beam}}
%
\end{pgfscope}%
\begin{pgfscope}%
\pgfpathrectangle{\pgfqpoint{0.750000in}{0.700000in}}{\pgfqpoint{4.650000in}{5.600000in}} %
\pgfusepath{clip}%
\pgfsetbuttcap%
\pgfsetroundjoin%
\pgfsetlinewidth{0.501875pt}%
\definecolor{currentstroke}{rgb}{0.000000,0.000000,0.000000}%
\pgfsetstrokecolor{currentstroke}%
\pgfsetdash{{1.000000pt}{3.000000pt}}{0.000000pt}%
\pgfpathmoveto{\pgfqpoint{0.750000in}{6.300000in}}%
\pgfpathlineto{\pgfqpoint{5.400000in}{6.300000in}}%
\pgfusepath{stroke}%
\end{pgfscope}%
\begin{pgfscope}%
\pgfsetbuttcap%
\pgfsetroundjoin%
\definecolor{currentfill}{rgb}{0.000000,0.000000,0.000000}%
\pgfsetfillcolor{currentfill}%
\pgfsetlinewidth{0.501875pt}%
\definecolor{currentstroke}{rgb}{0.000000,0.000000,0.000000}%
\pgfsetstrokecolor{currentstroke}%
\pgfsetdash{}{0pt}%
\pgfsys@defobject{currentmarker}{\pgfqpoint{0.000000in}{0.000000in}}{\pgfqpoint{0.055556in}{0.000000in}}{%
\pgfpathmoveto{\pgfqpoint{0.000000in}{0.000000in}}%
\pgfpathlineto{\pgfqpoint{0.055556in}{0.000000in}}%
\pgfusepath{stroke,fill}%
}%
\begin{pgfscope}%
\pgfsys@transformshift{0.750000in}{6.300000in}%
\pgfsys@useobject{currentmarker}{}%
\end{pgfscope}%
\end{pgfscope}%
\begin{pgfscope}%
\pgfsetbuttcap%
\pgfsetroundjoin%
\definecolor{currentfill}{rgb}{0.000000,0.000000,0.000000}%
\pgfsetfillcolor{currentfill}%
\pgfsetlinewidth{0.501875pt}%
\definecolor{currentstroke}{rgb}{0.000000,0.000000,0.000000}%
\pgfsetstrokecolor{currentstroke}%
\pgfsetdash{}{0pt}%
\pgfsys@defobject{currentmarker}{\pgfqpoint{-0.055556in}{0.000000in}}{\pgfqpoint{0.000000in}{0.000000in}}{%
\pgfpathmoveto{\pgfqpoint{0.000000in}{0.000000in}}%
\pgfpathlineto{\pgfqpoint{-0.055556in}{0.000000in}}%
\pgfusepath{stroke,fill}%
}%
\begin{pgfscope}%
\pgfsys@transformshift{5.400000in}{6.300000in}%
\pgfsys@useobject{currentmarker}{}%
\end{pgfscope}%
\end{pgfscope}%
\begin{pgfscope}%
\pgftext[left,bottom,x=0.485920in,y=6.246296in,rotate=0.000000]{{\rmfamily\fontsize{12.000000}{14.400000}\selectfont \(\displaystyle 0.0\)}}
%
\end{pgfscope}%
\begin{pgfscope}%
\pgfpathrectangle{\pgfqpoint{0.750000in}{0.700000in}}{\pgfqpoint{4.650000in}{5.600000in}} %
\pgfusepath{clip}%
\pgfsetbuttcap%
\pgfsetroundjoin%
\pgfsetlinewidth{0.501875pt}%
\definecolor{currentstroke}{rgb}{0.000000,0.000000,0.000000}%
\pgfsetstrokecolor{currentstroke}%
\pgfsetdash{{1.000000pt}{3.000000pt}}{0.000000pt}%
\pgfpathmoveto{\pgfqpoint{0.750000in}{4.900000in}}%
\pgfpathlineto{\pgfqpoint{5.400000in}{4.900000in}}%
\pgfusepath{stroke}%
\end{pgfscope}%
\begin{pgfscope}%
\pgfsetbuttcap%
\pgfsetroundjoin%
\definecolor{currentfill}{rgb}{0.000000,0.000000,0.000000}%
\pgfsetfillcolor{currentfill}%
\pgfsetlinewidth{0.501875pt}%
\definecolor{currentstroke}{rgb}{0.000000,0.000000,0.000000}%
\pgfsetstrokecolor{currentstroke}%
\pgfsetdash{}{0pt}%
\pgfsys@defobject{currentmarker}{\pgfqpoint{0.000000in}{0.000000in}}{\pgfqpoint{0.055556in}{0.000000in}}{%
\pgfpathmoveto{\pgfqpoint{0.000000in}{0.000000in}}%
\pgfpathlineto{\pgfqpoint{0.055556in}{0.000000in}}%
\pgfusepath{stroke,fill}%
}%
\begin{pgfscope}%
\pgfsys@transformshift{0.750000in}{4.900000in}%
\pgfsys@useobject{currentmarker}{}%
\end{pgfscope}%
\end{pgfscope}%
\begin{pgfscope}%
\pgfsetbuttcap%
\pgfsetroundjoin%
\definecolor{currentfill}{rgb}{0.000000,0.000000,0.000000}%
\pgfsetfillcolor{currentfill}%
\pgfsetlinewidth{0.501875pt}%
\definecolor{currentstroke}{rgb}{0.000000,0.000000,0.000000}%
\pgfsetstrokecolor{currentstroke}%
\pgfsetdash{}{0pt}%
\pgfsys@defobject{currentmarker}{\pgfqpoint{-0.055556in}{0.000000in}}{\pgfqpoint{0.000000in}{0.000000in}}{%
\pgfpathmoveto{\pgfqpoint{0.000000in}{0.000000in}}%
\pgfpathlineto{\pgfqpoint{-0.055556in}{0.000000in}}%
\pgfusepath{stroke,fill}%
}%
\begin{pgfscope}%
\pgfsys@transformshift{5.400000in}{4.900000in}%
\pgfsys@useobject{currentmarker}{}%
\end{pgfscope}%
\end{pgfscope}%
\begin{pgfscope}%
\pgftext[left,bottom,x=0.356290in,y=4.839352in,rotate=0.000000]{{\rmfamily\fontsize{12.000000}{14.400000}\selectfont \(\displaystyle -0.3\)}}
%
\end{pgfscope}%
\begin{pgfscope}%
\pgfpathrectangle{\pgfqpoint{0.750000in}{0.700000in}}{\pgfqpoint{4.650000in}{5.600000in}} %
\pgfusepath{clip}%
\pgfsetbuttcap%
\pgfsetroundjoin%
\pgfsetlinewidth{0.501875pt}%
\definecolor{currentstroke}{rgb}{0.000000,0.000000,0.000000}%
\pgfsetstrokecolor{currentstroke}%
\pgfsetdash{{1.000000pt}{3.000000pt}}{0.000000pt}%
\pgfpathmoveto{\pgfqpoint{0.750000in}{3.500000in}}%
\pgfpathlineto{\pgfqpoint{5.400000in}{3.500000in}}%
\pgfusepath{stroke}%
\end{pgfscope}%
\begin{pgfscope}%
\pgfsetbuttcap%
\pgfsetroundjoin%
\definecolor{currentfill}{rgb}{0.000000,0.000000,0.000000}%
\pgfsetfillcolor{currentfill}%
\pgfsetlinewidth{0.501875pt}%
\definecolor{currentstroke}{rgb}{0.000000,0.000000,0.000000}%
\pgfsetstrokecolor{currentstroke}%
\pgfsetdash{}{0pt}%
\pgfsys@defobject{currentmarker}{\pgfqpoint{0.000000in}{0.000000in}}{\pgfqpoint{0.055556in}{0.000000in}}{%
\pgfpathmoveto{\pgfqpoint{0.000000in}{0.000000in}}%
\pgfpathlineto{\pgfqpoint{0.055556in}{0.000000in}}%
\pgfusepath{stroke,fill}%
}%
\begin{pgfscope}%
\pgfsys@transformshift{0.750000in}{3.500000in}%
\pgfsys@useobject{currentmarker}{}%
\end{pgfscope}%
\end{pgfscope}%
\begin{pgfscope}%
\pgfsetbuttcap%
\pgfsetroundjoin%
\definecolor{currentfill}{rgb}{0.000000,0.000000,0.000000}%
\pgfsetfillcolor{currentfill}%
\pgfsetlinewidth{0.501875pt}%
\definecolor{currentstroke}{rgb}{0.000000,0.000000,0.000000}%
\pgfsetstrokecolor{currentstroke}%
\pgfsetdash{}{0pt}%
\pgfsys@defobject{currentmarker}{\pgfqpoint{-0.055556in}{0.000000in}}{\pgfqpoint{0.000000in}{0.000000in}}{%
\pgfpathmoveto{\pgfqpoint{0.000000in}{0.000000in}}%
\pgfpathlineto{\pgfqpoint{-0.055556in}{0.000000in}}%
\pgfusepath{stroke,fill}%
}%
\begin{pgfscope}%
\pgfsys@transformshift{5.400000in}{3.500000in}%
\pgfsys@useobject{currentmarker}{}%
\end{pgfscope}%
\end{pgfscope}%
\begin{pgfscope}%
\pgftext[left,bottom,x=0.356290in,y=3.439352in,rotate=0.000000]{{\rmfamily\fontsize{12.000000}{14.400000}\selectfont \(\displaystyle -0.6\)}}
%
\end{pgfscope}%
\begin{pgfscope}%
\pgfpathrectangle{\pgfqpoint{0.750000in}{0.700000in}}{\pgfqpoint{4.650000in}{5.600000in}} %
\pgfusepath{clip}%
\pgfsetbuttcap%
\pgfsetroundjoin%
\pgfsetlinewidth{0.501875pt}%
\definecolor{currentstroke}{rgb}{0.000000,0.000000,0.000000}%
\pgfsetstrokecolor{currentstroke}%
\pgfsetdash{{1.000000pt}{3.000000pt}}{0.000000pt}%
\pgfpathmoveto{\pgfqpoint{0.750000in}{2.100000in}}%
\pgfpathlineto{\pgfqpoint{5.400000in}{2.100000in}}%
\pgfusepath{stroke}%
\end{pgfscope}%
\begin{pgfscope}%
\pgfsetbuttcap%
\pgfsetroundjoin%
\definecolor{currentfill}{rgb}{0.000000,0.000000,0.000000}%
\pgfsetfillcolor{currentfill}%
\pgfsetlinewidth{0.501875pt}%
\definecolor{currentstroke}{rgb}{0.000000,0.000000,0.000000}%
\pgfsetstrokecolor{currentstroke}%
\pgfsetdash{}{0pt}%
\pgfsys@defobject{currentmarker}{\pgfqpoint{0.000000in}{0.000000in}}{\pgfqpoint{0.055556in}{0.000000in}}{%
\pgfpathmoveto{\pgfqpoint{0.000000in}{0.000000in}}%
\pgfpathlineto{\pgfqpoint{0.055556in}{0.000000in}}%
\pgfusepath{stroke,fill}%
}%
\begin{pgfscope}%
\pgfsys@transformshift{0.750000in}{2.100000in}%
\pgfsys@useobject{currentmarker}{}%
\end{pgfscope}%
\end{pgfscope}%
\begin{pgfscope}%
\pgfsetbuttcap%
\pgfsetroundjoin%
\definecolor{currentfill}{rgb}{0.000000,0.000000,0.000000}%
\pgfsetfillcolor{currentfill}%
\pgfsetlinewidth{0.501875pt}%
\definecolor{currentstroke}{rgb}{0.000000,0.000000,0.000000}%
\pgfsetstrokecolor{currentstroke}%
\pgfsetdash{}{0pt}%
\pgfsys@defobject{currentmarker}{\pgfqpoint{-0.055556in}{0.000000in}}{\pgfqpoint{0.000000in}{0.000000in}}{%
\pgfpathmoveto{\pgfqpoint{0.000000in}{0.000000in}}%
\pgfpathlineto{\pgfqpoint{-0.055556in}{0.000000in}}%
\pgfusepath{stroke,fill}%
}%
\begin{pgfscope}%
\pgfsys@transformshift{5.400000in}{2.100000in}%
\pgfsys@useobject{currentmarker}{}%
\end{pgfscope}%
\end{pgfscope}%
\begin{pgfscope}%
\pgftext[left,bottom,x=0.356290in,y=2.039352in,rotate=0.000000]{{\rmfamily\fontsize{12.000000}{14.400000}\selectfont \(\displaystyle -0.9\)}}
%
\end{pgfscope}%
\begin{pgfscope}%
\pgfpathrectangle{\pgfqpoint{0.750000in}{0.700000in}}{\pgfqpoint{4.650000in}{5.600000in}} %
\pgfusepath{clip}%
\pgfsetbuttcap%
\pgfsetroundjoin%
\pgfsetlinewidth{0.501875pt}%
\definecolor{currentstroke}{rgb}{0.000000,0.000000,0.000000}%
\pgfsetstrokecolor{currentstroke}%
\pgfsetdash{{1.000000pt}{3.000000pt}}{0.000000pt}%
\pgfpathmoveto{\pgfqpoint{0.750000in}{0.700000in}}%
\pgfpathlineto{\pgfqpoint{5.400000in}{0.700000in}}%
\pgfusepath{stroke}%
\end{pgfscope}%
\begin{pgfscope}%
\pgfsetbuttcap%
\pgfsetroundjoin%
\definecolor{currentfill}{rgb}{0.000000,0.000000,0.000000}%
\pgfsetfillcolor{currentfill}%
\pgfsetlinewidth{0.501875pt}%
\definecolor{currentstroke}{rgb}{0.000000,0.000000,0.000000}%
\pgfsetstrokecolor{currentstroke}%
\pgfsetdash{}{0pt}%
\pgfsys@defobject{currentmarker}{\pgfqpoint{0.000000in}{0.000000in}}{\pgfqpoint{0.055556in}{0.000000in}}{%
\pgfpathmoveto{\pgfqpoint{0.000000in}{0.000000in}}%
\pgfpathlineto{\pgfqpoint{0.055556in}{0.000000in}}%
\pgfusepath{stroke,fill}%
}%
\begin{pgfscope}%
\pgfsys@transformshift{0.750000in}{0.700000in}%
\pgfsys@useobject{currentmarker}{}%
\end{pgfscope}%
\end{pgfscope}%
\begin{pgfscope}%
\pgfsetbuttcap%
\pgfsetroundjoin%
\definecolor{currentfill}{rgb}{0.000000,0.000000,0.000000}%
\pgfsetfillcolor{currentfill}%
\pgfsetlinewidth{0.501875pt}%
\definecolor{currentstroke}{rgb}{0.000000,0.000000,0.000000}%
\pgfsetstrokecolor{currentstroke}%
\pgfsetdash{}{0pt}%
\pgfsys@defobject{currentmarker}{\pgfqpoint{-0.055556in}{0.000000in}}{\pgfqpoint{0.000000in}{0.000000in}}{%
\pgfpathmoveto{\pgfqpoint{0.000000in}{0.000000in}}%
\pgfpathlineto{\pgfqpoint{-0.055556in}{0.000000in}}%
\pgfusepath{stroke,fill}%
}%
\begin{pgfscope}%
\pgfsys@transformshift{5.400000in}{0.700000in}%
\pgfsys@useobject{currentmarker}{}%
\end{pgfscope}%
\end{pgfscope}%
\begin{pgfscope}%
\pgftext[left,bottom,x=0.356290in,y=0.639352in,rotate=0.000000]{{\rmfamily\fontsize{12.000000}{14.400000}\selectfont \(\displaystyle -1.2\)}}
%
\end{pgfscope}%
\begin{pgfscope}%
\pgftext[left,bottom,x=0.286846in,y=3.143030in,rotate=90.000000]{{\rmfamily\fontsize{12.000000}{14.400000}\selectfont Deflection}}
%
\end{pgfscope}%
\begin{pgfscope}%
\pgftext[left,bottom,x=0.750000in,y=6.327778in,rotate=0.000000]{{\rmfamily\fontsize{12.000000}{14.400000}\selectfont \(\displaystyle \times10^{-4}\)}}
%
\end{pgfscope}%
\begin{pgfscope}%
\pgfsetrectcap%
\pgfsetroundjoin%
\pgfsetlinewidth{1.003750pt}%
\definecolor{currentstroke}{rgb}{0.000000,0.000000,0.000000}%
\pgfsetstrokecolor{currentstroke}%
\pgfsetdash{}{0pt}%
\pgfpathmoveto{\pgfqpoint{0.750000in}{6.300000in}}%
\pgfpathlineto{\pgfqpoint{5.400000in}{6.300000in}}%
\pgfusepath{stroke}%
\end{pgfscope}%
\begin{pgfscope}%
\pgfsetrectcap%
\pgfsetroundjoin%
\pgfsetlinewidth{1.003750pt}%
\definecolor{currentstroke}{rgb}{0.000000,0.000000,0.000000}%
\pgfsetstrokecolor{currentstroke}%
\pgfsetdash{}{0pt}%
\pgfpathmoveto{\pgfqpoint{5.400000in}{0.700000in}}%
\pgfpathlineto{\pgfqpoint{5.400000in}{6.300000in}}%
\pgfusepath{stroke}%
\end{pgfscope}%
\begin{pgfscope}%
\pgfsetrectcap%
\pgfsetroundjoin%
\pgfsetlinewidth{1.003750pt}%
\definecolor{currentstroke}{rgb}{0.000000,0.000000,0.000000}%
\pgfsetstrokecolor{currentstroke}%
\pgfsetdash{}{0pt}%
\pgfpathmoveto{\pgfqpoint{0.750000in}{0.700000in}}%
\pgfpathlineto{\pgfqpoint{5.400000in}{0.700000in}}%
\pgfusepath{stroke}%
\end{pgfscope}%
\begin{pgfscope}%
\pgfsetrectcap%
\pgfsetroundjoin%
\pgfsetlinewidth{1.003750pt}%
\definecolor{currentstroke}{rgb}{0.000000,0.000000,0.000000}%
\pgfsetstrokecolor{currentstroke}%
\pgfsetdash{}{0pt}%
\pgfpathmoveto{\pgfqpoint{0.750000in}{0.700000in}}%
\pgfpathlineto{\pgfqpoint{0.750000in}{6.300000in}}%
\pgfusepath{stroke}%
\end{pgfscope}%
\begin{pgfscope}%
\pgftext[left,bottom,x=1.699647in,y=6.330556in,rotate=0.000000]{{\rmfamily\fontsize{14.400000}{17.280000}\selectfont Uniformly Loaded Elastic Beam}}
%
\end{pgfscope}%
\begin{pgfscope}%
\pgfsetrectcap%
\pgfsetroundjoin%
\definecolor{currentfill}{rgb}{1.000000,1.000000,1.000000}%
\pgfsetfillcolor{currentfill}%
\pgfsetlinewidth{1.003750pt}%
\definecolor{currentstroke}{rgb}{0.000000,0.000000,0.000000}%
\pgfsetstrokecolor{currentstroke}%
\pgfsetdash{}{0pt}%
\pgfpathmoveto{\pgfqpoint{1.710823in}{5.403334in}}%
\pgfpathlineto{\pgfqpoint{4.439177in}{5.403334in}}%
\pgfpathlineto{\pgfqpoint{4.439177in}{6.300000in}}%
\pgfpathlineto{\pgfqpoint{1.710823in}{6.300000in}}%
\pgfpathlineto{\pgfqpoint{1.710823in}{5.403334in}}%
\pgfpathclose%
\pgfusepath{stroke,fill}%
\end{pgfscope}%
\begin{pgfscope}%
\pgfsetrectcap%
\pgfsetroundjoin%
\pgfsetlinewidth{1.003750pt}%
\definecolor{currentstroke}{rgb}{0.000000,0.000000,1.000000}%
\pgfsetstrokecolor{currentstroke}%
\pgfsetdash{}{0pt}%
\pgfpathmoveto{\pgfqpoint{1.850823in}{6.150000in}}%
\pgfpathlineto{\pgfqpoint{2.130823in}{6.150000in}}%
\pgfusepath{stroke}%
\end{pgfscope}%
\begin{pgfscope}%
\pgfsetbuttcap%
\pgfsetroundjoin%
\definecolor{currentfill}{rgb}{0.000000,0.000000,1.000000}%
\pgfsetfillcolor{currentfill}%
\pgfsetlinewidth{0.501875pt}%
\definecolor{currentstroke}{rgb}{0.000000,0.000000,0.000000}%
\pgfsetstrokecolor{currentstroke}%
\pgfsetdash{}{0pt}%
\pgfsys@defobject{currentmarker}{\pgfqpoint{-0.041667in}{-0.041667in}}{\pgfqpoint{0.041667in}{0.041667in}}{%
\pgfpathmoveto{\pgfqpoint{0.000000in}{-0.041667in}}%
\pgfpathcurveto{\pgfqpoint{0.011050in}{-0.041667in}}{\pgfqpoint{0.021649in}{-0.037276in}}{\pgfqpoint{0.029463in}{-0.029463in}}%
\pgfpathcurveto{\pgfqpoint{0.037276in}{-0.021649in}}{\pgfqpoint{0.041667in}{-0.011050in}}{\pgfqpoint{0.041667in}{0.000000in}}%
\pgfpathcurveto{\pgfqpoint{0.041667in}{0.011050in}}{\pgfqpoint{0.037276in}{0.021649in}}{\pgfqpoint{0.029463in}{0.029463in}}%
\pgfpathcurveto{\pgfqpoint{0.021649in}{0.037276in}}{\pgfqpoint{0.011050in}{0.041667in}}{\pgfqpoint{0.000000in}{0.041667in}}%
\pgfpathcurveto{\pgfqpoint{-0.011050in}{0.041667in}}{\pgfqpoint{-0.021649in}{0.037276in}}{\pgfqpoint{-0.029463in}{0.029463in}}%
\pgfpathcurveto{\pgfqpoint{-0.037276in}{0.021649in}}{\pgfqpoint{-0.041667in}{0.011050in}}{\pgfqpoint{-0.041667in}{0.000000in}}%
\pgfpathcurveto{\pgfqpoint{-0.041667in}{-0.011050in}}{\pgfqpoint{-0.037276in}{-0.021649in}}{\pgfqpoint{-0.029463in}{-0.029463in}}%
\pgfpathcurveto{\pgfqpoint{-0.021649in}{-0.037276in}}{\pgfqpoint{-0.011050in}{-0.041667in}}{\pgfqpoint{0.000000in}{-0.041667in}}%
\pgfpathclose%
\pgfusepath{stroke,fill}%
}%
\begin{pgfscope}%
\pgfsys@transformshift{1.850823in}{6.150000in}%
\pgfsys@useobject{currentmarker}{}%
\end{pgfscope}%
\begin{pgfscope}%
\pgfsys@transformshift{2.130823in}{6.150000in}%
\pgfsys@useobject{currentmarker}{}%
\end{pgfscope}%
\end{pgfscope}%
\begin{pgfscope}%
\pgftext[left,bottom,x=2.350823in,y=6.041111in,rotate=0.000000]{{\rmfamily\fontsize{14.400000}{17.280000}\selectfont Abaqus Elastic Beam}}
%
\end{pgfscope}%
\begin{pgfscope}%
\pgfsetrectcap%
\pgfsetroundjoin%
\pgfsetlinewidth{1.003750pt}%
\definecolor{currentstroke}{rgb}{0.000000,0.500000,0.000000}%
\pgfsetstrokecolor{currentstroke}%
\pgfsetdash{}{0pt}%
\pgfpathmoveto{\pgfqpoint{1.850823in}{5.871111in}}%
\pgfpathlineto{\pgfqpoint{2.130823in}{5.871111in}}%
\pgfusepath{stroke}%
\end{pgfscope}%
\begin{pgfscope}%
\pgfsetbuttcap%
\pgfsetmiterjoin%
\definecolor{currentfill}{rgb}{0.000000,0.500000,0.000000}%
\pgfsetfillcolor{currentfill}%
\pgfsetlinewidth{0.501875pt}%
\definecolor{currentstroke}{rgb}{0.000000,0.000000,0.000000}%
\pgfsetstrokecolor{currentstroke}%
\pgfsetdash{}{0pt}%
\pgfsys@defobject{currentmarker}{\pgfqpoint{-0.041667in}{-0.041667in}}{\pgfqpoint{0.041667in}{0.041667in}}{%
\pgfpathmoveto{\pgfqpoint{0.000000in}{0.041667in}}%
\pgfpathlineto{\pgfqpoint{-0.041667in}{-0.041667in}}%
\pgfpathlineto{\pgfqpoint{0.041667in}{-0.041667in}}%
\pgfpathclose%
\pgfusepath{stroke,fill}%
}%
\begin{pgfscope}%
\pgfsys@transformshift{1.850823in}{5.871111in}%
\pgfsys@useobject{currentmarker}{}%
\end{pgfscope}%
\begin{pgfscope}%
\pgfsys@transformshift{2.130823in}{5.871111in}%
\pgfsys@useobject{currentmarker}{}%
\end{pgfscope}%
\end{pgfscope}%
\begin{pgfscope}%
\pgftext[left,bottom,x=2.350823in,y=5.762223in,rotate=0.000000]{{\rmfamily\fontsize{14.400000}{17.280000}\selectfont 50 nodes, horizon 0.20}}
%
\end{pgfscope}%
\begin{pgfscope}%
\pgfsetrectcap%
\pgfsetroundjoin%
\pgfsetlinewidth{1.003750pt}%
\definecolor{currentstroke}{rgb}{1.000000,0.000000,0.000000}%
\pgfsetstrokecolor{currentstroke}%
\pgfsetdash{}{0pt}%
\pgfpathmoveto{\pgfqpoint{1.850823in}{5.592223in}}%
\pgfpathlineto{\pgfqpoint{2.130823in}{5.592223in}}%
\pgfusepath{stroke}%
\end{pgfscope}%
\begin{pgfscope}%
\pgfsetbuttcap%
\pgfsetmiterjoin%
\definecolor{currentfill}{rgb}{1.000000,0.000000,0.000000}%
\pgfsetfillcolor{currentfill}%
\pgfsetlinewidth{0.501875pt}%
\definecolor{currentstroke}{rgb}{0.000000,0.000000,0.000000}%
\pgfsetstrokecolor{currentstroke}%
\pgfsetdash{}{0pt}%
\pgfsys@defobject{currentmarker}{\pgfqpoint{-0.041667in}{-0.041667in}}{\pgfqpoint{0.041667in}{0.041667in}}{%
\pgfpathmoveto{\pgfqpoint{0.041667in}{-0.000000in}}%
\pgfpathlineto{\pgfqpoint{-0.041667in}{0.041667in}}%
\pgfpathlineto{\pgfqpoint{-0.041667in}{-0.041667in}}%
\pgfpathclose%
\pgfusepath{stroke,fill}%
}%
\begin{pgfscope}%
\pgfsys@transformshift{1.850823in}{5.592223in}%
\pgfsys@useobject{currentmarker}{}%
\end{pgfscope}%
\begin{pgfscope}%
\pgfsys@transformshift{2.130823in}{5.592223in}%
\pgfsys@useobject{currentmarker}{}%
\end{pgfscope}%
\end{pgfscope}%
\begin{pgfscope}%
\pgftext[left,bottom,x=2.350823in,y=5.483334in,rotate=0.000000]{{\rmfamily\fontsize{14.400000}{17.280000}\selectfont 100 nodes, horizon 0.20}}
%
\end{pgfscope}%
\end{pgfpicture}%
\makeatother%
\endgroup%
}
  \caption{The uniform-load elastic beam is accurately modeled with few nodes}
  \label{fig:elastic_g2000}
\end{figure}

As an elastic-perfectly-plastic beam exceeds the elastic limit of its material, plastic zones begin to grow on the top and bottom of the beam's cross section.
This behavior is mimicked by the plasticity of the longest bond-pairs, producing the results shown in \cref{fig:eppu_h10_g2000}.
To accurately capture this phenomenon and model beam plasticity, a finer discretization is required.
\begin{figure}[h]
  \centering
  \resizebox{0.5\linewidth}{!}{%% Creator: Matplotlib, PGF backend
%%
%% To include the figure in your LaTeX document, write
%%   \input{<filename>.pgf}
%%
%% Make sure the required packages are loaded in your preamble
%%   \usepackage{pgf}
%%
%% Figures using additional raster images can only be included by \input if
%% they are in the same directory as the main LaTeX file. For loading figures
%% from other directories you can use the `import` package
%%   \usepackage{import}
%% and then include the figures with
%%   \import{<path to file>}{<filename>.pgf}
%%
%% Matplotlib used the following preamble
%%
\begingroup%
\makeatletter%
\begin{pgfpicture}%
\pgfpathrectangle{\pgfpointorigin}{\pgfqpoint{6.000000in}{7.000000in}}%
\pgfusepath{use as bounding box}%
\begin{pgfscope}%
\pgfsetrectcap%
\pgfsetroundjoin%
\definecolor{currentfill}{rgb}{1.000000,1.000000,1.000000}%
\pgfsetfillcolor{currentfill}%
\pgfsetlinewidth{0.000000pt}%
\definecolor{currentstroke}{rgb}{1.000000,1.000000,1.000000}%
\pgfsetstrokecolor{currentstroke}%
\pgfsetdash{}{0pt}%
\pgfpathmoveto{\pgfqpoint{0.000000in}{0.000000in}}%
\pgfpathlineto{\pgfqpoint{6.000000in}{0.000000in}}%
\pgfpathlineto{\pgfqpoint{6.000000in}{7.000000in}}%
\pgfpathlineto{\pgfqpoint{0.000000in}{7.000000in}}%
\pgfpathclose%
\pgfusepath{fill}%
\end{pgfscope}%
\begin{pgfscope}%
\pgfsetrectcap%
\pgfsetroundjoin%
\definecolor{currentfill}{rgb}{1.000000,1.000000,1.000000}%
\pgfsetfillcolor{currentfill}%
\pgfsetlinewidth{0.000000pt}%
\definecolor{currentstroke}{rgb}{0.000000,0.000000,0.000000}%
\pgfsetstrokecolor{currentstroke}%
\pgfsetdash{}{0pt}%
\pgfpathmoveto{\pgfqpoint{0.750000in}{0.700000in}}%
\pgfpathlineto{\pgfqpoint{5.400000in}{0.700000in}}%
\pgfpathlineto{\pgfqpoint{5.400000in}{6.300000in}}%
\pgfpathlineto{\pgfqpoint{0.750000in}{6.300000in}}%
\pgfpathclose%
\pgfusepath{fill}%
\end{pgfscope}%
\begin{pgfscope}%
\pgfpathrectangle{\pgfqpoint{0.750000in}{0.700000in}}{\pgfqpoint{4.650000in}{5.600000in}} %
\pgfusepath{clip}%
\pgfsetrectcap%
\pgfsetroundjoin%
\pgfsetlinewidth{1.003750pt}%
\definecolor{currentstroke}{rgb}{0.000000,0.000000,1.000000}%
\pgfsetstrokecolor{currentstroke}%
\pgfsetdash{}{0pt}%
\pgfpathmoveto{\pgfqpoint{0.750000in}{6.300000in}}%
\pgfpathlineto{\pgfqpoint{0.989475in}{5.555416in}}%
\pgfpathlineto{\pgfqpoint{1.119675in}{5.157984in}}%
\pgfpathlineto{\pgfqpoint{1.228950in}{4.831380in}}%
\pgfpathlineto{\pgfqpoint{1.326600in}{4.546372in}}%
\pgfpathlineto{\pgfqpoint{1.414950in}{4.295076in}}%
\pgfpathlineto{\pgfqpoint{1.498650in}{4.063524in}}%
\pgfpathlineto{\pgfqpoint{1.577700in}{3.851252in}}%
\pgfpathlineto{\pgfqpoint{1.652100in}{3.657636in}}%
\pgfpathlineto{\pgfqpoint{1.724175in}{3.476172in}}%
\pgfpathlineto{\pgfqpoint{1.793925in}{3.306620in}}%
\pgfpathlineto{\pgfqpoint{1.861350in}{3.148676in}}%
\pgfpathlineto{\pgfqpoint{1.924125in}{3.007116in}}%
\pgfpathlineto{\pgfqpoint{1.984575in}{2.875996in}}%
\pgfpathlineto{\pgfqpoint{2.045025in}{2.750152in}}%
\pgfpathlineto{\pgfqpoint{2.103150in}{2.634308in}}%
\pgfpathlineto{\pgfqpoint{2.156625in}{2.532416in}}%
\pgfpathlineto{\pgfqpoint{2.210100in}{2.435220in}}%
\pgfpathlineto{\pgfqpoint{2.261250in}{2.346820in}}%
\pgfpathlineto{\pgfqpoint{2.310075in}{2.266760in}}%
\pgfpathlineto{\pgfqpoint{2.354250in}{2.198040in}}%
\pgfpathlineto{\pgfqpoint{2.400750in}{2.129640in}}%
\pgfpathlineto{\pgfqpoint{2.444925in}{2.068480in}}%
\pgfpathlineto{\pgfqpoint{2.486775in}{2.014040in}}%
\pgfpathlineto{\pgfqpoint{2.528625in}{1.963080in}}%
\pgfpathlineto{\pgfqpoint{2.570475in}{1.915640in}}%
\pgfpathlineto{\pgfqpoint{2.612325in}{1.871800in}}%
\pgfpathlineto{\pgfqpoint{2.654175in}{1.831600in}}%
\pgfpathlineto{\pgfqpoint{2.686725in}{1.802840in}}%
\pgfpathlineto{\pgfqpoint{2.726250in}{1.770960in}}%
\pgfpathlineto{\pgfqpoint{2.765775in}{1.742400in}}%
\pgfpathlineto{\pgfqpoint{2.802975in}{1.718640in}}%
\pgfpathlineto{\pgfqpoint{2.835525in}{1.700280in}}%
\pgfpathlineto{\pgfqpoint{2.865750in}{1.685280in}}%
\pgfpathlineto{\pgfqpoint{2.905275in}{1.668720in}}%
\pgfpathlineto{\pgfqpoint{2.944800in}{1.655600in}}%
\pgfpathlineto{\pgfqpoint{2.975025in}{1.647880in}}%
\pgfpathlineto{\pgfqpoint{3.005250in}{1.642200in}}%
\pgfpathlineto{\pgfqpoint{3.035475in}{1.638520in}}%
\pgfpathlineto{\pgfqpoint{3.065700in}{1.636880in}}%
\pgfpathlineto{\pgfqpoint{3.098250in}{1.637400in}}%
\pgfpathlineto{\pgfqpoint{3.128475in}{1.639960in}}%
\pgfpathlineto{\pgfqpoint{3.158700in}{1.644560in}}%
\pgfpathlineto{\pgfqpoint{3.191250in}{1.651800in}}%
\pgfpathlineto{\pgfqpoint{3.221475in}{1.660600in}}%
\pgfpathlineto{\pgfqpoint{3.254025in}{1.672320in}}%
\pgfpathlineto{\pgfqpoint{3.286575in}{1.686360in}}%
\pgfpathlineto{\pgfqpoint{3.326100in}{1.706560in}}%
\pgfpathlineto{\pgfqpoint{3.358650in}{1.725720in}}%
\pgfpathlineto{\pgfqpoint{3.398175in}{1.752120in}}%
\pgfpathlineto{\pgfqpoint{3.430725in}{1.776360in}}%
\pgfpathlineto{\pgfqpoint{3.463275in}{1.802840in}}%
\pgfpathlineto{\pgfqpoint{3.505125in}{1.840200in}}%
\pgfpathlineto{\pgfqpoint{3.546975in}{1.881240in}}%
\pgfpathlineto{\pgfqpoint{3.588825in}{1.925880in}}%
\pgfpathlineto{\pgfqpoint{3.633000in}{1.976880in}}%
\pgfpathlineto{\pgfqpoint{3.677175in}{2.031800in}}%
\pgfpathlineto{\pgfqpoint{3.721350in}{2.090560in}}%
\pgfpathlineto{\pgfqpoint{3.765525in}{2.153120in}}%
\pgfpathlineto{\pgfqpoint{3.812025in}{2.222920in}}%
\pgfpathlineto{\pgfqpoint{3.860850in}{2.300540in}}%
\pgfpathlineto{\pgfqpoint{3.909675in}{2.382436in}}%
\pgfpathlineto{\pgfqpoint{3.960825in}{2.472684in}}%
\pgfpathlineto{\pgfqpoint{4.014300in}{2.571736in}}%
\pgfpathlineto{\pgfqpoint{4.070100in}{2.680020in}}%
\pgfpathlineto{\pgfqpoint{4.128225in}{2.797920in}}%
\pgfpathlineto{\pgfqpoint{4.186350in}{2.920796in}}%
\pgfpathlineto{\pgfqpoint{4.246800in}{3.053700in}}%
\pgfpathlineto{\pgfqpoint{4.309575in}{3.197052in}}%
\pgfpathlineto{\pgfqpoint{4.374675in}{3.351236in}}%
\pgfpathlineto{\pgfqpoint{4.442100in}{3.516604in}}%
\pgfpathlineto{\pgfqpoint{4.514175in}{3.699456in}}%
\pgfpathlineto{\pgfqpoint{4.588575in}{3.894420in}}%
\pgfpathlineto{\pgfqpoint{4.665300in}{4.101648in}}%
\pgfpathlineto{\pgfqpoint{4.746675in}{4.327756in}}%
\pgfpathlineto{\pgfqpoint{4.832700in}{4.573200in}}%
\pgfpathlineto{\pgfqpoint{4.928025in}{4.852000in}}%
\pgfpathlineto{\pgfqpoint{5.032650in}{5.165008in}}%
\pgfpathlineto{\pgfqpoint{5.153550in}{5.533944in}}%
\pgfpathlineto{\pgfqpoint{5.302350in}{5.995179in}}%
\pgfpathlineto{\pgfqpoint{5.400000in}{6.300000in}}%
\pgfpathlineto{\pgfqpoint{5.400000in}{6.300000in}}%
\pgfusepath{stroke}%
\end{pgfscope}%
\begin{pgfscope}%
\pgfpathrectangle{\pgfqpoint{0.750000in}{0.700000in}}{\pgfqpoint{4.650000in}{5.600000in}} %
\pgfusepath{clip}%
\pgfsetbuttcap%
\pgfsetroundjoin%
\definecolor{currentfill}{rgb}{0.000000,0.000000,1.000000}%
\pgfsetfillcolor{currentfill}%
\pgfsetlinewidth{0.501875pt}%
\definecolor{currentstroke}{rgb}{0.000000,0.000000,0.000000}%
\pgfsetstrokecolor{currentstroke}%
\pgfsetdash{}{0pt}%
\pgfsys@defobject{currentmarker}{\pgfqpoint{-0.041667in}{-0.041667in}}{\pgfqpoint{0.041667in}{0.041667in}}{%
\pgfpathmoveto{\pgfqpoint{0.000000in}{-0.041667in}}%
\pgfpathcurveto{\pgfqpoint{0.011050in}{-0.041667in}}{\pgfqpoint{0.021649in}{-0.037276in}}{\pgfqpoint{0.029463in}{-0.029463in}}%
\pgfpathcurveto{\pgfqpoint{0.037276in}{-0.021649in}}{\pgfqpoint{0.041667in}{-0.011050in}}{\pgfqpoint{0.041667in}{0.000000in}}%
\pgfpathcurveto{\pgfqpoint{0.041667in}{0.011050in}}{\pgfqpoint{0.037276in}{0.021649in}}{\pgfqpoint{0.029463in}{0.029463in}}%
\pgfpathcurveto{\pgfqpoint{0.021649in}{0.037276in}}{\pgfqpoint{0.011050in}{0.041667in}}{\pgfqpoint{0.000000in}{0.041667in}}%
\pgfpathcurveto{\pgfqpoint{-0.011050in}{0.041667in}}{\pgfqpoint{-0.021649in}{0.037276in}}{\pgfqpoint{-0.029463in}{0.029463in}}%
\pgfpathcurveto{\pgfqpoint{-0.037276in}{0.021649in}}{\pgfqpoint{-0.041667in}{0.011050in}}{\pgfqpoint{-0.041667in}{0.000000in}}%
\pgfpathcurveto{\pgfqpoint{-0.041667in}{-0.011050in}}{\pgfqpoint{-0.037276in}{-0.021649in}}{\pgfqpoint{-0.029463in}{-0.029463in}}%
\pgfpathcurveto{\pgfqpoint{-0.021649in}{-0.037276in}}{\pgfqpoint{-0.011050in}{-0.041667in}}{\pgfqpoint{0.000000in}{-0.041667in}}%
\pgfpathclose%
\pgfusepath{stroke,fill}%
}%
\begin{pgfscope}%
\pgfsys@transformshift{0.773250in}{6.227369in}%
\pgfsys@useobject{currentmarker}{}%
\end{pgfscope}%
\begin{pgfscope}%
\pgfsys@transformshift{1.238250in}{4.803936in}%
\pgfsys@useobject{currentmarker}{}%
\end{pgfscope}%
\begin{pgfscope}%
\pgfsys@transformshift{1.703250in}{3.528216in}%
\pgfsys@useobject{currentmarker}{}%
\end{pgfscope}%
\begin{pgfscope}%
\pgfsys@transformshift{2.168250in}{2.510880in}%
\pgfsys@useobject{currentmarker}{}%
\end{pgfscope}%
\begin{pgfscope}%
\pgfsys@transformshift{2.633250in}{1.851240in}%
\pgfsys@useobject{currentmarker}{}%
\end{pgfscope}%
\begin{pgfscope}%
\pgfsys@transformshift{3.098250in}{1.637400in}%
\pgfsys@useobject{currentmarker}{}%
\end{pgfscope}%
\begin{pgfscope}%
\pgfsys@transformshift{3.563250in}{1.898160in}%
\pgfsys@useobject{currentmarker}{}%
\end{pgfscope}%
\begin{pgfscope}%
\pgfsys@transformshift{4.028250in}{2.598344in}%
\pgfsys@useobject{currentmarker}{}%
\end{pgfscope}%
\begin{pgfscope}%
\pgfsys@transformshift{4.493250in}{3.645744in}%
\pgfsys@useobject{currentmarker}{}%
\end{pgfscope}%
\begin{pgfscope}%
\pgfsys@transformshift{4.958250in}{4.941728in}%
\pgfsys@useobject{currentmarker}{}%
\end{pgfscope}%
\end{pgfscope}%
\begin{pgfscope}%
\pgfpathrectangle{\pgfqpoint{0.750000in}{0.700000in}}{\pgfqpoint{4.650000in}{5.600000in}} %
\pgfusepath{clip}%
\pgfsetrectcap%
\pgfsetroundjoin%
\pgfsetlinewidth{1.003750pt}%
\definecolor{currentstroke}{rgb}{0.000000,0.500000,0.000000}%
\pgfsetstrokecolor{currentstroke}%
\pgfsetdash{}{0pt}%
\pgfpathmoveto{\pgfqpoint{0.750000in}{6.300000in}}%
\pgfpathlineto{\pgfqpoint{1.029000in}{5.381331in}}%
\pgfpathlineto{\pgfqpoint{1.168500in}{4.929915in}}%
\pgfpathlineto{\pgfqpoint{1.261500in}{4.634467in}}%
\pgfpathlineto{\pgfqpoint{1.354500in}{4.344651in}}%
\pgfpathlineto{\pgfqpoint{1.447500in}{4.061441in}}%
\pgfpathlineto{\pgfqpoint{1.540500in}{3.785761in}}%
\pgfpathlineto{\pgfqpoint{1.633500in}{3.518485in}}%
\pgfpathlineto{\pgfqpoint{1.726500in}{3.260440in}}%
\pgfpathlineto{\pgfqpoint{1.819500in}{3.012390in}}%
\pgfpathlineto{\pgfqpoint{1.912500in}{2.775066in}}%
\pgfpathlineto{\pgfqpoint{2.005500in}{2.549141in}}%
\pgfpathlineto{\pgfqpoint{2.052000in}{2.440631in}}%
\pgfpathlineto{\pgfqpoint{2.098500in}{2.335193in}}%
\pgfpathlineto{\pgfqpoint{2.145000in}{2.232874in}}%
\pgfpathlineto{\pgfqpoint{2.191500in}{2.133777in}}%
\pgfpathlineto{\pgfqpoint{2.238000in}{2.038038in}}%
\pgfpathlineto{\pgfqpoint{2.284500in}{1.945809in}}%
\pgfpathlineto{\pgfqpoint{2.331000in}{1.857167in}}%
\pgfpathlineto{\pgfqpoint{2.377500in}{1.772254in}}%
\pgfpathlineto{\pgfqpoint{2.424000in}{1.691190in}}%
\pgfpathlineto{\pgfqpoint{2.470500in}{1.614026in}}%
\pgfpathlineto{\pgfqpoint{2.517000in}{1.541015in}}%
\pgfpathlineto{\pgfqpoint{2.563500in}{1.472365in}}%
\pgfpathlineto{\pgfqpoint{2.610000in}{1.408312in}}%
\pgfpathlineto{\pgfqpoint{2.656500in}{1.349118in}}%
\pgfpathlineto{\pgfqpoint{2.703000in}{1.294912in}}%
\pgfpathlineto{\pgfqpoint{2.749500in}{1.245847in}}%
\pgfpathlineto{\pgfqpoint{2.796000in}{1.202188in}}%
\pgfpathlineto{\pgfqpoint{2.842500in}{1.164352in}}%
\pgfpathlineto{\pgfqpoint{2.889000in}{1.132778in}}%
\pgfpathlineto{\pgfqpoint{2.935500in}{1.107851in}}%
\pgfpathlineto{\pgfqpoint{2.982000in}{1.089841in}}%
\pgfpathlineto{\pgfqpoint{3.028500in}{1.078955in}}%
\pgfpathlineto{\pgfqpoint{3.075000in}{1.075313in}}%
\pgfpathlineto{\pgfqpoint{3.121500in}{1.078955in}}%
\pgfpathlineto{\pgfqpoint{3.168000in}{1.089841in}}%
\pgfpathlineto{\pgfqpoint{3.214500in}{1.107851in}}%
\pgfpathlineto{\pgfqpoint{3.261000in}{1.132778in}}%
\pgfpathlineto{\pgfqpoint{3.307500in}{1.164352in}}%
\pgfpathlineto{\pgfqpoint{3.354000in}{1.202188in}}%
\pgfpathlineto{\pgfqpoint{3.400500in}{1.245847in}}%
\pgfpathlineto{\pgfqpoint{3.447000in}{1.294912in}}%
\pgfpathlineto{\pgfqpoint{3.493500in}{1.349118in}}%
\pgfpathlineto{\pgfqpoint{3.540000in}{1.408312in}}%
\pgfpathlineto{\pgfqpoint{3.586500in}{1.472365in}}%
\pgfpathlineto{\pgfqpoint{3.633000in}{1.541015in}}%
\pgfpathlineto{\pgfqpoint{3.679500in}{1.614026in}}%
\pgfpathlineto{\pgfqpoint{3.726000in}{1.691190in}}%
\pgfpathlineto{\pgfqpoint{3.772500in}{1.772254in}}%
\pgfpathlineto{\pgfqpoint{3.819000in}{1.857167in}}%
\pgfpathlineto{\pgfqpoint{3.865500in}{1.945809in}}%
\pgfpathlineto{\pgfqpoint{3.912000in}{2.038038in}}%
\pgfpathlineto{\pgfqpoint{3.958500in}{2.133777in}}%
\pgfpathlineto{\pgfqpoint{4.005000in}{2.232874in}}%
\pgfpathlineto{\pgfqpoint{4.051500in}{2.335193in}}%
\pgfpathlineto{\pgfqpoint{4.098000in}{2.440631in}}%
\pgfpathlineto{\pgfqpoint{4.144500in}{2.549141in}}%
\pgfpathlineto{\pgfqpoint{4.191000in}{2.660645in}}%
\pgfpathlineto{\pgfqpoint{4.284000in}{2.892339in}}%
\pgfpathlineto{\pgfqpoint{4.377000in}{3.135120in}}%
\pgfpathlineto{\pgfqpoint{4.470000in}{3.388260in}}%
\pgfpathlineto{\pgfqpoint{4.563000in}{3.651020in}}%
\pgfpathlineto{\pgfqpoint{4.656000in}{3.922604in}}%
\pgfpathlineto{\pgfqpoint{4.749000in}{4.202161in}}%
\pgfpathlineto{\pgfqpoint{4.842000in}{4.488793in}}%
\pgfpathlineto{\pgfqpoint{4.935000in}{4.781546in}}%
\pgfpathlineto{\pgfqpoint{5.074500in}{5.229952in}}%
\pgfpathlineto{\pgfqpoint{5.214000in}{5.686363in}}%
\pgfpathlineto{\pgfqpoint{5.400000in}{6.300000in}}%
\pgfpathlineto{\pgfqpoint{5.400000in}{6.300000in}}%
\pgfusepath{stroke}%
\end{pgfscope}%
\begin{pgfscope}%
\pgfpathrectangle{\pgfqpoint{0.750000in}{0.700000in}}{\pgfqpoint{4.650000in}{5.600000in}} %
\pgfusepath{clip}%
\pgfsetbuttcap%
\pgfsetmiterjoin%
\definecolor{currentfill}{rgb}{0.000000,0.500000,0.000000}%
\pgfsetfillcolor{currentfill}%
\pgfsetlinewidth{0.501875pt}%
\definecolor{currentstroke}{rgb}{0.000000,0.000000,0.000000}%
\pgfsetstrokecolor{currentstroke}%
\pgfsetdash{}{0pt}%
\pgfsys@defobject{currentmarker}{\pgfqpoint{-0.041667in}{-0.041667in}}{\pgfqpoint{0.041667in}{0.041667in}}{%
\pgfpathmoveto{\pgfqpoint{0.000000in}{0.041667in}}%
\pgfpathlineto{\pgfqpoint{-0.041667in}{-0.041667in}}%
\pgfpathlineto{\pgfqpoint{0.041667in}{-0.041667in}}%
\pgfpathclose%
\pgfusepath{stroke,fill}%
}%
\begin{pgfscope}%
\pgfsys@transformshift{0.843000in}{5.992974in}%
\pgfsys@useobject{currentmarker}{}%
\end{pgfscope}%
\begin{pgfscope}%
\pgfsys@transformshift{1.308000in}{4.488793in}%
\pgfsys@useobject{currentmarker}{}%
\end{pgfscope}%
\begin{pgfscope}%
\pgfsys@transformshift{1.773000in}{3.135120in}%
\pgfsys@useobject{currentmarker}{}%
\end{pgfscope}%
\begin{pgfscope}%
\pgfsys@transformshift{2.238000in}{2.038038in}%
\pgfsys@useobject{currentmarker}{}%
\end{pgfscope}%
\begin{pgfscope}%
\pgfsys@transformshift{2.703000in}{1.294912in}%
\pgfsys@useobject{currentmarker}{}%
\end{pgfscope}%
\begin{pgfscope}%
\pgfsys@transformshift{3.168000in}{1.089841in}%
\pgfsys@useobject{currentmarker}{}%
\end{pgfscope}%
\begin{pgfscope}%
\pgfsys@transformshift{3.633000in}{1.541015in}%
\pgfsys@useobject{currentmarker}{}%
\end{pgfscope}%
\begin{pgfscope}%
\pgfsys@transformshift{4.098000in}{2.440631in}%
\pgfsys@useobject{currentmarker}{}%
\end{pgfscope}%
\begin{pgfscope}%
\pgfsys@transformshift{4.563000in}{3.651020in}%
\pgfsys@useobject{currentmarker}{}%
\end{pgfscope}%
\begin{pgfscope}%
\pgfsys@transformshift{5.028000in}{5.079429in}%
\pgfsys@useobject{currentmarker}{}%
\end{pgfscope}%
\end{pgfscope}%
\begin{pgfscope}%
\pgfpathrectangle{\pgfqpoint{0.750000in}{0.700000in}}{\pgfqpoint{4.650000in}{5.600000in}} %
\pgfusepath{clip}%
\pgfsetrectcap%
\pgfsetroundjoin%
\pgfsetlinewidth{1.003750pt}%
\definecolor{currentstroke}{rgb}{1.000000,0.000000,0.000000}%
\pgfsetstrokecolor{currentstroke}%
\pgfsetdash{}{0pt}%
\pgfpathmoveto{\pgfqpoint{0.750000in}{6.300000in}}%
\pgfpathlineto{\pgfqpoint{1.005750in}{5.489523in}}%
\pgfpathlineto{\pgfqpoint{1.145250in}{5.055530in}}%
\pgfpathlineto{\pgfqpoint{1.261500in}{4.701770in}}%
\pgfpathlineto{\pgfqpoint{1.354500in}{4.425409in}}%
\pgfpathlineto{\pgfqpoint{1.447500in}{4.156015in}}%
\pgfpathlineto{\pgfqpoint{1.540500in}{3.894506in}}%
\pgfpathlineto{\pgfqpoint{1.610250in}{3.704077in}}%
\pgfpathlineto{\pgfqpoint{1.680000in}{3.518925in}}%
\pgfpathlineto{\pgfqpoint{1.749750in}{3.339380in}}%
\pgfpathlineto{\pgfqpoint{1.819500in}{3.165758in}}%
\pgfpathlineto{\pgfqpoint{1.889250in}{2.998371in}}%
\pgfpathlineto{\pgfqpoint{1.959000in}{2.837486in}}%
\pgfpathlineto{\pgfqpoint{2.028750in}{2.683366in}}%
\pgfpathlineto{\pgfqpoint{2.098500in}{2.536278in}}%
\pgfpathlineto{\pgfqpoint{2.168250in}{2.396470in}}%
\pgfpathlineto{\pgfqpoint{2.214750in}{2.307468in}}%
\pgfpathlineto{\pgfqpoint{2.261250in}{2.221937in}}%
\pgfpathlineto{\pgfqpoint{2.307750in}{2.139967in}}%
\pgfpathlineto{\pgfqpoint{2.354250in}{2.061691in}}%
\pgfpathlineto{\pgfqpoint{2.400750in}{1.987213in}}%
\pgfpathlineto{\pgfqpoint{2.447250in}{1.916638in}}%
\pgfpathlineto{\pgfqpoint{2.493750in}{1.850123in}}%
\pgfpathlineto{\pgfqpoint{2.540250in}{1.787802in}}%
\pgfpathlineto{\pgfqpoint{2.586750in}{1.729814in}}%
\pgfpathlineto{\pgfqpoint{2.633250in}{1.676301in}}%
\pgfpathlineto{\pgfqpoint{2.679750in}{1.627429in}}%
\pgfpathlineto{\pgfqpoint{2.726250in}{1.583389in}}%
\pgfpathlineto{\pgfqpoint{2.772750in}{1.544340in}}%
\pgfpathlineto{\pgfqpoint{2.819250in}{1.510460in}}%
\pgfpathlineto{\pgfqpoint{2.842500in}{1.495491in}}%
\pgfpathlineto{\pgfqpoint{2.865750in}{1.481864in}}%
\pgfpathlineto{\pgfqpoint{2.889000in}{1.469604in}}%
\pgfpathlineto{\pgfqpoint{2.912250in}{1.458731in}}%
\pgfpathlineto{\pgfqpoint{2.935500in}{1.449265in}}%
\pgfpathlineto{\pgfqpoint{2.958750in}{1.441222in}}%
\pgfpathlineto{\pgfqpoint{2.982000in}{1.434623in}}%
\pgfpathlineto{\pgfqpoint{3.005250in}{1.429480in}}%
\pgfpathlineto{\pgfqpoint{3.028500in}{1.425799in}}%
\pgfpathlineto{\pgfqpoint{3.051750in}{1.423587in}}%
\pgfpathlineto{\pgfqpoint{3.075000in}{1.422849in}}%
\pgfpathlineto{\pgfqpoint{3.098250in}{1.423587in}}%
\pgfpathlineto{\pgfqpoint{3.121500in}{1.425799in}}%
\pgfpathlineto{\pgfqpoint{3.144750in}{1.429480in}}%
\pgfpathlineto{\pgfqpoint{3.168000in}{1.434623in}}%
\pgfpathlineto{\pgfqpoint{3.191250in}{1.441222in}}%
\pgfpathlineto{\pgfqpoint{3.214500in}{1.449265in}}%
\pgfpathlineto{\pgfqpoint{3.237750in}{1.458731in}}%
\pgfpathlineto{\pgfqpoint{3.261000in}{1.469604in}}%
\pgfpathlineto{\pgfqpoint{3.284250in}{1.481864in}}%
\pgfpathlineto{\pgfqpoint{3.307500in}{1.495491in}}%
\pgfpathlineto{\pgfqpoint{3.354000in}{1.526746in}}%
\pgfpathlineto{\pgfqpoint{3.400500in}{1.563230in}}%
\pgfpathlineto{\pgfqpoint{3.447000in}{1.604794in}}%
\pgfpathlineto{\pgfqpoint{3.493500in}{1.651273in}}%
\pgfpathlineto{\pgfqpoint{3.540000in}{1.702491in}}%
\pgfpathlineto{\pgfqpoint{3.586500in}{1.758257in}}%
\pgfpathlineto{\pgfqpoint{3.633000in}{1.818430in}}%
\pgfpathlineto{\pgfqpoint{3.679500in}{1.882864in}}%
\pgfpathlineto{\pgfqpoint{3.726000in}{1.951434in}}%
\pgfpathlineto{\pgfqpoint{3.772500in}{2.023970in}}%
\pgfpathlineto{\pgfqpoint{3.819000in}{2.100361in}}%
\pgfpathlineto{\pgfqpoint{3.865500in}{2.180498in}}%
\pgfpathlineto{\pgfqpoint{3.912000in}{2.264263in}}%
\pgfpathlineto{\pgfqpoint{3.958500in}{2.351541in}}%
\pgfpathlineto{\pgfqpoint{4.005000in}{2.442244in}}%
\pgfpathlineto{\pgfqpoint{4.074750in}{2.584516in}}%
\pgfpathlineto{\pgfqpoint{4.144500in}{2.733971in}}%
\pgfpathlineto{\pgfqpoint{4.214250in}{2.890376in}}%
\pgfpathlineto{\pgfqpoint{4.284000in}{3.053456in}}%
\pgfpathlineto{\pgfqpoint{4.353750in}{3.222955in}}%
\pgfpathlineto{\pgfqpoint{4.423500in}{3.398586in}}%
\pgfpathlineto{\pgfqpoint{4.493250in}{3.580035in}}%
\pgfpathlineto{\pgfqpoint{4.563000in}{3.766984in}}%
\pgfpathlineto{\pgfqpoint{4.632750in}{3.959095in}}%
\pgfpathlineto{\pgfqpoint{4.725750in}{4.222659in}}%
\pgfpathlineto{\pgfqpoint{4.818750in}{4.493883in}}%
\pgfpathlineto{\pgfqpoint{4.911750in}{4.771835in}}%
\pgfpathlineto{\pgfqpoint{5.028000in}{5.127236in}}%
\pgfpathlineto{\pgfqpoint{5.167500in}{5.562489in}}%
\pgfpathlineto{\pgfqpoint{5.400000in}{6.300000in}}%
\pgfpathlineto{\pgfqpoint{5.400000in}{6.300000in}}%
\pgfusepath{stroke}%
\end{pgfscope}%
\begin{pgfscope}%
\pgfpathrectangle{\pgfqpoint{0.750000in}{0.700000in}}{\pgfqpoint{4.650000in}{5.600000in}} %
\pgfusepath{clip}%
\pgfsetbuttcap%
\pgfsetmiterjoin%
\definecolor{currentfill}{rgb}{1.000000,0.000000,0.000000}%
\pgfsetfillcolor{currentfill}%
\pgfsetlinewidth{0.501875pt}%
\definecolor{currentstroke}{rgb}{0.000000,0.000000,0.000000}%
\pgfsetstrokecolor{currentstroke}%
\pgfsetdash{}{0pt}%
\pgfsys@defobject{currentmarker}{\pgfqpoint{-0.041667in}{-0.041667in}}{\pgfqpoint{0.041667in}{0.041667in}}{%
\pgfpathmoveto{\pgfqpoint{0.041667in}{-0.000000in}}%
\pgfpathlineto{\pgfqpoint{-0.041667in}{0.041667in}}%
\pgfpathlineto{\pgfqpoint{-0.041667in}{-0.041667in}}%
\pgfpathclose%
\pgfusepath{stroke,fill}%
}%
\begin{pgfscope}%
\pgfsys@transformshift{0.982500in}{5.562489in}%
\pgfsys@useobject{currentmarker}{}%
\end{pgfscope}%
\begin{pgfscope}%
\pgfsys@transformshift{1.447500in}{4.156015in}%
\pgfsys@useobject{currentmarker}{}%
\end{pgfscope}%
\begin{pgfscope}%
\pgfsys@transformshift{1.912500in}{2.944007in}%
\pgfsys@useobject{currentmarker}{}%
\end{pgfscope}%
\begin{pgfscope}%
\pgfsys@transformshift{2.377500in}{2.023970in}%
\pgfsys@useobject{currentmarker}{}%
\end{pgfscope}%
\begin{pgfscope}%
\pgfsys@transformshift{2.842500in}{1.495491in}%
\pgfsys@useobject{currentmarker}{}%
\end{pgfscope}%
\begin{pgfscope}%
\pgfsys@transformshift{3.307500in}{1.495491in}%
\pgfsys@useobject{currentmarker}{}%
\end{pgfscope}%
\begin{pgfscope}%
\pgfsys@transformshift{3.772500in}{2.023970in}%
\pgfsys@useobject{currentmarker}{}%
\end{pgfscope}%
\begin{pgfscope}%
\pgfsys@transformshift{4.237500in}{2.944007in}%
\pgfsys@useobject{currentmarker}{}%
\end{pgfscope}%
\begin{pgfscope}%
\pgfsys@transformshift{4.702500in}{4.156015in}%
\pgfsys@useobject{currentmarker}{}%
\end{pgfscope}%
\begin{pgfscope}%
\pgfsys@transformshift{5.167500in}{5.562489in}%
\pgfsys@useobject{currentmarker}{}%
\end{pgfscope}%
\end{pgfscope}%
\begin{pgfscope}%
\pgfpathrectangle{\pgfqpoint{0.750000in}{0.700000in}}{\pgfqpoint{4.650000in}{5.600000in}} %
\pgfusepath{clip}%
\pgfsetrectcap%
\pgfsetroundjoin%
\pgfsetlinewidth{1.003750pt}%
\definecolor{currentstroke}{rgb}{0.000000,0.750000,0.750000}%
\pgfsetstrokecolor{currentstroke}%
\pgfsetdash{}{0pt}%
\pgfpathmoveto{\pgfqpoint{0.750000in}{6.300000in}}%
\pgfpathlineto{\pgfqpoint{0.991800in}{5.551673in}}%
\pgfpathlineto{\pgfqpoint{1.122000in}{5.155943in}}%
\pgfpathlineto{\pgfqpoint{1.233600in}{4.823776in}}%
\pgfpathlineto{\pgfqpoint{1.335900in}{4.526649in}}%
\pgfpathlineto{\pgfqpoint{1.428900in}{4.263778in}}%
\pgfpathlineto{\pgfqpoint{1.512600in}{4.033890in}}%
\pgfpathlineto{\pgfqpoint{1.587000in}{3.835413in}}%
\pgfpathlineto{\pgfqpoint{1.661400in}{3.642905in}}%
\pgfpathlineto{\pgfqpoint{1.735800in}{3.456775in}}%
\pgfpathlineto{\pgfqpoint{1.800900in}{3.299448in}}%
\pgfpathlineto{\pgfqpoint{1.866000in}{3.147549in}}%
\pgfpathlineto{\pgfqpoint{1.931100in}{3.001311in}}%
\pgfpathlineto{\pgfqpoint{1.996200in}{2.860953in}}%
\pgfpathlineto{\pgfqpoint{2.052000in}{2.745488in}}%
\pgfpathlineto{\pgfqpoint{2.107800in}{2.634618in}}%
\pgfpathlineto{\pgfqpoint{2.163600in}{2.528482in}}%
\pgfpathlineto{\pgfqpoint{2.219400in}{2.427221in}}%
\pgfpathlineto{\pgfqpoint{2.275200in}{2.330979in}}%
\pgfpathlineto{\pgfqpoint{2.321700in}{2.254732in}}%
\pgfpathlineto{\pgfqpoint{2.368200in}{2.182182in}}%
\pgfpathlineto{\pgfqpoint{2.414700in}{2.113429in}}%
\pgfpathlineto{\pgfqpoint{2.461200in}{2.048584in}}%
\pgfpathlineto{\pgfqpoint{2.507700in}{1.987754in}}%
\pgfpathlineto{\pgfqpoint{2.554200in}{1.931056in}}%
\pgfpathlineto{\pgfqpoint{2.591400in}{1.888753in}}%
\pgfpathlineto{\pgfqpoint{2.628600in}{1.849235in}}%
\pgfpathlineto{\pgfqpoint{2.665800in}{1.812558in}}%
\pgfpathlineto{\pgfqpoint{2.703000in}{1.778793in}}%
\pgfpathlineto{\pgfqpoint{2.740200in}{1.748004in}}%
\pgfpathlineto{\pgfqpoint{2.777400in}{1.720245in}}%
\pgfpathlineto{\pgfqpoint{2.814600in}{1.695584in}}%
\pgfpathlineto{\pgfqpoint{2.851800in}{1.674078in}}%
\pgfpathlineto{\pgfqpoint{2.889000in}{1.655772in}}%
\pgfpathlineto{\pgfqpoint{2.916900in}{1.644180in}}%
\pgfpathlineto{\pgfqpoint{2.944800in}{1.634436in}}%
\pgfpathlineto{\pgfqpoint{2.972700in}{1.626558in}}%
\pgfpathlineto{\pgfqpoint{3.000600in}{1.620558in}}%
\pgfpathlineto{\pgfqpoint{3.028500in}{1.616447in}}%
\pgfpathlineto{\pgfqpoint{3.056400in}{1.614232in}}%
\pgfpathlineto{\pgfqpoint{3.084300in}{1.613916in}}%
\pgfpathlineto{\pgfqpoint{3.112200in}{1.615498in}}%
\pgfpathlineto{\pgfqpoint{3.140100in}{1.618977in}}%
\pgfpathlineto{\pgfqpoint{3.168000in}{1.624348in}}%
\pgfpathlineto{\pgfqpoint{3.195900in}{1.631602in}}%
\pgfpathlineto{\pgfqpoint{3.223800in}{1.640725in}}%
\pgfpathlineto{\pgfqpoint{3.251700in}{1.651704in}}%
\pgfpathlineto{\pgfqpoint{3.279600in}{1.664521in}}%
\pgfpathlineto{\pgfqpoint{3.316800in}{1.684434in}}%
\pgfpathlineto{\pgfqpoint{3.354000in}{1.707523in}}%
\pgfpathlineto{\pgfqpoint{3.391200in}{1.733741in}}%
\pgfpathlineto{\pgfqpoint{3.428400in}{1.763023in}}%
\pgfpathlineto{\pgfqpoint{3.465600in}{1.795308in}}%
\pgfpathlineto{\pgfqpoint{3.502800in}{1.830537in}}%
\pgfpathlineto{\pgfqpoint{3.540000in}{1.868642in}}%
\pgfpathlineto{\pgfqpoint{3.577200in}{1.909559in}}%
\pgfpathlineto{\pgfqpoint{3.614400in}{1.953234in}}%
\pgfpathlineto{\pgfqpoint{3.660900in}{2.011596in}}%
\pgfpathlineto{\pgfqpoint{3.707400in}{2.074047in}}%
\pgfpathlineto{\pgfqpoint{3.753900in}{2.140468in}}%
\pgfpathlineto{\pgfqpoint{3.800400in}{2.210751in}}%
\pgfpathlineto{\pgfqpoint{3.846900in}{2.284794in}}%
\pgfpathlineto{\pgfqpoint{3.893400in}{2.362491in}}%
\pgfpathlineto{\pgfqpoint{3.949200in}{2.460425in}}%
\pgfpathlineto{\pgfqpoint{4.005000in}{2.563327in}}%
\pgfpathlineto{\pgfqpoint{4.060800in}{2.671054in}}%
\pgfpathlineto{\pgfqpoint{4.116600in}{2.783472in}}%
\pgfpathlineto{\pgfqpoint{4.172400in}{2.900443in}}%
\pgfpathlineto{\pgfqpoint{4.237500in}{3.042502in}}%
\pgfpathlineto{\pgfqpoint{4.302600in}{3.190381in}}%
\pgfpathlineto{\pgfqpoint{4.367700in}{3.343855in}}%
\pgfpathlineto{\pgfqpoint{4.432800in}{3.502688in}}%
\pgfpathlineto{\pgfqpoint{4.507200in}{3.690450in}}%
\pgfpathlineto{\pgfqpoint{4.581600in}{3.884489in}}%
\pgfpathlineto{\pgfqpoint{4.656000in}{4.084393in}}%
\pgfpathlineto{\pgfqpoint{4.739700in}{4.315753in}}%
\pgfpathlineto{\pgfqpoint{4.832700in}{4.580086in}}%
\pgfpathlineto{\pgfqpoint{4.925700in}{4.851159in}}%
\pgfpathlineto{\pgfqpoint{5.028000in}{5.155943in}}%
\pgfpathlineto{\pgfqpoint{5.148900in}{5.523186in}}%
\pgfpathlineto{\pgfqpoint{5.297700in}{5.982337in}}%
\pgfpathlineto{\pgfqpoint{5.400000in}{6.300000in}}%
\pgfpathlineto{\pgfqpoint{5.400000in}{6.300000in}}%
\pgfusepath{stroke}%
\end{pgfscope}%
\begin{pgfscope}%
\pgfpathrectangle{\pgfqpoint{0.750000in}{0.700000in}}{\pgfqpoint{4.650000in}{5.600000in}} %
\pgfusepath{clip}%
\pgfsetbuttcap%
\pgfsetmiterjoin%
\definecolor{currentfill}{rgb}{0.000000,0.750000,0.750000}%
\pgfsetfillcolor{currentfill}%
\pgfsetlinewidth{0.501875pt}%
\definecolor{currentstroke}{rgb}{0.000000,0.000000,0.000000}%
\pgfsetstrokecolor{currentstroke}%
\pgfsetdash{}{0pt}%
\pgfsys@defobject{currentmarker}{\pgfqpoint{-0.041667in}{-0.041667in}}{\pgfqpoint{0.041667in}{0.041667in}}{%
\pgfpathmoveto{\pgfqpoint{-0.000000in}{-0.041667in}}%
\pgfpathlineto{\pgfqpoint{0.041667in}{0.041667in}}%
\pgfpathlineto{\pgfqpoint{-0.041667in}{0.041667in}}%
\pgfpathclose%
\pgfusepath{stroke,fill}%
}%
\begin{pgfscope}%
\pgfsys@transformshift{1.094100in}{5.240095in}%
\pgfsys@useobject{currentmarker}{}%
\end{pgfscope}%
\begin{pgfscope}%
\pgfsys@transformshift{1.559100in}{3.909165in}%
\pgfsys@useobject{currentmarker}{}%
\end{pgfscope}%
\begin{pgfscope}%
\pgfsys@transformshift{2.024100in}{2.802653in}%
\pgfsys@useobject{currentmarker}{}%
\end{pgfscope}%
\begin{pgfscope}%
\pgfsys@transformshift{2.489100in}{2.011596in}%
\pgfsys@useobject{currentmarker}{}%
\end{pgfscope}%
\begin{pgfscope}%
\pgfsys@transformshift{2.954100in}{1.631602in}%
\pgfsys@useobject{currentmarker}{}%
\end{pgfscope}%
\begin{pgfscope}%
\pgfsys@transformshift{3.419100in}{1.755420in}%
\pgfsys@useobject{currentmarker}{}%
\end{pgfscope}%
\begin{pgfscope}%
\pgfsys@transformshift{3.884100in}{2.346663in}%
\pgfsys@useobject{currentmarker}{}%
\end{pgfscope}%
\begin{pgfscope}%
\pgfsys@transformshift{4.349100in}{3.299448in}%
\pgfsys@useobject{currentmarker}{}%
\end{pgfscope}%
\begin{pgfscope}%
\pgfsys@transformshift{4.814100in}{4.526649in}%
\pgfsys@useobject{currentmarker}{}%
\end{pgfscope}%
\begin{pgfscope}%
\pgfsys@transformshift{5.279100in}{5.924682in}%
\pgfsys@useobject{currentmarker}{}%
\end{pgfscope}%
\end{pgfscope}%
\begin{pgfscope}%
\pgfpathrectangle{\pgfqpoint{0.750000in}{0.700000in}}{\pgfqpoint{4.650000in}{5.600000in}} %
\pgfusepath{clip}%
\pgfsetbuttcap%
\pgfsetroundjoin%
\pgfsetlinewidth{0.501875pt}%
\definecolor{currentstroke}{rgb}{0.000000,0.000000,0.000000}%
\pgfsetstrokecolor{currentstroke}%
\pgfsetdash{{1.000000pt}{3.000000pt}}{0.000000pt}%
\pgfpathmoveto{\pgfqpoint{0.750000in}{0.700000in}}%
\pgfpathlineto{\pgfqpoint{0.750000in}{6.300000in}}%
\pgfusepath{stroke}%
\end{pgfscope}%
\begin{pgfscope}%
\pgfsetbuttcap%
\pgfsetroundjoin%
\definecolor{currentfill}{rgb}{0.000000,0.000000,0.000000}%
\pgfsetfillcolor{currentfill}%
\pgfsetlinewidth{0.501875pt}%
\definecolor{currentstroke}{rgb}{0.000000,0.000000,0.000000}%
\pgfsetstrokecolor{currentstroke}%
\pgfsetdash{}{0pt}%
\pgfsys@defobject{currentmarker}{\pgfqpoint{0.000000in}{0.000000in}}{\pgfqpoint{0.000000in}{0.055556in}}{%
\pgfpathmoveto{\pgfqpoint{0.000000in}{0.000000in}}%
\pgfpathlineto{\pgfqpoint{0.000000in}{0.055556in}}%
\pgfusepath{stroke,fill}%
}%
\begin{pgfscope}%
\pgfsys@transformshift{0.750000in}{0.700000in}%
\pgfsys@useobject{currentmarker}{}%
\end{pgfscope}%
\end{pgfscope}%
\begin{pgfscope}%
\pgfsetbuttcap%
\pgfsetroundjoin%
\definecolor{currentfill}{rgb}{0.000000,0.000000,0.000000}%
\pgfsetfillcolor{currentfill}%
\pgfsetlinewidth{0.501875pt}%
\definecolor{currentstroke}{rgb}{0.000000,0.000000,0.000000}%
\pgfsetstrokecolor{currentstroke}%
\pgfsetdash{}{0pt}%
\pgfsys@defobject{currentmarker}{\pgfqpoint{0.000000in}{-0.055556in}}{\pgfqpoint{0.000000in}{0.000000in}}{%
\pgfpathmoveto{\pgfqpoint{0.000000in}{0.000000in}}%
\pgfpathlineto{\pgfqpoint{0.000000in}{-0.055556in}}%
\pgfusepath{stroke,fill}%
}%
\begin{pgfscope}%
\pgfsys@transformshift{0.750000in}{6.300000in}%
\pgfsys@useobject{currentmarker}{}%
\end{pgfscope}%
\end{pgfscope}%
\begin{pgfscope}%
\pgftext[left,bottom,x=0.645738in,y=0.537037in,rotate=0.000000]{{\rmfamily\fontsize{12.000000}{14.400000}\selectfont \(\displaystyle 0.0\)}}
%
\end{pgfscope}%
\begin{pgfscope}%
\pgfpathrectangle{\pgfqpoint{0.750000in}{0.700000in}}{\pgfqpoint{4.650000in}{5.600000in}} %
\pgfusepath{clip}%
\pgfsetbuttcap%
\pgfsetroundjoin%
\pgfsetlinewidth{0.501875pt}%
\definecolor{currentstroke}{rgb}{0.000000,0.000000,0.000000}%
\pgfsetstrokecolor{currentstroke}%
\pgfsetdash{{1.000000pt}{3.000000pt}}{0.000000pt}%
\pgfpathmoveto{\pgfqpoint{1.912500in}{0.700000in}}%
\pgfpathlineto{\pgfqpoint{1.912500in}{6.300000in}}%
\pgfusepath{stroke}%
\end{pgfscope}%
\begin{pgfscope}%
\pgfsetbuttcap%
\pgfsetroundjoin%
\definecolor{currentfill}{rgb}{0.000000,0.000000,0.000000}%
\pgfsetfillcolor{currentfill}%
\pgfsetlinewidth{0.501875pt}%
\definecolor{currentstroke}{rgb}{0.000000,0.000000,0.000000}%
\pgfsetstrokecolor{currentstroke}%
\pgfsetdash{}{0pt}%
\pgfsys@defobject{currentmarker}{\pgfqpoint{0.000000in}{0.000000in}}{\pgfqpoint{0.000000in}{0.055556in}}{%
\pgfpathmoveto{\pgfqpoint{0.000000in}{0.000000in}}%
\pgfpathlineto{\pgfqpoint{0.000000in}{0.055556in}}%
\pgfusepath{stroke,fill}%
}%
\begin{pgfscope}%
\pgfsys@transformshift{1.912500in}{0.700000in}%
\pgfsys@useobject{currentmarker}{}%
\end{pgfscope}%
\end{pgfscope}%
\begin{pgfscope}%
\pgfsetbuttcap%
\pgfsetroundjoin%
\definecolor{currentfill}{rgb}{0.000000,0.000000,0.000000}%
\pgfsetfillcolor{currentfill}%
\pgfsetlinewidth{0.501875pt}%
\definecolor{currentstroke}{rgb}{0.000000,0.000000,0.000000}%
\pgfsetstrokecolor{currentstroke}%
\pgfsetdash{}{0pt}%
\pgfsys@defobject{currentmarker}{\pgfqpoint{0.000000in}{-0.055556in}}{\pgfqpoint{0.000000in}{0.000000in}}{%
\pgfpathmoveto{\pgfqpoint{0.000000in}{0.000000in}}%
\pgfpathlineto{\pgfqpoint{0.000000in}{-0.055556in}}%
\pgfusepath{stroke,fill}%
}%
\begin{pgfscope}%
\pgfsys@transformshift{1.912500in}{6.300000in}%
\pgfsys@useobject{currentmarker}{}%
\end{pgfscope}%
\end{pgfscope}%
\begin{pgfscope}%
\pgftext[left,bottom,x=1.808238in,y=0.537037in,rotate=0.000000]{{\rmfamily\fontsize{12.000000}{14.400000}\selectfont \(\displaystyle 0.5\)}}
%
\end{pgfscope}%
\begin{pgfscope}%
\pgfpathrectangle{\pgfqpoint{0.750000in}{0.700000in}}{\pgfqpoint{4.650000in}{5.600000in}} %
\pgfusepath{clip}%
\pgfsetbuttcap%
\pgfsetroundjoin%
\pgfsetlinewidth{0.501875pt}%
\definecolor{currentstroke}{rgb}{0.000000,0.000000,0.000000}%
\pgfsetstrokecolor{currentstroke}%
\pgfsetdash{{1.000000pt}{3.000000pt}}{0.000000pt}%
\pgfpathmoveto{\pgfqpoint{3.075000in}{0.700000in}}%
\pgfpathlineto{\pgfqpoint{3.075000in}{6.300000in}}%
\pgfusepath{stroke}%
\end{pgfscope}%
\begin{pgfscope}%
\pgfsetbuttcap%
\pgfsetroundjoin%
\definecolor{currentfill}{rgb}{0.000000,0.000000,0.000000}%
\pgfsetfillcolor{currentfill}%
\pgfsetlinewidth{0.501875pt}%
\definecolor{currentstroke}{rgb}{0.000000,0.000000,0.000000}%
\pgfsetstrokecolor{currentstroke}%
\pgfsetdash{}{0pt}%
\pgfsys@defobject{currentmarker}{\pgfqpoint{0.000000in}{0.000000in}}{\pgfqpoint{0.000000in}{0.055556in}}{%
\pgfpathmoveto{\pgfqpoint{0.000000in}{0.000000in}}%
\pgfpathlineto{\pgfqpoint{0.000000in}{0.055556in}}%
\pgfusepath{stroke,fill}%
}%
\begin{pgfscope}%
\pgfsys@transformshift{3.075000in}{0.700000in}%
\pgfsys@useobject{currentmarker}{}%
\end{pgfscope}%
\end{pgfscope}%
\begin{pgfscope}%
\pgfsetbuttcap%
\pgfsetroundjoin%
\definecolor{currentfill}{rgb}{0.000000,0.000000,0.000000}%
\pgfsetfillcolor{currentfill}%
\pgfsetlinewidth{0.501875pt}%
\definecolor{currentstroke}{rgb}{0.000000,0.000000,0.000000}%
\pgfsetstrokecolor{currentstroke}%
\pgfsetdash{}{0pt}%
\pgfsys@defobject{currentmarker}{\pgfqpoint{0.000000in}{-0.055556in}}{\pgfqpoint{0.000000in}{0.000000in}}{%
\pgfpathmoveto{\pgfqpoint{0.000000in}{0.000000in}}%
\pgfpathlineto{\pgfqpoint{0.000000in}{-0.055556in}}%
\pgfusepath{stroke,fill}%
}%
\begin{pgfscope}%
\pgfsys@transformshift{3.075000in}{6.300000in}%
\pgfsys@useobject{currentmarker}{}%
\end{pgfscope}%
\end{pgfscope}%
\begin{pgfscope}%
\pgftext[left,bottom,x=2.970738in,y=0.537037in,rotate=0.000000]{{\rmfamily\fontsize{12.000000}{14.400000}\selectfont \(\displaystyle 1.0\)}}
%
\end{pgfscope}%
\begin{pgfscope}%
\pgfpathrectangle{\pgfqpoint{0.750000in}{0.700000in}}{\pgfqpoint{4.650000in}{5.600000in}} %
\pgfusepath{clip}%
\pgfsetbuttcap%
\pgfsetroundjoin%
\pgfsetlinewidth{0.501875pt}%
\definecolor{currentstroke}{rgb}{0.000000,0.000000,0.000000}%
\pgfsetstrokecolor{currentstroke}%
\pgfsetdash{{1.000000pt}{3.000000pt}}{0.000000pt}%
\pgfpathmoveto{\pgfqpoint{4.237500in}{0.700000in}}%
\pgfpathlineto{\pgfqpoint{4.237500in}{6.300000in}}%
\pgfusepath{stroke}%
\end{pgfscope}%
\begin{pgfscope}%
\pgfsetbuttcap%
\pgfsetroundjoin%
\definecolor{currentfill}{rgb}{0.000000,0.000000,0.000000}%
\pgfsetfillcolor{currentfill}%
\pgfsetlinewidth{0.501875pt}%
\definecolor{currentstroke}{rgb}{0.000000,0.000000,0.000000}%
\pgfsetstrokecolor{currentstroke}%
\pgfsetdash{}{0pt}%
\pgfsys@defobject{currentmarker}{\pgfqpoint{0.000000in}{0.000000in}}{\pgfqpoint{0.000000in}{0.055556in}}{%
\pgfpathmoveto{\pgfqpoint{0.000000in}{0.000000in}}%
\pgfpathlineto{\pgfqpoint{0.000000in}{0.055556in}}%
\pgfusepath{stroke,fill}%
}%
\begin{pgfscope}%
\pgfsys@transformshift{4.237500in}{0.700000in}%
\pgfsys@useobject{currentmarker}{}%
\end{pgfscope}%
\end{pgfscope}%
\begin{pgfscope}%
\pgfsetbuttcap%
\pgfsetroundjoin%
\definecolor{currentfill}{rgb}{0.000000,0.000000,0.000000}%
\pgfsetfillcolor{currentfill}%
\pgfsetlinewidth{0.501875pt}%
\definecolor{currentstroke}{rgb}{0.000000,0.000000,0.000000}%
\pgfsetstrokecolor{currentstroke}%
\pgfsetdash{}{0pt}%
\pgfsys@defobject{currentmarker}{\pgfqpoint{0.000000in}{-0.055556in}}{\pgfqpoint{0.000000in}{0.000000in}}{%
\pgfpathmoveto{\pgfqpoint{0.000000in}{0.000000in}}%
\pgfpathlineto{\pgfqpoint{0.000000in}{-0.055556in}}%
\pgfusepath{stroke,fill}%
}%
\begin{pgfscope}%
\pgfsys@transformshift{4.237500in}{6.300000in}%
\pgfsys@useobject{currentmarker}{}%
\end{pgfscope}%
\end{pgfscope}%
\begin{pgfscope}%
\pgftext[left,bottom,x=4.133238in,y=0.537037in,rotate=0.000000]{{\rmfamily\fontsize{12.000000}{14.400000}\selectfont \(\displaystyle 1.5\)}}
%
\end{pgfscope}%
\begin{pgfscope}%
\pgfpathrectangle{\pgfqpoint{0.750000in}{0.700000in}}{\pgfqpoint{4.650000in}{5.600000in}} %
\pgfusepath{clip}%
\pgfsetbuttcap%
\pgfsetroundjoin%
\pgfsetlinewidth{0.501875pt}%
\definecolor{currentstroke}{rgb}{0.000000,0.000000,0.000000}%
\pgfsetstrokecolor{currentstroke}%
\pgfsetdash{{1.000000pt}{3.000000pt}}{0.000000pt}%
\pgfpathmoveto{\pgfqpoint{5.400000in}{0.700000in}}%
\pgfpathlineto{\pgfqpoint{5.400000in}{6.300000in}}%
\pgfusepath{stroke}%
\end{pgfscope}%
\begin{pgfscope}%
\pgfsetbuttcap%
\pgfsetroundjoin%
\definecolor{currentfill}{rgb}{0.000000,0.000000,0.000000}%
\pgfsetfillcolor{currentfill}%
\pgfsetlinewidth{0.501875pt}%
\definecolor{currentstroke}{rgb}{0.000000,0.000000,0.000000}%
\pgfsetstrokecolor{currentstroke}%
\pgfsetdash{}{0pt}%
\pgfsys@defobject{currentmarker}{\pgfqpoint{0.000000in}{0.000000in}}{\pgfqpoint{0.000000in}{0.055556in}}{%
\pgfpathmoveto{\pgfqpoint{0.000000in}{0.000000in}}%
\pgfpathlineto{\pgfqpoint{0.000000in}{0.055556in}}%
\pgfusepath{stroke,fill}%
}%
\begin{pgfscope}%
\pgfsys@transformshift{5.400000in}{0.700000in}%
\pgfsys@useobject{currentmarker}{}%
\end{pgfscope}%
\end{pgfscope}%
\begin{pgfscope}%
\pgfsetbuttcap%
\pgfsetroundjoin%
\definecolor{currentfill}{rgb}{0.000000,0.000000,0.000000}%
\pgfsetfillcolor{currentfill}%
\pgfsetlinewidth{0.501875pt}%
\definecolor{currentstroke}{rgb}{0.000000,0.000000,0.000000}%
\pgfsetstrokecolor{currentstroke}%
\pgfsetdash{}{0pt}%
\pgfsys@defobject{currentmarker}{\pgfqpoint{0.000000in}{-0.055556in}}{\pgfqpoint{0.000000in}{0.000000in}}{%
\pgfpathmoveto{\pgfqpoint{0.000000in}{0.000000in}}%
\pgfpathlineto{\pgfqpoint{0.000000in}{-0.055556in}}%
\pgfusepath{stroke,fill}%
}%
\begin{pgfscope}%
\pgfsys@transformshift{5.400000in}{6.300000in}%
\pgfsys@useobject{currentmarker}{}%
\end{pgfscope}%
\end{pgfscope}%
\begin{pgfscope}%
\pgftext[left,bottom,x=5.295738in,y=0.537037in,rotate=0.000000]{{\rmfamily\fontsize{12.000000}{14.400000}\selectfont \(\displaystyle 2.0\)}}
%
\end{pgfscope}%
\begin{pgfscope}%
\pgftext[left,bottom,x=2.319809in,y=0.319445in,rotate=0.000000]{{\rmfamily\fontsize{12.000000}{14.400000}\selectfont Distance along Beam}}
%
\end{pgfscope}%
\begin{pgfscope}%
\pgfpathrectangle{\pgfqpoint{0.750000in}{0.700000in}}{\pgfqpoint{4.650000in}{5.600000in}} %
\pgfusepath{clip}%
\pgfsetbuttcap%
\pgfsetroundjoin%
\pgfsetlinewidth{0.501875pt}%
\definecolor{currentstroke}{rgb}{0.000000,0.000000,0.000000}%
\pgfsetstrokecolor{currentstroke}%
\pgfsetdash{{1.000000pt}{3.000000pt}}{0.000000pt}%
\pgfpathmoveto{\pgfqpoint{0.750000in}{6.300000in}}%
\pgfpathlineto{\pgfqpoint{5.400000in}{6.300000in}}%
\pgfusepath{stroke}%
\end{pgfscope}%
\begin{pgfscope}%
\pgfsetbuttcap%
\pgfsetroundjoin%
\definecolor{currentfill}{rgb}{0.000000,0.000000,0.000000}%
\pgfsetfillcolor{currentfill}%
\pgfsetlinewidth{0.501875pt}%
\definecolor{currentstroke}{rgb}{0.000000,0.000000,0.000000}%
\pgfsetstrokecolor{currentstroke}%
\pgfsetdash{}{0pt}%
\pgfsys@defobject{currentmarker}{\pgfqpoint{0.000000in}{0.000000in}}{\pgfqpoint{0.055556in}{0.000000in}}{%
\pgfpathmoveto{\pgfqpoint{0.000000in}{0.000000in}}%
\pgfpathlineto{\pgfqpoint{0.055556in}{0.000000in}}%
\pgfusepath{stroke,fill}%
}%
\begin{pgfscope}%
\pgfsys@transformshift{0.750000in}{6.300000in}%
\pgfsys@useobject{currentmarker}{}%
\end{pgfscope}%
\end{pgfscope}%
\begin{pgfscope}%
\pgfsetbuttcap%
\pgfsetroundjoin%
\definecolor{currentfill}{rgb}{0.000000,0.000000,0.000000}%
\pgfsetfillcolor{currentfill}%
\pgfsetlinewidth{0.501875pt}%
\definecolor{currentstroke}{rgb}{0.000000,0.000000,0.000000}%
\pgfsetstrokecolor{currentstroke}%
\pgfsetdash{}{0pt}%
\pgfsys@defobject{currentmarker}{\pgfqpoint{-0.055556in}{0.000000in}}{\pgfqpoint{0.000000in}{0.000000in}}{%
\pgfpathmoveto{\pgfqpoint{0.000000in}{0.000000in}}%
\pgfpathlineto{\pgfqpoint{-0.055556in}{0.000000in}}%
\pgfusepath{stroke,fill}%
}%
\begin{pgfscope}%
\pgfsys@transformshift{5.400000in}{6.300000in}%
\pgfsys@useobject{currentmarker}{}%
\end{pgfscope}%
\end{pgfscope}%
\begin{pgfscope}%
\pgftext[left,bottom,x=0.485920in,y=6.246296in,rotate=0.000000]{{\rmfamily\fontsize{12.000000}{14.400000}\selectfont \(\displaystyle 0.0\)}}
%
\end{pgfscope}%
\begin{pgfscope}%
\pgfpathrectangle{\pgfqpoint{0.750000in}{0.700000in}}{\pgfqpoint{4.650000in}{5.600000in}} %
\pgfusepath{clip}%
\pgfsetbuttcap%
\pgfsetroundjoin%
\pgfsetlinewidth{0.501875pt}%
\definecolor{currentstroke}{rgb}{0.000000,0.000000,0.000000}%
\pgfsetstrokecolor{currentstroke}%
\pgfsetdash{{1.000000pt}{3.000000pt}}{0.000000pt}%
\pgfpathmoveto{\pgfqpoint{0.750000in}{5.100000in}}%
\pgfpathlineto{\pgfqpoint{5.400000in}{5.100000in}}%
\pgfusepath{stroke}%
\end{pgfscope}%
\begin{pgfscope}%
\pgfsetbuttcap%
\pgfsetroundjoin%
\definecolor{currentfill}{rgb}{0.000000,0.000000,0.000000}%
\pgfsetfillcolor{currentfill}%
\pgfsetlinewidth{0.501875pt}%
\definecolor{currentstroke}{rgb}{0.000000,0.000000,0.000000}%
\pgfsetstrokecolor{currentstroke}%
\pgfsetdash{}{0pt}%
\pgfsys@defobject{currentmarker}{\pgfqpoint{0.000000in}{0.000000in}}{\pgfqpoint{0.055556in}{0.000000in}}{%
\pgfpathmoveto{\pgfqpoint{0.000000in}{0.000000in}}%
\pgfpathlineto{\pgfqpoint{0.055556in}{0.000000in}}%
\pgfusepath{stroke,fill}%
}%
\begin{pgfscope}%
\pgfsys@transformshift{0.750000in}{5.100000in}%
\pgfsys@useobject{currentmarker}{}%
\end{pgfscope}%
\end{pgfscope}%
\begin{pgfscope}%
\pgfsetbuttcap%
\pgfsetroundjoin%
\definecolor{currentfill}{rgb}{0.000000,0.000000,0.000000}%
\pgfsetfillcolor{currentfill}%
\pgfsetlinewidth{0.501875pt}%
\definecolor{currentstroke}{rgb}{0.000000,0.000000,0.000000}%
\pgfsetstrokecolor{currentstroke}%
\pgfsetdash{}{0pt}%
\pgfsys@defobject{currentmarker}{\pgfqpoint{-0.055556in}{0.000000in}}{\pgfqpoint{0.000000in}{0.000000in}}{%
\pgfpathmoveto{\pgfqpoint{0.000000in}{0.000000in}}%
\pgfpathlineto{\pgfqpoint{-0.055556in}{0.000000in}}%
\pgfusepath{stroke,fill}%
}%
\begin{pgfscope}%
\pgfsys@transformshift{5.400000in}{5.100000in}%
\pgfsys@useobject{currentmarker}{}%
\end{pgfscope}%
\end{pgfscope}%
\begin{pgfscope}%
\pgftext[left,bottom,x=0.356290in,y=5.039352in,rotate=0.000000]{{\rmfamily\fontsize{12.000000}{14.400000}\selectfont \(\displaystyle -0.3\)}}
%
\end{pgfscope}%
\begin{pgfscope}%
\pgfpathrectangle{\pgfqpoint{0.750000in}{0.700000in}}{\pgfqpoint{4.650000in}{5.600000in}} %
\pgfusepath{clip}%
\pgfsetbuttcap%
\pgfsetroundjoin%
\pgfsetlinewidth{0.501875pt}%
\definecolor{currentstroke}{rgb}{0.000000,0.000000,0.000000}%
\pgfsetstrokecolor{currentstroke}%
\pgfsetdash{{1.000000pt}{3.000000pt}}{0.000000pt}%
\pgfpathmoveto{\pgfqpoint{0.750000in}{3.900000in}}%
\pgfpathlineto{\pgfqpoint{5.400000in}{3.900000in}}%
\pgfusepath{stroke}%
\end{pgfscope}%
\begin{pgfscope}%
\pgfsetbuttcap%
\pgfsetroundjoin%
\definecolor{currentfill}{rgb}{0.000000,0.000000,0.000000}%
\pgfsetfillcolor{currentfill}%
\pgfsetlinewidth{0.501875pt}%
\definecolor{currentstroke}{rgb}{0.000000,0.000000,0.000000}%
\pgfsetstrokecolor{currentstroke}%
\pgfsetdash{}{0pt}%
\pgfsys@defobject{currentmarker}{\pgfqpoint{0.000000in}{0.000000in}}{\pgfqpoint{0.055556in}{0.000000in}}{%
\pgfpathmoveto{\pgfqpoint{0.000000in}{0.000000in}}%
\pgfpathlineto{\pgfqpoint{0.055556in}{0.000000in}}%
\pgfusepath{stroke,fill}%
}%
\begin{pgfscope}%
\pgfsys@transformshift{0.750000in}{3.900000in}%
\pgfsys@useobject{currentmarker}{}%
\end{pgfscope}%
\end{pgfscope}%
\begin{pgfscope}%
\pgfsetbuttcap%
\pgfsetroundjoin%
\definecolor{currentfill}{rgb}{0.000000,0.000000,0.000000}%
\pgfsetfillcolor{currentfill}%
\pgfsetlinewidth{0.501875pt}%
\definecolor{currentstroke}{rgb}{0.000000,0.000000,0.000000}%
\pgfsetstrokecolor{currentstroke}%
\pgfsetdash{}{0pt}%
\pgfsys@defobject{currentmarker}{\pgfqpoint{-0.055556in}{0.000000in}}{\pgfqpoint{0.000000in}{0.000000in}}{%
\pgfpathmoveto{\pgfqpoint{0.000000in}{0.000000in}}%
\pgfpathlineto{\pgfqpoint{-0.055556in}{0.000000in}}%
\pgfusepath{stroke,fill}%
}%
\begin{pgfscope}%
\pgfsys@transformshift{5.400000in}{3.900000in}%
\pgfsys@useobject{currentmarker}{}%
\end{pgfscope}%
\end{pgfscope}%
\begin{pgfscope}%
\pgftext[left,bottom,x=0.356290in,y=3.839352in,rotate=0.000000]{{\rmfamily\fontsize{12.000000}{14.400000}\selectfont \(\displaystyle -0.6\)}}
%
\end{pgfscope}%
\begin{pgfscope}%
\pgfpathrectangle{\pgfqpoint{0.750000in}{0.700000in}}{\pgfqpoint{4.650000in}{5.600000in}} %
\pgfusepath{clip}%
\pgfsetbuttcap%
\pgfsetroundjoin%
\pgfsetlinewidth{0.501875pt}%
\definecolor{currentstroke}{rgb}{0.000000,0.000000,0.000000}%
\pgfsetstrokecolor{currentstroke}%
\pgfsetdash{{1.000000pt}{3.000000pt}}{0.000000pt}%
\pgfpathmoveto{\pgfqpoint{0.750000in}{2.700000in}}%
\pgfpathlineto{\pgfqpoint{5.400000in}{2.700000in}}%
\pgfusepath{stroke}%
\end{pgfscope}%
\begin{pgfscope}%
\pgfsetbuttcap%
\pgfsetroundjoin%
\definecolor{currentfill}{rgb}{0.000000,0.000000,0.000000}%
\pgfsetfillcolor{currentfill}%
\pgfsetlinewidth{0.501875pt}%
\definecolor{currentstroke}{rgb}{0.000000,0.000000,0.000000}%
\pgfsetstrokecolor{currentstroke}%
\pgfsetdash{}{0pt}%
\pgfsys@defobject{currentmarker}{\pgfqpoint{0.000000in}{0.000000in}}{\pgfqpoint{0.055556in}{0.000000in}}{%
\pgfpathmoveto{\pgfqpoint{0.000000in}{0.000000in}}%
\pgfpathlineto{\pgfqpoint{0.055556in}{0.000000in}}%
\pgfusepath{stroke,fill}%
}%
\begin{pgfscope}%
\pgfsys@transformshift{0.750000in}{2.700000in}%
\pgfsys@useobject{currentmarker}{}%
\end{pgfscope}%
\end{pgfscope}%
\begin{pgfscope}%
\pgfsetbuttcap%
\pgfsetroundjoin%
\definecolor{currentfill}{rgb}{0.000000,0.000000,0.000000}%
\pgfsetfillcolor{currentfill}%
\pgfsetlinewidth{0.501875pt}%
\definecolor{currentstroke}{rgb}{0.000000,0.000000,0.000000}%
\pgfsetstrokecolor{currentstroke}%
\pgfsetdash{}{0pt}%
\pgfsys@defobject{currentmarker}{\pgfqpoint{-0.055556in}{0.000000in}}{\pgfqpoint{0.000000in}{0.000000in}}{%
\pgfpathmoveto{\pgfqpoint{0.000000in}{0.000000in}}%
\pgfpathlineto{\pgfqpoint{-0.055556in}{0.000000in}}%
\pgfusepath{stroke,fill}%
}%
\begin{pgfscope}%
\pgfsys@transformshift{5.400000in}{2.700000in}%
\pgfsys@useobject{currentmarker}{}%
\end{pgfscope}%
\end{pgfscope}%
\begin{pgfscope}%
\pgftext[left,bottom,x=0.356290in,y=2.639352in,rotate=0.000000]{{\rmfamily\fontsize{12.000000}{14.400000}\selectfont \(\displaystyle -0.9\)}}
%
\end{pgfscope}%
\begin{pgfscope}%
\pgfpathrectangle{\pgfqpoint{0.750000in}{0.700000in}}{\pgfqpoint{4.650000in}{5.600000in}} %
\pgfusepath{clip}%
\pgfsetbuttcap%
\pgfsetroundjoin%
\pgfsetlinewidth{0.501875pt}%
\definecolor{currentstroke}{rgb}{0.000000,0.000000,0.000000}%
\pgfsetstrokecolor{currentstroke}%
\pgfsetdash{{1.000000pt}{3.000000pt}}{0.000000pt}%
\pgfpathmoveto{\pgfqpoint{0.750000in}{1.500000in}}%
\pgfpathlineto{\pgfqpoint{5.400000in}{1.500000in}}%
\pgfusepath{stroke}%
\end{pgfscope}%
\begin{pgfscope}%
\pgfsetbuttcap%
\pgfsetroundjoin%
\definecolor{currentfill}{rgb}{0.000000,0.000000,0.000000}%
\pgfsetfillcolor{currentfill}%
\pgfsetlinewidth{0.501875pt}%
\definecolor{currentstroke}{rgb}{0.000000,0.000000,0.000000}%
\pgfsetstrokecolor{currentstroke}%
\pgfsetdash{}{0pt}%
\pgfsys@defobject{currentmarker}{\pgfqpoint{0.000000in}{0.000000in}}{\pgfqpoint{0.055556in}{0.000000in}}{%
\pgfpathmoveto{\pgfqpoint{0.000000in}{0.000000in}}%
\pgfpathlineto{\pgfqpoint{0.055556in}{0.000000in}}%
\pgfusepath{stroke,fill}%
}%
\begin{pgfscope}%
\pgfsys@transformshift{0.750000in}{1.500000in}%
\pgfsys@useobject{currentmarker}{}%
\end{pgfscope}%
\end{pgfscope}%
\begin{pgfscope}%
\pgfsetbuttcap%
\pgfsetroundjoin%
\definecolor{currentfill}{rgb}{0.000000,0.000000,0.000000}%
\pgfsetfillcolor{currentfill}%
\pgfsetlinewidth{0.501875pt}%
\definecolor{currentstroke}{rgb}{0.000000,0.000000,0.000000}%
\pgfsetstrokecolor{currentstroke}%
\pgfsetdash{}{0pt}%
\pgfsys@defobject{currentmarker}{\pgfqpoint{-0.055556in}{0.000000in}}{\pgfqpoint{0.000000in}{0.000000in}}{%
\pgfpathmoveto{\pgfqpoint{0.000000in}{0.000000in}}%
\pgfpathlineto{\pgfqpoint{-0.055556in}{0.000000in}}%
\pgfusepath{stroke,fill}%
}%
\begin{pgfscope}%
\pgfsys@transformshift{5.400000in}{1.500000in}%
\pgfsys@useobject{currentmarker}{}%
\end{pgfscope}%
\end{pgfscope}%
\begin{pgfscope}%
\pgftext[left,bottom,x=0.356290in,y=1.439352in,rotate=0.000000]{{\rmfamily\fontsize{12.000000}{14.400000}\selectfont \(\displaystyle -1.2\)}}
%
\end{pgfscope}%
\begin{pgfscope}%
\pgftext[left,bottom,x=0.286846in,y=3.143030in,rotate=90.000000]{{\rmfamily\fontsize{12.000000}{14.400000}\selectfont Deflection}}
%
\end{pgfscope}%
\begin{pgfscope}%
\pgftext[left,bottom,x=0.750000in,y=6.327778in,rotate=0.000000]{{\rmfamily\fontsize{12.000000}{14.400000}\selectfont \(\displaystyle \times10^{-4}\)}}
%
\end{pgfscope}%
\begin{pgfscope}%
\pgfsetrectcap%
\pgfsetroundjoin%
\pgfsetlinewidth{1.003750pt}%
\definecolor{currentstroke}{rgb}{0.000000,0.000000,0.000000}%
\pgfsetstrokecolor{currentstroke}%
\pgfsetdash{}{0pt}%
\pgfpathmoveto{\pgfqpoint{0.750000in}{6.300000in}}%
\pgfpathlineto{\pgfqpoint{5.400000in}{6.300000in}}%
\pgfusepath{stroke}%
\end{pgfscope}%
\begin{pgfscope}%
\pgfsetrectcap%
\pgfsetroundjoin%
\pgfsetlinewidth{1.003750pt}%
\definecolor{currentstroke}{rgb}{0.000000,0.000000,0.000000}%
\pgfsetstrokecolor{currentstroke}%
\pgfsetdash{}{0pt}%
\pgfpathmoveto{\pgfqpoint{5.400000in}{0.700000in}}%
\pgfpathlineto{\pgfqpoint{5.400000in}{6.300000in}}%
\pgfusepath{stroke}%
\end{pgfscope}%
\begin{pgfscope}%
\pgfsetrectcap%
\pgfsetroundjoin%
\pgfsetlinewidth{1.003750pt}%
\definecolor{currentstroke}{rgb}{0.000000,0.000000,0.000000}%
\pgfsetstrokecolor{currentstroke}%
\pgfsetdash{}{0pt}%
\pgfpathmoveto{\pgfqpoint{0.750000in}{0.700000in}}%
\pgfpathlineto{\pgfqpoint{5.400000in}{0.700000in}}%
\pgfusepath{stroke}%
\end{pgfscope}%
\begin{pgfscope}%
\pgfsetrectcap%
\pgfsetroundjoin%
\pgfsetlinewidth{1.003750pt}%
\definecolor{currentstroke}{rgb}{0.000000,0.000000,0.000000}%
\pgfsetstrokecolor{currentstroke}%
\pgfsetdash{}{0pt}%
\pgfpathmoveto{\pgfqpoint{0.750000in}{0.700000in}}%
\pgfpathlineto{\pgfqpoint{0.750000in}{6.300000in}}%
\pgfusepath{stroke}%
\end{pgfscope}%
\begin{pgfscope}%
\pgftext[left,bottom,x=1.789981in,y=6.330556in,rotate=0.000000]{{\rmfamily\fontsize{14.400000}{17.280000}\selectfont Uniformly Loaded EPP Beam}}
%
\end{pgfscope}%
\begin{pgfscope}%
\pgfsetrectcap%
\pgfsetroundjoin%
\definecolor{currentfill}{rgb}{1.000000,1.000000,1.000000}%
\pgfsetfillcolor{currentfill}%
\pgfsetlinewidth{1.003750pt}%
\definecolor{currentstroke}{rgb}{0.000000,0.000000,0.000000}%
\pgfsetstrokecolor{currentstroke}%
\pgfsetdash{}{0pt}%
\pgfpathmoveto{\pgfqpoint{1.710823in}{5.124445in}}%
\pgfpathlineto{\pgfqpoint{4.439177in}{5.124445in}}%
\pgfpathlineto{\pgfqpoint{4.439177in}{6.300000in}}%
\pgfpathlineto{\pgfqpoint{1.710823in}{6.300000in}}%
\pgfpathlineto{\pgfqpoint{1.710823in}{5.124445in}}%
\pgfpathclose%
\pgfusepath{stroke,fill}%
\end{pgfscope}%
\begin{pgfscope}%
\pgfsetrectcap%
\pgfsetroundjoin%
\pgfsetlinewidth{1.003750pt}%
\definecolor{currentstroke}{rgb}{0.000000,0.000000,1.000000}%
\pgfsetstrokecolor{currentstroke}%
\pgfsetdash{}{0pt}%
\pgfpathmoveto{\pgfqpoint{1.850823in}{6.150000in}}%
\pgfpathlineto{\pgfqpoint{2.130823in}{6.150000in}}%
\pgfusepath{stroke}%
\end{pgfscope}%
\begin{pgfscope}%
\pgfsetbuttcap%
\pgfsetroundjoin%
\definecolor{currentfill}{rgb}{0.000000,0.000000,1.000000}%
\pgfsetfillcolor{currentfill}%
\pgfsetlinewidth{0.501875pt}%
\definecolor{currentstroke}{rgb}{0.000000,0.000000,0.000000}%
\pgfsetstrokecolor{currentstroke}%
\pgfsetdash{}{0pt}%
\pgfsys@defobject{currentmarker}{\pgfqpoint{-0.041667in}{-0.041667in}}{\pgfqpoint{0.041667in}{0.041667in}}{%
\pgfpathmoveto{\pgfqpoint{0.000000in}{-0.041667in}}%
\pgfpathcurveto{\pgfqpoint{0.011050in}{-0.041667in}}{\pgfqpoint{0.021649in}{-0.037276in}}{\pgfqpoint{0.029463in}{-0.029463in}}%
\pgfpathcurveto{\pgfqpoint{0.037276in}{-0.021649in}}{\pgfqpoint{0.041667in}{-0.011050in}}{\pgfqpoint{0.041667in}{0.000000in}}%
\pgfpathcurveto{\pgfqpoint{0.041667in}{0.011050in}}{\pgfqpoint{0.037276in}{0.021649in}}{\pgfqpoint{0.029463in}{0.029463in}}%
\pgfpathcurveto{\pgfqpoint{0.021649in}{0.037276in}}{\pgfqpoint{0.011050in}{0.041667in}}{\pgfqpoint{0.000000in}{0.041667in}}%
\pgfpathcurveto{\pgfqpoint{-0.011050in}{0.041667in}}{\pgfqpoint{-0.021649in}{0.037276in}}{\pgfqpoint{-0.029463in}{0.029463in}}%
\pgfpathcurveto{\pgfqpoint{-0.037276in}{0.021649in}}{\pgfqpoint{-0.041667in}{0.011050in}}{\pgfqpoint{-0.041667in}{0.000000in}}%
\pgfpathcurveto{\pgfqpoint{-0.041667in}{-0.011050in}}{\pgfqpoint{-0.037276in}{-0.021649in}}{\pgfqpoint{-0.029463in}{-0.029463in}}%
\pgfpathcurveto{\pgfqpoint{-0.021649in}{-0.037276in}}{\pgfqpoint{-0.011050in}{-0.041667in}}{\pgfqpoint{0.000000in}{-0.041667in}}%
\pgfpathclose%
\pgfusepath{stroke,fill}%
}%
\begin{pgfscope}%
\pgfsys@transformshift{1.850823in}{6.150000in}%
\pgfsys@useobject{currentmarker}{}%
\end{pgfscope}%
\begin{pgfscope}%
\pgfsys@transformshift{2.130823in}{6.150000in}%
\pgfsys@useobject{currentmarker}{}%
\end{pgfscope}%
\end{pgfscope}%
\begin{pgfscope}%
\pgftext[left,bottom,x=2.350823in,y=6.041111in,rotate=0.000000]{{\rmfamily\fontsize{14.400000}{17.280000}\selectfont Abaqus EPP Beam}}
%
\end{pgfscope}%
\begin{pgfscope}%
\pgfsetrectcap%
\pgfsetroundjoin%
\pgfsetlinewidth{1.003750pt}%
\definecolor{currentstroke}{rgb}{0.000000,0.500000,0.000000}%
\pgfsetstrokecolor{currentstroke}%
\pgfsetdash{}{0pt}%
\pgfpathmoveto{\pgfqpoint{1.850823in}{5.871111in}}%
\pgfpathlineto{\pgfqpoint{2.130823in}{5.871111in}}%
\pgfusepath{stroke}%
\end{pgfscope}%
\begin{pgfscope}%
\pgfsetbuttcap%
\pgfsetmiterjoin%
\definecolor{currentfill}{rgb}{0.000000,0.500000,0.000000}%
\pgfsetfillcolor{currentfill}%
\pgfsetlinewidth{0.501875pt}%
\definecolor{currentstroke}{rgb}{0.000000,0.000000,0.000000}%
\pgfsetstrokecolor{currentstroke}%
\pgfsetdash{}{0pt}%
\pgfsys@defobject{currentmarker}{\pgfqpoint{-0.041667in}{-0.041667in}}{\pgfqpoint{0.041667in}{0.041667in}}{%
\pgfpathmoveto{\pgfqpoint{0.000000in}{0.041667in}}%
\pgfpathlineto{\pgfqpoint{-0.041667in}{-0.041667in}}%
\pgfpathlineto{\pgfqpoint{0.041667in}{-0.041667in}}%
\pgfpathclose%
\pgfusepath{stroke,fill}%
}%
\begin{pgfscope}%
\pgfsys@transformshift{1.850823in}{5.871111in}%
\pgfsys@useobject{currentmarker}{}%
\end{pgfscope}%
\begin{pgfscope}%
\pgfsys@transformshift{2.130823in}{5.871111in}%
\pgfsys@useobject{currentmarker}{}%
\end{pgfscope}%
\end{pgfscope}%
\begin{pgfscope}%
\pgftext[left,bottom,x=2.350823in,y=5.762223in,rotate=0.000000]{{\rmfamily\fontsize{14.400000}{17.280000}\selectfont 100 nodes, horizon 0.10}}
%
\end{pgfscope}%
\begin{pgfscope}%
\pgfsetrectcap%
\pgfsetroundjoin%
\pgfsetlinewidth{1.003750pt}%
\definecolor{currentstroke}{rgb}{1.000000,0.000000,0.000000}%
\pgfsetstrokecolor{currentstroke}%
\pgfsetdash{}{0pt}%
\pgfpathmoveto{\pgfqpoint{1.850823in}{5.592223in}}%
\pgfpathlineto{\pgfqpoint{2.130823in}{5.592223in}}%
\pgfusepath{stroke}%
\end{pgfscope}%
\begin{pgfscope}%
\pgfsetbuttcap%
\pgfsetmiterjoin%
\definecolor{currentfill}{rgb}{1.000000,0.000000,0.000000}%
\pgfsetfillcolor{currentfill}%
\pgfsetlinewidth{0.501875pt}%
\definecolor{currentstroke}{rgb}{0.000000,0.000000,0.000000}%
\pgfsetstrokecolor{currentstroke}%
\pgfsetdash{}{0pt}%
\pgfsys@defobject{currentmarker}{\pgfqpoint{-0.041667in}{-0.041667in}}{\pgfqpoint{0.041667in}{0.041667in}}{%
\pgfpathmoveto{\pgfqpoint{0.041667in}{-0.000000in}}%
\pgfpathlineto{\pgfqpoint{-0.041667in}{0.041667in}}%
\pgfpathlineto{\pgfqpoint{-0.041667in}{-0.041667in}}%
\pgfpathclose%
\pgfusepath{stroke,fill}%
}%
\begin{pgfscope}%
\pgfsys@transformshift{1.850823in}{5.592223in}%
\pgfsys@useobject{currentmarker}{}%
\end{pgfscope}%
\begin{pgfscope}%
\pgfsys@transformshift{2.130823in}{5.592223in}%
\pgfsys@useobject{currentmarker}{}%
\end{pgfscope}%
\end{pgfscope}%
\begin{pgfscope}%
\pgftext[left,bottom,x=2.350823in,y=5.483334in,rotate=0.000000]{{\rmfamily\fontsize{14.400000}{17.280000}\selectfont 200 nodes, horizon 0.10}}
%
\end{pgfscope}%
\begin{pgfscope}%
\pgfsetrectcap%
\pgfsetroundjoin%
\pgfsetlinewidth{1.003750pt}%
\definecolor{currentstroke}{rgb}{0.000000,0.750000,0.750000}%
\pgfsetstrokecolor{currentstroke}%
\pgfsetdash{}{0pt}%
\pgfpathmoveto{\pgfqpoint{1.850823in}{5.313334in}}%
\pgfpathlineto{\pgfqpoint{2.130823in}{5.313334in}}%
\pgfusepath{stroke}%
\end{pgfscope}%
\begin{pgfscope}%
\pgfsetbuttcap%
\pgfsetmiterjoin%
\definecolor{currentfill}{rgb}{0.000000,0.750000,0.750000}%
\pgfsetfillcolor{currentfill}%
\pgfsetlinewidth{0.501875pt}%
\definecolor{currentstroke}{rgb}{0.000000,0.000000,0.000000}%
\pgfsetstrokecolor{currentstroke}%
\pgfsetdash{}{0pt}%
\pgfsys@defobject{currentmarker}{\pgfqpoint{-0.041667in}{-0.041667in}}{\pgfqpoint{0.041667in}{0.041667in}}{%
\pgfpathmoveto{\pgfqpoint{-0.000000in}{-0.041667in}}%
\pgfpathlineto{\pgfqpoint{0.041667in}{0.041667in}}%
\pgfpathlineto{\pgfqpoint{-0.041667in}{0.041667in}}%
\pgfpathclose%
\pgfusepath{stroke,fill}%
}%
\begin{pgfscope}%
\pgfsys@transformshift{1.850823in}{5.313334in}%
\pgfsys@useobject{currentmarker}{}%
\end{pgfscope}%
\begin{pgfscope}%
\pgfsys@transformshift{2.130823in}{5.313334in}%
\pgfsys@useobject{currentmarker}{}%
\end{pgfscope}%
\end{pgfscope}%
\begin{pgfscope}%
\pgftext[left,bottom,x=2.350823in,y=5.204445in,rotate=0.000000]{{\rmfamily\fontsize{14.400000}{17.280000}\selectfont 500 nodes, horizon 0.10}}
%
\end{pgfscope}%
\end{pgfpicture}%
\makeatother%
\endgroup%
}
  \caption{The elastic perfectly-plastic beam requires finer discretization}
  \label{fig:eppu_h10_g2000}
\end{figure}

A material that is plastically deformed does not return to its original state when unloaded.
For a beam in bending, the residual deformations can be seen in a beam that has been loaded beyond the onset of plastic deformation and then unloaded.
This result is observed in the bond-pair plasticity model, shown in \cref{fig:ResidualPlasticityN,fig:ResidualPlasticityH}. 
Accurate residual deformation modeling requires both a relatively small horizon and a fairly large number of nodes.

%
\begin{figure}[h]
  \centering
  \resizebox{0.5\linewidth}{!}{%% Creator: Matplotlib, PGF backend
%%
%% To include the figure in your LaTeX document, write
%%   \input{<filename>.pgf}
%%
%% Make sure the required packages are loaded in your preamble
%%   \usepackage{pgf}
%%
%% Figures using additional raster images can only be included by \input if
%% they are in the same directory as the main LaTeX file. For loading figures
%% from other directories you can use the `import` package
%%   \usepackage{import}
%% and then include the figures with
%%   \import{<path to file>}{<filename>.pgf}
%%
%% Matplotlib used the following preamble
%%
\begingroup%
\makeatletter%
\begin{pgfpicture}%
\pgfpathrectangle{\pgfpointorigin}{\pgfqpoint{6.000000in}{7.000000in}}%
\pgfusepath{use as bounding box}%
\begin{pgfscope}%
\pgfsetrectcap%
\pgfsetroundjoin%
\definecolor{currentfill}{rgb}{1.000000,1.000000,1.000000}%
\pgfsetfillcolor{currentfill}%
\pgfsetlinewidth{0.000000pt}%
\definecolor{currentstroke}{rgb}{1.000000,1.000000,1.000000}%
\pgfsetstrokecolor{currentstroke}%
\pgfsetdash{}{0pt}%
\pgfpathmoveto{\pgfqpoint{0.000000in}{0.000000in}}%
\pgfpathlineto{\pgfqpoint{6.000000in}{0.000000in}}%
\pgfpathlineto{\pgfqpoint{6.000000in}{7.000000in}}%
\pgfpathlineto{\pgfqpoint{0.000000in}{7.000000in}}%
\pgfpathclose%
\pgfusepath{fill}%
\end{pgfscope}%
\begin{pgfscope}%
\pgfsetrectcap%
\pgfsetroundjoin%
\definecolor{currentfill}{rgb}{1.000000,1.000000,1.000000}%
\pgfsetfillcolor{currentfill}%
\pgfsetlinewidth{0.000000pt}%
\definecolor{currentstroke}{rgb}{0.000000,0.000000,0.000000}%
\pgfsetstrokecolor{currentstroke}%
\pgfsetdash{}{0pt}%
\pgfpathmoveto{\pgfqpoint{0.750000in}{0.700000in}}%
\pgfpathlineto{\pgfqpoint{5.400000in}{0.700000in}}%
\pgfpathlineto{\pgfqpoint{5.400000in}{6.300000in}}%
\pgfpathlineto{\pgfqpoint{0.750000in}{6.300000in}}%
\pgfpathclose%
\pgfusepath{fill}%
\end{pgfscope}%
\begin{pgfscope}%
\pgfpathrectangle{\pgfqpoint{0.750000in}{0.700000in}}{\pgfqpoint{4.650000in}{5.600000in}} %
\pgfusepath{clip}%
\pgfsetrectcap%
\pgfsetroundjoin%
\pgfsetlinewidth{1.003750pt}%
\definecolor{currentstroke}{rgb}{0.000000,0.000000,1.000000}%
\pgfsetstrokecolor{currentstroke}%
\pgfsetdash{}{0pt}%
\pgfpathmoveto{\pgfqpoint{0.750000in}{6.300000in}}%
\pgfpathlineto{\pgfqpoint{2.168250in}{4.709869in}}%
\pgfpathlineto{\pgfqpoint{2.261250in}{4.610004in}}%
\pgfpathlineto{\pgfqpoint{2.331000in}{4.538555in}}%
\pgfpathlineto{\pgfqpoint{2.400750in}{4.470935in}}%
\pgfpathlineto{\pgfqpoint{2.447250in}{4.428340in}}%
\pgfpathlineto{\pgfqpoint{2.493750in}{4.387968in}}%
\pgfpathlineto{\pgfqpoint{2.540250in}{4.350010in}}%
\pgfpathlineto{\pgfqpoint{2.586750in}{4.314625in}}%
\pgfpathlineto{\pgfqpoint{2.633250in}{4.281970in}}%
\pgfpathlineto{\pgfqpoint{2.679750in}{4.252168in}}%
\pgfpathlineto{\pgfqpoint{2.726250in}{4.225375in}}%
\pgfpathlineto{\pgfqpoint{2.772750in}{4.201680in}}%
\pgfpathlineto{\pgfqpoint{2.819250in}{4.181188in}}%
\pgfpathlineto{\pgfqpoint{2.865750in}{4.163985in}}%
\pgfpathlineto{\pgfqpoint{2.912250in}{4.150125in}}%
\pgfpathlineto{\pgfqpoint{2.958750in}{4.139695in}}%
\pgfpathlineto{\pgfqpoint{3.005250in}{4.132713in}}%
\pgfpathlineto{\pgfqpoint{3.051750in}{4.129213in}}%
\pgfpathlineto{\pgfqpoint{3.098250in}{4.129213in}}%
\pgfpathlineto{\pgfqpoint{3.144750in}{4.132713in}}%
\pgfpathlineto{\pgfqpoint{3.191250in}{4.139695in}}%
\pgfpathlineto{\pgfqpoint{3.237750in}{4.150125in}}%
\pgfpathlineto{\pgfqpoint{3.284250in}{4.163985in}}%
\pgfpathlineto{\pgfqpoint{3.330750in}{4.181188in}}%
\pgfpathlineto{\pgfqpoint{3.377250in}{4.201680in}}%
\pgfpathlineto{\pgfqpoint{3.423750in}{4.225375in}}%
\pgfpathlineto{\pgfqpoint{3.470250in}{4.252168in}}%
\pgfpathlineto{\pgfqpoint{3.516750in}{4.281970in}}%
\pgfpathlineto{\pgfqpoint{3.563250in}{4.314625in}}%
\pgfpathlineto{\pgfqpoint{3.609750in}{4.350010in}}%
\pgfpathlineto{\pgfqpoint{3.656250in}{4.387985in}}%
\pgfpathlineto{\pgfqpoint{3.702750in}{4.428340in}}%
\pgfpathlineto{\pgfqpoint{3.749250in}{4.470935in}}%
\pgfpathlineto{\pgfqpoint{3.819000in}{4.538555in}}%
\pgfpathlineto{\pgfqpoint{3.888750in}{4.610006in}}%
\pgfpathlineto{\pgfqpoint{3.958500in}{4.684507in}}%
\pgfpathlineto{\pgfqpoint{4.051500in}{4.787078in}}%
\pgfpathlineto{\pgfqpoint{4.470000in}{5.256517in}}%
\pgfpathlineto{\pgfqpoint{5.400000in}{6.300000in}}%
\pgfpathlineto{\pgfqpoint{5.400000in}{6.300000in}}%
\pgfusepath{stroke}%
\end{pgfscope}%
\begin{pgfscope}%
\pgfpathrectangle{\pgfqpoint{0.750000in}{0.700000in}}{\pgfqpoint{4.650000in}{5.600000in}} %
\pgfusepath{clip}%
\pgfsetrectcap%
\pgfsetroundjoin%
\pgfsetlinewidth{1.003750pt}%
\definecolor{currentstroke}{rgb}{0.000000,0.500000,0.000000}%
\pgfsetstrokecolor{currentstroke}%
\pgfsetdash{}{0pt}%
\pgfpathmoveto{\pgfqpoint{0.750000in}{6.300000in}}%
\pgfpathlineto{\pgfqpoint{2.331000in}{2.162459in}}%
\pgfpathlineto{\pgfqpoint{2.424000in}{1.926583in}}%
\pgfpathlineto{\pgfqpoint{2.517000in}{1.697901in}}%
\pgfpathlineto{\pgfqpoint{2.563500in}{1.588027in}}%
\pgfpathlineto{\pgfqpoint{2.610000in}{1.482246in}}%
\pgfpathlineto{\pgfqpoint{2.656500in}{1.381554in}}%
\pgfpathlineto{\pgfqpoint{2.703000in}{1.286387in}}%
\pgfpathlineto{\pgfqpoint{2.749500in}{1.197288in}}%
\pgfpathlineto{\pgfqpoint{2.796000in}{1.115316in}}%
\pgfpathlineto{\pgfqpoint{2.842500in}{1.042198in}}%
\pgfpathlineto{\pgfqpoint{2.889000in}{0.979780in}}%
\pgfpathlineto{\pgfqpoint{2.935500in}{0.929681in}}%
\pgfpathlineto{\pgfqpoint{2.982000in}{0.893038in}}%
\pgfpathlineto{\pgfqpoint{3.028500in}{0.870713in}}%
\pgfpathlineto{\pgfqpoint{3.075000in}{0.863217in}}%
\pgfpathlineto{\pgfqpoint{3.121500in}{0.870713in}}%
\pgfpathlineto{\pgfqpoint{3.168000in}{0.893038in}}%
\pgfpathlineto{\pgfqpoint{3.214500in}{0.929681in}}%
\pgfpathlineto{\pgfqpoint{3.261000in}{0.979780in}}%
\pgfpathlineto{\pgfqpoint{3.307500in}{1.042198in}}%
\pgfpathlineto{\pgfqpoint{3.354000in}{1.115316in}}%
\pgfpathlineto{\pgfqpoint{3.400500in}{1.197288in}}%
\pgfpathlineto{\pgfqpoint{3.447000in}{1.286387in}}%
\pgfpathlineto{\pgfqpoint{3.493500in}{1.381554in}}%
\pgfpathlineto{\pgfqpoint{3.540000in}{1.482246in}}%
\pgfpathlineto{\pgfqpoint{3.586500in}{1.588027in}}%
\pgfpathlineto{\pgfqpoint{3.633000in}{1.697901in}}%
\pgfpathlineto{\pgfqpoint{3.726000in}{1.926583in}}%
\pgfpathlineto{\pgfqpoint{3.819000in}{2.162459in}}%
\pgfpathlineto{\pgfqpoint{3.958500in}{2.524837in}}%
\pgfpathlineto{\pgfqpoint{4.935000in}{5.082819in}}%
\pgfpathlineto{\pgfqpoint{5.400000in}{6.300000in}}%
\pgfpathlineto{\pgfqpoint{5.400000in}{6.300000in}}%
\pgfusepath{stroke}%
\end{pgfscope}%
\begin{pgfscope}%
\pgfpathrectangle{\pgfqpoint{0.750000in}{0.700000in}}{\pgfqpoint{4.650000in}{5.600000in}} %
\pgfusepath{clip}%
\pgfsetbuttcap%
\pgfsetmiterjoin%
\definecolor{currentfill}{rgb}{0.000000,0.500000,0.000000}%
\pgfsetfillcolor{currentfill}%
\pgfsetlinewidth{0.501875pt}%
\definecolor{currentstroke}{rgb}{0.000000,0.000000,0.000000}%
\pgfsetstrokecolor{currentstroke}%
\pgfsetdash{}{0pt}%
\pgfsys@defobject{currentmarker}{\pgfqpoint{-0.041667in}{-0.041667in}}{\pgfqpoint{0.041667in}{0.041667in}}{%
\pgfpathmoveto{\pgfqpoint{0.000000in}{0.041667in}}%
\pgfpathlineto{\pgfqpoint{-0.041667in}{-0.041667in}}%
\pgfpathlineto{\pgfqpoint{0.041667in}{-0.041667in}}%
\pgfpathclose%
\pgfusepath{stroke,fill}%
}%
\begin{pgfscope}%
\pgfsys@transformshift{1.122000in}{5.326292in}%
\pgfsys@useobject{currentmarker}{}%
\end{pgfscope}%
\begin{pgfscope}%
\pgfsys@transformshift{1.587000in}{4.108728in}%
\pgfsys@useobject{currentmarker}{}%
\end{pgfscope}%
\begin{pgfscope}%
\pgfsys@transformshift{2.052000in}{2.890680in}%
\pgfsys@useobject{currentmarker}{}%
\end{pgfscope}%
\begin{pgfscope}%
\pgfsys@transformshift{2.517000in}{1.697901in}%
\pgfsys@useobject{currentmarker}{}%
\end{pgfscope}%
\begin{pgfscope}%
\pgfsys@transformshift{2.982000in}{0.893038in}%
\pgfsys@useobject{currentmarker}{}%
\end{pgfscope}%
\begin{pgfscope}%
\pgfsys@transformshift{3.447000in}{1.286387in}%
\pgfsys@useobject{currentmarker}{}%
\end{pgfscope}%
\begin{pgfscope}%
\pgfsys@transformshift{3.912000in}{2.403244in}%
\pgfsys@useobject{currentmarker}{}%
\end{pgfscope}%
\begin{pgfscope}%
\pgfsys@transformshift{4.377000in}{3.621570in}%
\pgfsys@useobject{currentmarker}{}%
\end{pgfscope}%
\begin{pgfscope}%
\pgfsys@transformshift{4.842000in}{4.839325in}%
\pgfsys@useobject{currentmarker}{}%
\end{pgfscope}%
\begin{pgfscope}%
\pgfsys@transformshift{5.307000in}{6.056597in}%
\pgfsys@useobject{currentmarker}{}%
\end{pgfscope}%
\end{pgfscope}%
\begin{pgfscope}%
\pgfpathrectangle{\pgfqpoint{0.750000in}{0.700000in}}{\pgfqpoint{4.650000in}{5.600000in}} %
\pgfusepath{clip}%
\pgfsetrectcap%
\pgfsetroundjoin%
\pgfsetlinewidth{1.003750pt}%
\definecolor{currentstroke}{rgb}{1.000000,0.000000,0.000000}%
\pgfsetstrokecolor{currentstroke}%
\pgfsetdash{}{0pt}%
\pgfpathmoveto{\pgfqpoint{0.750000in}{6.300000in}}%
\pgfpathlineto{\pgfqpoint{2.307750in}{3.673852in}}%
\pgfpathlineto{\pgfqpoint{2.400750in}{3.522013in}}%
\pgfpathlineto{\pgfqpoint{2.470500in}{3.411590in}}%
\pgfpathlineto{\pgfqpoint{2.540250in}{3.305722in}}%
\pgfpathlineto{\pgfqpoint{2.586750in}{3.238378in}}%
\pgfpathlineto{\pgfqpoint{2.633250in}{3.174162in}}%
\pgfpathlineto{\pgfqpoint{2.679750in}{3.113661in}}%
\pgfpathlineto{\pgfqpoint{2.726250in}{3.057585in}}%
\pgfpathlineto{\pgfqpoint{2.772750in}{3.006515in}}%
\pgfpathlineto{\pgfqpoint{2.819250in}{2.961132in}}%
\pgfpathlineto{\pgfqpoint{2.842500in}{2.940692in}}%
\pgfpathlineto{\pgfqpoint{2.865750in}{2.921856in}}%
\pgfpathlineto{\pgfqpoint{2.889000in}{2.904731in}}%
\pgfpathlineto{\pgfqpoint{2.912250in}{2.889399in}}%
\pgfpathlineto{\pgfqpoint{2.935500in}{2.875934in}}%
\pgfpathlineto{\pgfqpoint{2.958750in}{2.864407in}}%
\pgfpathlineto{\pgfqpoint{2.982000in}{2.854902in}}%
\pgfpathlineto{\pgfqpoint{3.005250in}{2.847466in}}%
\pgfpathlineto{\pgfqpoint{3.028500in}{2.842125in}}%
\pgfpathlineto{\pgfqpoint{3.051750in}{2.838908in}}%
\pgfpathlineto{\pgfqpoint{3.075000in}{2.837833in}}%
\pgfpathlineto{\pgfqpoint{3.098250in}{2.838908in}}%
\pgfpathlineto{\pgfqpoint{3.121500in}{2.842125in}}%
\pgfpathlineto{\pgfqpoint{3.144750in}{2.847466in}}%
\pgfpathlineto{\pgfqpoint{3.168000in}{2.854902in}}%
\pgfpathlineto{\pgfqpoint{3.191250in}{2.864407in}}%
\pgfpathlineto{\pgfqpoint{3.214500in}{2.875934in}}%
\pgfpathlineto{\pgfqpoint{3.237750in}{2.889399in}}%
\pgfpathlineto{\pgfqpoint{3.261000in}{2.904731in}}%
\pgfpathlineto{\pgfqpoint{3.284250in}{2.921856in}}%
\pgfpathlineto{\pgfqpoint{3.307500in}{2.940692in}}%
\pgfpathlineto{\pgfqpoint{3.330750in}{2.961132in}}%
\pgfpathlineto{\pgfqpoint{3.377250in}{3.006515in}}%
\pgfpathlineto{\pgfqpoint{3.423750in}{3.057585in}}%
\pgfpathlineto{\pgfqpoint{3.470250in}{3.113661in}}%
\pgfpathlineto{\pgfqpoint{3.516750in}{3.174162in}}%
\pgfpathlineto{\pgfqpoint{3.563250in}{3.238378in}}%
\pgfpathlineto{\pgfqpoint{3.609750in}{3.305722in}}%
\pgfpathlineto{\pgfqpoint{3.679500in}{3.411590in}}%
\pgfpathlineto{\pgfqpoint{3.749250in}{3.522013in}}%
\pgfpathlineto{\pgfqpoint{3.842250in}{3.673852in}}%
\pgfpathlineto{\pgfqpoint{3.981750in}{3.907163in}}%
\pgfpathlineto{\pgfqpoint{5.121000in}{5.829335in}}%
\pgfpathlineto{\pgfqpoint{5.400000in}{6.300000in}}%
\pgfpathlineto{\pgfqpoint{5.400000in}{6.300000in}}%
\pgfusepath{stroke}%
\end{pgfscope}%
\begin{pgfscope}%
\pgfpathrectangle{\pgfqpoint{0.750000in}{0.700000in}}{\pgfqpoint{4.650000in}{5.600000in}} %
\pgfusepath{clip}%
\pgfsetbuttcap%
\pgfsetmiterjoin%
\definecolor{currentfill}{rgb}{1.000000,0.000000,0.000000}%
\pgfsetfillcolor{currentfill}%
\pgfsetlinewidth{0.501875pt}%
\definecolor{currentstroke}{rgb}{0.000000,0.000000,0.000000}%
\pgfsetstrokecolor{currentstroke}%
\pgfsetdash{}{0pt}%
\pgfsys@defobject{currentmarker}{\pgfqpoint{-0.041667in}{-0.041667in}}{\pgfqpoint{0.041667in}{0.041667in}}{%
\pgfpathmoveto{\pgfqpoint{0.041667in}{-0.000000in}}%
\pgfpathlineto{\pgfqpoint{-0.041667in}{0.041667in}}%
\pgfpathlineto{\pgfqpoint{-0.041667in}{-0.041667in}}%
\pgfpathclose%
\pgfusepath{stroke,fill}%
}%
\begin{pgfscope}%
\pgfsys@transformshift{0.796500in}{6.221557in}%
\pgfsys@useobject{currentmarker}{}%
\end{pgfscope}%
\begin{pgfscope}%
\pgfsys@transformshift{1.261500in}{5.437096in}%
\pgfsys@useobject{currentmarker}{}%
\end{pgfscope}%
\begin{pgfscope}%
\pgfsys@transformshift{1.726500in}{4.652570in}%
\pgfsys@useobject{currentmarker}{}%
\end{pgfscope}%
\begin{pgfscope}%
\pgfsys@transformshift{2.191500in}{3.868035in}%
\pgfsys@useobject{currentmarker}{}%
\end{pgfscope}%
\begin{pgfscope}%
\pgfsys@transformshift{2.656500in}{3.143402in}%
\pgfsys@useobject{currentmarker}{}%
\end{pgfscope}%
\begin{pgfscope}%
\pgfsys@transformshift{3.121500in}{2.842125in}%
\pgfsys@useobject{currentmarker}{}%
\end{pgfscope}%
\begin{pgfscope}%
\pgfsys@transformshift{3.586500in}{3.271685in}%
\pgfsys@useobject{currentmarker}{}%
\end{pgfscope}%
\begin{pgfscope}%
\pgfsys@transformshift{4.051500in}{4.024823in}%
\pgfsys@useobject{currentmarker}{}%
\end{pgfscope}%
\begin{pgfscope}%
\pgfsys@transformshift{4.516500in}{4.809479in}%
\pgfsys@useobject{currentmarker}{}%
\end{pgfscope}%
\begin{pgfscope}%
\pgfsys@transformshift{4.981500in}{5.593993in}%
\pgfsys@useobject{currentmarker}{}%
\end{pgfscope}%
\end{pgfscope}%
\begin{pgfscope}%
\pgfpathrectangle{\pgfqpoint{0.750000in}{0.700000in}}{\pgfqpoint{4.650000in}{5.600000in}} %
\pgfusepath{clip}%
\pgfsetrectcap%
\pgfsetroundjoin%
\pgfsetlinewidth{1.003750pt}%
\definecolor{currentstroke}{rgb}{0.000000,0.750000,0.750000}%
\pgfsetstrokecolor{currentstroke}%
\pgfsetdash{}{0pt}%
\pgfpathmoveto{\pgfqpoint{0.750000in}{6.300000in}}%
\pgfpathlineto{\pgfqpoint{2.321700in}{4.431354in}}%
\pgfpathlineto{\pgfqpoint{2.414700in}{4.324776in}}%
\pgfpathlineto{\pgfqpoint{2.489100in}{4.242679in}}%
\pgfpathlineto{\pgfqpoint{2.554200in}{4.174200in}}%
\pgfpathlineto{\pgfqpoint{2.610000in}{4.118760in}}%
\pgfpathlineto{\pgfqpoint{2.656500in}{4.075348in}}%
\pgfpathlineto{\pgfqpoint{2.703000in}{4.034912in}}%
\pgfpathlineto{\pgfqpoint{2.749500in}{3.997892in}}%
\pgfpathlineto{\pgfqpoint{2.786700in}{3.970993in}}%
\pgfpathlineto{\pgfqpoint{2.823900in}{3.946772in}}%
\pgfpathlineto{\pgfqpoint{2.861100in}{3.925414in}}%
\pgfpathlineto{\pgfqpoint{2.898300in}{3.907101in}}%
\pgfpathlineto{\pgfqpoint{2.926200in}{3.895492in}}%
\pgfpathlineto{\pgfqpoint{2.954100in}{3.885770in}}%
\pgfpathlineto{\pgfqpoint{2.982000in}{3.877991in}}%
\pgfpathlineto{\pgfqpoint{3.009900in}{3.872204in}}%
\pgfpathlineto{\pgfqpoint{3.037800in}{3.868448in}}%
\pgfpathlineto{\pgfqpoint{3.065700in}{3.866736in}}%
\pgfpathlineto{\pgfqpoint{3.093600in}{3.867078in}}%
\pgfpathlineto{\pgfqpoint{3.121500in}{3.869473in}}%
\pgfpathlineto{\pgfqpoint{3.149400in}{3.873908in}}%
\pgfpathlineto{\pgfqpoint{3.177300in}{3.880365in}}%
\pgfpathlineto{\pgfqpoint{3.205200in}{3.888797in}}%
\pgfpathlineto{\pgfqpoint{3.233100in}{3.899155in}}%
\pgfpathlineto{\pgfqpoint{3.261000in}{3.911379in}}%
\pgfpathlineto{\pgfqpoint{3.298200in}{3.930478in}}%
\pgfpathlineto{\pgfqpoint{3.335400in}{3.952565in}}%
\pgfpathlineto{\pgfqpoint{3.372600in}{3.977476in}}%
\pgfpathlineto{\pgfqpoint{3.409800in}{4.005007in}}%
\pgfpathlineto{\pgfqpoint{3.447000in}{4.034912in}}%
\pgfpathlineto{\pgfqpoint{3.493500in}{4.075347in}}%
\pgfpathlineto{\pgfqpoint{3.540000in}{4.118760in}}%
\pgfpathlineto{\pgfqpoint{3.595800in}{4.174200in}}%
\pgfpathlineto{\pgfqpoint{3.660900in}{4.242678in}}%
\pgfpathlineto{\pgfqpoint{3.735300in}{4.324776in}}%
\pgfpathlineto{\pgfqpoint{3.828300in}{4.431354in}}%
\pgfpathlineto{\pgfqpoint{3.958500in}{4.584603in}}%
\pgfpathlineto{\pgfqpoint{4.442100in}{5.160064in}}%
\pgfpathlineto{\pgfqpoint{5.400000in}{6.300000in}}%
\pgfpathlineto{\pgfqpoint{5.400000in}{6.300000in}}%
\pgfusepath{stroke}%
\end{pgfscope}%
\begin{pgfscope}%
\pgfpathrectangle{\pgfqpoint{0.750000in}{0.700000in}}{\pgfqpoint{4.650000in}{5.600000in}} %
\pgfusepath{clip}%
\pgfsetbuttcap%
\pgfsetmiterjoin%
\definecolor{currentfill}{rgb}{0.000000,0.750000,0.750000}%
\pgfsetfillcolor{currentfill}%
\pgfsetlinewidth{0.501875pt}%
\definecolor{currentstroke}{rgb}{0.000000,0.000000,0.000000}%
\pgfsetstrokecolor{currentstroke}%
\pgfsetdash{}{0pt}%
\pgfsys@defobject{currentmarker}{\pgfqpoint{-0.041667in}{-0.041667in}}{\pgfqpoint{0.041667in}{0.041667in}}{%
\pgfpathmoveto{\pgfqpoint{-0.000000in}{-0.041667in}}%
\pgfpathlineto{\pgfqpoint{0.041667in}{0.041667in}}%
\pgfpathlineto{\pgfqpoint{-0.041667in}{0.041667in}}%
\pgfpathclose%
\pgfusepath{stroke,fill}%
}%
\begin{pgfscope}%
\pgfsys@transformshift{1.001100in}{6.001186in}%
\pgfsys@useobject{currentmarker}{}%
\end{pgfscope}%
\begin{pgfscope}%
\pgfsys@transformshift{1.466100in}{5.447819in}%
\pgfsys@useobject{currentmarker}{}%
\end{pgfscope}%
\begin{pgfscope}%
\pgfsys@transformshift{1.931100in}{4.894446in}%
\pgfsys@useobject{currentmarker}{}%
\end{pgfscope}%
\begin{pgfscope}%
\pgfsys@transformshift{2.396100in}{4.345801in}%
\pgfsys@useobject{currentmarker}{}%
\end{pgfscope}%
\begin{pgfscope}%
\pgfsys@transformshift{2.861100in}{3.925414in}%
\pgfsys@useobject{currentmarker}{}%
\end{pgfscope}%
\begin{pgfscope}%
\pgfsys@transformshift{3.326100in}{3.946771in}%
\pgfsys@useobject{currentmarker}{}%
\end{pgfscope}%
\begin{pgfscope}%
\pgfsys@transformshift{3.791100in}{4.388312in}%
\pgfsys@useobject{currentmarker}{}%
\end{pgfscope}%
\begin{pgfscope}%
\pgfsys@transformshift{4.256100in}{4.938714in}%
\pgfsys@useobject{currentmarker}{}%
\end{pgfscope}%
\begin{pgfscope}%
\pgfsys@transformshift{4.721100in}{5.492089in}%
\pgfsys@useobject{currentmarker}{}%
\end{pgfscope}%
\begin{pgfscope}%
\pgfsys@transformshift{5.186100in}{6.045455in}%
\pgfsys@useobject{currentmarker}{}%
\end{pgfscope}%
\end{pgfscope}%
\begin{pgfscope}%
\pgfpathrectangle{\pgfqpoint{0.750000in}{0.700000in}}{\pgfqpoint{4.650000in}{5.600000in}} %
\pgfusepath{clip}%
\pgfsetbuttcap%
\pgfsetroundjoin%
\pgfsetlinewidth{0.501875pt}%
\definecolor{currentstroke}{rgb}{0.000000,0.000000,0.000000}%
\pgfsetstrokecolor{currentstroke}%
\pgfsetdash{{1.000000pt}{3.000000pt}}{0.000000pt}%
\pgfpathmoveto{\pgfqpoint{0.750000in}{0.700000in}}%
\pgfpathlineto{\pgfqpoint{0.750000in}{6.300000in}}%
\pgfusepath{stroke}%
\end{pgfscope}%
\begin{pgfscope}%
\pgfsetbuttcap%
\pgfsetroundjoin%
\definecolor{currentfill}{rgb}{0.000000,0.000000,0.000000}%
\pgfsetfillcolor{currentfill}%
\pgfsetlinewidth{0.501875pt}%
\definecolor{currentstroke}{rgb}{0.000000,0.000000,0.000000}%
\pgfsetstrokecolor{currentstroke}%
\pgfsetdash{}{0pt}%
\pgfsys@defobject{currentmarker}{\pgfqpoint{0.000000in}{0.000000in}}{\pgfqpoint{0.000000in}{0.055556in}}{%
\pgfpathmoveto{\pgfqpoint{0.000000in}{0.000000in}}%
\pgfpathlineto{\pgfqpoint{0.000000in}{0.055556in}}%
\pgfusepath{stroke,fill}%
}%
\begin{pgfscope}%
\pgfsys@transformshift{0.750000in}{0.700000in}%
\pgfsys@useobject{currentmarker}{}%
\end{pgfscope}%
\end{pgfscope}%
\begin{pgfscope}%
\pgfsetbuttcap%
\pgfsetroundjoin%
\definecolor{currentfill}{rgb}{0.000000,0.000000,0.000000}%
\pgfsetfillcolor{currentfill}%
\pgfsetlinewidth{0.501875pt}%
\definecolor{currentstroke}{rgb}{0.000000,0.000000,0.000000}%
\pgfsetstrokecolor{currentstroke}%
\pgfsetdash{}{0pt}%
\pgfsys@defobject{currentmarker}{\pgfqpoint{0.000000in}{-0.055556in}}{\pgfqpoint{0.000000in}{0.000000in}}{%
\pgfpathmoveto{\pgfqpoint{0.000000in}{0.000000in}}%
\pgfpathlineto{\pgfqpoint{0.000000in}{-0.055556in}}%
\pgfusepath{stroke,fill}%
}%
\begin{pgfscope}%
\pgfsys@transformshift{0.750000in}{6.300000in}%
\pgfsys@useobject{currentmarker}{}%
\end{pgfscope}%
\end{pgfscope}%
\begin{pgfscope}%
\pgftext[left,bottom,x=0.645738in,y=0.537037in,rotate=0.000000]{{\rmfamily\fontsize{12.000000}{14.400000}\selectfont \(\displaystyle 0.0\)}}
%
\end{pgfscope}%
\begin{pgfscope}%
\pgfpathrectangle{\pgfqpoint{0.750000in}{0.700000in}}{\pgfqpoint{4.650000in}{5.600000in}} %
\pgfusepath{clip}%
\pgfsetbuttcap%
\pgfsetroundjoin%
\pgfsetlinewidth{0.501875pt}%
\definecolor{currentstroke}{rgb}{0.000000,0.000000,0.000000}%
\pgfsetstrokecolor{currentstroke}%
\pgfsetdash{{1.000000pt}{3.000000pt}}{0.000000pt}%
\pgfpathmoveto{\pgfqpoint{1.912500in}{0.700000in}}%
\pgfpathlineto{\pgfqpoint{1.912500in}{6.300000in}}%
\pgfusepath{stroke}%
\end{pgfscope}%
\begin{pgfscope}%
\pgfsetbuttcap%
\pgfsetroundjoin%
\definecolor{currentfill}{rgb}{0.000000,0.000000,0.000000}%
\pgfsetfillcolor{currentfill}%
\pgfsetlinewidth{0.501875pt}%
\definecolor{currentstroke}{rgb}{0.000000,0.000000,0.000000}%
\pgfsetstrokecolor{currentstroke}%
\pgfsetdash{}{0pt}%
\pgfsys@defobject{currentmarker}{\pgfqpoint{0.000000in}{0.000000in}}{\pgfqpoint{0.000000in}{0.055556in}}{%
\pgfpathmoveto{\pgfqpoint{0.000000in}{0.000000in}}%
\pgfpathlineto{\pgfqpoint{0.000000in}{0.055556in}}%
\pgfusepath{stroke,fill}%
}%
\begin{pgfscope}%
\pgfsys@transformshift{1.912500in}{0.700000in}%
\pgfsys@useobject{currentmarker}{}%
\end{pgfscope}%
\end{pgfscope}%
\begin{pgfscope}%
\pgfsetbuttcap%
\pgfsetroundjoin%
\definecolor{currentfill}{rgb}{0.000000,0.000000,0.000000}%
\pgfsetfillcolor{currentfill}%
\pgfsetlinewidth{0.501875pt}%
\definecolor{currentstroke}{rgb}{0.000000,0.000000,0.000000}%
\pgfsetstrokecolor{currentstroke}%
\pgfsetdash{}{0pt}%
\pgfsys@defobject{currentmarker}{\pgfqpoint{0.000000in}{-0.055556in}}{\pgfqpoint{0.000000in}{0.000000in}}{%
\pgfpathmoveto{\pgfqpoint{0.000000in}{0.000000in}}%
\pgfpathlineto{\pgfqpoint{0.000000in}{-0.055556in}}%
\pgfusepath{stroke,fill}%
}%
\begin{pgfscope}%
\pgfsys@transformshift{1.912500in}{6.300000in}%
\pgfsys@useobject{currentmarker}{}%
\end{pgfscope}%
\end{pgfscope}%
\begin{pgfscope}%
\pgftext[left,bottom,x=1.808238in,y=0.537037in,rotate=0.000000]{{\rmfamily\fontsize{12.000000}{14.400000}\selectfont \(\displaystyle 0.5\)}}
%
\end{pgfscope}%
\begin{pgfscope}%
\pgfpathrectangle{\pgfqpoint{0.750000in}{0.700000in}}{\pgfqpoint{4.650000in}{5.600000in}} %
\pgfusepath{clip}%
\pgfsetbuttcap%
\pgfsetroundjoin%
\pgfsetlinewidth{0.501875pt}%
\definecolor{currentstroke}{rgb}{0.000000,0.000000,0.000000}%
\pgfsetstrokecolor{currentstroke}%
\pgfsetdash{{1.000000pt}{3.000000pt}}{0.000000pt}%
\pgfpathmoveto{\pgfqpoint{3.075000in}{0.700000in}}%
\pgfpathlineto{\pgfqpoint{3.075000in}{6.300000in}}%
\pgfusepath{stroke}%
\end{pgfscope}%
\begin{pgfscope}%
\pgfsetbuttcap%
\pgfsetroundjoin%
\definecolor{currentfill}{rgb}{0.000000,0.000000,0.000000}%
\pgfsetfillcolor{currentfill}%
\pgfsetlinewidth{0.501875pt}%
\definecolor{currentstroke}{rgb}{0.000000,0.000000,0.000000}%
\pgfsetstrokecolor{currentstroke}%
\pgfsetdash{}{0pt}%
\pgfsys@defobject{currentmarker}{\pgfqpoint{0.000000in}{0.000000in}}{\pgfqpoint{0.000000in}{0.055556in}}{%
\pgfpathmoveto{\pgfqpoint{0.000000in}{0.000000in}}%
\pgfpathlineto{\pgfqpoint{0.000000in}{0.055556in}}%
\pgfusepath{stroke,fill}%
}%
\begin{pgfscope}%
\pgfsys@transformshift{3.075000in}{0.700000in}%
\pgfsys@useobject{currentmarker}{}%
\end{pgfscope}%
\end{pgfscope}%
\begin{pgfscope}%
\pgfsetbuttcap%
\pgfsetroundjoin%
\definecolor{currentfill}{rgb}{0.000000,0.000000,0.000000}%
\pgfsetfillcolor{currentfill}%
\pgfsetlinewidth{0.501875pt}%
\definecolor{currentstroke}{rgb}{0.000000,0.000000,0.000000}%
\pgfsetstrokecolor{currentstroke}%
\pgfsetdash{}{0pt}%
\pgfsys@defobject{currentmarker}{\pgfqpoint{0.000000in}{-0.055556in}}{\pgfqpoint{0.000000in}{0.000000in}}{%
\pgfpathmoveto{\pgfqpoint{0.000000in}{0.000000in}}%
\pgfpathlineto{\pgfqpoint{0.000000in}{-0.055556in}}%
\pgfusepath{stroke,fill}%
}%
\begin{pgfscope}%
\pgfsys@transformshift{3.075000in}{6.300000in}%
\pgfsys@useobject{currentmarker}{}%
\end{pgfscope}%
\end{pgfscope}%
\begin{pgfscope}%
\pgftext[left,bottom,x=2.970738in,y=0.537037in,rotate=0.000000]{{\rmfamily\fontsize{12.000000}{14.400000}\selectfont \(\displaystyle 1.0\)}}
%
\end{pgfscope}%
\begin{pgfscope}%
\pgfpathrectangle{\pgfqpoint{0.750000in}{0.700000in}}{\pgfqpoint{4.650000in}{5.600000in}} %
\pgfusepath{clip}%
\pgfsetbuttcap%
\pgfsetroundjoin%
\pgfsetlinewidth{0.501875pt}%
\definecolor{currentstroke}{rgb}{0.000000,0.000000,0.000000}%
\pgfsetstrokecolor{currentstroke}%
\pgfsetdash{{1.000000pt}{3.000000pt}}{0.000000pt}%
\pgfpathmoveto{\pgfqpoint{4.237500in}{0.700000in}}%
\pgfpathlineto{\pgfqpoint{4.237500in}{6.300000in}}%
\pgfusepath{stroke}%
\end{pgfscope}%
\begin{pgfscope}%
\pgfsetbuttcap%
\pgfsetroundjoin%
\definecolor{currentfill}{rgb}{0.000000,0.000000,0.000000}%
\pgfsetfillcolor{currentfill}%
\pgfsetlinewidth{0.501875pt}%
\definecolor{currentstroke}{rgb}{0.000000,0.000000,0.000000}%
\pgfsetstrokecolor{currentstroke}%
\pgfsetdash{}{0pt}%
\pgfsys@defobject{currentmarker}{\pgfqpoint{0.000000in}{0.000000in}}{\pgfqpoint{0.000000in}{0.055556in}}{%
\pgfpathmoveto{\pgfqpoint{0.000000in}{0.000000in}}%
\pgfpathlineto{\pgfqpoint{0.000000in}{0.055556in}}%
\pgfusepath{stroke,fill}%
}%
\begin{pgfscope}%
\pgfsys@transformshift{4.237500in}{0.700000in}%
\pgfsys@useobject{currentmarker}{}%
\end{pgfscope}%
\end{pgfscope}%
\begin{pgfscope}%
\pgfsetbuttcap%
\pgfsetroundjoin%
\definecolor{currentfill}{rgb}{0.000000,0.000000,0.000000}%
\pgfsetfillcolor{currentfill}%
\pgfsetlinewidth{0.501875pt}%
\definecolor{currentstroke}{rgb}{0.000000,0.000000,0.000000}%
\pgfsetstrokecolor{currentstroke}%
\pgfsetdash{}{0pt}%
\pgfsys@defobject{currentmarker}{\pgfqpoint{0.000000in}{-0.055556in}}{\pgfqpoint{0.000000in}{0.000000in}}{%
\pgfpathmoveto{\pgfqpoint{0.000000in}{0.000000in}}%
\pgfpathlineto{\pgfqpoint{0.000000in}{-0.055556in}}%
\pgfusepath{stroke,fill}%
}%
\begin{pgfscope}%
\pgfsys@transformshift{4.237500in}{6.300000in}%
\pgfsys@useobject{currentmarker}{}%
\end{pgfscope}%
\end{pgfscope}%
\begin{pgfscope}%
\pgftext[left,bottom,x=4.133238in,y=0.537037in,rotate=0.000000]{{\rmfamily\fontsize{12.000000}{14.400000}\selectfont \(\displaystyle 1.5\)}}
%
\end{pgfscope}%
\begin{pgfscope}%
\pgfpathrectangle{\pgfqpoint{0.750000in}{0.700000in}}{\pgfqpoint{4.650000in}{5.600000in}} %
\pgfusepath{clip}%
\pgfsetbuttcap%
\pgfsetroundjoin%
\pgfsetlinewidth{0.501875pt}%
\definecolor{currentstroke}{rgb}{0.000000,0.000000,0.000000}%
\pgfsetstrokecolor{currentstroke}%
\pgfsetdash{{1.000000pt}{3.000000pt}}{0.000000pt}%
\pgfpathmoveto{\pgfqpoint{5.400000in}{0.700000in}}%
\pgfpathlineto{\pgfqpoint{5.400000in}{6.300000in}}%
\pgfusepath{stroke}%
\end{pgfscope}%
\begin{pgfscope}%
\pgfsetbuttcap%
\pgfsetroundjoin%
\definecolor{currentfill}{rgb}{0.000000,0.000000,0.000000}%
\pgfsetfillcolor{currentfill}%
\pgfsetlinewidth{0.501875pt}%
\definecolor{currentstroke}{rgb}{0.000000,0.000000,0.000000}%
\pgfsetstrokecolor{currentstroke}%
\pgfsetdash{}{0pt}%
\pgfsys@defobject{currentmarker}{\pgfqpoint{0.000000in}{0.000000in}}{\pgfqpoint{0.000000in}{0.055556in}}{%
\pgfpathmoveto{\pgfqpoint{0.000000in}{0.000000in}}%
\pgfpathlineto{\pgfqpoint{0.000000in}{0.055556in}}%
\pgfusepath{stroke,fill}%
}%
\begin{pgfscope}%
\pgfsys@transformshift{5.400000in}{0.700000in}%
\pgfsys@useobject{currentmarker}{}%
\end{pgfscope}%
\end{pgfscope}%
\begin{pgfscope}%
\pgfsetbuttcap%
\pgfsetroundjoin%
\definecolor{currentfill}{rgb}{0.000000,0.000000,0.000000}%
\pgfsetfillcolor{currentfill}%
\pgfsetlinewidth{0.501875pt}%
\definecolor{currentstroke}{rgb}{0.000000,0.000000,0.000000}%
\pgfsetstrokecolor{currentstroke}%
\pgfsetdash{}{0pt}%
\pgfsys@defobject{currentmarker}{\pgfqpoint{0.000000in}{-0.055556in}}{\pgfqpoint{0.000000in}{0.000000in}}{%
\pgfpathmoveto{\pgfqpoint{0.000000in}{0.000000in}}%
\pgfpathlineto{\pgfqpoint{0.000000in}{-0.055556in}}%
\pgfusepath{stroke,fill}%
}%
\begin{pgfscope}%
\pgfsys@transformshift{5.400000in}{6.300000in}%
\pgfsys@useobject{currentmarker}{}%
\end{pgfscope}%
\end{pgfscope}%
\begin{pgfscope}%
\pgftext[left,bottom,x=5.295738in,y=0.537037in,rotate=0.000000]{{\rmfamily\fontsize{12.000000}{14.400000}\selectfont \(\displaystyle 2.0\)}}
%
\end{pgfscope}%
\begin{pgfscope}%
\pgftext[left,bottom,x=2.319809in,y=0.319445in,rotate=0.000000]{{\rmfamily\fontsize{12.000000}{14.400000}\selectfont Distance along Beam}}
%
\end{pgfscope}%
\begin{pgfscope}%
\pgfpathrectangle{\pgfqpoint{0.750000in}{0.700000in}}{\pgfqpoint{4.650000in}{5.600000in}} %
\pgfusepath{clip}%
\pgfsetbuttcap%
\pgfsetroundjoin%
\pgfsetlinewidth{0.501875pt}%
\definecolor{currentstroke}{rgb}{0.000000,0.000000,0.000000}%
\pgfsetstrokecolor{currentstroke}%
\pgfsetdash{{1.000000pt}{3.000000pt}}{0.000000pt}%
\pgfpathmoveto{\pgfqpoint{0.750000in}{6.300000in}}%
\pgfpathlineto{\pgfqpoint{5.400000in}{6.300000in}}%
\pgfusepath{stroke}%
\end{pgfscope}%
\begin{pgfscope}%
\pgfsetbuttcap%
\pgfsetroundjoin%
\definecolor{currentfill}{rgb}{0.000000,0.000000,0.000000}%
\pgfsetfillcolor{currentfill}%
\pgfsetlinewidth{0.501875pt}%
\definecolor{currentstroke}{rgb}{0.000000,0.000000,0.000000}%
\pgfsetstrokecolor{currentstroke}%
\pgfsetdash{}{0pt}%
\pgfsys@defobject{currentmarker}{\pgfqpoint{0.000000in}{0.000000in}}{\pgfqpoint{0.055556in}{0.000000in}}{%
\pgfpathmoveto{\pgfqpoint{0.000000in}{0.000000in}}%
\pgfpathlineto{\pgfqpoint{0.055556in}{0.000000in}}%
\pgfusepath{stroke,fill}%
}%
\begin{pgfscope}%
\pgfsys@transformshift{0.750000in}{6.300000in}%
\pgfsys@useobject{currentmarker}{}%
\end{pgfscope}%
\end{pgfscope}%
\begin{pgfscope}%
\pgfsetbuttcap%
\pgfsetroundjoin%
\definecolor{currentfill}{rgb}{0.000000,0.000000,0.000000}%
\pgfsetfillcolor{currentfill}%
\pgfsetlinewidth{0.501875pt}%
\definecolor{currentstroke}{rgb}{0.000000,0.000000,0.000000}%
\pgfsetstrokecolor{currentstroke}%
\pgfsetdash{}{0pt}%
\pgfsys@defobject{currentmarker}{\pgfqpoint{-0.055556in}{0.000000in}}{\pgfqpoint{0.000000in}{0.000000in}}{%
\pgfpathmoveto{\pgfqpoint{0.000000in}{0.000000in}}%
\pgfpathlineto{\pgfqpoint{-0.055556in}{0.000000in}}%
\pgfusepath{stroke,fill}%
}%
\begin{pgfscope}%
\pgfsys@transformshift{5.400000in}{6.300000in}%
\pgfsys@useobject{currentmarker}{}%
\end{pgfscope}%
\end{pgfscope}%
\begin{pgfscope}%
\pgftext[left,bottom,x=0.485920in,y=6.246296in,rotate=0.000000]{{\rmfamily\fontsize{12.000000}{14.400000}\selectfont \(\displaystyle 0.0\)}}
%
\end{pgfscope}%
\begin{pgfscope}%
\pgfpathrectangle{\pgfqpoint{0.750000in}{0.700000in}}{\pgfqpoint{4.650000in}{5.600000in}} %
\pgfusepath{clip}%
\pgfsetbuttcap%
\pgfsetroundjoin%
\pgfsetlinewidth{0.501875pt}%
\definecolor{currentstroke}{rgb}{0.000000,0.000000,0.000000}%
\pgfsetstrokecolor{currentstroke}%
\pgfsetdash{{1.000000pt}{3.000000pt}}{0.000000pt}%
\pgfpathmoveto{\pgfqpoint{0.750000in}{4.900000in}}%
\pgfpathlineto{\pgfqpoint{5.400000in}{4.900000in}}%
\pgfusepath{stroke}%
\end{pgfscope}%
\begin{pgfscope}%
\pgfsetbuttcap%
\pgfsetroundjoin%
\definecolor{currentfill}{rgb}{0.000000,0.000000,0.000000}%
\pgfsetfillcolor{currentfill}%
\pgfsetlinewidth{0.501875pt}%
\definecolor{currentstroke}{rgb}{0.000000,0.000000,0.000000}%
\pgfsetstrokecolor{currentstroke}%
\pgfsetdash{}{0pt}%
\pgfsys@defobject{currentmarker}{\pgfqpoint{0.000000in}{0.000000in}}{\pgfqpoint{0.055556in}{0.000000in}}{%
\pgfpathmoveto{\pgfqpoint{0.000000in}{0.000000in}}%
\pgfpathlineto{\pgfqpoint{0.055556in}{0.000000in}}%
\pgfusepath{stroke,fill}%
}%
\begin{pgfscope}%
\pgfsys@transformshift{0.750000in}{4.900000in}%
\pgfsys@useobject{currentmarker}{}%
\end{pgfscope}%
\end{pgfscope}%
\begin{pgfscope}%
\pgfsetbuttcap%
\pgfsetroundjoin%
\definecolor{currentfill}{rgb}{0.000000,0.000000,0.000000}%
\pgfsetfillcolor{currentfill}%
\pgfsetlinewidth{0.501875pt}%
\definecolor{currentstroke}{rgb}{0.000000,0.000000,0.000000}%
\pgfsetstrokecolor{currentstroke}%
\pgfsetdash{}{0pt}%
\pgfsys@defobject{currentmarker}{\pgfqpoint{-0.055556in}{0.000000in}}{\pgfqpoint{0.000000in}{0.000000in}}{%
\pgfpathmoveto{\pgfqpoint{0.000000in}{0.000000in}}%
\pgfpathlineto{\pgfqpoint{-0.055556in}{0.000000in}}%
\pgfusepath{stroke,fill}%
}%
\begin{pgfscope}%
\pgfsys@transformshift{5.400000in}{4.900000in}%
\pgfsys@useobject{currentmarker}{}%
\end{pgfscope}%
\end{pgfscope}%
\begin{pgfscope}%
\pgftext[left,bottom,x=0.356290in,y=4.839352in,rotate=0.000000]{{\rmfamily\fontsize{12.000000}{14.400000}\selectfont \(\displaystyle -0.8\)}}
%
\end{pgfscope}%
\begin{pgfscope}%
\pgfpathrectangle{\pgfqpoint{0.750000in}{0.700000in}}{\pgfqpoint{4.650000in}{5.600000in}} %
\pgfusepath{clip}%
\pgfsetbuttcap%
\pgfsetroundjoin%
\pgfsetlinewidth{0.501875pt}%
\definecolor{currentstroke}{rgb}{0.000000,0.000000,0.000000}%
\pgfsetstrokecolor{currentstroke}%
\pgfsetdash{{1.000000pt}{3.000000pt}}{0.000000pt}%
\pgfpathmoveto{\pgfqpoint{0.750000in}{3.500000in}}%
\pgfpathlineto{\pgfqpoint{5.400000in}{3.500000in}}%
\pgfusepath{stroke}%
\end{pgfscope}%
\begin{pgfscope}%
\pgfsetbuttcap%
\pgfsetroundjoin%
\definecolor{currentfill}{rgb}{0.000000,0.000000,0.000000}%
\pgfsetfillcolor{currentfill}%
\pgfsetlinewidth{0.501875pt}%
\definecolor{currentstroke}{rgb}{0.000000,0.000000,0.000000}%
\pgfsetstrokecolor{currentstroke}%
\pgfsetdash{}{0pt}%
\pgfsys@defobject{currentmarker}{\pgfqpoint{0.000000in}{0.000000in}}{\pgfqpoint{0.055556in}{0.000000in}}{%
\pgfpathmoveto{\pgfqpoint{0.000000in}{0.000000in}}%
\pgfpathlineto{\pgfqpoint{0.055556in}{0.000000in}}%
\pgfusepath{stroke,fill}%
}%
\begin{pgfscope}%
\pgfsys@transformshift{0.750000in}{3.500000in}%
\pgfsys@useobject{currentmarker}{}%
\end{pgfscope}%
\end{pgfscope}%
\begin{pgfscope}%
\pgfsetbuttcap%
\pgfsetroundjoin%
\definecolor{currentfill}{rgb}{0.000000,0.000000,0.000000}%
\pgfsetfillcolor{currentfill}%
\pgfsetlinewidth{0.501875pt}%
\definecolor{currentstroke}{rgb}{0.000000,0.000000,0.000000}%
\pgfsetstrokecolor{currentstroke}%
\pgfsetdash{}{0pt}%
\pgfsys@defobject{currentmarker}{\pgfqpoint{-0.055556in}{0.000000in}}{\pgfqpoint{0.000000in}{0.000000in}}{%
\pgfpathmoveto{\pgfqpoint{0.000000in}{0.000000in}}%
\pgfpathlineto{\pgfqpoint{-0.055556in}{0.000000in}}%
\pgfusepath{stroke,fill}%
}%
\begin{pgfscope}%
\pgfsys@transformshift{5.400000in}{3.500000in}%
\pgfsys@useobject{currentmarker}{}%
\end{pgfscope}%
\end{pgfscope}%
\begin{pgfscope}%
\pgftext[left,bottom,x=0.356290in,y=3.439352in,rotate=0.000000]{{\rmfamily\fontsize{12.000000}{14.400000}\selectfont \(\displaystyle -1.6\)}}
%
\end{pgfscope}%
\begin{pgfscope}%
\pgfpathrectangle{\pgfqpoint{0.750000in}{0.700000in}}{\pgfqpoint{4.650000in}{5.600000in}} %
\pgfusepath{clip}%
\pgfsetbuttcap%
\pgfsetroundjoin%
\pgfsetlinewidth{0.501875pt}%
\definecolor{currentstroke}{rgb}{0.000000,0.000000,0.000000}%
\pgfsetstrokecolor{currentstroke}%
\pgfsetdash{{1.000000pt}{3.000000pt}}{0.000000pt}%
\pgfpathmoveto{\pgfqpoint{0.750000in}{2.100000in}}%
\pgfpathlineto{\pgfqpoint{5.400000in}{2.100000in}}%
\pgfusepath{stroke}%
\end{pgfscope}%
\begin{pgfscope}%
\pgfsetbuttcap%
\pgfsetroundjoin%
\definecolor{currentfill}{rgb}{0.000000,0.000000,0.000000}%
\pgfsetfillcolor{currentfill}%
\pgfsetlinewidth{0.501875pt}%
\definecolor{currentstroke}{rgb}{0.000000,0.000000,0.000000}%
\pgfsetstrokecolor{currentstroke}%
\pgfsetdash{}{0pt}%
\pgfsys@defobject{currentmarker}{\pgfqpoint{0.000000in}{0.000000in}}{\pgfqpoint{0.055556in}{0.000000in}}{%
\pgfpathmoveto{\pgfqpoint{0.000000in}{0.000000in}}%
\pgfpathlineto{\pgfqpoint{0.055556in}{0.000000in}}%
\pgfusepath{stroke,fill}%
}%
\begin{pgfscope}%
\pgfsys@transformshift{0.750000in}{2.100000in}%
\pgfsys@useobject{currentmarker}{}%
\end{pgfscope}%
\end{pgfscope}%
\begin{pgfscope}%
\pgfsetbuttcap%
\pgfsetroundjoin%
\definecolor{currentfill}{rgb}{0.000000,0.000000,0.000000}%
\pgfsetfillcolor{currentfill}%
\pgfsetlinewidth{0.501875pt}%
\definecolor{currentstroke}{rgb}{0.000000,0.000000,0.000000}%
\pgfsetstrokecolor{currentstroke}%
\pgfsetdash{}{0pt}%
\pgfsys@defobject{currentmarker}{\pgfqpoint{-0.055556in}{0.000000in}}{\pgfqpoint{0.000000in}{0.000000in}}{%
\pgfpathmoveto{\pgfqpoint{0.000000in}{0.000000in}}%
\pgfpathlineto{\pgfqpoint{-0.055556in}{0.000000in}}%
\pgfusepath{stroke,fill}%
}%
\begin{pgfscope}%
\pgfsys@transformshift{5.400000in}{2.100000in}%
\pgfsys@useobject{currentmarker}{}%
\end{pgfscope}%
\end{pgfscope}%
\begin{pgfscope}%
\pgftext[left,bottom,x=0.356290in,y=2.039352in,rotate=0.000000]{{\rmfamily\fontsize{12.000000}{14.400000}\selectfont \(\displaystyle -2.4\)}}
%
\end{pgfscope}%
\begin{pgfscope}%
\pgfpathrectangle{\pgfqpoint{0.750000in}{0.700000in}}{\pgfqpoint{4.650000in}{5.600000in}} %
\pgfusepath{clip}%
\pgfsetbuttcap%
\pgfsetroundjoin%
\pgfsetlinewidth{0.501875pt}%
\definecolor{currentstroke}{rgb}{0.000000,0.000000,0.000000}%
\pgfsetstrokecolor{currentstroke}%
\pgfsetdash{{1.000000pt}{3.000000pt}}{0.000000pt}%
\pgfpathmoveto{\pgfqpoint{0.750000in}{0.700000in}}%
\pgfpathlineto{\pgfqpoint{5.400000in}{0.700000in}}%
\pgfusepath{stroke}%
\end{pgfscope}%
\begin{pgfscope}%
\pgfsetbuttcap%
\pgfsetroundjoin%
\definecolor{currentfill}{rgb}{0.000000,0.000000,0.000000}%
\pgfsetfillcolor{currentfill}%
\pgfsetlinewidth{0.501875pt}%
\definecolor{currentstroke}{rgb}{0.000000,0.000000,0.000000}%
\pgfsetstrokecolor{currentstroke}%
\pgfsetdash{}{0pt}%
\pgfsys@defobject{currentmarker}{\pgfqpoint{0.000000in}{0.000000in}}{\pgfqpoint{0.055556in}{0.000000in}}{%
\pgfpathmoveto{\pgfqpoint{0.000000in}{0.000000in}}%
\pgfpathlineto{\pgfqpoint{0.055556in}{0.000000in}}%
\pgfusepath{stroke,fill}%
}%
\begin{pgfscope}%
\pgfsys@transformshift{0.750000in}{0.700000in}%
\pgfsys@useobject{currentmarker}{}%
\end{pgfscope}%
\end{pgfscope}%
\begin{pgfscope}%
\pgfsetbuttcap%
\pgfsetroundjoin%
\definecolor{currentfill}{rgb}{0.000000,0.000000,0.000000}%
\pgfsetfillcolor{currentfill}%
\pgfsetlinewidth{0.501875pt}%
\definecolor{currentstroke}{rgb}{0.000000,0.000000,0.000000}%
\pgfsetstrokecolor{currentstroke}%
\pgfsetdash{}{0pt}%
\pgfsys@defobject{currentmarker}{\pgfqpoint{-0.055556in}{0.000000in}}{\pgfqpoint{0.000000in}{0.000000in}}{%
\pgfpathmoveto{\pgfqpoint{0.000000in}{0.000000in}}%
\pgfpathlineto{\pgfqpoint{-0.055556in}{0.000000in}}%
\pgfusepath{stroke,fill}%
}%
\begin{pgfscope}%
\pgfsys@transformshift{5.400000in}{0.700000in}%
\pgfsys@useobject{currentmarker}{}%
\end{pgfscope}%
\end{pgfscope}%
\begin{pgfscope}%
\pgftext[left,bottom,x=0.356290in,y=0.639352in,rotate=0.000000]{{\rmfamily\fontsize{12.000000}{14.400000}\selectfont \(\displaystyle -3.2\)}}
%
\end{pgfscope}%
\begin{pgfscope}%
\pgftext[left,bottom,x=0.286846in,y=3.143030in,rotate=90.000000]{{\rmfamily\fontsize{12.000000}{14.400000}\selectfont Deflection}}
%
\end{pgfscope}%
\begin{pgfscope}%
\pgftext[left,bottom,x=0.750000in,y=6.327778in,rotate=0.000000]{{\rmfamily\fontsize{12.000000}{14.400000}\selectfont \(\displaystyle \times10^{-5}\)}}
%
\end{pgfscope}%
\begin{pgfscope}%
\pgfsetrectcap%
\pgfsetroundjoin%
\pgfsetlinewidth{1.003750pt}%
\definecolor{currentstroke}{rgb}{0.000000,0.000000,0.000000}%
\pgfsetstrokecolor{currentstroke}%
\pgfsetdash{}{0pt}%
\pgfpathmoveto{\pgfqpoint{0.750000in}{6.300000in}}%
\pgfpathlineto{\pgfqpoint{5.400000in}{6.300000in}}%
\pgfusepath{stroke}%
\end{pgfscope}%
\begin{pgfscope}%
\pgfsetrectcap%
\pgfsetroundjoin%
\pgfsetlinewidth{1.003750pt}%
\definecolor{currentstroke}{rgb}{0.000000,0.000000,0.000000}%
\pgfsetstrokecolor{currentstroke}%
\pgfsetdash{}{0pt}%
\pgfpathmoveto{\pgfqpoint{5.400000in}{0.700000in}}%
\pgfpathlineto{\pgfqpoint{5.400000in}{6.300000in}}%
\pgfusepath{stroke}%
\end{pgfscope}%
\begin{pgfscope}%
\pgfsetrectcap%
\pgfsetroundjoin%
\pgfsetlinewidth{1.003750pt}%
\definecolor{currentstroke}{rgb}{0.000000,0.000000,0.000000}%
\pgfsetstrokecolor{currentstroke}%
\pgfsetdash{}{0pt}%
\pgfpathmoveto{\pgfqpoint{0.750000in}{0.700000in}}%
\pgfpathlineto{\pgfqpoint{5.400000in}{0.700000in}}%
\pgfusepath{stroke}%
\end{pgfscope}%
\begin{pgfscope}%
\pgfsetrectcap%
\pgfsetroundjoin%
\pgfsetlinewidth{1.003750pt}%
\definecolor{currentstroke}{rgb}{0.000000,0.000000,0.000000}%
\pgfsetstrokecolor{currentstroke}%
\pgfsetdash{}{0pt}%
\pgfpathmoveto{\pgfqpoint{0.750000in}{0.700000in}}%
\pgfpathlineto{\pgfqpoint{0.750000in}{6.300000in}}%
\pgfusepath{stroke}%
\end{pgfscope}%
\begin{pgfscope}%
\pgftext[left,bottom,x=2.161226in,y=6.369444in,rotate=0.000000]{{\rmfamily\fontsize{14.400000}{17.280000}\selectfont Unloaded EPP Beam}}
%
\end{pgfscope}%
\begin{pgfscope}%
\pgfsetrectcap%
\pgfsetroundjoin%
\definecolor{currentfill}{rgb}{1.000000,1.000000,1.000000}%
\pgfsetfillcolor{currentfill}%
\pgfsetlinewidth{1.003750pt}%
\definecolor{currentstroke}{rgb}{0.000000,0.000000,0.000000}%
\pgfsetstrokecolor{currentstroke}%
\pgfsetdash{}{0pt}%
\pgfpathmoveto{\pgfqpoint{1.710823in}{5.124445in}}%
\pgfpathlineto{\pgfqpoint{4.439177in}{5.124445in}}%
\pgfpathlineto{\pgfqpoint{4.439177in}{6.300000in}}%
\pgfpathlineto{\pgfqpoint{1.710823in}{6.300000in}}%
\pgfpathlineto{\pgfqpoint{1.710823in}{5.124445in}}%
\pgfpathclose%
\pgfusepath{stroke,fill}%
\end{pgfscope}%
\begin{pgfscope}%
\pgfsetrectcap%
\pgfsetroundjoin%
\pgfsetlinewidth{1.003750pt}%
\definecolor{currentstroke}{rgb}{0.000000,0.000000,1.000000}%
\pgfsetstrokecolor{currentstroke}%
\pgfsetdash{}{0pt}%
\pgfpathmoveto{\pgfqpoint{1.850823in}{6.150000in}}%
\pgfpathlineto{\pgfqpoint{2.130823in}{6.150000in}}%
\pgfusepath{stroke}%
\end{pgfscope}%
\begin{pgfscope}%
\pgftext[left,bottom,x=2.350823in,y=6.041111in,rotate=0.000000]{{\rmfamily\fontsize{14.400000}{17.280000}\selectfont Abaqus EPP Beam}}
%
\end{pgfscope}%
\begin{pgfscope}%
\pgfsetrectcap%
\pgfsetroundjoin%
\pgfsetlinewidth{1.003750pt}%
\definecolor{currentstroke}{rgb}{0.000000,0.500000,0.000000}%
\pgfsetstrokecolor{currentstroke}%
\pgfsetdash{}{0pt}%
\pgfpathmoveto{\pgfqpoint{1.850823in}{5.871111in}}%
\pgfpathlineto{\pgfqpoint{2.130823in}{5.871111in}}%
\pgfusepath{stroke}%
\end{pgfscope}%
\begin{pgfscope}%
\pgfsetbuttcap%
\pgfsetmiterjoin%
\definecolor{currentfill}{rgb}{0.000000,0.500000,0.000000}%
\pgfsetfillcolor{currentfill}%
\pgfsetlinewidth{0.501875pt}%
\definecolor{currentstroke}{rgb}{0.000000,0.000000,0.000000}%
\pgfsetstrokecolor{currentstroke}%
\pgfsetdash{}{0pt}%
\pgfsys@defobject{currentmarker}{\pgfqpoint{-0.041667in}{-0.041667in}}{\pgfqpoint{0.041667in}{0.041667in}}{%
\pgfpathmoveto{\pgfqpoint{0.000000in}{0.041667in}}%
\pgfpathlineto{\pgfqpoint{-0.041667in}{-0.041667in}}%
\pgfpathlineto{\pgfqpoint{0.041667in}{-0.041667in}}%
\pgfpathclose%
\pgfusepath{stroke,fill}%
}%
\begin{pgfscope}%
\pgfsys@transformshift{1.850823in}{5.871111in}%
\pgfsys@useobject{currentmarker}{}%
\end{pgfscope}%
\begin{pgfscope}%
\pgfsys@transformshift{2.130823in}{5.871111in}%
\pgfsys@useobject{currentmarker}{}%
\end{pgfscope}%
\end{pgfscope}%
\begin{pgfscope}%
\pgftext[left,bottom,x=2.350823in,y=5.762223in,rotate=0.000000]{{\rmfamily\fontsize{14.400000}{17.280000}\selectfont 100 nodes, horizon 0.10}}
%
\end{pgfscope}%
\begin{pgfscope}%
\pgfsetrectcap%
\pgfsetroundjoin%
\pgfsetlinewidth{1.003750pt}%
\definecolor{currentstroke}{rgb}{1.000000,0.000000,0.000000}%
\pgfsetstrokecolor{currentstroke}%
\pgfsetdash{}{0pt}%
\pgfpathmoveto{\pgfqpoint{1.850823in}{5.592223in}}%
\pgfpathlineto{\pgfqpoint{2.130823in}{5.592223in}}%
\pgfusepath{stroke}%
\end{pgfscope}%
\begin{pgfscope}%
\pgfsetbuttcap%
\pgfsetmiterjoin%
\definecolor{currentfill}{rgb}{1.000000,0.000000,0.000000}%
\pgfsetfillcolor{currentfill}%
\pgfsetlinewidth{0.501875pt}%
\definecolor{currentstroke}{rgb}{0.000000,0.000000,0.000000}%
\pgfsetstrokecolor{currentstroke}%
\pgfsetdash{}{0pt}%
\pgfsys@defobject{currentmarker}{\pgfqpoint{-0.041667in}{-0.041667in}}{\pgfqpoint{0.041667in}{0.041667in}}{%
\pgfpathmoveto{\pgfqpoint{0.041667in}{-0.000000in}}%
\pgfpathlineto{\pgfqpoint{-0.041667in}{0.041667in}}%
\pgfpathlineto{\pgfqpoint{-0.041667in}{-0.041667in}}%
\pgfpathclose%
\pgfusepath{stroke,fill}%
}%
\begin{pgfscope}%
\pgfsys@transformshift{1.850823in}{5.592223in}%
\pgfsys@useobject{currentmarker}{}%
\end{pgfscope}%
\begin{pgfscope}%
\pgfsys@transformshift{2.130823in}{5.592223in}%
\pgfsys@useobject{currentmarker}{}%
\end{pgfscope}%
\end{pgfscope}%
\begin{pgfscope}%
\pgftext[left,bottom,x=2.350823in,y=5.483334in,rotate=0.000000]{{\rmfamily\fontsize{14.400000}{17.280000}\selectfont 200 nodes, horizon 0.10}}
%
\end{pgfscope}%
\begin{pgfscope}%
\pgfsetrectcap%
\pgfsetroundjoin%
\pgfsetlinewidth{1.003750pt}%
\definecolor{currentstroke}{rgb}{0.000000,0.750000,0.750000}%
\pgfsetstrokecolor{currentstroke}%
\pgfsetdash{}{0pt}%
\pgfpathmoveto{\pgfqpoint{1.850823in}{5.313334in}}%
\pgfpathlineto{\pgfqpoint{2.130823in}{5.313334in}}%
\pgfusepath{stroke}%
\end{pgfscope}%
\begin{pgfscope}%
\pgfsetbuttcap%
\pgfsetmiterjoin%
\definecolor{currentfill}{rgb}{0.000000,0.750000,0.750000}%
\pgfsetfillcolor{currentfill}%
\pgfsetlinewidth{0.501875pt}%
\definecolor{currentstroke}{rgb}{0.000000,0.000000,0.000000}%
\pgfsetstrokecolor{currentstroke}%
\pgfsetdash{}{0pt}%
\pgfsys@defobject{currentmarker}{\pgfqpoint{-0.041667in}{-0.041667in}}{\pgfqpoint{0.041667in}{0.041667in}}{%
\pgfpathmoveto{\pgfqpoint{-0.000000in}{-0.041667in}}%
\pgfpathlineto{\pgfqpoint{0.041667in}{0.041667in}}%
\pgfpathlineto{\pgfqpoint{-0.041667in}{0.041667in}}%
\pgfpathclose%
\pgfusepath{stroke,fill}%
}%
\begin{pgfscope}%
\pgfsys@transformshift{1.850823in}{5.313334in}%
\pgfsys@useobject{currentmarker}{}%
\end{pgfscope}%
\begin{pgfscope}%
\pgfsys@transformshift{2.130823in}{5.313334in}%
\pgfsys@useobject{currentmarker}{}%
\end{pgfscope}%
\end{pgfscope}%
\begin{pgfscope}%
\pgftext[left,bottom,x=2.350823in,y=5.204445in,rotate=0.000000]{{\rmfamily\fontsize{14.400000}{17.280000}\selectfont 500 nodes, horizon 0.10}}
%
\end{pgfscope}%
\end{pgfpicture}%
\makeatother%
\endgroup%
}
  \caption{The need for fine discretization is even more apparent when representing residual plastic deformation}
  \label{fig:ResidualPlasticityN}
\end{figure}
%
\begin{figure}[h]
  \centering
  \resizebox{0.5\linewidth}{!}{%% Creator: Matplotlib, PGF backend
%%
%% To include the figure in your LaTeX document, write
%%   \input{<filename>.pgf}
%%
%% Make sure the required packages are loaded in your preamble
%%   \usepackage{pgf}
%%
%% Figures using additional raster images can only be included by \input if
%% they are in the same directory as the main LaTeX file. For loading figures
%% from other directories you can use the `import` package
%%   \usepackage{import}
%% and then include the figures with
%%   \import{<path to file>}{<filename>.pgf}
%%
%% Matplotlib used the following preamble
%%
\begingroup%
\makeatletter%
\begin{pgfpicture}%
\pgfpathrectangle{\pgfpointorigin}{\pgfqpoint{6.000000in}{7.000000in}}%
\pgfusepath{use as bounding box}%
\begin{pgfscope}%
\pgfsetrectcap%
\pgfsetroundjoin%
\definecolor{currentfill}{rgb}{1.000000,1.000000,1.000000}%
\pgfsetfillcolor{currentfill}%
\pgfsetlinewidth{0.000000pt}%
\definecolor{currentstroke}{rgb}{1.000000,1.000000,1.000000}%
\pgfsetstrokecolor{currentstroke}%
\pgfsetdash{}{0pt}%
\pgfpathmoveto{\pgfqpoint{0.000000in}{0.000000in}}%
\pgfpathlineto{\pgfqpoint{6.000000in}{0.000000in}}%
\pgfpathlineto{\pgfqpoint{6.000000in}{7.000000in}}%
\pgfpathlineto{\pgfqpoint{0.000000in}{7.000000in}}%
\pgfpathclose%
\pgfusepath{fill}%
\end{pgfscope}%
\begin{pgfscope}%
\pgfsetrectcap%
\pgfsetroundjoin%
\definecolor{currentfill}{rgb}{1.000000,1.000000,1.000000}%
\pgfsetfillcolor{currentfill}%
\pgfsetlinewidth{0.000000pt}%
\definecolor{currentstroke}{rgb}{0.000000,0.000000,0.000000}%
\pgfsetstrokecolor{currentstroke}%
\pgfsetdash{}{0pt}%
\pgfpathmoveto{\pgfqpoint{0.750000in}{0.700000in}}%
\pgfpathlineto{\pgfqpoint{5.400000in}{0.700000in}}%
\pgfpathlineto{\pgfqpoint{5.400000in}{6.300000in}}%
\pgfpathlineto{\pgfqpoint{0.750000in}{6.300000in}}%
\pgfpathclose%
\pgfusepath{fill}%
\end{pgfscope}%
\begin{pgfscope}%
\pgfpathrectangle{\pgfqpoint{0.750000in}{0.700000in}}{\pgfqpoint{4.650000in}{5.600000in}} %
\pgfusepath{clip}%
\pgfsetrectcap%
\pgfsetroundjoin%
\pgfsetlinewidth{1.003750pt}%
\definecolor{currentstroke}{rgb}{0.000000,0.000000,1.000000}%
\pgfsetstrokecolor{currentstroke}%
\pgfsetdash{}{0pt}%
\pgfpathmoveto{\pgfqpoint{0.750000in}{6.300000in}}%
\pgfpathlineto{\pgfqpoint{2.168250in}{2.385833in}}%
\pgfpathlineto{\pgfqpoint{2.238000in}{2.200408in}}%
\pgfpathlineto{\pgfqpoint{2.307750in}{2.021794in}}%
\pgfpathlineto{\pgfqpoint{2.354250in}{1.907489in}}%
\pgfpathlineto{\pgfqpoint{2.400750in}{1.797686in}}%
\pgfpathlineto{\pgfqpoint{2.447250in}{1.692837in}}%
\pgfpathlineto{\pgfqpoint{2.493750in}{1.593458in}}%
\pgfpathlineto{\pgfqpoint{2.540250in}{1.500025in}}%
\pgfpathlineto{\pgfqpoint{2.586750in}{1.412923in}}%
\pgfpathlineto{\pgfqpoint{2.633250in}{1.332542in}}%
\pgfpathlineto{\pgfqpoint{2.679750in}{1.259182in}}%
\pgfpathlineto{\pgfqpoint{2.726250in}{1.193231in}}%
\pgfpathlineto{\pgfqpoint{2.749500in}{1.163077in}}%
\pgfpathlineto{\pgfqpoint{2.772750in}{1.134905in}}%
\pgfpathlineto{\pgfqpoint{2.796000in}{1.108671in}}%
\pgfpathlineto{\pgfqpoint{2.819250in}{1.084462in}}%
\pgfpathlineto{\pgfqpoint{2.842500in}{1.062277in}}%
\pgfpathlineto{\pgfqpoint{2.865750in}{1.042117in}}%
\pgfpathlineto{\pgfqpoint{2.889000in}{1.024025in}}%
\pgfpathlineto{\pgfqpoint{2.912250in}{1.008000in}}%
\pgfpathlineto{\pgfqpoint{2.935500in}{0.994129in}}%
\pgfpathlineto{\pgfqpoint{2.958750in}{0.982326in}}%
\pgfpathlineto{\pgfqpoint{2.982000in}{0.972677in}}%
\pgfpathlineto{\pgfqpoint{3.005250in}{0.965138in}}%
\pgfpathlineto{\pgfqpoint{3.028500in}{0.959754in}}%
\pgfpathlineto{\pgfqpoint{3.051750in}{0.956523in}}%
\pgfpathlineto{\pgfqpoint{3.075000in}{0.955446in}}%
\pgfpathlineto{\pgfqpoint{3.098250in}{0.956523in}}%
\pgfpathlineto{\pgfqpoint{3.121500in}{0.959754in}}%
\pgfpathlineto{\pgfqpoint{3.144750in}{0.965138in}}%
\pgfpathlineto{\pgfqpoint{3.168000in}{0.972677in}}%
\pgfpathlineto{\pgfqpoint{3.191250in}{0.982326in}}%
\pgfpathlineto{\pgfqpoint{3.214500in}{0.994129in}}%
\pgfpathlineto{\pgfqpoint{3.237750in}{1.008000in}}%
\pgfpathlineto{\pgfqpoint{3.261000in}{1.024025in}}%
\pgfpathlineto{\pgfqpoint{3.284250in}{1.042117in}}%
\pgfpathlineto{\pgfqpoint{3.307500in}{1.062277in}}%
\pgfpathlineto{\pgfqpoint{3.330750in}{1.084462in}}%
\pgfpathlineto{\pgfqpoint{3.354000in}{1.108671in}}%
\pgfpathlineto{\pgfqpoint{3.377250in}{1.134905in}}%
\pgfpathlineto{\pgfqpoint{3.400500in}{1.163077in}}%
\pgfpathlineto{\pgfqpoint{3.423750in}{1.193231in}}%
\pgfpathlineto{\pgfqpoint{3.470250in}{1.259182in}}%
\pgfpathlineto{\pgfqpoint{3.516750in}{1.332542in}}%
\pgfpathlineto{\pgfqpoint{3.563250in}{1.412923in}}%
\pgfpathlineto{\pgfqpoint{3.609750in}{1.500025in}}%
\pgfpathlineto{\pgfqpoint{3.656250in}{1.593502in}}%
\pgfpathlineto{\pgfqpoint{3.702750in}{1.692837in}}%
\pgfpathlineto{\pgfqpoint{3.749250in}{1.797686in}}%
\pgfpathlineto{\pgfqpoint{3.795750in}{1.907489in}}%
\pgfpathlineto{\pgfqpoint{3.842250in}{2.021794in}}%
\pgfpathlineto{\pgfqpoint{3.912000in}{2.200408in}}%
\pgfpathlineto{\pgfqpoint{3.981750in}{2.385833in}}%
\pgfpathlineto{\pgfqpoint{4.074750in}{2.639862in}}%
\pgfpathlineto{\pgfqpoint{5.121000in}{5.529427in}}%
\pgfpathlineto{\pgfqpoint{5.400000in}{6.300000in}}%
\pgfpathlineto{\pgfqpoint{5.400000in}{6.300000in}}%
\pgfusepath{stroke}%
\end{pgfscope}%
\begin{pgfscope}%
\pgfpathrectangle{\pgfqpoint{0.750000in}{0.700000in}}{\pgfqpoint{4.650000in}{5.600000in}} %
\pgfusepath{clip}%
\pgfsetrectcap%
\pgfsetroundjoin%
\pgfsetlinewidth{1.003750pt}%
\definecolor{currentstroke}{rgb}{0.000000,0.500000,0.000000}%
\pgfsetstrokecolor{currentstroke}%
\pgfsetdash{}{0pt}%
\pgfpathmoveto{\pgfqpoint{0.754650in}{6.289209in}}%
\pgfpathlineto{\pgfqpoint{2.372850in}{2.536601in}}%
\pgfpathlineto{\pgfqpoint{2.451900in}{2.361175in}}%
\pgfpathlineto{\pgfqpoint{2.517000in}{2.222124in}}%
\pgfpathlineto{\pgfqpoint{2.568150in}{2.117694in}}%
\pgfpathlineto{\pgfqpoint{2.614650in}{2.027381in}}%
\pgfpathlineto{\pgfqpoint{2.656500in}{1.950516in}}%
\pgfpathlineto{\pgfqpoint{2.698350in}{1.878464in}}%
\pgfpathlineto{\pgfqpoint{2.735550in}{1.818936in}}%
\pgfpathlineto{\pgfqpoint{2.768100in}{1.770690in}}%
\pgfpathlineto{\pgfqpoint{2.800650in}{1.726330in}}%
\pgfpathlineto{\pgfqpoint{2.828550in}{1.691608in}}%
\pgfpathlineto{\pgfqpoint{2.856450in}{1.660113in}}%
\pgfpathlineto{\pgfqpoint{2.884350in}{1.631999in}}%
\pgfpathlineto{\pgfqpoint{2.912250in}{1.607412in}}%
\pgfpathlineto{\pgfqpoint{2.935500in}{1.589708in}}%
\pgfpathlineto{\pgfqpoint{2.958750in}{1.574615in}}%
\pgfpathlineto{\pgfqpoint{2.982000in}{1.562177in}}%
\pgfpathlineto{\pgfqpoint{3.005250in}{1.552453in}}%
\pgfpathlineto{\pgfqpoint{3.028500in}{1.545479in}}%
\pgfpathlineto{\pgfqpoint{3.051750in}{1.541288in}}%
\pgfpathlineto{\pgfqpoint{3.070350in}{1.539945in}}%
\pgfpathlineto{\pgfqpoint{3.088950in}{1.540391in}}%
\pgfpathlineto{\pgfqpoint{3.112200in}{1.543472in}}%
\pgfpathlineto{\pgfqpoint{3.135450in}{1.549339in}}%
\pgfpathlineto{\pgfqpoint{3.158700in}{1.557967in}}%
\pgfpathlineto{\pgfqpoint{3.181950in}{1.569330in}}%
\pgfpathlineto{\pgfqpoint{3.205200in}{1.583366in}}%
\pgfpathlineto{\pgfqpoint{3.228450in}{1.600036in}}%
\pgfpathlineto{\pgfqpoint{3.251700in}{1.619274in}}%
\pgfpathlineto{\pgfqpoint{3.279600in}{1.645642in}}%
\pgfpathlineto{\pgfqpoint{3.307500in}{1.675471in}}%
\pgfpathlineto{\pgfqpoint{3.335400in}{1.708598in}}%
\pgfpathlineto{\pgfqpoint{3.367950in}{1.751207in}}%
\pgfpathlineto{\pgfqpoint{3.400500in}{1.797824in}}%
\pgfpathlineto{\pgfqpoint{3.433050in}{1.848165in}}%
\pgfpathlineto{\pgfqpoint{3.470250in}{1.909886in}}%
\pgfpathlineto{\pgfqpoint{3.512100in}{1.984154in}}%
\pgfpathlineto{\pgfqpoint{3.553950in}{2.062956in}}%
\pgfpathlineto{\pgfqpoint{3.600450in}{2.155133in}}%
\pgfpathlineto{\pgfqpoint{3.656250in}{2.271131in}}%
\pgfpathlineto{\pgfqpoint{3.721350in}{2.412175in}}%
\pgfpathlineto{\pgfqpoint{3.800400in}{2.589232in}}%
\pgfpathlineto{\pgfqpoint{3.907350in}{2.834849in}}%
\pgfpathlineto{\pgfqpoint{4.260750in}{3.656386in}}%
\pgfpathlineto{\pgfqpoint{5.400000in}{6.300000in}}%
\pgfpathlineto{\pgfqpoint{5.400000in}{6.300000in}}%
\pgfusepath{stroke}%
\end{pgfscope}%
\begin{pgfscope}%
\pgfpathrectangle{\pgfqpoint{0.750000in}{0.700000in}}{\pgfqpoint{4.650000in}{5.600000in}} %
\pgfusepath{clip}%
\pgfsetbuttcap%
\pgfsetmiterjoin%
\definecolor{currentfill}{rgb}{0.000000,0.500000,0.000000}%
\pgfsetfillcolor{currentfill}%
\pgfsetlinewidth{0.501875pt}%
\definecolor{currentstroke}{rgb}{0.000000,0.000000,0.000000}%
\pgfsetstrokecolor{currentstroke}%
\pgfsetdash{}{0pt}%
\pgfsys@defobject{currentmarker}{\pgfqpoint{-0.041667in}{-0.041667in}}{\pgfqpoint{0.041667in}{0.041667in}}{%
\pgfpathmoveto{\pgfqpoint{0.000000in}{0.041667in}}%
\pgfpathlineto{\pgfqpoint{-0.041667in}{-0.041667in}}%
\pgfpathlineto{\pgfqpoint{0.041667in}{-0.041667in}}%
\pgfpathclose%
\pgfusepath{stroke,fill}%
}%
\begin{pgfscope}%
\pgfsys@transformshift{0.829050in}{6.116554in}%
\pgfsys@useobject{currentmarker}{}%
\end{pgfscope}%
\begin{pgfscope}%
\pgfsys@transformshift{1.294050in}{5.037460in}%
\pgfsys@useobject{currentmarker}{}%
\end{pgfscope}%
\begin{pgfscope}%
\pgfsys@transformshift{1.759050in}{3.958438in}%
\pgfsys@useobject{currentmarker}{}%
\end{pgfscope}%
\begin{pgfscope}%
\pgfsys@transformshift{2.224050in}{2.877941in}%
\pgfsys@useobject{currentmarker}{}%
\end{pgfscope}%
\begin{pgfscope}%
\pgfsys@transformshift{2.689050in}{1.894028in}%
\pgfsys@useobject{currentmarker}{}%
\end{pgfscope}%
\begin{pgfscope}%
\pgfsys@transformshift{3.154050in}{1.556022in}%
\pgfsys@useobject{currentmarker}{}%
\end{pgfscope}%
\begin{pgfscope}%
\pgfsys@transformshift{3.619050in}{2.193200in}%
\pgfsys@useobject{currentmarker}{}%
\end{pgfscope}%
\begin{pgfscope}%
\pgfsys@transformshift{4.084050in}{3.245902in}%
\pgfsys@useobject{currentmarker}{}%
\end{pgfscope}%
\begin{pgfscope}%
\pgfsys@transformshift{4.549050in}{4.325167in}%
\pgfsys@useobject{currentmarker}{}%
\end{pgfscope}%
\begin{pgfscope}%
\pgfsys@transformshift{5.014050in}{5.404355in}%
\pgfsys@useobject{currentmarker}{}%
\end{pgfscope}%
\end{pgfscope}%
\begin{pgfscope}%
\pgfpathrectangle{\pgfqpoint{0.750000in}{0.700000in}}{\pgfqpoint{4.650000in}{5.600000in}} %
\pgfusepath{clip}%
\pgfsetrectcap%
\pgfsetroundjoin%
\pgfsetlinewidth{1.003750pt}%
\definecolor{currentstroke}{rgb}{1.000000,0.000000,0.000000}%
\pgfsetstrokecolor{currentstroke}%
\pgfsetdash{}{0pt}%
\pgfpathmoveto{\pgfqpoint{0.750000in}{6.300000in}}%
\pgfpathlineto{\pgfqpoint{2.340300in}{2.499474in}}%
\pgfpathlineto{\pgfqpoint{2.424000in}{2.306579in}}%
\pgfpathlineto{\pgfqpoint{2.489100in}{2.161718in}}%
\pgfpathlineto{\pgfqpoint{2.544900in}{2.042596in}}%
\pgfpathlineto{\pgfqpoint{2.596050in}{1.938610in}}%
\pgfpathlineto{\pgfqpoint{2.642550in}{1.849297in}}%
\pgfpathlineto{\pgfqpoint{2.684400in}{1.773874in}}%
\pgfpathlineto{\pgfqpoint{2.721600in}{1.711301in}}%
\pgfpathlineto{\pgfqpoint{2.754150in}{1.660357in}}%
\pgfpathlineto{\pgfqpoint{2.786700in}{1.613266in}}%
\pgfpathlineto{\pgfqpoint{2.819250in}{1.570316in}}%
\pgfpathlineto{\pgfqpoint{2.847150in}{1.537014in}}%
\pgfpathlineto{\pgfqpoint{2.875050in}{1.507106in}}%
\pgfpathlineto{\pgfqpoint{2.902950in}{1.480764in}}%
\pgfpathlineto{\pgfqpoint{2.926200in}{1.461624in}}%
\pgfpathlineto{\pgfqpoint{2.949450in}{1.445120in}}%
\pgfpathlineto{\pgfqpoint{2.972700in}{1.431321in}}%
\pgfpathlineto{\pgfqpoint{2.995950in}{1.420272in}}%
\pgfpathlineto{\pgfqpoint{3.019200in}{1.412021in}}%
\pgfpathlineto{\pgfqpoint{3.042450in}{1.406597in}}%
\pgfpathlineto{\pgfqpoint{3.061050in}{1.404307in}}%
\pgfpathlineto{\pgfqpoint{3.079650in}{1.403847in}}%
\pgfpathlineto{\pgfqpoint{3.098250in}{1.405224in}}%
\pgfpathlineto{\pgfqpoint{3.116850in}{1.408425in}}%
\pgfpathlineto{\pgfqpoint{3.140100in}{1.414984in}}%
\pgfpathlineto{\pgfqpoint{3.163350in}{1.424357in}}%
\pgfpathlineto{\pgfqpoint{3.186600in}{1.436513in}}%
\pgfpathlineto{\pgfqpoint{3.209850in}{1.451402in}}%
\pgfpathlineto{\pgfqpoint{3.233100in}{1.468967in}}%
\pgfpathlineto{\pgfqpoint{3.256350in}{1.489139in}}%
\pgfpathlineto{\pgfqpoint{3.284250in}{1.516686in}}%
\pgfpathlineto{\pgfqpoint{3.312150in}{1.547744in}}%
\pgfpathlineto{\pgfqpoint{3.340050in}{1.582148in}}%
\pgfpathlineto{\pgfqpoint{3.372600in}{1.626307in}}%
\pgfpathlineto{\pgfqpoint{3.405150in}{1.674533in}}%
\pgfpathlineto{\pgfqpoint{3.437700in}{1.726528in}}%
\pgfpathlineto{\pgfqpoint{3.474900in}{1.790196in}}%
\pgfpathlineto{\pgfqpoint{3.516750in}{1.866723in}}%
\pgfpathlineto{\pgfqpoint{3.558600in}{1.947829in}}%
\pgfpathlineto{\pgfqpoint{3.605100in}{2.042595in}}%
\pgfpathlineto{\pgfqpoint{3.660900in}{2.161717in}}%
\pgfpathlineto{\pgfqpoint{3.721350in}{2.296058in}}%
\pgfpathlineto{\pgfqpoint{3.795750in}{2.466941in}}%
\pgfpathlineto{\pgfqpoint{3.898050in}{2.707922in}}%
\pgfpathlineto{\pgfqpoint{4.107300in}{3.208287in}}%
\pgfpathlineto{\pgfqpoint{5.400000in}{6.300000in}}%
\pgfpathlineto{\pgfqpoint{5.400000in}{6.300000in}}%
\pgfusepath{stroke}%
\end{pgfscope}%
\begin{pgfscope}%
\pgfpathrectangle{\pgfqpoint{0.750000in}{0.700000in}}{\pgfqpoint{4.650000in}{5.600000in}} %
\pgfusepath{clip}%
\pgfsetbuttcap%
\pgfsetmiterjoin%
\definecolor{currentfill}{rgb}{1.000000,0.000000,0.000000}%
\pgfsetfillcolor{currentfill}%
\pgfsetlinewidth{0.501875pt}%
\definecolor{currentstroke}{rgb}{0.000000,0.000000,0.000000}%
\pgfsetstrokecolor{currentstroke}%
\pgfsetdash{}{0pt}%
\pgfsys@defobject{currentmarker}{\pgfqpoint{-0.041667in}{-0.041667in}}{\pgfqpoint{0.041667in}{0.041667in}}{%
\pgfpathmoveto{\pgfqpoint{0.041667in}{-0.000000in}}%
\pgfpathlineto{\pgfqpoint{-0.041667in}{0.041667in}}%
\pgfpathlineto{\pgfqpoint{-0.041667in}{-0.041667in}}%
\pgfpathclose%
\pgfusepath{stroke,fill}%
}%
\begin{pgfscope}%
\pgfsys@transformshift{0.968550in}{5.777315in}%
\pgfsys@useobject{currentmarker}{}%
\end{pgfscope}%
\begin{pgfscope}%
\pgfsys@transformshift{1.433550in}{4.665210in}%
\pgfsys@useobject{currentmarker}{}%
\end{pgfscope}%
\begin{pgfscope}%
\pgfsys@transformshift{1.898550in}{3.553154in}%
\pgfsys@useobject{currentmarker}{}%
\end{pgfscope}%
\begin{pgfscope}%
\pgfsys@transformshift{2.363550in}{2.445321in}%
\pgfsys@useobject{currentmarker}{}%
\end{pgfscope}%
\begin{pgfscope}%
\pgfsys@transformshift{2.828550in}{1.558845in}%
\pgfsys@useobject{currentmarker}{}%
\end{pgfscope}%
\begin{pgfscope}%
\pgfsys@transformshift{3.293550in}{1.526656in}%
\pgfsys@useobject{currentmarker}{}%
\end{pgfscope}%
\begin{pgfscope}%
\pgfsys@transformshift{3.758550in}{2.380869in}%
\pgfsys@useobject{currentmarker}{}%
\end{pgfscope}%
\begin{pgfscope}%
\pgfsys@transformshift{4.223550in}{3.486422in}%
\pgfsys@useobject{currentmarker}{}%
\end{pgfscope}%
\begin{pgfscope}%
\pgfsys@transformshift{4.688550in}{4.598485in}%
\pgfsys@useobject{currentmarker}{}%
\end{pgfscope}%
\begin{pgfscope}%
\pgfsys@transformshift{5.153550in}{5.710589in}%
\pgfsys@useobject{currentmarker}{}%
\end{pgfscope}%
\end{pgfscope}%
\begin{pgfscope}%
\pgfpathrectangle{\pgfqpoint{0.750000in}{0.700000in}}{\pgfqpoint{4.650000in}{5.600000in}} %
\pgfusepath{clip}%
\pgfsetrectcap%
\pgfsetroundjoin%
\pgfsetlinewidth{1.003750pt}%
\definecolor{currentstroke}{rgb}{0.000000,0.750000,0.750000}%
\pgfsetstrokecolor{currentstroke}%
\pgfsetdash{}{0pt}%
\pgfpathmoveto{\pgfqpoint{0.750000in}{6.300000in}}%
\pgfpathlineto{\pgfqpoint{2.312400in}{2.288285in}}%
\pgfpathlineto{\pgfqpoint{2.400750in}{2.068569in}}%
\pgfpathlineto{\pgfqpoint{2.470500in}{1.900737in}}%
\pgfpathlineto{\pgfqpoint{2.530950in}{1.761107in}}%
\pgfpathlineto{\pgfqpoint{2.582100in}{1.648451in}}%
\pgfpathlineto{\pgfqpoint{2.628600in}{1.551362in}}%
\pgfpathlineto{\pgfqpoint{2.670450in}{1.469056in}}%
\pgfpathlineto{\pgfqpoint{2.707650in}{1.400483in}}%
\pgfpathlineto{\pgfqpoint{2.744850in}{1.336694in}}%
\pgfpathlineto{\pgfqpoint{2.777400in}{1.285181in}}%
\pgfpathlineto{\pgfqpoint{2.809950in}{1.238003in}}%
\pgfpathlineto{\pgfqpoint{2.837850in}{1.201246in}}%
\pgfpathlineto{\pgfqpoint{2.865750in}{1.168050in}}%
\pgfpathlineto{\pgfqpoint{2.893650in}{1.138608in}}%
\pgfpathlineto{\pgfqpoint{2.916900in}{1.117044in}}%
\pgfpathlineto{\pgfqpoint{2.940150in}{1.098266in}}%
\pgfpathlineto{\pgfqpoint{2.963400in}{1.082341in}}%
\pgfpathlineto{\pgfqpoint{2.986650in}{1.069335in}}%
\pgfpathlineto{\pgfqpoint{3.009900in}{1.059307in}}%
\pgfpathlineto{\pgfqpoint{3.028500in}{1.053452in}}%
\pgfpathlineto{\pgfqpoint{3.047100in}{1.049537in}}%
\pgfpathlineto{\pgfqpoint{3.065700in}{1.047575in}}%
\pgfpathlineto{\pgfqpoint{3.084300in}{1.047575in}}%
\pgfpathlineto{\pgfqpoint{3.102900in}{1.049537in}}%
\pgfpathlineto{\pgfqpoint{3.121500in}{1.053452in}}%
\pgfpathlineto{\pgfqpoint{3.140100in}{1.059307in}}%
\pgfpathlineto{\pgfqpoint{3.158700in}{1.067090in}}%
\pgfpathlineto{\pgfqpoint{3.181950in}{1.079505in}}%
\pgfpathlineto{\pgfqpoint{3.205200in}{1.094850in}}%
\pgfpathlineto{\pgfqpoint{3.228450in}{1.113063in}}%
\pgfpathlineto{\pgfqpoint{3.251700in}{1.134076in}}%
\pgfpathlineto{\pgfqpoint{3.274950in}{1.157810in}}%
\pgfpathlineto{\pgfqpoint{3.302850in}{1.189775in}}%
\pgfpathlineto{\pgfqpoint{3.330750in}{1.225362in}}%
\pgfpathlineto{\pgfqpoint{3.358650in}{1.264413in}}%
\pgfpathlineto{\pgfqpoint{3.391200in}{1.314109in}}%
\pgfpathlineto{\pgfqpoint{3.423750in}{1.367957in}}%
\pgfpathlineto{\pgfqpoint{3.460950in}{1.434193in}}%
\pgfpathlineto{\pgfqpoint{3.498150in}{1.505003in}}%
\pgfpathlineto{\pgfqpoint{3.540000in}{1.589530in}}%
\pgfpathlineto{\pgfqpoint{3.586500in}{1.688763in}}%
\pgfpathlineto{\pgfqpoint{3.637650in}{1.803396in}}%
\pgfpathlineto{\pgfqpoint{3.698100in}{1.944873in}}%
\pgfpathlineto{\pgfqpoint{3.767850in}{2.114272in}}%
\pgfpathlineto{\pgfqpoint{3.856200in}{2.335247in}}%
\pgfpathlineto{\pgfqpoint{3.986400in}{2.667587in}}%
\pgfpathlineto{\pgfqpoint{5.037300in}{5.368022in}}%
\pgfpathlineto{\pgfqpoint{5.400000in}{6.300000in}}%
\pgfpathlineto{\pgfqpoint{5.400000in}{6.300000in}}%
\pgfusepath{stroke}%
\end{pgfscope}%
\begin{pgfscope}%
\pgfpathrectangle{\pgfqpoint{0.750000in}{0.700000in}}{\pgfqpoint{4.650000in}{5.600000in}} %
\pgfusepath{clip}%
\pgfsetbuttcap%
\pgfsetmiterjoin%
\definecolor{currentfill}{rgb}{0.000000,0.750000,0.750000}%
\pgfsetfillcolor{currentfill}%
\pgfsetlinewidth{0.501875pt}%
\definecolor{currentstroke}{rgb}{0.000000,0.000000,0.000000}%
\pgfsetstrokecolor{currentstroke}%
\pgfsetdash{}{0pt}%
\pgfsys@defobject{currentmarker}{\pgfqpoint{-0.041667in}{-0.041667in}}{\pgfqpoint{0.041667in}{0.041667in}}{%
\pgfpathmoveto{\pgfqpoint{-0.000000in}{-0.041667in}}%
\pgfpathlineto{\pgfqpoint{0.041667in}{0.041667in}}%
\pgfpathlineto{\pgfqpoint{-0.041667in}{0.041667in}}%
\pgfpathclose%
\pgfusepath{stroke,fill}%
}%
\begin{pgfscope}%
\pgfsys@transformshift{1.108050in}{5.379970in}%
\pgfsys@useobject{currentmarker}{}%
\end{pgfscope}%
\begin{pgfscope}%
\pgfsys@transformshift{1.573050in}{4.185115in}%
\pgfsys@useobject{currentmarker}{}%
\end{pgfscope}%
\begin{pgfscope}%
\pgfsys@transformshift{2.038050in}{2.990232in}%
\pgfsys@useobject{currentmarker}{}%
\end{pgfscope}%
\begin{pgfscope}%
\pgfsys@transformshift{2.503050in}{1.824772in}%
\pgfsys@useobject{currentmarker}{}%
\end{pgfscope}%
\begin{pgfscope}%
\pgfsys@transformshift{2.968050in}{1.079505in}%
\pgfsys@useobject{currentmarker}{}%
\end{pgfscope}%
\begin{pgfscope}%
\pgfsys@transformshift{3.433050in}{1.384067in}%
\pgfsys@useobject{currentmarker}{}%
\end{pgfscope}%
\begin{pgfscope}%
\pgfsys@transformshift{3.898050in}{2.441515in}%
\pgfsys@useobject{currentmarker}{}%
\end{pgfscope}%
\begin{pgfscope}%
\pgfsys@transformshift{4.363050in}{3.635473in}%
\pgfsys@useobject{currentmarker}{}%
\end{pgfscope}%
\begin{pgfscope}%
\pgfsys@transformshift{4.828050in}{4.830339in}%
\pgfsys@useobject{currentmarker}{}%
\end{pgfscope}%
\begin{pgfscope}%
\pgfsys@transformshift{5.293050in}{6.025187in}%
\pgfsys@useobject{currentmarker}{}%
\end{pgfscope}%
\end{pgfscope}%
\begin{pgfscope}%
\pgfpathrectangle{\pgfqpoint{0.750000in}{0.700000in}}{\pgfqpoint{4.650000in}{5.600000in}} %
\pgfusepath{clip}%
\pgfsetbuttcap%
\pgfsetroundjoin%
\pgfsetlinewidth{0.501875pt}%
\definecolor{currentstroke}{rgb}{0.000000,0.000000,0.000000}%
\pgfsetstrokecolor{currentstroke}%
\pgfsetdash{{1.000000pt}{3.000000pt}}{0.000000pt}%
\pgfpathmoveto{\pgfqpoint{0.750000in}{0.700000in}}%
\pgfpathlineto{\pgfqpoint{0.750000in}{6.300000in}}%
\pgfusepath{stroke}%
\end{pgfscope}%
\begin{pgfscope}%
\pgfsetbuttcap%
\pgfsetroundjoin%
\definecolor{currentfill}{rgb}{0.000000,0.000000,0.000000}%
\pgfsetfillcolor{currentfill}%
\pgfsetlinewidth{0.501875pt}%
\definecolor{currentstroke}{rgb}{0.000000,0.000000,0.000000}%
\pgfsetstrokecolor{currentstroke}%
\pgfsetdash{}{0pt}%
\pgfsys@defobject{currentmarker}{\pgfqpoint{0.000000in}{0.000000in}}{\pgfqpoint{0.000000in}{0.055556in}}{%
\pgfpathmoveto{\pgfqpoint{0.000000in}{0.000000in}}%
\pgfpathlineto{\pgfqpoint{0.000000in}{0.055556in}}%
\pgfusepath{stroke,fill}%
}%
\begin{pgfscope}%
\pgfsys@transformshift{0.750000in}{0.700000in}%
\pgfsys@useobject{currentmarker}{}%
\end{pgfscope}%
\end{pgfscope}%
\begin{pgfscope}%
\pgfsetbuttcap%
\pgfsetroundjoin%
\definecolor{currentfill}{rgb}{0.000000,0.000000,0.000000}%
\pgfsetfillcolor{currentfill}%
\pgfsetlinewidth{0.501875pt}%
\definecolor{currentstroke}{rgb}{0.000000,0.000000,0.000000}%
\pgfsetstrokecolor{currentstroke}%
\pgfsetdash{}{0pt}%
\pgfsys@defobject{currentmarker}{\pgfqpoint{0.000000in}{-0.055556in}}{\pgfqpoint{0.000000in}{0.000000in}}{%
\pgfpathmoveto{\pgfqpoint{0.000000in}{0.000000in}}%
\pgfpathlineto{\pgfqpoint{0.000000in}{-0.055556in}}%
\pgfusepath{stroke,fill}%
}%
\begin{pgfscope}%
\pgfsys@transformshift{0.750000in}{6.300000in}%
\pgfsys@useobject{currentmarker}{}%
\end{pgfscope}%
\end{pgfscope}%
\begin{pgfscope}%
\pgftext[left,bottom,x=0.645738in,y=0.537037in,rotate=0.000000]{{\rmfamily\fontsize{12.000000}{14.400000}\selectfont \(\displaystyle 0.0\)}}
%
\end{pgfscope}%
\begin{pgfscope}%
\pgfpathrectangle{\pgfqpoint{0.750000in}{0.700000in}}{\pgfqpoint{4.650000in}{5.600000in}} %
\pgfusepath{clip}%
\pgfsetbuttcap%
\pgfsetroundjoin%
\pgfsetlinewidth{0.501875pt}%
\definecolor{currentstroke}{rgb}{0.000000,0.000000,0.000000}%
\pgfsetstrokecolor{currentstroke}%
\pgfsetdash{{1.000000pt}{3.000000pt}}{0.000000pt}%
\pgfpathmoveto{\pgfqpoint{1.912500in}{0.700000in}}%
\pgfpathlineto{\pgfqpoint{1.912500in}{6.300000in}}%
\pgfusepath{stroke}%
\end{pgfscope}%
\begin{pgfscope}%
\pgfsetbuttcap%
\pgfsetroundjoin%
\definecolor{currentfill}{rgb}{0.000000,0.000000,0.000000}%
\pgfsetfillcolor{currentfill}%
\pgfsetlinewidth{0.501875pt}%
\definecolor{currentstroke}{rgb}{0.000000,0.000000,0.000000}%
\pgfsetstrokecolor{currentstroke}%
\pgfsetdash{}{0pt}%
\pgfsys@defobject{currentmarker}{\pgfqpoint{0.000000in}{0.000000in}}{\pgfqpoint{0.000000in}{0.055556in}}{%
\pgfpathmoveto{\pgfqpoint{0.000000in}{0.000000in}}%
\pgfpathlineto{\pgfqpoint{0.000000in}{0.055556in}}%
\pgfusepath{stroke,fill}%
}%
\begin{pgfscope}%
\pgfsys@transformshift{1.912500in}{0.700000in}%
\pgfsys@useobject{currentmarker}{}%
\end{pgfscope}%
\end{pgfscope}%
\begin{pgfscope}%
\pgfsetbuttcap%
\pgfsetroundjoin%
\definecolor{currentfill}{rgb}{0.000000,0.000000,0.000000}%
\pgfsetfillcolor{currentfill}%
\pgfsetlinewidth{0.501875pt}%
\definecolor{currentstroke}{rgb}{0.000000,0.000000,0.000000}%
\pgfsetstrokecolor{currentstroke}%
\pgfsetdash{}{0pt}%
\pgfsys@defobject{currentmarker}{\pgfqpoint{0.000000in}{-0.055556in}}{\pgfqpoint{0.000000in}{0.000000in}}{%
\pgfpathmoveto{\pgfqpoint{0.000000in}{0.000000in}}%
\pgfpathlineto{\pgfqpoint{0.000000in}{-0.055556in}}%
\pgfusepath{stroke,fill}%
}%
\begin{pgfscope}%
\pgfsys@transformshift{1.912500in}{6.300000in}%
\pgfsys@useobject{currentmarker}{}%
\end{pgfscope}%
\end{pgfscope}%
\begin{pgfscope}%
\pgftext[left,bottom,x=1.808238in,y=0.537037in,rotate=0.000000]{{\rmfamily\fontsize{12.000000}{14.400000}\selectfont \(\displaystyle 0.5\)}}
%
\end{pgfscope}%
\begin{pgfscope}%
\pgfpathrectangle{\pgfqpoint{0.750000in}{0.700000in}}{\pgfqpoint{4.650000in}{5.600000in}} %
\pgfusepath{clip}%
\pgfsetbuttcap%
\pgfsetroundjoin%
\pgfsetlinewidth{0.501875pt}%
\definecolor{currentstroke}{rgb}{0.000000,0.000000,0.000000}%
\pgfsetstrokecolor{currentstroke}%
\pgfsetdash{{1.000000pt}{3.000000pt}}{0.000000pt}%
\pgfpathmoveto{\pgfqpoint{3.075000in}{0.700000in}}%
\pgfpathlineto{\pgfqpoint{3.075000in}{6.300000in}}%
\pgfusepath{stroke}%
\end{pgfscope}%
\begin{pgfscope}%
\pgfsetbuttcap%
\pgfsetroundjoin%
\definecolor{currentfill}{rgb}{0.000000,0.000000,0.000000}%
\pgfsetfillcolor{currentfill}%
\pgfsetlinewidth{0.501875pt}%
\definecolor{currentstroke}{rgb}{0.000000,0.000000,0.000000}%
\pgfsetstrokecolor{currentstroke}%
\pgfsetdash{}{0pt}%
\pgfsys@defobject{currentmarker}{\pgfqpoint{0.000000in}{0.000000in}}{\pgfqpoint{0.000000in}{0.055556in}}{%
\pgfpathmoveto{\pgfqpoint{0.000000in}{0.000000in}}%
\pgfpathlineto{\pgfqpoint{0.000000in}{0.055556in}}%
\pgfusepath{stroke,fill}%
}%
\begin{pgfscope}%
\pgfsys@transformshift{3.075000in}{0.700000in}%
\pgfsys@useobject{currentmarker}{}%
\end{pgfscope}%
\end{pgfscope}%
\begin{pgfscope}%
\pgfsetbuttcap%
\pgfsetroundjoin%
\definecolor{currentfill}{rgb}{0.000000,0.000000,0.000000}%
\pgfsetfillcolor{currentfill}%
\pgfsetlinewidth{0.501875pt}%
\definecolor{currentstroke}{rgb}{0.000000,0.000000,0.000000}%
\pgfsetstrokecolor{currentstroke}%
\pgfsetdash{}{0pt}%
\pgfsys@defobject{currentmarker}{\pgfqpoint{0.000000in}{-0.055556in}}{\pgfqpoint{0.000000in}{0.000000in}}{%
\pgfpathmoveto{\pgfqpoint{0.000000in}{0.000000in}}%
\pgfpathlineto{\pgfqpoint{0.000000in}{-0.055556in}}%
\pgfusepath{stroke,fill}%
}%
\begin{pgfscope}%
\pgfsys@transformshift{3.075000in}{6.300000in}%
\pgfsys@useobject{currentmarker}{}%
\end{pgfscope}%
\end{pgfscope}%
\begin{pgfscope}%
\pgftext[left,bottom,x=2.970738in,y=0.537037in,rotate=0.000000]{{\rmfamily\fontsize{12.000000}{14.400000}\selectfont \(\displaystyle 1.0\)}}
%
\end{pgfscope}%
\begin{pgfscope}%
\pgfpathrectangle{\pgfqpoint{0.750000in}{0.700000in}}{\pgfqpoint{4.650000in}{5.600000in}} %
\pgfusepath{clip}%
\pgfsetbuttcap%
\pgfsetroundjoin%
\pgfsetlinewidth{0.501875pt}%
\definecolor{currentstroke}{rgb}{0.000000,0.000000,0.000000}%
\pgfsetstrokecolor{currentstroke}%
\pgfsetdash{{1.000000pt}{3.000000pt}}{0.000000pt}%
\pgfpathmoveto{\pgfqpoint{4.237500in}{0.700000in}}%
\pgfpathlineto{\pgfqpoint{4.237500in}{6.300000in}}%
\pgfusepath{stroke}%
\end{pgfscope}%
\begin{pgfscope}%
\pgfsetbuttcap%
\pgfsetroundjoin%
\definecolor{currentfill}{rgb}{0.000000,0.000000,0.000000}%
\pgfsetfillcolor{currentfill}%
\pgfsetlinewidth{0.501875pt}%
\definecolor{currentstroke}{rgb}{0.000000,0.000000,0.000000}%
\pgfsetstrokecolor{currentstroke}%
\pgfsetdash{}{0pt}%
\pgfsys@defobject{currentmarker}{\pgfqpoint{0.000000in}{0.000000in}}{\pgfqpoint{0.000000in}{0.055556in}}{%
\pgfpathmoveto{\pgfqpoint{0.000000in}{0.000000in}}%
\pgfpathlineto{\pgfqpoint{0.000000in}{0.055556in}}%
\pgfusepath{stroke,fill}%
}%
\begin{pgfscope}%
\pgfsys@transformshift{4.237500in}{0.700000in}%
\pgfsys@useobject{currentmarker}{}%
\end{pgfscope}%
\end{pgfscope}%
\begin{pgfscope}%
\pgfsetbuttcap%
\pgfsetroundjoin%
\definecolor{currentfill}{rgb}{0.000000,0.000000,0.000000}%
\pgfsetfillcolor{currentfill}%
\pgfsetlinewidth{0.501875pt}%
\definecolor{currentstroke}{rgb}{0.000000,0.000000,0.000000}%
\pgfsetstrokecolor{currentstroke}%
\pgfsetdash{}{0pt}%
\pgfsys@defobject{currentmarker}{\pgfqpoint{0.000000in}{-0.055556in}}{\pgfqpoint{0.000000in}{0.000000in}}{%
\pgfpathmoveto{\pgfqpoint{0.000000in}{0.000000in}}%
\pgfpathlineto{\pgfqpoint{0.000000in}{-0.055556in}}%
\pgfusepath{stroke,fill}%
}%
\begin{pgfscope}%
\pgfsys@transformshift{4.237500in}{6.300000in}%
\pgfsys@useobject{currentmarker}{}%
\end{pgfscope}%
\end{pgfscope}%
\begin{pgfscope}%
\pgftext[left,bottom,x=4.133238in,y=0.537037in,rotate=0.000000]{{\rmfamily\fontsize{12.000000}{14.400000}\selectfont \(\displaystyle 1.5\)}}
%
\end{pgfscope}%
\begin{pgfscope}%
\pgfpathrectangle{\pgfqpoint{0.750000in}{0.700000in}}{\pgfqpoint{4.650000in}{5.600000in}} %
\pgfusepath{clip}%
\pgfsetbuttcap%
\pgfsetroundjoin%
\pgfsetlinewidth{0.501875pt}%
\definecolor{currentstroke}{rgb}{0.000000,0.000000,0.000000}%
\pgfsetstrokecolor{currentstroke}%
\pgfsetdash{{1.000000pt}{3.000000pt}}{0.000000pt}%
\pgfpathmoveto{\pgfqpoint{5.400000in}{0.700000in}}%
\pgfpathlineto{\pgfqpoint{5.400000in}{6.300000in}}%
\pgfusepath{stroke}%
\end{pgfscope}%
\begin{pgfscope}%
\pgfsetbuttcap%
\pgfsetroundjoin%
\definecolor{currentfill}{rgb}{0.000000,0.000000,0.000000}%
\pgfsetfillcolor{currentfill}%
\pgfsetlinewidth{0.501875pt}%
\definecolor{currentstroke}{rgb}{0.000000,0.000000,0.000000}%
\pgfsetstrokecolor{currentstroke}%
\pgfsetdash{}{0pt}%
\pgfsys@defobject{currentmarker}{\pgfqpoint{0.000000in}{0.000000in}}{\pgfqpoint{0.000000in}{0.055556in}}{%
\pgfpathmoveto{\pgfqpoint{0.000000in}{0.000000in}}%
\pgfpathlineto{\pgfqpoint{0.000000in}{0.055556in}}%
\pgfusepath{stroke,fill}%
}%
\begin{pgfscope}%
\pgfsys@transformshift{5.400000in}{0.700000in}%
\pgfsys@useobject{currentmarker}{}%
\end{pgfscope}%
\end{pgfscope}%
\begin{pgfscope}%
\pgfsetbuttcap%
\pgfsetroundjoin%
\definecolor{currentfill}{rgb}{0.000000,0.000000,0.000000}%
\pgfsetfillcolor{currentfill}%
\pgfsetlinewidth{0.501875pt}%
\definecolor{currentstroke}{rgb}{0.000000,0.000000,0.000000}%
\pgfsetstrokecolor{currentstroke}%
\pgfsetdash{}{0pt}%
\pgfsys@defobject{currentmarker}{\pgfqpoint{0.000000in}{-0.055556in}}{\pgfqpoint{0.000000in}{0.000000in}}{%
\pgfpathmoveto{\pgfqpoint{0.000000in}{0.000000in}}%
\pgfpathlineto{\pgfqpoint{0.000000in}{-0.055556in}}%
\pgfusepath{stroke,fill}%
}%
\begin{pgfscope}%
\pgfsys@transformshift{5.400000in}{6.300000in}%
\pgfsys@useobject{currentmarker}{}%
\end{pgfscope}%
\end{pgfscope}%
\begin{pgfscope}%
\pgftext[left,bottom,x=5.295738in,y=0.537037in,rotate=0.000000]{{\rmfamily\fontsize{12.000000}{14.400000}\selectfont \(\displaystyle 2.0\)}}
%
\end{pgfscope}%
\begin{pgfscope}%
\pgftext[left,bottom,x=2.319809in,y=0.319445in,rotate=0.000000]{{\rmfamily\fontsize{12.000000}{14.400000}\selectfont Distance along Beam}}
%
\end{pgfscope}%
\begin{pgfscope}%
\pgfpathrectangle{\pgfqpoint{0.750000in}{0.700000in}}{\pgfqpoint{4.650000in}{5.600000in}} %
\pgfusepath{clip}%
\pgfsetbuttcap%
\pgfsetroundjoin%
\pgfsetlinewidth{0.501875pt}%
\definecolor{currentstroke}{rgb}{0.000000,0.000000,0.000000}%
\pgfsetstrokecolor{currentstroke}%
\pgfsetdash{{1.000000pt}{3.000000pt}}{0.000000pt}%
\pgfpathmoveto{\pgfqpoint{0.750000in}{6.300000in}}%
\pgfpathlineto{\pgfqpoint{5.400000in}{6.300000in}}%
\pgfusepath{stroke}%
\end{pgfscope}%
\begin{pgfscope}%
\pgfsetbuttcap%
\pgfsetroundjoin%
\definecolor{currentfill}{rgb}{0.000000,0.000000,0.000000}%
\pgfsetfillcolor{currentfill}%
\pgfsetlinewidth{0.501875pt}%
\definecolor{currentstroke}{rgb}{0.000000,0.000000,0.000000}%
\pgfsetstrokecolor{currentstroke}%
\pgfsetdash{}{0pt}%
\pgfsys@defobject{currentmarker}{\pgfqpoint{0.000000in}{0.000000in}}{\pgfqpoint{0.055556in}{0.000000in}}{%
\pgfpathmoveto{\pgfqpoint{0.000000in}{0.000000in}}%
\pgfpathlineto{\pgfqpoint{0.055556in}{0.000000in}}%
\pgfusepath{stroke,fill}%
}%
\begin{pgfscope}%
\pgfsys@transformshift{0.750000in}{6.300000in}%
\pgfsys@useobject{currentmarker}{}%
\end{pgfscope}%
\end{pgfscope}%
\begin{pgfscope}%
\pgfsetbuttcap%
\pgfsetroundjoin%
\definecolor{currentfill}{rgb}{0.000000,0.000000,0.000000}%
\pgfsetfillcolor{currentfill}%
\pgfsetlinewidth{0.501875pt}%
\definecolor{currentstroke}{rgb}{0.000000,0.000000,0.000000}%
\pgfsetstrokecolor{currentstroke}%
\pgfsetdash{}{0pt}%
\pgfsys@defobject{currentmarker}{\pgfqpoint{-0.055556in}{0.000000in}}{\pgfqpoint{0.000000in}{0.000000in}}{%
\pgfpathmoveto{\pgfqpoint{0.000000in}{0.000000in}}%
\pgfpathlineto{\pgfqpoint{-0.055556in}{0.000000in}}%
\pgfusepath{stroke,fill}%
}%
\begin{pgfscope}%
\pgfsys@transformshift{5.400000in}{6.300000in}%
\pgfsys@useobject{currentmarker}{}%
\end{pgfscope}%
\end{pgfscope}%
\begin{pgfscope}%
\pgftext[left,bottom,x=0.485920in,y=6.246296in,rotate=0.000000]{{\rmfamily\fontsize{12.000000}{14.400000}\selectfont \(\displaystyle 0.0\)}}
%
\end{pgfscope}%
\begin{pgfscope}%
\pgfpathrectangle{\pgfqpoint{0.750000in}{0.700000in}}{\pgfqpoint{4.650000in}{5.600000in}} %
\pgfusepath{clip}%
\pgfsetbuttcap%
\pgfsetroundjoin%
\pgfsetlinewidth{0.501875pt}%
\definecolor{currentstroke}{rgb}{0.000000,0.000000,0.000000}%
\pgfsetstrokecolor{currentstroke}%
\pgfsetdash{{1.000000pt}{3.000000pt}}{0.000000pt}%
\pgfpathmoveto{\pgfqpoint{0.750000in}{5.007692in}}%
\pgfpathlineto{\pgfqpoint{5.400000in}{5.007692in}}%
\pgfusepath{stroke}%
\end{pgfscope}%
\begin{pgfscope}%
\pgfsetbuttcap%
\pgfsetroundjoin%
\definecolor{currentfill}{rgb}{0.000000,0.000000,0.000000}%
\pgfsetfillcolor{currentfill}%
\pgfsetlinewidth{0.501875pt}%
\definecolor{currentstroke}{rgb}{0.000000,0.000000,0.000000}%
\pgfsetstrokecolor{currentstroke}%
\pgfsetdash{}{0pt}%
\pgfsys@defobject{currentmarker}{\pgfqpoint{0.000000in}{0.000000in}}{\pgfqpoint{0.055556in}{0.000000in}}{%
\pgfpathmoveto{\pgfqpoint{0.000000in}{0.000000in}}%
\pgfpathlineto{\pgfqpoint{0.055556in}{0.000000in}}%
\pgfusepath{stroke,fill}%
}%
\begin{pgfscope}%
\pgfsys@transformshift{0.750000in}{5.007692in}%
\pgfsys@useobject{currentmarker}{}%
\end{pgfscope}%
\end{pgfscope}%
\begin{pgfscope}%
\pgfsetbuttcap%
\pgfsetroundjoin%
\definecolor{currentfill}{rgb}{0.000000,0.000000,0.000000}%
\pgfsetfillcolor{currentfill}%
\pgfsetlinewidth{0.501875pt}%
\definecolor{currentstroke}{rgb}{0.000000,0.000000,0.000000}%
\pgfsetstrokecolor{currentstroke}%
\pgfsetdash{}{0pt}%
\pgfsys@defobject{currentmarker}{\pgfqpoint{-0.055556in}{0.000000in}}{\pgfqpoint{0.000000in}{0.000000in}}{%
\pgfpathmoveto{\pgfqpoint{0.000000in}{0.000000in}}%
\pgfpathlineto{\pgfqpoint{-0.055556in}{0.000000in}}%
\pgfusepath{stroke,fill}%
}%
\begin{pgfscope}%
\pgfsys@transformshift{5.400000in}{5.007692in}%
\pgfsys@useobject{currentmarker}{}%
\end{pgfscope}%
\end{pgfscope}%
\begin{pgfscope}%
\pgftext[left,bottom,x=0.356290in,y=4.947044in,rotate=0.000000]{{\rmfamily\fontsize{12.000000}{14.400000}\selectfont \(\displaystyle -0.3\)}}
%
\end{pgfscope}%
\begin{pgfscope}%
\pgfpathrectangle{\pgfqpoint{0.750000in}{0.700000in}}{\pgfqpoint{4.650000in}{5.600000in}} %
\pgfusepath{clip}%
\pgfsetbuttcap%
\pgfsetroundjoin%
\pgfsetlinewidth{0.501875pt}%
\definecolor{currentstroke}{rgb}{0.000000,0.000000,0.000000}%
\pgfsetstrokecolor{currentstroke}%
\pgfsetdash{{1.000000pt}{3.000000pt}}{0.000000pt}%
\pgfpathmoveto{\pgfqpoint{0.750000in}{3.715385in}}%
\pgfpathlineto{\pgfqpoint{5.400000in}{3.715385in}}%
\pgfusepath{stroke}%
\end{pgfscope}%
\begin{pgfscope}%
\pgfsetbuttcap%
\pgfsetroundjoin%
\definecolor{currentfill}{rgb}{0.000000,0.000000,0.000000}%
\pgfsetfillcolor{currentfill}%
\pgfsetlinewidth{0.501875pt}%
\definecolor{currentstroke}{rgb}{0.000000,0.000000,0.000000}%
\pgfsetstrokecolor{currentstroke}%
\pgfsetdash{}{0pt}%
\pgfsys@defobject{currentmarker}{\pgfqpoint{0.000000in}{0.000000in}}{\pgfqpoint{0.055556in}{0.000000in}}{%
\pgfpathmoveto{\pgfqpoint{0.000000in}{0.000000in}}%
\pgfpathlineto{\pgfqpoint{0.055556in}{0.000000in}}%
\pgfusepath{stroke,fill}%
}%
\begin{pgfscope}%
\pgfsys@transformshift{0.750000in}{3.715385in}%
\pgfsys@useobject{currentmarker}{}%
\end{pgfscope}%
\end{pgfscope}%
\begin{pgfscope}%
\pgfsetbuttcap%
\pgfsetroundjoin%
\definecolor{currentfill}{rgb}{0.000000,0.000000,0.000000}%
\pgfsetfillcolor{currentfill}%
\pgfsetlinewidth{0.501875pt}%
\definecolor{currentstroke}{rgb}{0.000000,0.000000,0.000000}%
\pgfsetstrokecolor{currentstroke}%
\pgfsetdash{}{0pt}%
\pgfsys@defobject{currentmarker}{\pgfqpoint{-0.055556in}{0.000000in}}{\pgfqpoint{0.000000in}{0.000000in}}{%
\pgfpathmoveto{\pgfqpoint{0.000000in}{0.000000in}}%
\pgfpathlineto{\pgfqpoint{-0.055556in}{0.000000in}}%
\pgfusepath{stroke,fill}%
}%
\begin{pgfscope}%
\pgfsys@transformshift{5.400000in}{3.715385in}%
\pgfsys@useobject{currentmarker}{}%
\end{pgfscope}%
\end{pgfscope}%
\begin{pgfscope}%
\pgftext[left,bottom,x=0.356290in,y=3.654737in,rotate=0.000000]{{\rmfamily\fontsize{12.000000}{14.400000}\selectfont \(\displaystyle -0.6\)}}
%
\end{pgfscope}%
\begin{pgfscope}%
\pgfpathrectangle{\pgfqpoint{0.750000in}{0.700000in}}{\pgfqpoint{4.650000in}{5.600000in}} %
\pgfusepath{clip}%
\pgfsetbuttcap%
\pgfsetroundjoin%
\pgfsetlinewidth{0.501875pt}%
\definecolor{currentstroke}{rgb}{0.000000,0.000000,0.000000}%
\pgfsetstrokecolor{currentstroke}%
\pgfsetdash{{1.000000pt}{3.000000pt}}{0.000000pt}%
\pgfpathmoveto{\pgfqpoint{0.750000in}{2.423077in}}%
\pgfpathlineto{\pgfqpoint{5.400000in}{2.423077in}}%
\pgfusepath{stroke}%
\end{pgfscope}%
\begin{pgfscope}%
\pgfsetbuttcap%
\pgfsetroundjoin%
\definecolor{currentfill}{rgb}{0.000000,0.000000,0.000000}%
\pgfsetfillcolor{currentfill}%
\pgfsetlinewidth{0.501875pt}%
\definecolor{currentstroke}{rgb}{0.000000,0.000000,0.000000}%
\pgfsetstrokecolor{currentstroke}%
\pgfsetdash{}{0pt}%
\pgfsys@defobject{currentmarker}{\pgfqpoint{0.000000in}{0.000000in}}{\pgfqpoint{0.055556in}{0.000000in}}{%
\pgfpathmoveto{\pgfqpoint{0.000000in}{0.000000in}}%
\pgfpathlineto{\pgfqpoint{0.055556in}{0.000000in}}%
\pgfusepath{stroke,fill}%
}%
\begin{pgfscope}%
\pgfsys@transformshift{0.750000in}{2.423077in}%
\pgfsys@useobject{currentmarker}{}%
\end{pgfscope}%
\end{pgfscope}%
\begin{pgfscope}%
\pgfsetbuttcap%
\pgfsetroundjoin%
\definecolor{currentfill}{rgb}{0.000000,0.000000,0.000000}%
\pgfsetfillcolor{currentfill}%
\pgfsetlinewidth{0.501875pt}%
\definecolor{currentstroke}{rgb}{0.000000,0.000000,0.000000}%
\pgfsetstrokecolor{currentstroke}%
\pgfsetdash{}{0pt}%
\pgfsys@defobject{currentmarker}{\pgfqpoint{-0.055556in}{0.000000in}}{\pgfqpoint{0.000000in}{0.000000in}}{%
\pgfpathmoveto{\pgfqpoint{0.000000in}{0.000000in}}%
\pgfpathlineto{\pgfqpoint{-0.055556in}{0.000000in}}%
\pgfusepath{stroke,fill}%
}%
\begin{pgfscope}%
\pgfsys@transformshift{5.400000in}{2.423077in}%
\pgfsys@useobject{currentmarker}{}%
\end{pgfscope}%
\end{pgfscope}%
\begin{pgfscope}%
\pgftext[left,bottom,x=0.356290in,y=2.362429in,rotate=0.000000]{{\rmfamily\fontsize{12.000000}{14.400000}\selectfont \(\displaystyle -0.9\)}}
%
\end{pgfscope}%
\begin{pgfscope}%
\pgfpathrectangle{\pgfqpoint{0.750000in}{0.700000in}}{\pgfqpoint{4.650000in}{5.600000in}} %
\pgfusepath{clip}%
\pgfsetbuttcap%
\pgfsetroundjoin%
\pgfsetlinewidth{0.501875pt}%
\definecolor{currentstroke}{rgb}{0.000000,0.000000,0.000000}%
\pgfsetstrokecolor{currentstroke}%
\pgfsetdash{{1.000000pt}{3.000000pt}}{0.000000pt}%
\pgfpathmoveto{\pgfqpoint{0.750000in}{1.130769in}}%
\pgfpathlineto{\pgfqpoint{5.400000in}{1.130769in}}%
\pgfusepath{stroke}%
\end{pgfscope}%
\begin{pgfscope}%
\pgfsetbuttcap%
\pgfsetroundjoin%
\definecolor{currentfill}{rgb}{0.000000,0.000000,0.000000}%
\pgfsetfillcolor{currentfill}%
\pgfsetlinewidth{0.501875pt}%
\definecolor{currentstroke}{rgb}{0.000000,0.000000,0.000000}%
\pgfsetstrokecolor{currentstroke}%
\pgfsetdash{}{0pt}%
\pgfsys@defobject{currentmarker}{\pgfqpoint{0.000000in}{0.000000in}}{\pgfqpoint{0.055556in}{0.000000in}}{%
\pgfpathmoveto{\pgfqpoint{0.000000in}{0.000000in}}%
\pgfpathlineto{\pgfqpoint{0.055556in}{0.000000in}}%
\pgfusepath{stroke,fill}%
}%
\begin{pgfscope}%
\pgfsys@transformshift{0.750000in}{1.130769in}%
\pgfsys@useobject{currentmarker}{}%
\end{pgfscope}%
\end{pgfscope}%
\begin{pgfscope}%
\pgfsetbuttcap%
\pgfsetroundjoin%
\definecolor{currentfill}{rgb}{0.000000,0.000000,0.000000}%
\pgfsetfillcolor{currentfill}%
\pgfsetlinewidth{0.501875pt}%
\definecolor{currentstroke}{rgb}{0.000000,0.000000,0.000000}%
\pgfsetstrokecolor{currentstroke}%
\pgfsetdash{}{0pt}%
\pgfsys@defobject{currentmarker}{\pgfqpoint{-0.055556in}{0.000000in}}{\pgfqpoint{0.000000in}{0.000000in}}{%
\pgfpathmoveto{\pgfqpoint{0.000000in}{0.000000in}}%
\pgfpathlineto{\pgfqpoint{-0.055556in}{0.000000in}}%
\pgfusepath{stroke,fill}%
}%
\begin{pgfscope}%
\pgfsys@transformshift{5.400000in}{1.130769in}%
\pgfsys@useobject{currentmarker}{}%
\end{pgfscope}%
\end{pgfscope}%
\begin{pgfscope}%
\pgftext[left,bottom,x=0.356290in,y=1.070121in,rotate=0.000000]{{\rmfamily\fontsize{12.000000}{14.400000}\selectfont \(\displaystyle -1.2\)}}
%
\end{pgfscope}%
\begin{pgfscope}%
\pgftext[left,bottom,x=0.286846in,y=3.143030in,rotate=90.000000]{{\rmfamily\fontsize{12.000000}{14.400000}\selectfont Deflection}}
%
\end{pgfscope}%
\begin{pgfscope}%
\pgftext[left,bottom,x=0.750000in,y=6.327778in,rotate=0.000000]{{\rmfamily\fontsize{12.000000}{14.400000}\selectfont \(\displaystyle \times10^{-5}\)}}
%
\end{pgfscope}%
\begin{pgfscope}%
\pgfsetrectcap%
\pgfsetroundjoin%
\pgfsetlinewidth{1.003750pt}%
\definecolor{currentstroke}{rgb}{0.000000,0.000000,0.000000}%
\pgfsetstrokecolor{currentstroke}%
\pgfsetdash{}{0pt}%
\pgfpathmoveto{\pgfqpoint{0.750000in}{6.300000in}}%
\pgfpathlineto{\pgfqpoint{5.400000in}{6.300000in}}%
\pgfusepath{stroke}%
\end{pgfscope}%
\begin{pgfscope}%
\pgfsetrectcap%
\pgfsetroundjoin%
\pgfsetlinewidth{1.003750pt}%
\definecolor{currentstroke}{rgb}{0.000000,0.000000,0.000000}%
\pgfsetstrokecolor{currentstroke}%
\pgfsetdash{}{0pt}%
\pgfpathmoveto{\pgfqpoint{5.400000in}{0.700000in}}%
\pgfpathlineto{\pgfqpoint{5.400000in}{6.300000in}}%
\pgfusepath{stroke}%
\end{pgfscope}%
\begin{pgfscope}%
\pgfsetrectcap%
\pgfsetroundjoin%
\pgfsetlinewidth{1.003750pt}%
\definecolor{currentstroke}{rgb}{0.000000,0.000000,0.000000}%
\pgfsetstrokecolor{currentstroke}%
\pgfsetdash{}{0pt}%
\pgfpathmoveto{\pgfqpoint{0.750000in}{0.700000in}}%
\pgfpathlineto{\pgfqpoint{5.400000in}{0.700000in}}%
\pgfusepath{stroke}%
\end{pgfscope}%
\begin{pgfscope}%
\pgfsetrectcap%
\pgfsetroundjoin%
\pgfsetlinewidth{1.003750pt}%
\definecolor{currentstroke}{rgb}{0.000000,0.000000,0.000000}%
\pgfsetstrokecolor{currentstroke}%
\pgfsetdash{}{0pt}%
\pgfpathmoveto{\pgfqpoint{0.750000in}{0.700000in}}%
\pgfpathlineto{\pgfqpoint{0.750000in}{6.300000in}}%
\pgfusepath{stroke}%
\end{pgfscope}%
\begin{pgfscope}%
\pgftext[left,bottom,x=2.161226in,y=6.369444in,rotate=0.000000]{{\rmfamily\fontsize{14.400000}{17.280000}\selectfont Unloaded EPP Beam}}
%
\end{pgfscope}%
\begin{pgfscope}%
\pgfsetrectcap%
\pgfsetroundjoin%
\definecolor{currentfill}{rgb}{1.000000,1.000000,1.000000}%
\pgfsetfillcolor{currentfill}%
\pgfsetlinewidth{1.003750pt}%
\definecolor{currentstroke}{rgb}{0.000000,0.000000,0.000000}%
\pgfsetstrokecolor{currentstroke}%
\pgfsetdash{}{0pt}%
\pgfpathmoveto{\pgfqpoint{1.661865in}{5.124445in}}%
\pgfpathlineto{\pgfqpoint{4.488135in}{5.124445in}}%
\pgfpathlineto{\pgfqpoint{4.488135in}{6.300000in}}%
\pgfpathlineto{\pgfqpoint{1.661865in}{6.300000in}}%
\pgfpathlineto{\pgfqpoint{1.661865in}{5.124445in}}%
\pgfpathclose%
\pgfusepath{stroke,fill}%
\end{pgfscope}%
\begin{pgfscope}%
\pgfsetrectcap%
\pgfsetroundjoin%
\pgfsetlinewidth{1.003750pt}%
\definecolor{currentstroke}{rgb}{0.000000,0.000000,1.000000}%
\pgfsetstrokecolor{currentstroke}%
\pgfsetdash{}{0pt}%
\pgfpathmoveto{\pgfqpoint{1.801865in}{6.150000in}}%
\pgfpathlineto{\pgfqpoint{2.081865in}{6.150000in}}%
\pgfusepath{stroke}%
\end{pgfscope}%
\begin{pgfscope}%
\pgftext[left,bottom,x=2.301865in,y=6.041111in,rotate=0.000000]{{\rmfamily\fontsize{14.400000}{17.280000}\selectfont Abaqus EPP Beam}}
%
\end{pgfscope}%
\begin{pgfscope}%
\pgfsetrectcap%
\pgfsetroundjoin%
\pgfsetlinewidth{1.003750pt}%
\definecolor{currentstroke}{rgb}{0.000000,0.500000,0.000000}%
\pgfsetstrokecolor{currentstroke}%
\pgfsetdash{}{0pt}%
\pgfpathmoveto{\pgfqpoint{1.801865in}{5.871111in}}%
\pgfpathlineto{\pgfqpoint{2.081865in}{5.871111in}}%
\pgfusepath{stroke}%
\end{pgfscope}%
\begin{pgfscope}%
\pgfsetbuttcap%
\pgfsetmiterjoin%
\definecolor{currentfill}{rgb}{0.000000,0.500000,0.000000}%
\pgfsetfillcolor{currentfill}%
\pgfsetlinewidth{0.501875pt}%
\definecolor{currentstroke}{rgb}{0.000000,0.000000,0.000000}%
\pgfsetstrokecolor{currentstroke}%
\pgfsetdash{}{0pt}%
\pgfsys@defobject{currentmarker}{\pgfqpoint{-0.041667in}{-0.041667in}}{\pgfqpoint{0.041667in}{0.041667in}}{%
\pgfpathmoveto{\pgfqpoint{0.000000in}{0.041667in}}%
\pgfpathlineto{\pgfqpoint{-0.041667in}{-0.041667in}}%
\pgfpathlineto{\pgfqpoint{0.041667in}{-0.041667in}}%
\pgfpathclose%
\pgfusepath{stroke,fill}%
}%
\begin{pgfscope}%
\pgfsys@transformshift{1.801865in}{5.871111in}%
\pgfsys@useobject{currentmarker}{}%
\end{pgfscope}%
\begin{pgfscope}%
\pgfsys@transformshift{2.081865in}{5.871111in}%
\pgfsys@useobject{currentmarker}{}%
\end{pgfscope}%
\end{pgfscope}%
\begin{pgfscope}%
\pgftext[left,bottom,x=2.301865in,y=5.762223in,rotate=0.000000]{{\rmfamily\fontsize{14.400000}{17.280000}\selectfont 1000 nodes, horizon 0.20}}
%
\end{pgfscope}%
\begin{pgfscope}%
\pgfsetrectcap%
\pgfsetroundjoin%
\pgfsetlinewidth{1.003750pt}%
\definecolor{currentstroke}{rgb}{1.000000,0.000000,0.000000}%
\pgfsetstrokecolor{currentstroke}%
\pgfsetdash{}{0pt}%
\pgfpathmoveto{\pgfqpoint{1.801865in}{5.592223in}}%
\pgfpathlineto{\pgfqpoint{2.081865in}{5.592223in}}%
\pgfusepath{stroke}%
\end{pgfscope}%
\begin{pgfscope}%
\pgfsetbuttcap%
\pgfsetmiterjoin%
\definecolor{currentfill}{rgb}{1.000000,0.000000,0.000000}%
\pgfsetfillcolor{currentfill}%
\pgfsetlinewidth{0.501875pt}%
\definecolor{currentstroke}{rgb}{0.000000,0.000000,0.000000}%
\pgfsetstrokecolor{currentstroke}%
\pgfsetdash{}{0pt}%
\pgfsys@defobject{currentmarker}{\pgfqpoint{-0.041667in}{-0.041667in}}{\pgfqpoint{0.041667in}{0.041667in}}{%
\pgfpathmoveto{\pgfqpoint{0.041667in}{-0.000000in}}%
\pgfpathlineto{\pgfqpoint{-0.041667in}{0.041667in}}%
\pgfpathlineto{\pgfqpoint{-0.041667in}{-0.041667in}}%
\pgfpathclose%
\pgfusepath{stroke,fill}%
}%
\begin{pgfscope}%
\pgfsys@transformshift{1.801865in}{5.592223in}%
\pgfsys@useobject{currentmarker}{}%
\end{pgfscope}%
\begin{pgfscope}%
\pgfsys@transformshift{2.081865in}{5.592223in}%
\pgfsys@useobject{currentmarker}{}%
\end{pgfscope}%
\end{pgfscope}%
\begin{pgfscope}%
\pgftext[left,bottom,x=2.301865in,y=5.483334in,rotate=0.000000]{{\rmfamily\fontsize{14.400000}{17.280000}\selectfont 1000 nodes, horizon 0.15}}
%
\end{pgfscope}%
\begin{pgfscope}%
\pgfsetrectcap%
\pgfsetroundjoin%
\pgfsetlinewidth{1.003750pt}%
\definecolor{currentstroke}{rgb}{0.000000,0.750000,0.750000}%
\pgfsetstrokecolor{currentstroke}%
\pgfsetdash{}{0pt}%
\pgfpathmoveto{\pgfqpoint{1.801865in}{5.313334in}}%
\pgfpathlineto{\pgfqpoint{2.081865in}{5.313334in}}%
\pgfusepath{stroke}%
\end{pgfscope}%
\begin{pgfscope}%
\pgfsetbuttcap%
\pgfsetmiterjoin%
\definecolor{currentfill}{rgb}{0.000000,0.750000,0.750000}%
\pgfsetfillcolor{currentfill}%
\pgfsetlinewidth{0.501875pt}%
\definecolor{currentstroke}{rgb}{0.000000,0.000000,0.000000}%
\pgfsetstrokecolor{currentstroke}%
\pgfsetdash{}{0pt}%
\pgfsys@defobject{currentmarker}{\pgfqpoint{-0.041667in}{-0.041667in}}{\pgfqpoint{0.041667in}{0.041667in}}{%
\pgfpathmoveto{\pgfqpoint{-0.000000in}{-0.041667in}}%
\pgfpathlineto{\pgfqpoint{0.041667in}{0.041667in}}%
\pgfpathlineto{\pgfqpoint{-0.041667in}{0.041667in}}%
\pgfpathclose%
\pgfusepath{stroke,fill}%
}%
\begin{pgfscope}%
\pgfsys@transformshift{1.801865in}{5.313334in}%
\pgfsys@useobject{currentmarker}{}%
\end{pgfscope}%
\begin{pgfscope}%
\pgfsys@transformshift{2.081865in}{5.313334in}%
\pgfsys@useobject{currentmarker}{}%
\end{pgfscope}%
\end{pgfscope}%
\begin{pgfscope}%
\pgftext[left,bottom,x=2.301865in,y=5.204445in,rotate=0.000000]{{\rmfamily\fontsize{14.400000}{17.280000}\selectfont 1000 nodes, horizon 0.10}}
%
\end{pgfscope}%
\end{pgfpicture}%
\makeatother%
\endgroup%
}
  \caption{Accurately modeling residual plastic deformation also requires a small horizon}
  \label{fig:ResidualPlasticityH}
\end{figure}

It is more difficult to verify the brittle material model because brittle failure is unstable.
When a crack begins, moment is transferred to other bond pairs, and failure progresses until every pair of bonds surrounding a node are broken, creating a hinge at that node.
This is borne out by the results in \cref{fig:brittleBeam}, in which ``Nodal Health'' represents the fraction of bond-pairs about each node that have not failed.

\begin{figure}[h]
  \centering
  \resizebox{0.5\linewidth}{!}{%% Creator: Matplotlib, PGF backend
%%
%% To include the figure in your LaTeX document, write
%%   \input{<filename>.pgf}
%%
%% Make sure the required packages are loaded in your preamble
%%   \usepackage{pgf}
%%
%% Figures using additional raster images can only be included by \input if
%% they are in the same directory as the main LaTeX file. For loading figures
%% from other directories you can use the `import` package
%%   \usepackage{import}
%% and then include the figures with
%%   \import{<path to file>}{<filename>.pgf}
%%
%% Matplotlib used the following preamble
%%
\begingroup%
\makeatletter%
\begin{pgfpicture}%
\pgfpathrectangle{\pgfpointorigin}{\pgfqpoint{8.000000in}{6.000000in}}%
\pgfusepath{use as bounding box}%
\begin{pgfscope}%
\pgfsetrectcap%
\pgfsetroundjoin%
\definecolor{currentfill}{rgb}{1.000000,1.000000,1.000000}%
\pgfsetfillcolor{currentfill}%
\pgfsetlinewidth{0.000000pt}%
\definecolor{currentstroke}{rgb}{1.000000,1.000000,1.000000}%
\pgfsetstrokecolor{currentstroke}%
\pgfsetdash{}{0pt}%
\pgfpathmoveto{\pgfqpoint{0.000000in}{0.000000in}}%
\pgfpathlineto{\pgfqpoint{8.000000in}{0.000000in}}%
\pgfpathlineto{\pgfqpoint{8.000000in}{6.000000in}}%
\pgfpathlineto{\pgfqpoint{0.000000in}{6.000000in}}%
\pgfpathclose%
\pgfusepath{fill}%
\end{pgfscope}%
\begin{pgfscope}%
\pgfsetrectcap%
\pgfsetroundjoin%
\definecolor{currentfill}{rgb}{1.000000,1.000000,1.000000}%
\pgfsetfillcolor{currentfill}%
\pgfsetlinewidth{0.000000pt}%
\definecolor{currentstroke}{rgb}{0.000000,0.000000,0.000000}%
\pgfsetstrokecolor{currentstroke}%
\pgfsetdash{}{0pt}%
\pgfpathmoveto{\pgfqpoint{1.200000in}{4.356522in}}%
\pgfpathlineto{\pgfqpoint{7.200000in}{4.356522in}}%
\pgfpathlineto{\pgfqpoint{7.200000in}{5.400000in}}%
\pgfpathlineto{\pgfqpoint{1.200000in}{5.400000in}}%
\pgfpathclose%
\pgfusepath{fill}%
\end{pgfscope}%
\begin{pgfscope}%
\pgfpathrectangle{\pgfqpoint{1.200000in}{4.356522in}}{\pgfqpoint{6.000000in}{1.043478in}} %
\pgfusepath{clip}%
\pgfsetrectcap%
\pgfsetroundjoin%
\pgfsetlinewidth{1.003750pt}%
\definecolor{currentstroke}{rgb}{0.000000,0.000000,1.000000}%
\pgfsetstrokecolor{currentstroke}%
\pgfsetdash{}{0pt}%
\pgfpathmoveto{\pgfqpoint{1.200000in}{5.400000in}}%
\pgfpathlineto{\pgfqpoint{1.620000in}{5.192233in}}%
\pgfpathlineto{\pgfqpoint{1.830000in}{5.090808in}}%
\pgfpathlineto{\pgfqpoint{2.010000in}{5.006240in}}%
\pgfpathlineto{\pgfqpoint{2.160000in}{4.937927in}}%
\pgfpathlineto{\pgfqpoint{2.310000in}{4.871977in}}%
\pgfpathlineto{\pgfqpoint{2.460000in}{4.808763in}}%
\pgfpathlineto{\pgfqpoint{2.580000in}{4.760413in}}%
\pgfpathlineto{\pgfqpoint{2.700000in}{4.714246in}}%
\pgfpathlineto{\pgfqpoint{2.820000in}{4.670452in}}%
\pgfpathlineto{\pgfqpoint{2.940000in}{4.629223in}}%
\pgfpathlineto{\pgfqpoint{3.060000in}{4.590750in}}%
\pgfpathlineto{\pgfqpoint{3.180000in}{4.555226in}}%
\pgfpathlineto{\pgfqpoint{3.270000in}{4.530632in}}%
\pgfpathlineto{\pgfqpoint{3.360000in}{4.507885in}}%
\pgfpathlineto{\pgfqpoint{3.450000in}{4.487067in}}%
\pgfpathlineto{\pgfqpoint{3.540000in}{4.468254in}}%
\pgfpathlineto{\pgfqpoint{3.630000in}{4.451537in}}%
\pgfpathlineto{\pgfqpoint{3.720000in}{4.436979in}}%
\pgfpathlineto{\pgfqpoint{3.810000in}{4.424671in}}%
\pgfpathlineto{\pgfqpoint{3.900000in}{4.414852in}}%
\pgfpathlineto{\pgfqpoint{3.960000in}{4.409363in}}%
\pgfpathlineto{\pgfqpoint{4.050000in}{4.403498in}}%
\pgfpathlineto{\pgfqpoint{4.140000in}{4.400235in}}%
\pgfpathlineto{\pgfqpoint{4.200000in}{4.398261in}}%
\pgfpathlineto{\pgfqpoint{4.260000in}{4.400235in}}%
\pgfpathlineto{\pgfqpoint{4.350000in}{4.403498in}}%
\pgfpathlineto{\pgfqpoint{4.440000in}{4.409363in}}%
\pgfpathlineto{\pgfqpoint{4.530000in}{4.417779in}}%
\pgfpathlineto{\pgfqpoint{4.620000in}{4.428519in}}%
\pgfpathlineto{\pgfqpoint{4.710000in}{4.441584in}}%
\pgfpathlineto{\pgfqpoint{4.800000in}{4.456878in}}%
\pgfpathlineto{\pgfqpoint{4.890000in}{4.474297in}}%
\pgfpathlineto{\pgfqpoint{4.980000in}{4.493787in}}%
\pgfpathlineto{\pgfqpoint{5.070000in}{4.515258in}}%
\pgfpathlineto{\pgfqpoint{5.160000in}{4.538629in}}%
\pgfpathlineto{\pgfqpoint{5.250000in}{4.563820in}}%
\pgfpathlineto{\pgfqpoint{5.370000in}{4.600099in}}%
\pgfpathlineto{\pgfqpoint{5.490000in}{4.639279in}}%
\pgfpathlineto{\pgfqpoint{5.610000in}{4.681167in}}%
\pgfpathlineto{\pgfqpoint{5.730000in}{4.725572in}}%
\pgfpathlineto{\pgfqpoint{5.850000in}{4.772304in}}%
\pgfpathlineto{\pgfqpoint{5.970000in}{4.821169in}}%
\pgfpathlineto{\pgfqpoint{6.120000in}{4.884960in}}%
\pgfpathlineto{\pgfqpoint{6.270000in}{4.951413in}}%
\pgfpathlineto{\pgfqpoint{6.420000in}{5.020153in}}%
\pgfpathlineto{\pgfqpoint{6.600000in}{5.105136in}}%
\pgfpathlineto{\pgfqpoint{6.810000in}{5.206907in}}%
\pgfpathlineto{\pgfqpoint{7.110000in}{5.355250in}}%
\pgfpathlineto{\pgfqpoint{7.200000in}{5.400000in}}%
\pgfpathlineto{\pgfqpoint{7.200000in}{5.400000in}}%
\pgfusepath{stroke}%
\end{pgfscope}%
\begin{pgfscope}%
\pgfsetbuttcap%
\pgfsetroundjoin%
\definecolor{currentfill}{rgb}{0.000000,0.000000,0.000000}%
\pgfsetfillcolor{currentfill}%
\pgfsetlinewidth{0.501875pt}%
\definecolor{currentstroke}{rgb}{0.000000,0.000000,0.000000}%
\pgfsetstrokecolor{currentstroke}%
\pgfsetdash{}{0pt}%
\pgfsys@defobject{currentmarker}{\pgfqpoint{0.000000in}{0.000000in}}{\pgfqpoint{0.055556in}{0.000000in}}{%
\pgfpathmoveto{\pgfqpoint{0.000000in}{0.000000in}}%
\pgfpathlineto{\pgfqpoint{0.055556in}{0.000000in}}%
\pgfusepath{stroke,fill}%
}%
\begin{pgfscope}%
\pgfsys@transformshift{1.200000in}{4.356522in}%
\pgfsys@useobject{currentmarker}{}%
\end{pgfscope}%
\end{pgfscope}%
\begin{pgfscope}%
\pgftext[left,bottom,x=0.479905in,y=4.295874in,rotate=0.000000]{{\rmfamily\fontsize{12.000000}{14.400000}\selectfont \(\displaystyle -0.00007\)}}
%
\end{pgfscope}%
\begin{pgfscope}%
\pgfsetbuttcap%
\pgfsetroundjoin%
\definecolor{currentfill}{rgb}{0.000000,0.000000,0.000000}%
\pgfsetfillcolor{currentfill}%
\pgfsetlinewidth{0.501875pt}%
\definecolor{currentstroke}{rgb}{0.000000,0.000000,0.000000}%
\pgfsetstrokecolor{currentstroke}%
\pgfsetdash{}{0pt}%
\pgfsys@defobject{currentmarker}{\pgfqpoint{0.000000in}{0.000000in}}{\pgfqpoint{0.055556in}{0.000000in}}{%
\pgfpathmoveto{\pgfqpoint{0.000000in}{0.000000in}}%
\pgfpathlineto{\pgfqpoint{0.055556in}{0.000000in}}%
\pgfusepath{stroke,fill}%
}%
\begin{pgfscope}%
\pgfsys@transformshift{1.200000in}{4.505590in}%
\pgfsys@useobject{currentmarker}{}%
\end{pgfscope}%
\end{pgfscope}%
\begin{pgfscope}%
\pgftext[left,bottom,x=0.479905in,y=4.444942in,rotate=0.000000]{{\rmfamily\fontsize{12.000000}{14.400000}\selectfont \(\displaystyle -0.00006\)}}
%
\end{pgfscope}%
\begin{pgfscope}%
\pgfsetbuttcap%
\pgfsetroundjoin%
\definecolor{currentfill}{rgb}{0.000000,0.000000,0.000000}%
\pgfsetfillcolor{currentfill}%
\pgfsetlinewidth{0.501875pt}%
\definecolor{currentstroke}{rgb}{0.000000,0.000000,0.000000}%
\pgfsetstrokecolor{currentstroke}%
\pgfsetdash{}{0pt}%
\pgfsys@defobject{currentmarker}{\pgfqpoint{0.000000in}{0.000000in}}{\pgfqpoint{0.055556in}{0.000000in}}{%
\pgfpathmoveto{\pgfqpoint{0.000000in}{0.000000in}}%
\pgfpathlineto{\pgfqpoint{0.055556in}{0.000000in}}%
\pgfusepath{stroke,fill}%
}%
\begin{pgfscope}%
\pgfsys@transformshift{1.200000in}{4.654658in}%
\pgfsys@useobject{currentmarker}{}%
\end{pgfscope}%
\end{pgfscope}%
\begin{pgfscope}%
\pgftext[left,bottom,x=0.479905in,y=4.594010in,rotate=0.000000]{{\rmfamily\fontsize{12.000000}{14.400000}\selectfont \(\displaystyle -0.00005\)}}
%
\end{pgfscope}%
\begin{pgfscope}%
\pgfsetbuttcap%
\pgfsetroundjoin%
\definecolor{currentfill}{rgb}{0.000000,0.000000,0.000000}%
\pgfsetfillcolor{currentfill}%
\pgfsetlinewidth{0.501875pt}%
\definecolor{currentstroke}{rgb}{0.000000,0.000000,0.000000}%
\pgfsetstrokecolor{currentstroke}%
\pgfsetdash{}{0pt}%
\pgfsys@defobject{currentmarker}{\pgfqpoint{0.000000in}{0.000000in}}{\pgfqpoint{0.055556in}{0.000000in}}{%
\pgfpathmoveto{\pgfqpoint{0.000000in}{0.000000in}}%
\pgfpathlineto{\pgfqpoint{0.055556in}{0.000000in}}%
\pgfusepath{stroke,fill}%
}%
\begin{pgfscope}%
\pgfsys@transformshift{1.200000in}{4.803727in}%
\pgfsys@useobject{currentmarker}{}%
\end{pgfscope}%
\end{pgfscope}%
\begin{pgfscope}%
\pgftext[left,bottom,x=0.479905in,y=4.743079in,rotate=0.000000]{{\rmfamily\fontsize{12.000000}{14.400000}\selectfont \(\displaystyle -0.00004\)}}
%
\end{pgfscope}%
\begin{pgfscope}%
\pgfsetbuttcap%
\pgfsetroundjoin%
\definecolor{currentfill}{rgb}{0.000000,0.000000,0.000000}%
\pgfsetfillcolor{currentfill}%
\pgfsetlinewidth{0.501875pt}%
\definecolor{currentstroke}{rgb}{0.000000,0.000000,0.000000}%
\pgfsetstrokecolor{currentstroke}%
\pgfsetdash{}{0pt}%
\pgfsys@defobject{currentmarker}{\pgfqpoint{0.000000in}{0.000000in}}{\pgfqpoint{0.055556in}{0.000000in}}{%
\pgfpathmoveto{\pgfqpoint{0.000000in}{0.000000in}}%
\pgfpathlineto{\pgfqpoint{0.055556in}{0.000000in}}%
\pgfusepath{stroke,fill}%
}%
\begin{pgfscope}%
\pgfsys@transformshift{1.200000in}{4.952795in}%
\pgfsys@useobject{currentmarker}{}%
\end{pgfscope}%
\end{pgfscope}%
\begin{pgfscope}%
\pgftext[left,bottom,x=0.479905in,y=4.892147in,rotate=0.000000]{{\rmfamily\fontsize{12.000000}{14.400000}\selectfont \(\displaystyle -0.00003\)}}
%
\end{pgfscope}%
\begin{pgfscope}%
\pgfsetbuttcap%
\pgfsetroundjoin%
\definecolor{currentfill}{rgb}{0.000000,0.000000,0.000000}%
\pgfsetfillcolor{currentfill}%
\pgfsetlinewidth{0.501875pt}%
\definecolor{currentstroke}{rgb}{0.000000,0.000000,0.000000}%
\pgfsetstrokecolor{currentstroke}%
\pgfsetdash{}{0pt}%
\pgfsys@defobject{currentmarker}{\pgfqpoint{0.000000in}{0.000000in}}{\pgfqpoint{0.055556in}{0.000000in}}{%
\pgfpathmoveto{\pgfqpoint{0.000000in}{0.000000in}}%
\pgfpathlineto{\pgfqpoint{0.055556in}{0.000000in}}%
\pgfusepath{stroke,fill}%
}%
\begin{pgfscope}%
\pgfsys@transformshift{1.200000in}{5.101863in}%
\pgfsys@useobject{currentmarker}{}%
\end{pgfscope}%
\end{pgfscope}%
\begin{pgfscope}%
\pgftext[left,bottom,x=0.479905in,y=5.041215in,rotate=0.000000]{{\rmfamily\fontsize{12.000000}{14.400000}\selectfont \(\displaystyle -0.00002\)}}
%
\end{pgfscope}%
\begin{pgfscope}%
\pgfsetbuttcap%
\pgfsetroundjoin%
\definecolor{currentfill}{rgb}{0.000000,0.000000,0.000000}%
\pgfsetfillcolor{currentfill}%
\pgfsetlinewidth{0.501875pt}%
\definecolor{currentstroke}{rgb}{0.000000,0.000000,0.000000}%
\pgfsetstrokecolor{currentstroke}%
\pgfsetdash{}{0pt}%
\pgfsys@defobject{currentmarker}{\pgfqpoint{0.000000in}{0.000000in}}{\pgfqpoint{0.055556in}{0.000000in}}{%
\pgfpathmoveto{\pgfqpoint{0.000000in}{0.000000in}}%
\pgfpathlineto{\pgfqpoint{0.055556in}{0.000000in}}%
\pgfusepath{stroke,fill}%
}%
\begin{pgfscope}%
\pgfsys@transformshift{1.200000in}{5.250932in}%
\pgfsys@useobject{currentmarker}{}%
\end{pgfscope}%
\end{pgfscope}%
\begin{pgfscope}%
\pgftext[left,bottom,x=0.479905in,y=5.190284in,rotate=0.000000]{{\rmfamily\fontsize{12.000000}{14.400000}\selectfont \(\displaystyle -0.00001\)}}
%
\end{pgfscope}%
\begin{pgfscope}%
\pgfsetbuttcap%
\pgfsetroundjoin%
\definecolor{currentfill}{rgb}{0.000000,0.000000,0.000000}%
\pgfsetfillcolor{currentfill}%
\pgfsetlinewidth{0.501875pt}%
\definecolor{currentstroke}{rgb}{0.000000,0.000000,0.000000}%
\pgfsetstrokecolor{currentstroke}%
\pgfsetdash{}{0pt}%
\pgfsys@defobject{currentmarker}{\pgfqpoint{0.000000in}{0.000000in}}{\pgfqpoint{0.055556in}{0.000000in}}{%
\pgfpathmoveto{\pgfqpoint{0.000000in}{0.000000in}}%
\pgfpathlineto{\pgfqpoint{0.055556in}{0.000000in}}%
\pgfusepath{stroke,fill}%
}%
\begin{pgfscope}%
\pgfsys@transformshift{1.200000in}{5.400000in}%
\pgfsys@useobject{currentmarker}{}%
\end{pgfscope}%
\end{pgfscope}%
\begin{pgfscope}%
\pgftext[left,bottom,x=0.609535in,y=5.346296in,rotate=0.000000]{{\rmfamily\fontsize{12.000000}{14.400000}\selectfont \(\displaystyle 0.00000\)}}
%
\end{pgfscope}%
\begin{pgfscope}%
\pgfsetrectcap%
\pgfsetroundjoin%
\pgfsetlinewidth{1.003750pt}%
\definecolor{currentstroke}{rgb}{0.000000,0.000000,0.000000}%
\pgfsetstrokecolor{currentstroke}%
\pgfsetdash{}{0pt}%
\pgfpathmoveto{\pgfqpoint{1.200000in}{5.400000in}}%
\pgfpathlineto{\pgfqpoint{7.200000in}{5.400000in}}%
\pgfusepath{stroke}%
\end{pgfscope}%
\begin{pgfscope}%
\pgfsetrectcap%
\pgfsetroundjoin%
\pgfsetlinewidth{1.003750pt}%
\definecolor{currentstroke}{rgb}{0.000000,0.000000,0.000000}%
\pgfsetstrokecolor{currentstroke}%
\pgfsetdash{}{0pt}%
\pgfpathmoveto{\pgfqpoint{7.200000in}{4.356522in}}%
\pgfpathlineto{\pgfqpoint{7.200000in}{5.400000in}}%
\pgfusepath{stroke}%
\end{pgfscope}%
\begin{pgfscope}%
\pgfsetrectcap%
\pgfsetroundjoin%
\pgfsetlinewidth{1.003750pt}%
\definecolor{currentstroke}{rgb}{0.000000,0.000000,0.000000}%
\pgfsetstrokecolor{currentstroke}%
\pgfsetdash{}{0pt}%
\pgfpathmoveto{\pgfqpoint{1.200000in}{4.356522in}}%
\pgfpathlineto{\pgfqpoint{7.200000in}{4.356522in}}%
\pgfusepath{stroke}%
\end{pgfscope}%
\begin{pgfscope}%
\pgfsetrectcap%
\pgfsetroundjoin%
\pgfsetlinewidth{1.003750pt}%
\definecolor{currentstroke}{rgb}{0.000000,0.000000,0.000000}%
\pgfsetstrokecolor{currentstroke}%
\pgfsetdash{}{0pt}%
\pgfpathmoveto{\pgfqpoint{1.200000in}{4.356522in}}%
\pgfpathlineto{\pgfqpoint{1.200000in}{5.400000in}}%
\pgfusepath{stroke}%
\end{pgfscope}%
\begin{pgfscope}%
\pgftext[left,bottom,x=2.790087in,y=5.430556in,rotate=0.000000]{{\rmfamily\fontsize{14.400000}{17.280000}\selectfont Brittle Beam Failure Progression}}
%
\end{pgfscope}%
\begin{pgfscope}%
\pgfpathrectangle{\pgfqpoint{1.200000in}{4.356522in}}{\pgfqpoint{6.000000in}{1.043478in}} %
\pgfusepath{clip}%
\pgfsetrectcap%
\pgfsetroundjoin%
\pgfsetlinewidth{1.003750pt}%
\definecolor{currentstroke}{rgb}{0.000000,0.000000,1.000000}%
\pgfsetstrokecolor{currentstroke}%
\pgfsetdash{}{0pt}%
\pgfpathmoveto{\pgfqpoint{1.200000in}{5.400000in}}%
\pgfpathlineto{\pgfqpoint{4.170000in}{5.400000in}}%
\pgfpathlineto{\pgfqpoint{4.200000in}{5.295652in}}%
\pgfpathlineto{\pgfqpoint{4.230000in}{5.400000in}}%
\pgfpathlineto{\pgfqpoint{7.200000in}{5.400000in}}%
\pgfpathlineto{\pgfqpoint{7.200000in}{5.400000in}}%
\pgfusepath{stroke}%
\end{pgfscope}%
\begin{pgfscope}%
\pgfsetbuttcap%
\pgfsetroundjoin%
\definecolor{currentfill}{rgb}{0.000000,0.000000,0.000000}%
\pgfsetfillcolor{currentfill}%
\pgfsetlinewidth{0.501875pt}%
\definecolor{currentstroke}{rgb}{0.000000,0.000000,0.000000}%
\pgfsetstrokecolor{currentstroke}%
\pgfsetdash{}{0pt}%
\pgfsys@defobject{currentmarker}{\pgfqpoint{-0.055556in}{0.000000in}}{\pgfqpoint{0.000000in}{0.000000in}}{%
\pgfpathmoveto{\pgfqpoint{0.000000in}{0.000000in}}%
\pgfpathlineto{\pgfqpoint{-0.055556in}{0.000000in}}%
\pgfusepath{stroke,fill}%
}%
\begin{pgfscope}%
\pgfsys@transformshift{7.200000in}{4.356522in}%
\pgfsys@useobject{currentmarker}{}%
\end{pgfscope}%
\end{pgfscope}%
\begin{pgfscope}%
\pgftext[left,bottom,x=7.255556in,y=4.302818in,rotate=0.000000]{{\rmfamily\fontsize{12.000000}{14.400000}\selectfont \(\displaystyle 0.0\)}}
%
\end{pgfscope}%
\begin{pgfscope}%
\pgfsetbuttcap%
\pgfsetroundjoin%
\definecolor{currentfill}{rgb}{0.000000,0.000000,0.000000}%
\pgfsetfillcolor{currentfill}%
\pgfsetlinewidth{0.501875pt}%
\definecolor{currentstroke}{rgb}{0.000000,0.000000,0.000000}%
\pgfsetstrokecolor{currentstroke}%
\pgfsetdash{}{0pt}%
\pgfsys@defobject{currentmarker}{\pgfqpoint{-0.055556in}{0.000000in}}{\pgfqpoint{0.000000in}{0.000000in}}{%
\pgfpathmoveto{\pgfqpoint{0.000000in}{0.000000in}}%
\pgfpathlineto{\pgfqpoint{-0.055556in}{0.000000in}}%
\pgfusepath{stroke,fill}%
}%
\begin{pgfscope}%
\pgfsys@transformshift{7.200000in}{4.565217in}%
\pgfsys@useobject{currentmarker}{}%
\end{pgfscope}%
\end{pgfscope}%
\begin{pgfscope}%
\pgftext[left,bottom,x=7.255556in,y=4.511514in,rotate=0.000000]{{\rmfamily\fontsize{12.000000}{14.400000}\selectfont \(\displaystyle 0.2\)}}
%
\end{pgfscope}%
\begin{pgfscope}%
\pgfsetbuttcap%
\pgfsetroundjoin%
\definecolor{currentfill}{rgb}{0.000000,0.000000,0.000000}%
\pgfsetfillcolor{currentfill}%
\pgfsetlinewidth{0.501875pt}%
\definecolor{currentstroke}{rgb}{0.000000,0.000000,0.000000}%
\pgfsetstrokecolor{currentstroke}%
\pgfsetdash{}{0pt}%
\pgfsys@defobject{currentmarker}{\pgfqpoint{-0.055556in}{0.000000in}}{\pgfqpoint{0.000000in}{0.000000in}}{%
\pgfpathmoveto{\pgfqpoint{0.000000in}{0.000000in}}%
\pgfpathlineto{\pgfqpoint{-0.055556in}{0.000000in}}%
\pgfusepath{stroke,fill}%
}%
\begin{pgfscope}%
\pgfsys@transformshift{7.200000in}{4.773913in}%
\pgfsys@useobject{currentmarker}{}%
\end{pgfscope}%
\end{pgfscope}%
\begin{pgfscope}%
\pgftext[left,bottom,x=7.255556in,y=4.720209in,rotate=0.000000]{{\rmfamily\fontsize{12.000000}{14.400000}\selectfont \(\displaystyle 0.4\)}}
%
\end{pgfscope}%
\begin{pgfscope}%
\pgfsetbuttcap%
\pgfsetroundjoin%
\definecolor{currentfill}{rgb}{0.000000,0.000000,0.000000}%
\pgfsetfillcolor{currentfill}%
\pgfsetlinewidth{0.501875pt}%
\definecolor{currentstroke}{rgb}{0.000000,0.000000,0.000000}%
\pgfsetstrokecolor{currentstroke}%
\pgfsetdash{}{0pt}%
\pgfsys@defobject{currentmarker}{\pgfqpoint{-0.055556in}{0.000000in}}{\pgfqpoint{0.000000in}{0.000000in}}{%
\pgfpathmoveto{\pgfqpoint{0.000000in}{0.000000in}}%
\pgfpathlineto{\pgfqpoint{-0.055556in}{0.000000in}}%
\pgfusepath{stroke,fill}%
}%
\begin{pgfscope}%
\pgfsys@transformshift{7.200000in}{4.982609in}%
\pgfsys@useobject{currentmarker}{}%
\end{pgfscope}%
\end{pgfscope}%
\begin{pgfscope}%
\pgftext[left,bottom,x=7.255556in,y=4.928905in,rotate=0.000000]{{\rmfamily\fontsize{12.000000}{14.400000}\selectfont \(\displaystyle 0.6\)}}
%
\end{pgfscope}%
\begin{pgfscope}%
\pgfsetbuttcap%
\pgfsetroundjoin%
\definecolor{currentfill}{rgb}{0.000000,0.000000,0.000000}%
\pgfsetfillcolor{currentfill}%
\pgfsetlinewidth{0.501875pt}%
\definecolor{currentstroke}{rgb}{0.000000,0.000000,0.000000}%
\pgfsetstrokecolor{currentstroke}%
\pgfsetdash{}{0pt}%
\pgfsys@defobject{currentmarker}{\pgfqpoint{-0.055556in}{0.000000in}}{\pgfqpoint{0.000000in}{0.000000in}}{%
\pgfpathmoveto{\pgfqpoint{0.000000in}{0.000000in}}%
\pgfpathlineto{\pgfqpoint{-0.055556in}{0.000000in}}%
\pgfusepath{stroke,fill}%
}%
\begin{pgfscope}%
\pgfsys@transformshift{7.200000in}{5.191304in}%
\pgfsys@useobject{currentmarker}{}%
\end{pgfscope}%
\end{pgfscope}%
\begin{pgfscope}%
\pgftext[left,bottom,x=7.255556in,y=5.137601in,rotate=0.000000]{{\rmfamily\fontsize{12.000000}{14.400000}\selectfont \(\displaystyle 0.8\)}}
%
\end{pgfscope}%
\begin{pgfscope}%
\pgfsetbuttcap%
\pgfsetroundjoin%
\definecolor{currentfill}{rgb}{0.000000,0.000000,0.000000}%
\pgfsetfillcolor{currentfill}%
\pgfsetlinewidth{0.501875pt}%
\definecolor{currentstroke}{rgb}{0.000000,0.000000,0.000000}%
\pgfsetstrokecolor{currentstroke}%
\pgfsetdash{}{0pt}%
\pgfsys@defobject{currentmarker}{\pgfqpoint{-0.055556in}{0.000000in}}{\pgfqpoint{0.000000in}{0.000000in}}{%
\pgfpathmoveto{\pgfqpoint{0.000000in}{0.000000in}}%
\pgfpathlineto{\pgfqpoint{-0.055556in}{0.000000in}}%
\pgfusepath{stroke,fill}%
}%
\begin{pgfscope}%
\pgfsys@transformshift{7.200000in}{5.400000in}%
\pgfsys@useobject{currentmarker}{}%
\end{pgfscope}%
\end{pgfscope}%
\begin{pgfscope}%
\pgftext[left,bottom,x=7.255556in,y=5.346296in,rotate=0.000000]{{\rmfamily\fontsize{12.000000}{14.400000}\selectfont \(\displaystyle 1.0\)}}
%
\end{pgfscope}%
\begin{pgfscope}%
\pgfsetrectcap%
\pgfsetroundjoin%
\definecolor{currentfill}{rgb}{1.000000,1.000000,1.000000}%
\pgfsetfillcolor{currentfill}%
\pgfsetlinewidth{0.000000pt}%
\definecolor{currentstroke}{rgb}{0.000000,0.000000,0.000000}%
\pgfsetstrokecolor{currentstroke}%
\pgfsetdash{}{0pt}%
\pgfpathmoveto{\pgfqpoint{1.200000in}{3.104348in}}%
\pgfpathlineto{\pgfqpoint{7.200000in}{3.104348in}}%
\pgfpathlineto{\pgfqpoint{7.200000in}{4.147826in}}%
\pgfpathlineto{\pgfqpoint{1.200000in}{4.147826in}}%
\pgfpathclose%
\pgfusepath{fill}%
\end{pgfscope}%
\begin{pgfscope}%
\pgfpathrectangle{\pgfqpoint{1.200000in}{3.104348in}}{\pgfqpoint{6.000000in}{1.043478in}} %
\pgfusepath{clip}%
\pgfsetrectcap%
\pgfsetroundjoin%
\pgfsetlinewidth{1.003750pt}%
\definecolor{currentstroke}{rgb}{0.000000,0.000000,1.000000}%
\pgfsetstrokecolor{currentstroke}%
\pgfsetdash{}{0pt}%
\pgfpathmoveto{\pgfqpoint{1.200000in}{4.147826in}}%
\pgfpathlineto{\pgfqpoint{1.620000in}{3.945781in}}%
\pgfpathlineto{\pgfqpoint{1.830000in}{3.847008in}}%
\pgfpathlineto{\pgfqpoint{2.010000in}{3.764512in}}%
\pgfpathlineto{\pgfqpoint{2.190000in}{3.684636in}}%
\pgfpathlineto{\pgfqpoint{2.340000in}{3.620500in}}%
\pgfpathlineto{\pgfqpoint{2.490000in}{3.558937in}}%
\pgfpathlineto{\pgfqpoint{2.610000in}{3.511769in}}%
\pgfpathlineto{\pgfqpoint{2.730000in}{3.466640in}}%
\pgfpathlineto{\pgfqpoint{2.850000in}{3.423728in}}%
\pgfpathlineto{\pgfqpoint{2.970000in}{3.383206in}}%
\pgfpathlineto{\pgfqpoint{3.090000in}{3.345250in}}%
\pgfpathlineto{\pgfqpoint{3.210000in}{3.310035in}}%
\pgfpathlineto{\pgfqpoint{3.330000in}{3.277736in}}%
\pgfpathlineto{\pgfqpoint{3.420000in}{3.255532in}}%
\pgfpathlineto{\pgfqpoint{3.510000in}{3.235145in}}%
\pgfpathlineto{\pgfqpoint{3.600000in}{3.216650in}}%
\pgfpathlineto{\pgfqpoint{3.690000in}{3.200073in}}%
\pgfpathlineto{\pgfqpoint{3.780000in}{3.185521in}}%
\pgfpathlineto{\pgfqpoint{3.900000in}{3.169694in}}%
\pgfpathlineto{\pgfqpoint{4.020000in}{3.157066in}}%
\pgfpathlineto{\pgfqpoint{4.110000in}{3.150423in}}%
\pgfpathlineto{\pgfqpoint{4.200000in}{3.146087in}}%
\pgfpathlineto{\pgfqpoint{4.380000in}{3.157066in}}%
\pgfpathlineto{\pgfqpoint{4.500000in}{3.169694in}}%
\pgfpathlineto{\pgfqpoint{4.590000in}{3.181285in}}%
\pgfpathlineto{\pgfqpoint{4.680000in}{3.194991in}}%
\pgfpathlineto{\pgfqpoint{4.770000in}{3.210907in}}%
\pgfpathlineto{\pgfqpoint{4.860000in}{3.228765in}}%
\pgfpathlineto{\pgfqpoint{4.950000in}{3.248530in}}%
\pgfpathlineto{\pgfqpoint{5.040000in}{3.270137in}}%
\pgfpathlineto{\pgfqpoint{5.130000in}{3.293511in}}%
\pgfpathlineto{\pgfqpoint{5.250000in}{3.327289in}}%
\pgfpathlineto{\pgfqpoint{5.370000in}{3.363896in}}%
\pgfpathlineto{\pgfqpoint{5.490000in}{3.403157in}}%
\pgfpathlineto{\pgfqpoint{5.610000in}{3.444896in}}%
\pgfpathlineto{\pgfqpoint{5.730000in}{3.488939in}}%
\pgfpathlineto{\pgfqpoint{5.850000in}{3.535109in}}%
\pgfpathlineto{\pgfqpoint{6.000000in}{3.595547in}}%
\pgfpathlineto{\pgfqpoint{6.150000in}{3.658695in}}%
\pgfpathlineto{\pgfqpoint{6.300000in}{3.724209in}}%
\pgfpathlineto{\pgfqpoint{6.480000in}{3.805469in}}%
\pgfpathlineto{\pgfqpoint{6.690000in}{3.903170in}}%
\pgfpathlineto{\pgfqpoint{6.930000in}{4.017488in}}%
\pgfpathlineto{\pgfqpoint{7.200000in}{4.147826in}}%
\pgfpathlineto{\pgfqpoint{7.200000in}{4.147826in}}%
\pgfusepath{stroke}%
\end{pgfscope}%
\begin{pgfscope}%
\pgfsetbuttcap%
\pgfsetroundjoin%
\definecolor{currentfill}{rgb}{0.000000,0.000000,0.000000}%
\pgfsetfillcolor{currentfill}%
\pgfsetlinewidth{0.501875pt}%
\definecolor{currentstroke}{rgb}{0.000000,0.000000,0.000000}%
\pgfsetstrokecolor{currentstroke}%
\pgfsetdash{}{0pt}%
\pgfsys@defobject{currentmarker}{\pgfqpoint{0.000000in}{0.000000in}}{\pgfqpoint{0.055556in}{0.000000in}}{%
\pgfpathmoveto{\pgfqpoint{0.000000in}{0.000000in}}%
\pgfpathlineto{\pgfqpoint{0.055556in}{0.000000in}}%
\pgfusepath{stroke,fill}%
}%
\begin{pgfscope}%
\pgfsys@transformshift{1.200000in}{3.104348in}%
\pgfsys@useobject{currentmarker}{}%
\end{pgfscope}%
\end{pgfscope}%
\begin{pgfscope}%
\pgftext[left,bottom,x=0.479905in,y=3.043700in,rotate=0.000000]{{\rmfamily\fontsize{12.000000}{14.400000}\selectfont \(\displaystyle -0.00007\)}}
%
\end{pgfscope}%
\begin{pgfscope}%
\pgfsetbuttcap%
\pgfsetroundjoin%
\definecolor{currentfill}{rgb}{0.000000,0.000000,0.000000}%
\pgfsetfillcolor{currentfill}%
\pgfsetlinewidth{0.501875pt}%
\definecolor{currentstroke}{rgb}{0.000000,0.000000,0.000000}%
\pgfsetstrokecolor{currentstroke}%
\pgfsetdash{}{0pt}%
\pgfsys@defobject{currentmarker}{\pgfqpoint{0.000000in}{0.000000in}}{\pgfqpoint{0.055556in}{0.000000in}}{%
\pgfpathmoveto{\pgfqpoint{0.000000in}{0.000000in}}%
\pgfpathlineto{\pgfqpoint{0.055556in}{0.000000in}}%
\pgfusepath{stroke,fill}%
}%
\begin{pgfscope}%
\pgfsys@transformshift{1.200000in}{3.253416in}%
\pgfsys@useobject{currentmarker}{}%
\end{pgfscope}%
\end{pgfscope}%
\begin{pgfscope}%
\pgftext[left,bottom,x=0.479905in,y=3.192768in,rotate=0.000000]{{\rmfamily\fontsize{12.000000}{14.400000}\selectfont \(\displaystyle -0.00006\)}}
%
\end{pgfscope}%
\begin{pgfscope}%
\pgfsetbuttcap%
\pgfsetroundjoin%
\definecolor{currentfill}{rgb}{0.000000,0.000000,0.000000}%
\pgfsetfillcolor{currentfill}%
\pgfsetlinewidth{0.501875pt}%
\definecolor{currentstroke}{rgb}{0.000000,0.000000,0.000000}%
\pgfsetstrokecolor{currentstroke}%
\pgfsetdash{}{0pt}%
\pgfsys@defobject{currentmarker}{\pgfqpoint{0.000000in}{0.000000in}}{\pgfqpoint{0.055556in}{0.000000in}}{%
\pgfpathmoveto{\pgfqpoint{0.000000in}{0.000000in}}%
\pgfpathlineto{\pgfqpoint{0.055556in}{0.000000in}}%
\pgfusepath{stroke,fill}%
}%
\begin{pgfscope}%
\pgfsys@transformshift{1.200000in}{3.402484in}%
\pgfsys@useobject{currentmarker}{}%
\end{pgfscope}%
\end{pgfscope}%
\begin{pgfscope}%
\pgftext[left,bottom,x=0.479905in,y=3.341836in,rotate=0.000000]{{\rmfamily\fontsize{12.000000}{14.400000}\selectfont \(\displaystyle -0.00005\)}}
%
\end{pgfscope}%
\begin{pgfscope}%
\pgfsetbuttcap%
\pgfsetroundjoin%
\definecolor{currentfill}{rgb}{0.000000,0.000000,0.000000}%
\pgfsetfillcolor{currentfill}%
\pgfsetlinewidth{0.501875pt}%
\definecolor{currentstroke}{rgb}{0.000000,0.000000,0.000000}%
\pgfsetstrokecolor{currentstroke}%
\pgfsetdash{}{0pt}%
\pgfsys@defobject{currentmarker}{\pgfqpoint{0.000000in}{0.000000in}}{\pgfqpoint{0.055556in}{0.000000in}}{%
\pgfpathmoveto{\pgfqpoint{0.000000in}{0.000000in}}%
\pgfpathlineto{\pgfqpoint{0.055556in}{0.000000in}}%
\pgfusepath{stroke,fill}%
}%
\begin{pgfscope}%
\pgfsys@transformshift{1.200000in}{3.551553in}%
\pgfsys@useobject{currentmarker}{}%
\end{pgfscope}%
\end{pgfscope}%
\begin{pgfscope}%
\pgftext[left,bottom,x=0.479905in,y=3.490905in,rotate=0.000000]{{\rmfamily\fontsize{12.000000}{14.400000}\selectfont \(\displaystyle -0.00004\)}}
%
\end{pgfscope}%
\begin{pgfscope}%
\pgfsetbuttcap%
\pgfsetroundjoin%
\definecolor{currentfill}{rgb}{0.000000,0.000000,0.000000}%
\pgfsetfillcolor{currentfill}%
\pgfsetlinewidth{0.501875pt}%
\definecolor{currentstroke}{rgb}{0.000000,0.000000,0.000000}%
\pgfsetstrokecolor{currentstroke}%
\pgfsetdash{}{0pt}%
\pgfsys@defobject{currentmarker}{\pgfqpoint{0.000000in}{0.000000in}}{\pgfqpoint{0.055556in}{0.000000in}}{%
\pgfpathmoveto{\pgfqpoint{0.000000in}{0.000000in}}%
\pgfpathlineto{\pgfqpoint{0.055556in}{0.000000in}}%
\pgfusepath{stroke,fill}%
}%
\begin{pgfscope}%
\pgfsys@transformshift{1.200000in}{3.700621in}%
\pgfsys@useobject{currentmarker}{}%
\end{pgfscope}%
\end{pgfscope}%
\begin{pgfscope}%
\pgftext[left,bottom,x=0.479905in,y=3.639973in,rotate=0.000000]{{\rmfamily\fontsize{12.000000}{14.400000}\selectfont \(\displaystyle -0.00003\)}}
%
\end{pgfscope}%
\begin{pgfscope}%
\pgfsetbuttcap%
\pgfsetroundjoin%
\definecolor{currentfill}{rgb}{0.000000,0.000000,0.000000}%
\pgfsetfillcolor{currentfill}%
\pgfsetlinewidth{0.501875pt}%
\definecolor{currentstroke}{rgb}{0.000000,0.000000,0.000000}%
\pgfsetstrokecolor{currentstroke}%
\pgfsetdash{}{0pt}%
\pgfsys@defobject{currentmarker}{\pgfqpoint{0.000000in}{0.000000in}}{\pgfqpoint{0.055556in}{0.000000in}}{%
\pgfpathmoveto{\pgfqpoint{0.000000in}{0.000000in}}%
\pgfpathlineto{\pgfqpoint{0.055556in}{0.000000in}}%
\pgfusepath{stroke,fill}%
}%
\begin{pgfscope}%
\pgfsys@transformshift{1.200000in}{3.849689in}%
\pgfsys@useobject{currentmarker}{}%
\end{pgfscope}%
\end{pgfscope}%
\begin{pgfscope}%
\pgftext[left,bottom,x=0.479905in,y=3.789041in,rotate=0.000000]{{\rmfamily\fontsize{12.000000}{14.400000}\selectfont \(\displaystyle -0.00002\)}}
%
\end{pgfscope}%
\begin{pgfscope}%
\pgfsetbuttcap%
\pgfsetroundjoin%
\definecolor{currentfill}{rgb}{0.000000,0.000000,0.000000}%
\pgfsetfillcolor{currentfill}%
\pgfsetlinewidth{0.501875pt}%
\definecolor{currentstroke}{rgb}{0.000000,0.000000,0.000000}%
\pgfsetstrokecolor{currentstroke}%
\pgfsetdash{}{0pt}%
\pgfsys@defobject{currentmarker}{\pgfqpoint{0.000000in}{0.000000in}}{\pgfqpoint{0.055556in}{0.000000in}}{%
\pgfpathmoveto{\pgfqpoint{0.000000in}{0.000000in}}%
\pgfpathlineto{\pgfqpoint{0.055556in}{0.000000in}}%
\pgfusepath{stroke,fill}%
}%
\begin{pgfscope}%
\pgfsys@transformshift{1.200000in}{3.998758in}%
\pgfsys@useobject{currentmarker}{}%
\end{pgfscope}%
\end{pgfscope}%
\begin{pgfscope}%
\pgftext[left,bottom,x=0.479905in,y=3.938110in,rotate=0.000000]{{\rmfamily\fontsize{12.000000}{14.400000}\selectfont \(\displaystyle -0.00001\)}}
%
\end{pgfscope}%
\begin{pgfscope}%
\pgfsetbuttcap%
\pgfsetroundjoin%
\definecolor{currentfill}{rgb}{0.000000,0.000000,0.000000}%
\pgfsetfillcolor{currentfill}%
\pgfsetlinewidth{0.501875pt}%
\definecolor{currentstroke}{rgb}{0.000000,0.000000,0.000000}%
\pgfsetstrokecolor{currentstroke}%
\pgfsetdash{}{0pt}%
\pgfsys@defobject{currentmarker}{\pgfqpoint{0.000000in}{0.000000in}}{\pgfqpoint{0.055556in}{0.000000in}}{%
\pgfpathmoveto{\pgfqpoint{0.000000in}{0.000000in}}%
\pgfpathlineto{\pgfqpoint{0.055556in}{0.000000in}}%
\pgfusepath{stroke,fill}%
}%
\begin{pgfscope}%
\pgfsys@transformshift{1.200000in}{4.147826in}%
\pgfsys@useobject{currentmarker}{}%
\end{pgfscope}%
\end{pgfscope}%
\begin{pgfscope}%
\pgftext[left,bottom,x=0.609535in,y=4.094122in,rotate=0.000000]{{\rmfamily\fontsize{12.000000}{14.400000}\selectfont \(\displaystyle 0.00000\)}}
%
\end{pgfscope}%
\begin{pgfscope}%
\pgfsetrectcap%
\pgfsetroundjoin%
\pgfsetlinewidth{1.003750pt}%
\definecolor{currentstroke}{rgb}{0.000000,0.000000,0.000000}%
\pgfsetstrokecolor{currentstroke}%
\pgfsetdash{}{0pt}%
\pgfpathmoveto{\pgfqpoint{1.200000in}{4.147826in}}%
\pgfpathlineto{\pgfqpoint{7.200000in}{4.147826in}}%
\pgfusepath{stroke}%
\end{pgfscope}%
\begin{pgfscope}%
\pgfsetrectcap%
\pgfsetroundjoin%
\pgfsetlinewidth{1.003750pt}%
\definecolor{currentstroke}{rgb}{0.000000,0.000000,0.000000}%
\pgfsetstrokecolor{currentstroke}%
\pgfsetdash{}{0pt}%
\pgfpathmoveto{\pgfqpoint{7.200000in}{3.104348in}}%
\pgfpathlineto{\pgfqpoint{7.200000in}{4.147826in}}%
\pgfusepath{stroke}%
\end{pgfscope}%
\begin{pgfscope}%
\pgfsetrectcap%
\pgfsetroundjoin%
\pgfsetlinewidth{1.003750pt}%
\definecolor{currentstroke}{rgb}{0.000000,0.000000,0.000000}%
\pgfsetstrokecolor{currentstroke}%
\pgfsetdash{}{0pt}%
\pgfpathmoveto{\pgfqpoint{1.200000in}{3.104348in}}%
\pgfpathlineto{\pgfqpoint{7.200000in}{3.104348in}}%
\pgfusepath{stroke}%
\end{pgfscope}%
\begin{pgfscope}%
\pgfsetrectcap%
\pgfsetroundjoin%
\pgfsetlinewidth{1.003750pt}%
\definecolor{currentstroke}{rgb}{0.000000,0.000000,0.000000}%
\pgfsetstrokecolor{currentstroke}%
\pgfsetdash{}{0pt}%
\pgfpathmoveto{\pgfqpoint{1.200000in}{3.104348in}}%
\pgfpathlineto{\pgfqpoint{1.200000in}{4.147826in}}%
\pgfusepath{stroke}%
\end{pgfscope}%
\begin{pgfscope}%
\pgfpathrectangle{\pgfqpoint{1.200000in}{3.104348in}}{\pgfqpoint{6.000000in}{1.043478in}} %
\pgfusepath{clip}%
\pgfsetrectcap%
\pgfsetroundjoin%
\pgfsetlinewidth{1.003750pt}%
\definecolor{currentstroke}{rgb}{0.000000,0.000000,1.000000}%
\pgfsetstrokecolor{currentstroke}%
\pgfsetdash{}{0pt}%
\pgfpathmoveto{\pgfqpoint{1.200000in}{4.147826in}}%
\pgfpathlineto{\pgfqpoint{4.080000in}{4.147826in}}%
\pgfpathlineto{\pgfqpoint{4.110000in}{4.043478in}}%
\pgfpathlineto{\pgfqpoint{4.140000in}{4.043478in}}%
\pgfpathlineto{\pgfqpoint{4.200000in}{3.834783in}}%
\pgfpathlineto{\pgfqpoint{4.260000in}{4.043478in}}%
\pgfpathlineto{\pgfqpoint{4.290000in}{4.043478in}}%
\pgfpathlineto{\pgfqpoint{4.320000in}{4.147826in}}%
\pgfpathlineto{\pgfqpoint{7.200000in}{4.147826in}}%
\pgfpathlineto{\pgfqpoint{7.200000in}{4.147826in}}%
\pgfusepath{stroke}%
\end{pgfscope}%
\begin{pgfscope}%
\pgfsetbuttcap%
\pgfsetroundjoin%
\definecolor{currentfill}{rgb}{0.000000,0.000000,0.000000}%
\pgfsetfillcolor{currentfill}%
\pgfsetlinewidth{0.501875pt}%
\definecolor{currentstroke}{rgb}{0.000000,0.000000,0.000000}%
\pgfsetstrokecolor{currentstroke}%
\pgfsetdash{}{0pt}%
\pgfsys@defobject{currentmarker}{\pgfqpoint{-0.055556in}{0.000000in}}{\pgfqpoint{0.000000in}{0.000000in}}{%
\pgfpathmoveto{\pgfqpoint{0.000000in}{0.000000in}}%
\pgfpathlineto{\pgfqpoint{-0.055556in}{0.000000in}}%
\pgfusepath{stroke,fill}%
}%
\begin{pgfscope}%
\pgfsys@transformshift{7.200000in}{3.104348in}%
\pgfsys@useobject{currentmarker}{}%
\end{pgfscope}%
\end{pgfscope}%
\begin{pgfscope}%
\pgftext[left,bottom,x=7.255556in,y=3.050644in,rotate=0.000000]{{\rmfamily\fontsize{12.000000}{14.400000}\selectfont \(\displaystyle 0.0\)}}
%
\end{pgfscope}%
\begin{pgfscope}%
\pgfsetbuttcap%
\pgfsetroundjoin%
\definecolor{currentfill}{rgb}{0.000000,0.000000,0.000000}%
\pgfsetfillcolor{currentfill}%
\pgfsetlinewidth{0.501875pt}%
\definecolor{currentstroke}{rgb}{0.000000,0.000000,0.000000}%
\pgfsetstrokecolor{currentstroke}%
\pgfsetdash{}{0pt}%
\pgfsys@defobject{currentmarker}{\pgfqpoint{-0.055556in}{0.000000in}}{\pgfqpoint{0.000000in}{0.000000in}}{%
\pgfpathmoveto{\pgfqpoint{0.000000in}{0.000000in}}%
\pgfpathlineto{\pgfqpoint{-0.055556in}{0.000000in}}%
\pgfusepath{stroke,fill}%
}%
\begin{pgfscope}%
\pgfsys@transformshift{7.200000in}{3.313043in}%
\pgfsys@useobject{currentmarker}{}%
\end{pgfscope}%
\end{pgfscope}%
\begin{pgfscope}%
\pgftext[left,bottom,x=7.255556in,y=3.259340in,rotate=0.000000]{{\rmfamily\fontsize{12.000000}{14.400000}\selectfont \(\displaystyle 0.2\)}}
%
\end{pgfscope}%
\begin{pgfscope}%
\pgfsetbuttcap%
\pgfsetroundjoin%
\definecolor{currentfill}{rgb}{0.000000,0.000000,0.000000}%
\pgfsetfillcolor{currentfill}%
\pgfsetlinewidth{0.501875pt}%
\definecolor{currentstroke}{rgb}{0.000000,0.000000,0.000000}%
\pgfsetstrokecolor{currentstroke}%
\pgfsetdash{}{0pt}%
\pgfsys@defobject{currentmarker}{\pgfqpoint{-0.055556in}{0.000000in}}{\pgfqpoint{0.000000in}{0.000000in}}{%
\pgfpathmoveto{\pgfqpoint{0.000000in}{0.000000in}}%
\pgfpathlineto{\pgfqpoint{-0.055556in}{0.000000in}}%
\pgfusepath{stroke,fill}%
}%
\begin{pgfscope}%
\pgfsys@transformshift{7.200000in}{3.521739in}%
\pgfsys@useobject{currentmarker}{}%
\end{pgfscope}%
\end{pgfscope}%
\begin{pgfscope}%
\pgftext[left,bottom,x=7.255556in,y=3.468036in,rotate=0.000000]{{\rmfamily\fontsize{12.000000}{14.400000}\selectfont \(\displaystyle 0.4\)}}
%
\end{pgfscope}%
\begin{pgfscope}%
\pgfsetbuttcap%
\pgfsetroundjoin%
\definecolor{currentfill}{rgb}{0.000000,0.000000,0.000000}%
\pgfsetfillcolor{currentfill}%
\pgfsetlinewidth{0.501875pt}%
\definecolor{currentstroke}{rgb}{0.000000,0.000000,0.000000}%
\pgfsetstrokecolor{currentstroke}%
\pgfsetdash{}{0pt}%
\pgfsys@defobject{currentmarker}{\pgfqpoint{-0.055556in}{0.000000in}}{\pgfqpoint{0.000000in}{0.000000in}}{%
\pgfpathmoveto{\pgfqpoint{0.000000in}{0.000000in}}%
\pgfpathlineto{\pgfqpoint{-0.055556in}{0.000000in}}%
\pgfusepath{stroke,fill}%
}%
\begin{pgfscope}%
\pgfsys@transformshift{7.200000in}{3.730435in}%
\pgfsys@useobject{currentmarker}{}%
\end{pgfscope}%
\end{pgfscope}%
\begin{pgfscope}%
\pgftext[left,bottom,x=7.255556in,y=3.676731in,rotate=0.000000]{{\rmfamily\fontsize{12.000000}{14.400000}\selectfont \(\displaystyle 0.6\)}}
%
\end{pgfscope}%
\begin{pgfscope}%
\pgfsetbuttcap%
\pgfsetroundjoin%
\definecolor{currentfill}{rgb}{0.000000,0.000000,0.000000}%
\pgfsetfillcolor{currentfill}%
\pgfsetlinewidth{0.501875pt}%
\definecolor{currentstroke}{rgb}{0.000000,0.000000,0.000000}%
\pgfsetstrokecolor{currentstroke}%
\pgfsetdash{}{0pt}%
\pgfsys@defobject{currentmarker}{\pgfqpoint{-0.055556in}{0.000000in}}{\pgfqpoint{0.000000in}{0.000000in}}{%
\pgfpathmoveto{\pgfqpoint{0.000000in}{0.000000in}}%
\pgfpathlineto{\pgfqpoint{-0.055556in}{0.000000in}}%
\pgfusepath{stroke,fill}%
}%
\begin{pgfscope}%
\pgfsys@transformshift{7.200000in}{3.939130in}%
\pgfsys@useobject{currentmarker}{}%
\end{pgfscope}%
\end{pgfscope}%
\begin{pgfscope}%
\pgftext[left,bottom,x=7.255556in,y=3.885427in,rotate=0.000000]{{\rmfamily\fontsize{12.000000}{14.400000}\selectfont \(\displaystyle 0.8\)}}
%
\end{pgfscope}%
\begin{pgfscope}%
\pgfsetbuttcap%
\pgfsetroundjoin%
\definecolor{currentfill}{rgb}{0.000000,0.000000,0.000000}%
\pgfsetfillcolor{currentfill}%
\pgfsetlinewidth{0.501875pt}%
\definecolor{currentstroke}{rgb}{0.000000,0.000000,0.000000}%
\pgfsetstrokecolor{currentstroke}%
\pgfsetdash{}{0pt}%
\pgfsys@defobject{currentmarker}{\pgfqpoint{-0.055556in}{0.000000in}}{\pgfqpoint{0.000000in}{0.000000in}}{%
\pgfpathmoveto{\pgfqpoint{0.000000in}{0.000000in}}%
\pgfpathlineto{\pgfqpoint{-0.055556in}{0.000000in}}%
\pgfusepath{stroke,fill}%
}%
\begin{pgfscope}%
\pgfsys@transformshift{7.200000in}{4.147826in}%
\pgfsys@useobject{currentmarker}{}%
\end{pgfscope}%
\end{pgfscope}%
\begin{pgfscope}%
\pgftext[left,bottom,x=7.255556in,y=4.094122in,rotate=0.000000]{{\rmfamily\fontsize{12.000000}{14.400000}\selectfont \(\displaystyle 1.0\)}}
%
\end{pgfscope}%
\begin{pgfscope}%
\pgfsetrectcap%
\pgfsetroundjoin%
\definecolor{currentfill}{rgb}{1.000000,1.000000,1.000000}%
\pgfsetfillcolor{currentfill}%
\pgfsetlinewidth{0.000000pt}%
\definecolor{currentstroke}{rgb}{0.000000,0.000000,0.000000}%
\pgfsetstrokecolor{currentstroke}%
\pgfsetdash{}{0pt}%
\pgfpathmoveto{\pgfqpoint{1.200000in}{1.852174in}}%
\pgfpathlineto{\pgfqpoint{7.200000in}{1.852174in}}%
\pgfpathlineto{\pgfqpoint{7.200000in}{2.895652in}}%
\pgfpathlineto{\pgfqpoint{1.200000in}{2.895652in}}%
\pgfpathclose%
\pgfusepath{fill}%
\end{pgfscope}%
\begin{pgfscope}%
\pgfpathrectangle{\pgfqpoint{1.200000in}{1.852174in}}{\pgfqpoint{6.000000in}{1.043478in}} %
\pgfusepath{clip}%
\pgfsetrectcap%
\pgfsetroundjoin%
\pgfsetlinewidth{1.003750pt}%
\definecolor{currentstroke}{rgb}{0.000000,0.000000,1.000000}%
\pgfsetstrokecolor{currentstroke}%
\pgfsetdash{}{0pt}%
\pgfpathmoveto{\pgfqpoint{1.200000in}{2.895652in}}%
\pgfpathlineto{\pgfqpoint{1.650000in}{2.692382in}}%
\pgfpathlineto{\pgfqpoint{1.890000in}{2.586469in}}%
\pgfpathlineto{\pgfqpoint{2.100000in}{2.496399in}}%
\pgfpathlineto{\pgfqpoint{2.280000in}{2.421736in}}%
\pgfpathlineto{\pgfqpoint{2.430000in}{2.361676in}}%
\pgfpathlineto{\pgfqpoint{2.580000in}{2.303875in}}%
\pgfpathlineto{\pgfqpoint{2.730000in}{2.248611in}}%
\pgfpathlineto{\pgfqpoint{2.880000in}{2.196163in}}%
\pgfpathlineto{\pgfqpoint{3.000000in}{2.156419in}}%
\pgfpathlineto{\pgfqpoint{3.120000in}{2.118798in}}%
\pgfpathlineto{\pgfqpoint{3.240000in}{2.083442in}}%
\pgfpathlineto{\pgfqpoint{3.360000in}{2.050493in}}%
\pgfpathlineto{\pgfqpoint{3.480000in}{2.020091in}}%
\pgfpathlineto{\pgfqpoint{3.600000in}{1.992433in}}%
\pgfpathlineto{\pgfqpoint{3.720000in}{1.967472in}}%
\pgfpathlineto{\pgfqpoint{3.810000in}{1.950835in}}%
\pgfpathlineto{\pgfqpoint{3.900000in}{1.935906in}}%
\pgfpathlineto{\pgfqpoint{4.050000in}{1.914735in}}%
\pgfpathlineto{\pgfqpoint{4.140000in}{1.903129in}}%
\pgfpathlineto{\pgfqpoint{4.200000in}{1.893913in}}%
\pgfpathlineto{\pgfqpoint{4.260000in}{1.903129in}}%
\pgfpathlineto{\pgfqpoint{4.380000in}{1.918803in}}%
\pgfpathlineto{\pgfqpoint{4.560000in}{1.945650in}}%
\pgfpathlineto{\pgfqpoint{4.680000in}{1.967472in}}%
\pgfpathlineto{\pgfqpoint{4.800000in}{1.992433in}}%
\pgfpathlineto{\pgfqpoint{4.920000in}{2.020091in}}%
\pgfpathlineto{\pgfqpoint{5.040000in}{2.050493in}}%
\pgfpathlineto{\pgfqpoint{5.160000in}{2.083442in}}%
\pgfpathlineto{\pgfqpoint{5.280000in}{2.118798in}}%
\pgfpathlineto{\pgfqpoint{5.400000in}{2.156419in}}%
\pgfpathlineto{\pgfqpoint{5.520000in}{2.196163in}}%
\pgfpathlineto{\pgfqpoint{5.640000in}{2.237887in}}%
\pgfpathlineto{\pgfqpoint{5.790000in}{2.292610in}}%
\pgfpathlineto{\pgfqpoint{5.940000in}{2.349926in}}%
\pgfpathlineto{\pgfqpoint{6.090000in}{2.409557in}}%
\pgfpathlineto{\pgfqpoint{6.270000in}{2.483778in}}%
\pgfpathlineto{\pgfqpoint{6.450000in}{2.560450in}}%
\pgfpathlineto{\pgfqpoint{6.660000in}{2.652358in}}%
\pgfpathlineto{\pgfqpoint{6.930000in}{2.773231in}}%
\pgfpathlineto{\pgfqpoint{7.200000in}{2.895652in}}%
\pgfpathlineto{\pgfqpoint{7.200000in}{2.895652in}}%
\pgfusepath{stroke}%
\end{pgfscope}%
\begin{pgfscope}%
\pgfsetbuttcap%
\pgfsetroundjoin%
\definecolor{currentfill}{rgb}{0.000000,0.000000,0.000000}%
\pgfsetfillcolor{currentfill}%
\pgfsetlinewidth{0.501875pt}%
\definecolor{currentstroke}{rgb}{0.000000,0.000000,0.000000}%
\pgfsetstrokecolor{currentstroke}%
\pgfsetdash{}{0pt}%
\pgfsys@defobject{currentmarker}{\pgfqpoint{0.000000in}{0.000000in}}{\pgfqpoint{0.055556in}{0.000000in}}{%
\pgfpathmoveto{\pgfqpoint{0.000000in}{0.000000in}}%
\pgfpathlineto{\pgfqpoint{0.055556in}{0.000000in}}%
\pgfusepath{stroke,fill}%
}%
\begin{pgfscope}%
\pgfsys@transformshift{1.200000in}{1.852174in}%
\pgfsys@useobject{currentmarker}{}%
\end{pgfscope}%
\end{pgfscope}%
\begin{pgfscope}%
\pgftext[left,bottom,x=0.479905in,y=1.791526in,rotate=0.000000]{{\rmfamily\fontsize{12.000000}{14.400000}\selectfont \(\displaystyle -0.00007\)}}
%
\end{pgfscope}%
\begin{pgfscope}%
\pgfsetbuttcap%
\pgfsetroundjoin%
\definecolor{currentfill}{rgb}{0.000000,0.000000,0.000000}%
\pgfsetfillcolor{currentfill}%
\pgfsetlinewidth{0.501875pt}%
\definecolor{currentstroke}{rgb}{0.000000,0.000000,0.000000}%
\pgfsetstrokecolor{currentstroke}%
\pgfsetdash{}{0pt}%
\pgfsys@defobject{currentmarker}{\pgfqpoint{0.000000in}{0.000000in}}{\pgfqpoint{0.055556in}{0.000000in}}{%
\pgfpathmoveto{\pgfqpoint{0.000000in}{0.000000in}}%
\pgfpathlineto{\pgfqpoint{0.055556in}{0.000000in}}%
\pgfusepath{stroke,fill}%
}%
\begin{pgfscope}%
\pgfsys@transformshift{1.200000in}{2.001242in}%
\pgfsys@useobject{currentmarker}{}%
\end{pgfscope}%
\end{pgfscope}%
\begin{pgfscope}%
\pgftext[left,bottom,x=0.479905in,y=1.940594in,rotate=0.000000]{{\rmfamily\fontsize{12.000000}{14.400000}\selectfont \(\displaystyle -0.00006\)}}
%
\end{pgfscope}%
\begin{pgfscope}%
\pgfsetbuttcap%
\pgfsetroundjoin%
\definecolor{currentfill}{rgb}{0.000000,0.000000,0.000000}%
\pgfsetfillcolor{currentfill}%
\pgfsetlinewidth{0.501875pt}%
\definecolor{currentstroke}{rgb}{0.000000,0.000000,0.000000}%
\pgfsetstrokecolor{currentstroke}%
\pgfsetdash{}{0pt}%
\pgfsys@defobject{currentmarker}{\pgfqpoint{0.000000in}{0.000000in}}{\pgfqpoint{0.055556in}{0.000000in}}{%
\pgfpathmoveto{\pgfqpoint{0.000000in}{0.000000in}}%
\pgfpathlineto{\pgfqpoint{0.055556in}{0.000000in}}%
\pgfusepath{stroke,fill}%
}%
\begin{pgfscope}%
\pgfsys@transformshift{1.200000in}{2.150311in}%
\pgfsys@useobject{currentmarker}{}%
\end{pgfscope}%
\end{pgfscope}%
\begin{pgfscope}%
\pgftext[left,bottom,x=0.479905in,y=2.089663in,rotate=0.000000]{{\rmfamily\fontsize{12.000000}{14.400000}\selectfont \(\displaystyle -0.00005\)}}
%
\end{pgfscope}%
\begin{pgfscope}%
\pgfsetbuttcap%
\pgfsetroundjoin%
\definecolor{currentfill}{rgb}{0.000000,0.000000,0.000000}%
\pgfsetfillcolor{currentfill}%
\pgfsetlinewidth{0.501875pt}%
\definecolor{currentstroke}{rgb}{0.000000,0.000000,0.000000}%
\pgfsetstrokecolor{currentstroke}%
\pgfsetdash{}{0pt}%
\pgfsys@defobject{currentmarker}{\pgfqpoint{0.000000in}{0.000000in}}{\pgfqpoint{0.055556in}{0.000000in}}{%
\pgfpathmoveto{\pgfqpoint{0.000000in}{0.000000in}}%
\pgfpathlineto{\pgfqpoint{0.055556in}{0.000000in}}%
\pgfusepath{stroke,fill}%
}%
\begin{pgfscope}%
\pgfsys@transformshift{1.200000in}{2.299379in}%
\pgfsys@useobject{currentmarker}{}%
\end{pgfscope}%
\end{pgfscope}%
\begin{pgfscope}%
\pgftext[left,bottom,x=0.479905in,y=2.238731in,rotate=0.000000]{{\rmfamily\fontsize{12.000000}{14.400000}\selectfont \(\displaystyle -0.00004\)}}
%
\end{pgfscope}%
\begin{pgfscope}%
\pgfsetbuttcap%
\pgfsetroundjoin%
\definecolor{currentfill}{rgb}{0.000000,0.000000,0.000000}%
\pgfsetfillcolor{currentfill}%
\pgfsetlinewidth{0.501875pt}%
\definecolor{currentstroke}{rgb}{0.000000,0.000000,0.000000}%
\pgfsetstrokecolor{currentstroke}%
\pgfsetdash{}{0pt}%
\pgfsys@defobject{currentmarker}{\pgfqpoint{0.000000in}{0.000000in}}{\pgfqpoint{0.055556in}{0.000000in}}{%
\pgfpathmoveto{\pgfqpoint{0.000000in}{0.000000in}}%
\pgfpathlineto{\pgfqpoint{0.055556in}{0.000000in}}%
\pgfusepath{stroke,fill}%
}%
\begin{pgfscope}%
\pgfsys@transformshift{1.200000in}{2.448447in}%
\pgfsys@useobject{currentmarker}{}%
\end{pgfscope}%
\end{pgfscope}%
\begin{pgfscope}%
\pgftext[left,bottom,x=0.479905in,y=2.387799in,rotate=0.000000]{{\rmfamily\fontsize{12.000000}{14.400000}\selectfont \(\displaystyle -0.00003\)}}
%
\end{pgfscope}%
\begin{pgfscope}%
\pgfsetbuttcap%
\pgfsetroundjoin%
\definecolor{currentfill}{rgb}{0.000000,0.000000,0.000000}%
\pgfsetfillcolor{currentfill}%
\pgfsetlinewidth{0.501875pt}%
\definecolor{currentstroke}{rgb}{0.000000,0.000000,0.000000}%
\pgfsetstrokecolor{currentstroke}%
\pgfsetdash{}{0pt}%
\pgfsys@defobject{currentmarker}{\pgfqpoint{0.000000in}{0.000000in}}{\pgfqpoint{0.055556in}{0.000000in}}{%
\pgfpathmoveto{\pgfqpoint{0.000000in}{0.000000in}}%
\pgfpathlineto{\pgfqpoint{0.055556in}{0.000000in}}%
\pgfusepath{stroke,fill}%
}%
\begin{pgfscope}%
\pgfsys@transformshift{1.200000in}{2.597516in}%
\pgfsys@useobject{currentmarker}{}%
\end{pgfscope}%
\end{pgfscope}%
\begin{pgfscope}%
\pgftext[left,bottom,x=0.479905in,y=2.536867in,rotate=0.000000]{{\rmfamily\fontsize{12.000000}{14.400000}\selectfont \(\displaystyle -0.00002\)}}
%
\end{pgfscope}%
\begin{pgfscope}%
\pgfsetbuttcap%
\pgfsetroundjoin%
\definecolor{currentfill}{rgb}{0.000000,0.000000,0.000000}%
\pgfsetfillcolor{currentfill}%
\pgfsetlinewidth{0.501875pt}%
\definecolor{currentstroke}{rgb}{0.000000,0.000000,0.000000}%
\pgfsetstrokecolor{currentstroke}%
\pgfsetdash{}{0pt}%
\pgfsys@defobject{currentmarker}{\pgfqpoint{0.000000in}{0.000000in}}{\pgfqpoint{0.055556in}{0.000000in}}{%
\pgfpathmoveto{\pgfqpoint{0.000000in}{0.000000in}}%
\pgfpathlineto{\pgfqpoint{0.055556in}{0.000000in}}%
\pgfusepath{stroke,fill}%
}%
\begin{pgfscope}%
\pgfsys@transformshift{1.200000in}{2.746584in}%
\pgfsys@useobject{currentmarker}{}%
\end{pgfscope}%
\end{pgfscope}%
\begin{pgfscope}%
\pgftext[left,bottom,x=0.479905in,y=2.685936in,rotate=0.000000]{{\rmfamily\fontsize{12.000000}{14.400000}\selectfont \(\displaystyle -0.00001\)}}
%
\end{pgfscope}%
\begin{pgfscope}%
\pgfsetbuttcap%
\pgfsetroundjoin%
\definecolor{currentfill}{rgb}{0.000000,0.000000,0.000000}%
\pgfsetfillcolor{currentfill}%
\pgfsetlinewidth{0.501875pt}%
\definecolor{currentstroke}{rgb}{0.000000,0.000000,0.000000}%
\pgfsetstrokecolor{currentstroke}%
\pgfsetdash{}{0pt}%
\pgfsys@defobject{currentmarker}{\pgfqpoint{0.000000in}{0.000000in}}{\pgfqpoint{0.055556in}{0.000000in}}{%
\pgfpathmoveto{\pgfqpoint{0.000000in}{0.000000in}}%
\pgfpathlineto{\pgfqpoint{0.055556in}{0.000000in}}%
\pgfusepath{stroke,fill}%
}%
\begin{pgfscope}%
\pgfsys@transformshift{1.200000in}{2.895652in}%
\pgfsys@useobject{currentmarker}{}%
\end{pgfscope}%
\end{pgfscope}%
\begin{pgfscope}%
\pgftext[left,bottom,x=0.609535in,y=2.841949in,rotate=0.000000]{{\rmfamily\fontsize{12.000000}{14.400000}\selectfont \(\displaystyle 0.00000\)}}
%
\end{pgfscope}%
\begin{pgfscope}%
\pgfsetrectcap%
\pgfsetroundjoin%
\pgfsetlinewidth{1.003750pt}%
\definecolor{currentstroke}{rgb}{0.000000,0.000000,0.000000}%
\pgfsetstrokecolor{currentstroke}%
\pgfsetdash{}{0pt}%
\pgfpathmoveto{\pgfqpoint{1.200000in}{2.895652in}}%
\pgfpathlineto{\pgfqpoint{7.200000in}{2.895652in}}%
\pgfusepath{stroke}%
\end{pgfscope}%
\begin{pgfscope}%
\pgfsetrectcap%
\pgfsetroundjoin%
\pgfsetlinewidth{1.003750pt}%
\definecolor{currentstroke}{rgb}{0.000000,0.000000,0.000000}%
\pgfsetstrokecolor{currentstroke}%
\pgfsetdash{}{0pt}%
\pgfpathmoveto{\pgfqpoint{7.200000in}{1.852174in}}%
\pgfpathlineto{\pgfqpoint{7.200000in}{2.895652in}}%
\pgfusepath{stroke}%
\end{pgfscope}%
\begin{pgfscope}%
\pgfsetrectcap%
\pgfsetroundjoin%
\pgfsetlinewidth{1.003750pt}%
\definecolor{currentstroke}{rgb}{0.000000,0.000000,0.000000}%
\pgfsetstrokecolor{currentstroke}%
\pgfsetdash{}{0pt}%
\pgfpathmoveto{\pgfqpoint{1.200000in}{1.852174in}}%
\pgfpathlineto{\pgfqpoint{7.200000in}{1.852174in}}%
\pgfusepath{stroke}%
\end{pgfscope}%
\begin{pgfscope}%
\pgfsetrectcap%
\pgfsetroundjoin%
\pgfsetlinewidth{1.003750pt}%
\definecolor{currentstroke}{rgb}{0.000000,0.000000,0.000000}%
\pgfsetstrokecolor{currentstroke}%
\pgfsetdash{}{0pt}%
\pgfpathmoveto{\pgfqpoint{1.200000in}{1.852174in}}%
\pgfpathlineto{\pgfqpoint{1.200000in}{2.895652in}}%
\pgfusepath{stroke}%
\end{pgfscope}%
\begin{pgfscope}%
\pgfpathrectangle{\pgfqpoint{1.200000in}{1.852174in}}{\pgfqpoint{6.000000in}{1.043478in}} %
\pgfusepath{clip}%
\pgfsetrectcap%
\pgfsetroundjoin%
\pgfsetlinewidth{1.003750pt}%
\definecolor{currentstroke}{rgb}{0.000000,0.000000,1.000000}%
\pgfsetstrokecolor{currentstroke}%
\pgfsetdash{}{0pt}%
\pgfpathmoveto{\pgfqpoint{1.200000in}{2.895652in}}%
\pgfpathlineto{\pgfqpoint{4.080000in}{2.895652in}}%
\pgfpathlineto{\pgfqpoint{4.140000in}{2.686957in}}%
\pgfpathlineto{\pgfqpoint{4.170000in}{2.478261in}}%
\pgfpathlineto{\pgfqpoint{4.200000in}{2.165217in}}%
\pgfpathlineto{\pgfqpoint{4.230000in}{2.478261in}}%
\pgfpathlineto{\pgfqpoint{4.260000in}{2.686957in}}%
\pgfpathlineto{\pgfqpoint{4.320000in}{2.895652in}}%
\pgfpathlineto{\pgfqpoint{7.200000in}{2.895652in}}%
\pgfpathlineto{\pgfqpoint{7.200000in}{2.895652in}}%
\pgfusepath{stroke}%
\end{pgfscope}%
\begin{pgfscope}%
\pgfsetbuttcap%
\pgfsetroundjoin%
\definecolor{currentfill}{rgb}{0.000000,0.000000,0.000000}%
\pgfsetfillcolor{currentfill}%
\pgfsetlinewidth{0.501875pt}%
\definecolor{currentstroke}{rgb}{0.000000,0.000000,0.000000}%
\pgfsetstrokecolor{currentstroke}%
\pgfsetdash{}{0pt}%
\pgfsys@defobject{currentmarker}{\pgfqpoint{-0.055556in}{0.000000in}}{\pgfqpoint{0.000000in}{0.000000in}}{%
\pgfpathmoveto{\pgfqpoint{0.000000in}{0.000000in}}%
\pgfpathlineto{\pgfqpoint{-0.055556in}{0.000000in}}%
\pgfusepath{stroke,fill}%
}%
\begin{pgfscope}%
\pgfsys@transformshift{7.200000in}{1.852174in}%
\pgfsys@useobject{currentmarker}{}%
\end{pgfscope}%
\end{pgfscope}%
\begin{pgfscope}%
\pgftext[left,bottom,x=7.255556in,y=1.798470in,rotate=0.000000]{{\rmfamily\fontsize{12.000000}{14.400000}\selectfont \(\displaystyle 0.0\)}}
%
\end{pgfscope}%
\begin{pgfscope}%
\pgfsetbuttcap%
\pgfsetroundjoin%
\definecolor{currentfill}{rgb}{0.000000,0.000000,0.000000}%
\pgfsetfillcolor{currentfill}%
\pgfsetlinewidth{0.501875pt}%
\definecolor{currentstroke}{rgb}{0.000000,0.000000,0.000000}%
\pgfsetstrokecolor{currentstroke}%
\pgfsetdash{}{0pt}%
\pgfsys@defobject{currentmarker}{\pgfqpoint{-0.055556in}{0.000000in}}{\pgfqpoint{0.000000in}{0.000000in}}{%
\pgfpathmoveto{\pgfqpoint{0.000000in}{0.000000in}}%
\pgfpathlineto{\pgfqpoint{-0.055556in}{0.000000in}}%
\pgfusepath{stroke,fill}%
}%
\begin{pgfscope}%
\pgfsys@transformshift{7.200000in}{2.060870in}%
\pgfsys@useobject{currentmarker}{}%
\end{pgfscope}%
\end{pgfscope}%
\begin{pgfscope}%
\pgftext[left,bottom,x=7.255556in,y=2.007166in,rotate=0.000000]{{\rmfamily\fontsize{12.000000}{14.400000}\selectfont \(\displaystyle 0.2\)}}
%
\end{pgfscope}%
\begin{pgfscope}%
\pgfsetbuttcap%
\pgfsetroundjoin%
\definecolor{currentfill}{rgb}{0.000000,0.000000,0.000000}%
\pgfsetfillcolor{currentfill}%
\pgfsetlinewidth{0.501875pt}%
\definecolor{currentstroke}{rgb}{0.000000,0.000000,0.000000}%
\pgfsetstrokecolor{currentstroke}%
\pgfsetdash{}{0pt}%
\pgfsys@defobject{currentmarker}{\pgfqpoint{-0.055556in}{0.000000in}}{\pgfqpoint{0.000000in}{0.000000in}}{%
\pgfpathmoveto{\pgfqpoint{0.000000in}{0.000000in}}%
\pgfpathlineto{\pgfqpoint{-0.055556in}{0.000000in}}%
\pgfusepath{stroke,fill}%
}%
\begin{pgfscope}%
\pgfsys@transformshift{7.200000in}{2.269565in}%
\pgfsys@useobject{currentmarker}{}%
\end{pgfscope}%
\end{pgfscope}%
\begin{pgfscope}%
\pgftext[left,bottom,x=7.255556in,y=2.215862in,rotate=0.000000]{{\rmfamily\fontsize{12.000000}{14.400000}\selectfont \(\displaystyle 0.4\)}}
%
\end{pgfscope}%
\begin{pgfscope}%
\pgfsetbuttcap%
\pgfsetroundjoin%
\definecolor{currentfill}{rgb}{0.000000,0.000000,0.000000}%
\pgfsetfillcolor{currentfill}%
\pgfsetlinewidth{0.501875pt}%
\definecolor{currentstroke}{rgb}{0.000000,0.000000,0.000000}%
\pgfsetstrokecolor{currentstroke}%
\pgfsetdash{}{0pt}%
\pgfsys@defobject{currentmarker}{\pgfqpoint{-0.055556in}{0.000000in}}{\pgfqpoint{0.000000in}{0.000000in}}{%
\pgfpathmoveto{\pgfqpoint{0.000000in}{0.000000in}}%
\pgfpathlineto{\pgfqpoint{-0.055556in}{0.000000in}}%
\pgfusepath{stroke,fill}%
}%
\begin{pgfscope}%
\pgfsys@transformshift{7.200000in}{2.478261in}%
\pgfsys@useobject{currentmarker}{}%
\end{pgfscope}%
\end{pgfscope}%
\begin{pgfscope}%
\pgftext[left,bottom,x=7.255556in,y=2.424557in,rotate=0.000000]{{\rmfamily\fontsize{12.000000}{14.400000}\selectfont \(\displaystyle 0.6\)}}
%
\end{pgfscope}%
\begin{pgfscope}%
\pgfsetbuttcap%
\pgfsetroundjoin%
\definecolor{currentfill}{rgb}{0.000000,0.000000,0.000000}%
\pgfsetfillcolor{currentfill}%
\pgfsetlinewidth{0.501875pt}%
\definecolor{currentstroke}{rgb}{0.000000,0.000000,0.000000}%
\pgfsetstrokecolor{currentstroke}%
\pgfsetdash{}{0pt}%
\pgfsys@defobject{currentmarker}{\pgfqpoint{-0.055556in}{0.000000in}}{\pgfqpoint{0.000000in}{0.000000in}}{%
\pgfpathmoveto{\pgfqpoint{0.000000in}{0.000000in}}%
\pgfpathlineto{\pgfqpoint{-0.055556in}{0.000000in}}%
\pgfusepath{stroke,fill}%
}%
\begin{pgfscope}%
\pgfsys@transformshift{7.200000in}{2.686957in}%
\pgfsys@useobject{currentmarker}{}%
\end{pgfscope}%
\end{pgfscope}%
\begin{pgfscope}%
\pgftext[left,bottom,x=7.255556in,y=2.633253in,rotate=0.000000]{{\rmfamily\fontsize{12.000000}{14.400000}\selectfont \(\displaystyle 0.8\)}}
%
\end{pgfscope}%
\begin{pgfscope}%
\pgfsetbuttcap%
\pgfsetroundjoin%
\definecolor{currentfill}{rgb}{0.000000,0.000000,0.000000}%
\pgfsetfillcolor{currentfill}%
\pgfsetlinewidth{0.501875pt}%
\definecolor{currentstroke}{rgb}{0.000000,0.000000,0.000000}%
\pgfsetstrokecolor{currentstroke}%
\pgfsetdash{}{0pt}%
\pgfsys@defobject{currentmarker}{\pgfqpoint{-0.055556in}{0.000000in}}{\pgfqpoint{0.000000in}{0.000000in}}{%
\pgfpathmoveto{\pgfqpoint{0.000000in}{0.000000in}}%
\pgfpathlineto{\pgfqpoint{-0.055556in}{0.000000in}}%
\pgfusepath{stroke,fill}%
}%
\begin{pgfscope}%
\pgfsys@transformshift{7.200000in}{2.895652in}%
\pgfsys@useobject{currentmarker}{}%
\end{pgfscope}%
\end{pgfscope}%
\begin{pgfscope}%
\pgftext[left,bottom,x=7.255556in,y=2.841949in,rotate=0.000000]{{\rmfamily\fontsize{12.000000}{14.400000}\selectfont \(\displaystyle 1.0\)}}
%
\end{pgfscope}%
\begin{pgfscope}%
\pgfsetrectcap%
\pgfsetroundjoin%
\definecolor{currentfill}{rgb}{1.000000,1.000000,1.000000}%
\pgfsetfillcolor{currentfill}%
\pgfsetlinewidth{0.000000pt}%
\definecolor{currentstroke}{rgb}{0.000000,0.000000,0.000000}%
\pgfsetstrokecolor{currentstroke}%
\pgfsetdash{}{0pt}%
\pgfpathmoveto{\pgfqpoint{1.200000in}{0.600000in}}%
\pgfpathlineto{\pgfqpoint{7.200000in}{0.600000in}}%
\pgfpathlineto{\pgfqpoint{7.200000in}{1.643478in}}%
\pgfpathlineto{\pgfqpoint{1.200000in}{1.643478in}}%
\pgfpathclose%
\pgfusepath{fill}%
\end{pgfscope}%
\begin{pgfscope}%
\pgfpathrectangle{\pgfqpoint{1.200000in}{0.600000in}}{\pgfqpoint{6.000000in}{1.043478in}} %
\pgfusepath{clip}%
\pgfsetrectcap%
\pgfsetroundjoin%
\pgfsetlinewidth{1.003750pt}%
\definecolor{currentstroke}{rgb}{0.000000,0.000000,1.000000}%
\pgfsetstrokecolor{currentstroke}%
\pgfsetdash{}{0pt}%
\pgfpathmoveto{\pgfqpoint{1.200000in}{1.643478in}}%
\pgfpathlineto{\pgfqpoint{1.980000in}{1.367928in}}%
\pgfpathlineto{\pgfqpoint{2.370000in}{1.232413in}}%
\pgfpathlineto{\pgfqpoint{2.700000in}{1.119874in}}%
\pgfpathlineto{\pgfqpoint{3.000000in}{1.019771in}}%
\pgfpathlineto{\pgfqpoint{3.270000in}{0.931847in}}%
\pgfpathlineto{\pgfqpoint{3.540000in}{0.846328in}}%
\pgfpathlineto{\pgfqpoint{3.810000in}{0.763404in}}%
\pgfpathlineto{\pgfqpoint{3.990000in}{0.710682in}}%
\pgfpathlineto{\pgfqpoint{4.050000in}{0.692805in}}%
\pgfpathlineto{\pgfqpoint{4.110000in}{0.672147in}}%
\pgfpathlineto{\pgfqpoint{4.140000in}{0.660342in}}%
\pgfpathlineto{\pgfqpoint{4.170000in}{0.649915in}}%
\pgfpathlineto{\pgfqpoint{4.200000in}{0.641739in}}%
\pgfpathlineto{\pgfqpoint{4.230000in}{0.649376in}}%
\pgfpathlineto{\pgfqpoint{4.260000in}{0.659299in}}%
\pgfpathlineto{\pgfqpoint{4.320000in}{0.681650in}}%
\pgfpathlineto{\pgfqpoint{4.410000in}{0.709561in}}%
\pgfpathlineto{\pgfqpoint{4.590000in}{0.762507in}}%
\pgfpathlineto{\pgfqpoint{4.770000in}{0.817379in}}%
\pgfpathlineto{\pgfqpoint{5.100000in}{0.921382in}}%
\pgfpathlineto{\pgfqpoint{5.370000in}{1.009165in}}%
\pgfpathlineto{\pgfqpoint{5.670000in}{1.109152in}}%
\pgfpathlineto{\pgfqpoint{6.000000in}{1.221614in}}%
\pgfpathlineto{\pgfqpoint{6.390000in}{1.357105in}}%
\pgfpathlineto{\pgfqpoint{6.900000in}{1.536997in}}%
\pgfpathlineto{\pgfqpoint{7.200000in}{1.643478in}}%
\pgfpathlineto{\pgfqpoint{7.200000in}{1.643478in}}%
\pgfusepath{stroke}%
\end{pgfscope}%
\begin{pgfscope}%
\pgfsetbuttcap%
\pgfsetroundjoin%
\definecolor{currentfill}{rgb}{0.000000,0.000000,0.000000}%
\pgfsetfillcolor{currentfill}%
\pgfsetlinewidth{0.501875pt}%
\definecolor{currentstroke}{rgb}{0.000000,0.000000,0.000000}%
\pgfsetstrokecolor{currentstroke}%
\pgfsetdash{}{0pt}%
\pgfsys@defobject{currentmarker}{\pgfqpoint{0.000000in}{0.000000in}}{\pgfqpoint{0.000000in}{0.055556in}}{%
\pgfpathmoveto{\pgfqpoint{0.000000in}{0.000000in}}%
\pgfpathlineto{\pgfqpoint{0.000000in}{0.055556in}}%
\pgfusepath{stroke,fill}%
}%
\begin{pgfscope}%
\pgfsys@transformshift{1.200000in}{0.600000in}%
\pgfsys@useobject{currentmarker}{}%
\end{pgfscope}%
\end{pgfscope}%
\begin{pgfscope}%
\pgfsetbuttcap%
\pgfsetroundjoin%
\definecolor{currentfill}{rgb}{0.000000,0.000000,0.000000}%
\pgfsetfillcolor{currentfill}%
\pgfsetlinewidth{0.501875pt}%
\definecolor{currentstroke}{rgb}{0.000000,0.000000,0.000000}%
\pgfsetstrokecolor{currentstroke}%
\pgfsetdash{}{0pt}%
\pgfsys@defobject{currentmarker}{\pgfqpoint{0.000000in}{-0.055556in}}{\pgfqpoint{0.000000in}{0.000000in}}{%
\pgfpathmoveto{\pgfqpoint{0.000000in}{0.000000in}}%
\pgfpathlineto{\pgfqpoint{0.000000in}{-0.055556in}}%
\pgfusepath{stroke,fill}%
}%
\begin{pgfscope}%
\pgfsys@transformshift{1.200000in}{1.643478in}%
\pgfsys@useobject{currentmarker}{}%
\end{pgfscope}%
\end{pgfscope}%
\begin{pgfscope}%
\pgftext[left,bottom,x=1.095738in,y=0.437037in,rotate=0.000000]{{\rmfamily\fontsize{12.000000}{14.400000}\selectfont \(\displaystyle 0.0\)}}
%
\end{pgfscope}%
\begin{pgfscope}%
\pgfsetbuttcap%
\pgfsetroundjoin%
\definecolor{currentfill}{rgb}{0.000000,0.000000,0.000000}%
\pgfsetfillcolor{currentfill}%
\pgfsetlinewidth{0.501875pt}%
\definecolor{currentstroke}{rgb}{0.000000,0.000000,0.000000}%
\pgfsetstrokecolor{currentstroke}%
\pgfsetdash{}{0pt}%
\pgfsys@defobject{currentmarker}{\pgfqpoint{0.000000in}{0.000000in}}{\pgfqpoint{0.000000in}{0.055556in}}{%
\pgfpathmoveto{\pgfqpoint{0.000000in}{0.000000in}}%
\pgfpathlineto{\pgfqpoint{0.000000in}{0.055556in}}%
\pgfusepath{stroke,fill}%
}%
\begin{pgfscope}%
\pgfsys@transformshift{2.700000in}{0.600000in}%
\pgfsys@useobject{currentmarker}{}%
\end{pgfscope}%
\end{pgfscope}%
\begin{pgfscope}%
\pgfsetbuttcap%
\pgfsetroundjoin%
\definecolor{currentfill}{rgb}{0.000000,0.000000,0.000000}%
\pgfsetfillcolor{currentfill}%
\pgfsetlinewidth{0.501875pt}%
\definecolor{currentstroke}{rgb}{0.000000,0.000000,0.000000}%
\pgfsetstrokecolor{currentstroke}%
\pgfsetdash{}{0pt}%
\pgfsys@defobject{currentmarker}{\pgfqpoint{0.000000in}{-0.055556in}}{\pgfqpoint{0.000000in}{0.000000in}}{%
\pgfpathmoveto{\pgfqpoint{0.000000in}{0.000000in}}%
\pgfpathlineto{\pgfqpoint{0.000000in}{-0.055556in}}%
\pgfusepath{stroke,fill}%
}%
\begin{pgfscope}%
\pgfsys@transformshift{2.700000in}{1.643478in}%
\pgfsys@useobject{currentmarker}{}%
\end{pgfscope}%
\end{pgfscope}%
\begin{pgfscope}%
\pgftext[left,bottom,x=2.595738in,y=0.437037in,rotate=0.000000]{{\rmfamily\fontsize{12.000000}{14.400000}\selectfont \(\displaystyle 0.5\)}}
%
\end{pgfscope}%
\begin{pgfscope}%
\pgfsetbuttcap%
\pgfsetroundjoin%
\definecolor{currentfill}{rgb}{0.000000,0.000000,0.000000}%
\pgfsetfillcolor{currentfill}%
\pgfsetlinewidth{0.501875pt}%
\definecolor{currentstroke}{rgb}{0.000000,0.000000,0.000000}%
\pgfsetstrokecolor{currentstroke}%
\pgfsetdash{}{0pt}%
\pgfsys@defobject{currentmarker}{\pgfqpoint{0.000000in}{0.000000in}}{\pgfqpoint{0.000000in}{0.055556in}}{%
\pgfpathmoveto{\pgfqpoint{0.000000in}{0.000000in}}%
\pgfpathlineto{\pgfqpoint{0.000000in}{0.055556in}}%
\pgfusepath{stroke,fill}%
}%
\begin{pgfscope}%
\pgfsys@transformshift{4.200000in}{0.600000in}%
\pgfsys@useobject{currentmarker}{}%
\end{pgfscope}%
\end{pgfscope}%
\begin{pgfscope}%
\pgfsetbuttcap%
\pgfsetroundjoin%
\definecolor{currentfill}{rgb}{0.000000,0.000000,0.000000}%
\pgfsetfillcolor{currentfill}%
\pgfsetlinewidth{0.501875pt}%
\definecolor{currentstroke}{rgb}{0.000000,0.000000,0.000000}%
\pgfsetstrokecolor{currentstroke}%
\pgfsetdash{}{0pt}%
\pgfsys@defobject{currentmarker}{\pgfqpoint{0.000000in}{-0.055556in}}{\pgfqpoint{0.000000in}{0.000000in}}{%
\pgfpathmoveto{\pgfqpoint{0.000000in}{0.000000in}}%
\pgfpathlineto{\pgfqpoint{0.000000in}{-0.055556in}}%
\pgfusepath{stroke,fill}%
}%
\begin{pgfscope}%
\pgfsys@transformshift{4.200000in}{1.643478in}%
\pgfsys@useobject{currentmarker}{}%
\end{pgfscope}%
\end{pgfscope}%
\begin{pgfscope}%
\pgftext[left,bottom,x=4.095738in,y=0.437037in,rotate=0.000000]{{\rmfamily\fontsize{12.000000}{14.400000}\selectfont \(\displaystyle 1.0\)}}
%
\end{pgfscope}%
\begin{pgfscope}%
\pgfsetbuttcap%
\pgfsetroundjoin%
\definecolor{currentfill}{rgb}{0.000000,0.000000,0.000000}%
\pgfsetfillcolor{currentfill}%
\pgfsetlinewidth{0.501875pt}%
\definecolor{currentstroke}{rgb}{0.000000,0.000000,0.000000}%
\pgfsetstrokecolor{currentstroke}%
\pgfsetdash{}{0pt}%
\pgfsys@defobject{currentmarker}{\pgfqpoint{0.000000in}{0.000000in}}{\pgfqpoint{0.000000in}{0.055556in}}{%
\pgfpathmoveto{\pgfqpoint{0.000000in}{0.000000in}}%
\pgfpathlineto{\pgfqpoint{0.000000in}{0.055556in}}%
\pgfusepath{stroke,fill}%
}%
\begin{pgfscope}%
\pgfsys@transformshift{5.700000in}{0.600000in}%
\pgfsys@useobject{currentmarker}{}%
\end{pgfscope}%
\end{pgfscope}%
\begin{pgfscope}%
\pgfsetbuttcap%
\pgfsetroundjoin%
\definecolor{currentfill}{rgb}{0.000000,0.000000,0.000000}%
\pgfsetfillcolor{currentfill}%
\pgfsetlinewidth{0.501875pt}%
\definecolor{currentstroke}{rgb}{0.000000,0.000000,0.000000}%
\pgfsetstrokecolor{currentstroke}%
\pgfsetdash{}{0pt}%
\pgfsys@defobject{currentmarker}{\pgfqpoint{0.000000in}{-0.055556in}}{\pgfqpoint{0.000000in}{0.000000in}}{%
\pgfpathmoveto{\pgfqpoint{0.000000in}{0.000000in}}%
\pgfpathlineto{\pgfqpoint{0.000000in}{-0.055556in}}%
\pgfusepath{stroke,fill}%
}%
\begin{pgfscope}%
\pgfsys@transformshift{5.700000in}{1.643478in}%
\pgfsys@useobject{currentmarker}{}%
\end{pgfscope}%
\end{pgfscope}%
\begin{pgfscope}%
\pgftext[left,bottom,x=5.595738in,y=0.437037in,rotate=0.000000]{{\rmfamily\fontsize{12.000000}{14.400000}\selectfont \(\displaystyle 1.5\)}}
%
\end{pgfscope}%
\begin{pgfscope}%
\pgfsetbuttcap%
\pgfsetroundjoin%
\definecolor{currentfill}{rgb}{0.000000,0.000000,0.000000}%
\pgfsetfillcolor{currentfill}%
\pgfsetlinewidth{0.501875pt}%
\definecolor{currentstroke}{rgb}{0.000000,0.000000,0.000000}%
\pgfsetstrokecolor{currentstroke}%
\pgfsetdash{}{0pt}%
\pgfsys@defobject{currentmarker}{\pgfqpoint{0.000000in}{0.000000in}}{\pgfqpoint{0.000000in}{0.055556in}}{%
\pgfpathmoveto{\pgfqpoint{0.000000in}{0.000000in}}%
\pgfpathlineto{\pgfqpoint{0.000000in}{0.055556in}}%
\pgfusepath{stroke,fill}%
}%
\begin{pgfscope}%
\pgfsys@transformshift{7.200000in}{0.600000in}%
\pgfsys@useobject{currentmarker}{}%
\end{pgfscope}%
\end{pgfscope}%
\begin{pgfscope}%
\pgfsetbuttcap%
\pgfsetroundjoin%
\definecolor{currentfill}{rgb}{0.000000,0.000000,0.000000}%
\pgfsetfillcolor{currentfill}%
\pgfsetlinewidth{0.501875pt}%
\definecolor{currentstroke}{rgb}{0.000000,0.000000,0.000000}%
\pgfsetstrokecolor{currentstroke}%
\pgfsetdash{}{0pt}%
\pgfsys@defobject{currentmarker}{\pgfqpoint{0.000000in}{-0.055556in}}{\pgfqpoint{0.000000in}{0.000000in}}{%
\pgfpathmoveto{\pgfqpoint{0.000000in}{0.000000in}}%
\pgfpathlineto{\pgfqpoint{0.000000in}{-0.055556in}}%
\pgfusepath{stroke,fill}%
}%
\begin{pgfscope}%
\pgfsys@transformshift{7.200000in}{1.643478in}%
\pgfsys@useobject{currentmarker}{}%
\end{pgfscope}%
\end{pgfscope}%
\begin{pgfscope}%
\pgftext[left,bottom,x=7.095738in,y=0.437037in,rotate=0.000000]{{\rmfamily\fontsize{12.000000}{14.400000}\selectfont \(\displaystyle 2.0\)}}
%
\end{pgfscope}%
\begin{pgfscope}%
\pgfsetbuttcap%
\pgfsetroundjoin%
\definecolor{currentfill}{rgb}{0.000000,0.000000,0.000000}%
\pgfsetfillcolor{currentfill}%
\pgfsetlinewidth{0.501875pt}%
\definecolor{currentstroke}{rgb}{0.000000,0.000000,0.000000}%
\pgfsetstrokecolor{currentstroke}%
\pgfsetdash{}{0pt}%
\pgfsys@defobject{currentmarker}{\pgfqpoint{0.000000in}{0.000000in}}{\pgfqpoint{0.055556in}{0.000000in}}{%
\pgfpathmoveto{\pgfqpoint{0.000000in}{0.000000in}}%
\pgfpathlineto{\pgfqpoint{0.055556in}{0.000000in}}%
\pgfusepath{stroke,fill}%
}%
\begin{pgfscope}%
\pgfsys@transformshift{1.200000in}{0.600000in}%
\pgfsys@useobject{currentmarker}{}%
\end{pgfscope}%
\end{pgfscope}%
\begin{pgfscope}%
\pgftext[left,bottom,x=0.479905in,y=0.539352in,rotate=0.000000]{{\rmfamily\fontsize{12.000000}{14.400000}\selectfont \(\displaystyle -0.00007\)}}
%
\end{pgfscope}%
\begin{pgfscope}%
\pgfsetbuttcap%
\pgfsetroundjoin%
\definecolor{currentfill}{rgb}{0.000000,0.000000,0.000000}%
\pgfsetfillcolor{currentfill}%
\pgfsetlinewidth{0.501875pt}%
\definecolor{currentstroke}{rgb}{0.000000,0.000000,0.000000}%
\pgfsetstrokecolor{currentstroke}%
\pgfsetdash{}{0pt}%
\pgfsys@defobject{currentmarker}{\pgfqpoint{0.000000in}{0.000000in}}{\pgfqpoint{0.055556in}{0.000000in}}{%
\pgfpathmoveto{\pgfqpoint{0.000000in}{0.000000in}}%
\pgfpathlineto{\pgfqpoint{0.055556in}{0.000000in}}%
\pgfusepath{stroke,fill}%
}%
\begin{pgfscope}%
\pgfsys@transformshift{1.200000in}{0.749068in}%
\pgfsys@useobject{currentmarker}{}%
\end{pgfscope}%
\end{pgfscope}%
\begin{pgfscope}%
\pgftext[left,bottom,x=0.479905in,y=0.688420in,rotate=0.000000]{{\rmfamily\fontsize{12.000000}{14.400000}\selectfont \(\displaystyle -0.00006\)}}
%
\end{pgfscope}%
\begin{pgfscope}%
\pgfsetbuttcap%
\pgfsetroundjoin%
\definecolor{currentfill}{rgb}{0.000000,0.000000,0.000000}%
\pgfsetfillcolor{currentfill}%
\pgfsetlinewidth{0.501875pt}%
\definecolor{currentstroke}{rgb}{0.000000,0.000000,0.000000}%
\pgfsetstrokecolor{currentstroke}%
\pgfsetdash{}{0pt}%
\pgfsys@defobject{currentmarker}{\pgfqpoint{0.000000in}{0.000000in}}{\pgfqpoint{0.055556in}{0.000000in}}{%
\pgfpathmoveto{\pgfqpoint{0.000000in}{0.000000in}}%
\pgfpathlineto{\pgfqpoint{0.055556in}{0.000000in}}%
\pgfusepath{stroke,fill}%
}%
\begin{pgfscope}%
\pgfsys@transformshift{1.200000in}{0.898137in}%
\pgfsys@useobject{currentmarker}{}%
\end{pgfscope}%
\end{pgfscope}%
\begin{pgfscope}%
\pgftext[left,bottom,x=0.479905in,y=0.837489in,rotate=0.000000]{{\rmfamily\fontsize{12.000000}{14.400000}\selectfont \(\displaystyle -0.00005\)}}
%
\end{pgfscope}%
\begin{pgfscope}%
\pgfsetbuttcap%
\pgfsetroundjoin%
\definecolor{currentfill}{rgb}{0.000000,0.000000,0.000000}%
\pgfsetfillcolor{currentfill}%
\pgfsetlinewidth{0.501875pt}%
\definecolor{currentstroke}{rgb}{0.000000,0.000000,0.000000}%
\pgfsetstrokecolor{currentstroke}%
\pgfsetdash{}{0pt}%
\pgfsys@defobject{currentmarker}{\pgfqpoint{0.000000in}{0.000000in}}{\pgfqpoint{0.055556in}{0.000000in}}{%
\pgfpathmoveto{\pgfqpoint{0.000000in}{0.000000in}}%
\pgfpathlineto{\pgfqpoint{0.055556in}{0.000000in}}%
\pgfusepath{stroke,fill}%
}%
\begin{pgfscope}%
\pgfsys@transformshift{1.200000in}{1.047205in}%
\pgfsys@useobject{currentmarker}{}%
\end{pgfscope}%
\end{pgfscope}%
\begin{pgfscope}%
\pgftext[left,bottom,x=0.479905in,y=0.986557in,rotate=0.000000]{{\rmfamily\fontsize{12.000000}{14.400000}\selectfont \(\displaystyle -0.00004\)}}
%
\end{pgfscope}%
\begin{pgfscope}%
\pgfsetbuttcap%
\pgfsetroundjoin%
\definecolor{currentfill}{rgb}{0.000000,0.000000,0.000000}%
\pgfsetfillcolor{currentfill}%
\pgfsetlinewidth{0.501875pt}%
\definecolor{currentstroke}{rgb}{0.000000,0.000000,0.000000}%
\pgfsetstrokecolor{currentstroke}%
\pgfsetdash{}{0pt}%
\pgfsys@defobject{currentmarker}{\pgfqpoint{0.000000in}{0.000000in}}{\pgfqpoint{0.055556in}{0.000000in}}{%
\pgfpathmoveto{\pgfqpoint{0.000000in}{0.000000in}}%
\pgfpathlineto{\pgfqpoint{0.055556in}{0.000000in}}%
\pgfusepath{stroke,fill}%
}%
\begin{pgfscope}%
\pgfsys@transformshift{1.200000in}{1.196273in}%
\pgfsys@useobject{currentmarker}{}%
\end{pgfscope}%
\end{pgfscope}%
\begin{pgfscope}%
\pgftext[left,bottom,x=0.479905in,y=1.135625in,rotate=0.000000]{{\rmfamily\fontsize{12.000000}{14.400000}\selectfont \(\displaystyle -0.00003\)}}
%
\end{pgfscope}%
\begin{pgfscope}%
\pgfsetbuttcap%
\pgfsetroundjoin%
\definecolor{currentfill}{rgb}{0.000000,0.000000,0.000000}%
\pgfsetfillcolor{currentfill}%
\pgfsetlinewidth{0.501875pt}%
\definecolor{currentstroke}{rgb}{0.000000,0.000000,0.000000}%
\pgfsetstrokecolor{currentstroke}%
\pgfsetdash{}{0pt}%
\pgfsys@defobject{currentmarker}{\pgfqpoint{0.000000in}{0.000000in}}{\pgfqpoint{0.055556in}{0.000000in}}{%
\pgfpathmoveto{\pgfqpoint{0.000000in}{0.000000in}}%
\pgfpathlineto{\pgfqpoint{0.055556in}{0.000000in}}%
\pgfusepath{stroke,fill}%
}%
\begin{pgfscope}%
\pgfsys@transformshift{1.200000in}{1.345342in}%
\pgfsys@useobject{currentmarker}{}%
\end{pgfscope}%
\end{pgfscope}%
\begin{pgfscope}%
\pgftext[left,bottom,x=0.479905in,y=1.284694in,rotate=0.000000]{{\rmfamily\fontsize{12.000000}{14.400000}\selectfont \(\displaystyle -0.00002\)}}
%
\end{pgfscope}%
\begin{pgfscope}%
\pgfsetbuttcap%
\pgfsetroundjoin%
\definecolor{currentfill}{rgb}{0.000000,0.000000,0.000000}%
\pgfsetfillcolor{currentfill}%
\pgfsetlinewidth{0.501875pt}%
\definecolor{currentstroke}{rgb}{0.000000,0.000000,0.000000}%
\pgfsetstrokecolor{currentstroke}%
\pgfsetdash{}{0pt}%
\pgfsys@defobject{currentmarker}{\pgfqpoint{0.000000in}{0.000000in}}{\pgfqpoint{0.055556in}{0.000000in}}{%
\pgfpathmoveto{\pgfqpoint{0.000000in}{0.000000in}}%
\pgfpathlineto{\pgfqpoint{0.055556in}{0.000000in}}%
\pgfusepath{stroke,fill}%
}%
\begin{pgfscope}%
\pgfsys@transformshift{1.200000in}{1.494410in}%
\pgfsys@useobject{currentmarker}{}%
\end{pgfscope}%
\end{pgfscope}%
\begin{pgfscope}%
\pgftext[left,bottom,x=0.479905in,y=1.433762in,rotate=0.000000]{{\rmfamily\fontsize{12.000000}{14.400000}\selectfont \(\displaystyle -0.00001\)}}
%
\end{pgfscope}%
\begin{pgfscope}%
\pgfsetbuttcap%
\pgfsetroundjoin%
\definecolor{currentfill}{rgb}{0.000000,0.000000,0.000000}%
\pgfsetfillcolor{currentfill}%
\pgfsetlinewidth{0.501875pt}%
\definecolor{currentstroke}{rgb}{0.000000,0.000000,0.000000}%
\pgfsetstrokecolor{currentstroke}%
\pgfsetdash{}{0pt}%
\pgfsys@defobject{currentmarker}{\pgfqpoint{0.000000in}{0.000000in}}{\pgfqpoint{0.055556in}{0.000000in}}{%
\pgfpathmoveto{\pgfqpoint{0.000000in}{0.000000in}}%
\pgfpathlineto{\pgfqpoint{0.055556in}{0.000000in}}%
\pgfusepath{stroke,fill}%
}%
\begin{pgfscope}%
\pgfsys@transformshift{1.200000in}{1.643478in}%
\pgfsys@useobject{currentmarker}{}%
\end{pgfscope}%
\end{pgfscope}%
\begin{pgfscope}%
\pgftext[left,bottom,x=0.609535in,y=1.589775in,rotate=0.000000]{{\rmfamily\fontsize{12.000000}{14.400000}\selectfont \(\displaystyle 0.00000\)}}
%
\end{pgfscope}%
\begin{pgfscope}%
\pgfsetrectcap%
\pgfsetroundjoin%
\pgfsetlinewidth{1.003750pt}%
\definecolor{currentstroke}{rgb}{0.000000,0.000000,0.000000}%
\pgfsetstrokecolor{currentstroke}%
\pgfsetdash{}{0pt}%
\pgfpathmoveto{\pgfqpoint{1.200000in}{1.643478in}}%
\pgfpathlineto{\pgfqpoint{7.200000in}{1.643478in}}%
\pgfusepath{stroke}%
\end{pgfscope}%
\begin{pgfscope}%
\pgfsetrectcap%
\pgfsetroundjoin%
\pgfsetlinewidth{1.003750pt}%
\definecolor{currentstroke}{rgb}{0.000000,0.000000,0.000000}%
\pgfsetstrokecolor{currentstroke}%
\pgfsetdash{}{0pt}%
\pgfpathmoveto{\pgfqpoint{7.200000in}{0.600000in}}%
\pgfpathlineto{\pgfqpoint{7.200000in}{1.643478in}}%
\pgfusepath{stroke}%
\end{pgfscope}%
\begin{pgfscope}%
\pgfsetrectcap%
\pgfsetroundjoin%
\pgfsetlinewidth{1.003750pt}%
\definecolor{currentstroke}{rgb}{0.000000,0.000000,0.000000}%
\pgfsetstrokecolor{currentstroke}%
\pgfsetdash{}{0pt}%
\pgfpathmoveto{\pgfqpoint{1.200000in}{0.600000in}}%
\pgfpathlineto{\pgfqpoint{7.200000in}{0.600000in}}%
\pgfusepath{stroke}%
\end{pgfscope}%
\begin{pgfscope}%
\pgfsetrectcap%
\pgfsetroundjoin%
\pgfsetlinewidth{1.003750pt}%
\definecolor{currentstroke}{rgb}{0.000000,0.000000,0.000000}%
\pgfsetstrokecolor{currentstroke}%
\pgfsetdash{}{0pt}%
\pgfpathmoveto{\pgfqpoint{1.200000in}{0.600000in}}%
\pgfpathlineto{\pgfqpoint{1.200000in}{1.643478in}}%
\pgfusepath{stroke}%
\end{pgfscope}%
\begin{pgfscope}%
\pgfpathrectangle{\pgfqpoint{1.200000in}{0.600000in}}{\pgfqpoint{6.000000in}{1.043478in}} %
\pgfusepath{clip}%
\pgfsetrectcap%
\pgfsetroundjoin%
\pgfsetlinewidth{1.003750pt}%
\definecolor{currentstroke}{rgb}{0.000000,0.000000,1.000000}%
\pgfsetstrokecolor{currentstroke}%
\pgfsetdash{}{0pt}%
\pgfpathmoveto{\pgfqpoint{1.200000in}{1.643478in}}%
\pgfpathlineto{\pgfqpoint{3.990000in}{1.643478in}}%
\pgfpathlineto{\pgfqpoint{4.050000in}{1.434783in}}%
\pgfpathlineto{\pgfqpoint{4.140000in}{0.808696in}}%
\pgfpathlineto{\pgfqpoint{4.200000in}{0.600000in}}%
\pgfpathlineto{\pgfqpoint{4.260000in}{0.808696in}}%
\pgfpathlineto{\pgfqpoint{4.320000in}{1.226087in}}%
\pgfpathlineto{\pgfqpoint{4.350000in}{1.330435in}}%
\pgfpathlineto{\pgfqpoint{4.380000in}{1.539130in}}%
\pgfpathlineto{\pgfqpoint{4.410000in}{1.643478in}}%
\pgfpathlineto{\pgfqpoint{7.200000in}{1.643478in}}%
\pgfpathlineto{\pgfqpoint{7.200000in}{1.643478in}}%
\pgfusepath{stroke}%
\end{pgfscope}%
\begin{pgfscope}%
\pgfsetbuttcap%
\pgfsetroundjoin%
\definecolor{currentfill}{rgb}{0.000000,0.000000,0.000000}%
\pgfsetfillcolor{currentfill}%
\pgfsetlinewidth{0.501875pt}%
\definecolor{currentstroke}{rgb}{0.000000,0.000000,0.000000}%
\pgfsetstrokecolor{currentstroke}%
\pgfsetdash{}{0pt}%
\pgfsys@defobject{currentmarker}{\pgfqpoint{-0.055556in}{0.000000in}}{\pgfqpoint{0.000000in}{0.000000in}}{%
\pgfpathmoveto{\pgfqpoint{0.000000in}{0.000000in}}%
\pgfpathlineto{\pgfqpoint{-0.055556in}{0.000000in}}%
\pgfusepath{stroke,fill}%
}%
\begin{pgfscope}%
\pgfsys@transformshift{7.200000in}{0.600000in}%
\pgfsys@useobject{currentmarker}{}%
\end{pgfscope}%
\end{pgfscope}%
\begin{pgfscope}%
\pgftext[left,bottom,x=7.255556in,y=0.546296in,rotate=0.000000]{{\rmfamily\fontsize{12.000000}{14.400000}\selectfont \(\displaystyle 0.0\)}}
%
\end{pgfscope}%
\begin{pgfscope}%
\pgfsetbuttcap%
\pgfsetroundjoin%
\definecolor{currentfill}{rgb}{0.000000,0.000000,0.000000}%
\pgfsetfillcolor{currentfill}%
\pgfsetlinewidth{0.501875pt}%
\definecolor{currentstroke}{rgb}{0.000000,0.000000,0.000000}%
\pgfsetstrokecolor{currentstroke}%
\pgfsetdash{}{0pt}%
\pgfsys@defobject{currentmarker}{\pgfqpoint{-0.055556in}{0.000000in}}{\pgfqpoint{0.000000in}{0.000000in}}{%
\pgfpathmoveto{\pgfqpoint{0.000000in}{0.000000in}}%
\pgfpathlineto{\pgfqpoint{-0.055556in}{0.000000in}}%
\pgfusepath{stroke,fill}%
}%
\begin{pgfscope}%
\pgfsys@transformshift{7.200000in}{0.808696in}%
\pgfsys@useobject{currentmarker}{}%
\end{pgfscope}%
\end{pgfscope}%
\begin{pgfscope}%
\pgftext[left,bottom,x=7.255556in,y=0.754992in,rotate=0.000000]{{\rmfamily\fontsize{12.000000}{14.400000}\selectfont \(\displaystyle 0.2\)}}
%
\end{pgfscope}%
\begin{pgfscope}%
\pgfsetbuttcap%
\pgfsetroundjoin%
\definecolor{currentfill}{rgb}{0.000000,0.000000,0.000000}%
\pgfsetfillcolor{currentfill}%
\pgfsetlinewidth{0.501875pt}%
\definecolor{currentstroke}{rgb}{0.000000,0.000000,0.000000}%
\pgfsetstrokecolor{currentstroke}%
\pgfsetdash{}{0pt}%
\pgfsys@defobject{currentmarker}{\pgfqpoint{-0.055556in}{0.000000in}}{\pgfqpoint{0.000000in}{0.000000in}}{%
\pgfpathmoveto{\pgfqpoint{0.000000in}{0.000000in}}%
\pgfpathlineto{\pgfqpoint{-0.055556in}{0.000000in}}%
\pgfusepath{stroke,fill}%
}%
\begin{pgfscope}%
\pgfsys@transformshift{7.200000in}{1.017391in}%
\pgfsys@useobject{currentmarker}{}%
\end{pgfscope}%
\end{pgfscope}%
\begin{pgfscope}%
\pgftext[left,bottom,x=7.255556in,y=0.963688in,rotate=0.000000]{{\rmfamily\fontsize{12.000000}{14.400000}\selectfont \(\displaystyle 0.4\)}}
%
\end{pgfscope}%
\begin{pgfscope}%
\pgfsetbuttcap%
\pgfsetroundjoin%
\definecolor{currentfill}{rgb}{0.000000,0.000000,0.000000}%
\pgfsetfillcolor{currentfill}%
\pgfsetlinewidth{0.501875pt}%
\definecolor{currentstroke}{rgb}{0.000000,0.000000,0.000000}%
\pgfsetstrokecolor{currentstroke}%
\pgfsetdash{}{0pt}%
\pgfsys@defobject{currentmarker}{\pgfqpoint{-0.055556in}{0.000000in}}{\pgfqpoint{0.000000in}{0.000000in}}{%
\pgfpathmoveto{\pgfqpoint{0.000000in}{0.000000in}}%
\pgfpathlineto{\pgfqpoint{-0.055556in}{0.000000in}}%
\pgfusepath{stroke,fill}%
}%
\begin{pgfscope}%
\pgfsys@transformshift{7.200000in}{1.226087in}%
\pgfsys@useobject{currentmarker}{}%
\end{pgfscope}%
\end{pgfscope}%
\begin{pgfscope}%
\pgftext[left,bottom,x=7.255556in,y=1.172383in,rotate=0.000000]{{\rmfamily\fontsize{12.000000}{14.400000}\selectfont \(\displaystyle 0.6\)}}
%
\end{pgfscope}%
\begin{pgfscope}%
\pgfsetbuttcap%
\pgfsetroundjoin%
\definecolor{currentfill}{rgb}{0.000000,0.000000,0.000000}%
\pgfsetfillcolor{currentfill}%
\pgfsetlinewidth{0.501875pt}%
\definecolor{currentstroke}{rgb}{0.000000,0.000000,0.000000}%
\pgfsetstrokecolor{currentstroke}%
\pgfsetdash{}{0pt}%
\pgfsys@defobject{currentmarker}{\pgfqpoint{-0.055556in}{0.000000in}}{\pgfqpoint{0.000000in}{0.000000in}}{%
\pgfpathmoveto{\pgfqpoint{0.000000in}{0.000000in}}%
\pgfpathlineto{\pgfqpoint{-0.055556in}{0.000000in}}%
\pgfusepath{stroke,fill}%
}%
\begin{pgfscope}%
\pgfsys@transformshift{7.200000in}{1.434783in}%
\pgfsys@useobject{currentmarker}{}%
\end{pgfscope}%
\end{pgfscope}%
\begin{pgfscope}%
\pgftext[left,bottom,x=7.255556in,y=1.381079in,rotate=0.000000]{{\rmfamily\fontsize{12.000000}{14.400000}\selectfont \(\displaystyle 0.8\)}}
%
\end{pgfscope}%
\begin{pgfscope}%
\pgfsetbuttcap%
\pgfsetroundjoin%
\definecolor{currentfill}{rgb}{0.000000,0.000000,0.000000}%
\pgfsetfillcolor{currentfill}%
\pgfsetlinewidth{0.501875pt}%
\definecolor{currentstroke}{rgb}{0.000000,0.000000,0.000000}%
\pgfsetstrokecolor{currentstroke}%
\pgfsetdash{}{0pt}%
\pgfsys@defobject{currentmarker}{\pgfqpoint{-0.055556in}{0.000000in}}{\pgfqpoint{0.000000in}{0.000000in}}{%
\pgfpathmoveto{\pgfqpoint{0.000000in}{0.000000in}}%
\pgfpathlineto{\pgfqpoint{-0.055556in}{0.000000in}}%
\pgfusepath{stroke,fill}%
}%
\begin{pgfscope}%
\pgfsys@transformshift{7.200000in}{1.643478in}%
\pgfsys@useobject{currentmarker}{}%
\end{pgfscope}%
\end{pgfscope}%
\begin{pgfscope}%
\pgftext[left,bottom,x=7.255556in,y=1.589775in,rotate=0.000000]{{\rmfamily\fontsize{12.000000}{14.400000}\selectfont \(\displaystyle 1.0\)}}
%
\end{pgfscope}%
\begin{pgfscope}%
\pgftext[left,bottom,x=0.217870in,y=2.643030in,rotate=90.000000]{{\rmfamily\fontsize{12.000000}{14.400000}\selectfont Deflection}}
%
\end{pgfscope}%
\begin{pgfscope}%
\pgftext[left,bottom,x=7.737870in,y=2.549011in,rotate=90.000000]{{\rmfamily\fontsize{12.000000}{14.400000}\selectfont Node Health}}
%
\end{pgfscope}%
\end{pgfpicture}%
\makeatother%
\endgroup%
}
  \caption{A brittle beam with prescribed center displacement}
  \label{fig:brittleBeam}
\end{figure}

Unlike a local model, partial failure is observed at nodes near the plastic hinge, as pairs of bonds that straddle the hinge are broken.

\section{Conclusion}
\label{sec:Conclusion}

As far as we know, this the first peridynamic state based thin feature model, and results in accurate deformation results for simple beam tests.
The proposed damage model successfully reproduces the impact of nonlinear elasticity on deformation of a rectangular cantilever, and the framework is laid to allow application of the same model to I-beams.
It simplifies treatment of bending in beams, and is extensible to bending in plates.


\bibliographystyle{elsarticle-num}
\bibliography{jogrady_bibdesk}

\end{document} 
\end
