\documentclass{article}
\usepackage[left=1.0in,right=1.0in,top=1.0in,bottom=1.0in]{geometry}
\usepackage{fancyhdr}
\usepackage{color}
\usepackage{amsmath}
%\usepackage[normalem]{ulem}
\usepackage{cancel}
\definecolor{light-gray}{gray}{0.5}


\begin{document}

\begin{center}
    {\LARGE \bf Point-by-Point Response to Reviewer's Comments} \\
     Author's Responses in {\color{red} Red}
\end{center}

%%%% First Reviewer %%%%
\section{Reviewer}
%
\subsection*{General Statement}
Let me state at the outset that this is a welcome paper adressing a problem with much educational (and perhaps practical) value. It should certainly be published after revision to address some issues (some minor, some less so).
In this paper the authors use the basic concepts of the peridynamic theory (of 3D solids) to develop and study a similar model for beams. Their approach is not one which uses the 3D peridynamic theory to look at a beam as a 3D structure but rather one where a peridynamic-like approach is developed and applied to examining the 1D theory of bending (with of course 3D considerations used and averaged across the cross-section to develop the constitutive law). This seems to me to be a good way to proceed.
%
\subsection*{Specific Comments}

\begin{enumerate}
%
   \item There is some literature on peridynamics and beams which the authors are not aware of, e.g. A displacement based framework for the micromechanics of elastic beams by G. Failla, A. Sofi and M. Zingales which is to (or has already) appeared in IJSS.
   {\color{red}
     \begin{itemize}
        \item We were unable to find the specific paper mentioned, but a search revealed interesting and relevant work by Paola et al. that is definitely worth mentioning and has been added to section 1.1
     \end{itemize}}
%
  \item The authors mention in the Introduction that the Eringen model cannot reproduce the scale stiffening observed in nanoscale beams. Does their model reproduce the correct effects? Can they compare with experiments?
   {\color{red}
     \begin{itemize}
        \item In the last paragraph of Appendix B (previously Section 3), it is noted that:
        \begin{quote}
        It is therefore unsurprising that, like Eringen's nonlocal elasticity [Challamel2008], this peridynamic bending model fails to predict the stiffening associated with nanoscale cantilevers.''
        \end{quote}
        \item More generally, this model, like Eringen's model, predicts relative softening of nanoscale beams. Its advantages lie in the lack of spatial derivatives.
     \end{itemize}}
%
  \item In section 1.2 they say that �The term peridynamic was coined by Silling in [11]...� while this is correct, Silling did much more than coin the term in that paper. He laid out the theory there for the first time. He should be given credit for more than coining a term.
  {\color{red}
     \begin{itemize}
        \item Very true! We took Silling's creation of the field for common knowledge and have corrected the first line of 1.2 to more accurately credit him.
     \end{itemize}}
%
     \item It is not clear to me how the authors use the energy $w = \alpha[1 + cos\theta]$ in their derivation of equation (2). Since equation (2) relates force states and kinematic states it must be based on some constitutive hypothesis (presumably this energy function). The authors should provide more detail on the derivation of (2) and its relation to w.
  {\color{red}
     \begin{itemize}
        \item The derivation has been added along as Appendix A.
     \end{itemize}}
%
     \item I am confused in section 2.1 when the energy suddenly becomes $w = \omega(\xi)\alpha[1 + cos \theta]$ rather than the one they had earlier $w = \alpha[1 + cos\theta]$ . Isn�t it the former (with the $\omega$) the one that is fundamental to this peridynamic-like model and shouldn�t it be introduced at the outset? Shouldn�t it be used, e.g. in the discussion leading to (2) and not $w = \alpha[1 + cos\theta]$ .
  {\color{red}
     \begin{itemize}
        \item The weighting function is now included from before equation (2) in the definition of the model.
        \end{itemize}}
%
  \item In Section 4.2 the authors say the uniformly loaded beam is simply-supported at the two ends. Surely this is not adequate for the peridynamic model? Or if it is, there should be a careful explanation of why this is so. In general the formulation of the boundary conditions is intimately tied up with the sort of theory being used. For example in the strain gradient theory of elasticity the boundary conditions are different to those of the classical theory of elasticity. Or in the Timoshenko beam theory it is not enough the say that the displacement and bending moment vanish at a support.
%
     {\color{red}
     \begin{itemize}
        \item Added significant detail to the discussion of boundary conditions.
     \end{itemize}}
%
\end{enumerate}

\section{Reviewer}
\subsection*{General Statement}

%
\subsection*{Specific Comments}

\begin{enumerate}

  \item A remarkable feature of the proposed beam model is that it consists only of a material model � the equilibrium equation is the same as for a continuum. The authors should consider highlighting this fact as compared with traditional beam theories, which rely on deriving new equilibrium equations.
%
     {\color{red}
     \begin{itemize}
        \item A line added to the second paragraph of the Introduction makes this clearer:
        \begin{quote}
        \underline{Unlike many continuum beam theories that derive new equations of motion (such as fourth} \underline{order PDE's) from the 3D elastic constitutive model, the new} \cancel{This} model is not derived from prior ordinary peridynamic models based on bond extension, but \underline{is a material model that} directly resists bending deformation \underline{while maintaining the same conservation of momentum} \underline{equation as the 3D model}.
        \end{quote}
     \end{itemize}}
%
  \item In equation (1), define u.
%
     {\color{red}
     \begin{itemize}
        \item Added to the line above equation (1)
     \end{itemize}}
%
  \item I�m not sure I understand the significance of the particular choice of weighting function in equation (6). It seems like there could be many choices that reproduce EI in the standard theory. Please clarify.
%
     {\color{red}
     \begin{itemize}
        \item Added the following sentence, which should clarify this issue.
        \begin{quote}
While any function $\omega(\xi)$ that produces a convergent integral for $m$ will reproduce an elastic Euler beam, a physically meaningful choice of $\omega$ will allow us to extend our model to certain inelastic behaviors.
        \end{quote}
     \end{itemize}}
%
  \item The relationship with Eringen�s model is interesting but a bit of a distraction, and it could safely be moved to an appendix where it would not interrupt the main development.
%
     {\color{red}
     \begin{itemize}
        \item The section has been moved to Appendix B
     \end{itemize}}
%
  \item The symbol t (not bold) is used with multiple meanings, please disambiguate.
%
     {\color{red}
     \begin{itemize}
        \item Substituted $t_\text{1D}$ and $\sigma_\text{1D}$ for the the non-bold instances of t and $\sigma$ from Eringen's work.
     \end{itemize}}
%
  \item In section 4.1, it should be stated that the model is for a particular choice of horizon, which is $\delta = 2\xi$. Also, I don�t see where $\omega$ is defined.
%
     {\color{red}
     \begin{itemize}
        \item The model leaves $\delta$ and $\omega$ undetermined. The first equations of section 3.1 (formerly 4.1) have been altered and a note added to make that clear.
        \begin{quote}
          \underline{in which $\boldsymbol{\xi}_i$ is the $i^\textnormal{th}$ bond emanating from the point $\mathbf{x}$ to each of the $n$ points within distance} \underline{ $\delta$ of point $\mathbf{x}$.}
        \end{quote}
     \end{itemize}}
%
  \item In section 4.2, the sentence �Boundary conditions... require careful treatment...� is vague. Please elaborate on how boundary conditions are implemented.
%
     {\color{red}
     \begin{itemize}
        \item Added significant detail to the discussion of boundary conditions.
     \end{itemize}}
%
  \item In section 4.3, it should be stated that the EPP model is the one discussed in section 2.2, so readers can easily find this.
%
     {\color{red}
     \begin{itemize}
        \item Reference to EPP model equation added
     \end{itemize}}
%
  \item I don�t understand why the residual displacements in Figure 9 are greater than the loaded displacements in Figure 8. Are these two different loading conditions?
%
     {\color{red}
     \begin{itemize}
        \item The residual displacements are actually a little more then 10\% of the loaded displacements, so the plots are on difference scales. Text has been added to make that clearer.
     \end{itemize}}
%
  \item Please define the material and damage model used in Figure 11; also define �node health�. Alternatively, since damage modeling within the new beam theory is sure to be an interesting and deep subject that merits careful analysis, the authors may wish to consider writing a separate paper about this.
%
     {\color{red}
     \begin{itemize}
        \item Added a short description of the brittle model to the end of section 2.2. Node health is defined in the text referencing the model, but is edited for clarity
        \begin{quote}
          This is borne out by the results in figure 11, in which ``Nodal Health'' represents the fraction of bond-pairs about each node that have \underline{never exceeded their critical angle and therefore } \underline{have} not failed.
        \end{quote}
     \end{itemize}}
%
  \item Similarly, in section 5, the authors should not refer to �the proposed damage model� since there isn�t really a discussion of a damage model in the paper.
%
     {\color{red}
     \begin{itemize}
        \item Changed the wording of the conclusion:
        \begin{quote}
          The \cancel{proposed} \underline{perfect plasticity and simple brittle} damage models successfully reproduce the impact of nonlinear \cancel{elasticity} \underline{behaviors} on deformation of a rectangular cantilever, and the framework is laid to allow application of the same models to I-beams.
        \end{quote}
     \end{itemize}}
%
\end{enumerate}


\end{document}
